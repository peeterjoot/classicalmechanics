%
% Copyright � 2012 Peeter Joot.  All Rights Reserved.
% Licenced as described in the file LICENSE under the root directory of this GIT repository.
%
%
%\chapter{Runge-Lenz vector conservation}
\index{Runge-Lenz vector}
\label{chap:RungeLenz}
%\blogpage{http://sites.google.com/site/peeterjoot2/math2012/RungeLenz.pdf}
%\date{Feb 11, 2012}
%
\section{Motivation.}
%
Notes from Prof. Poppitz's phy354 classical mechanics lecture on the Runge-Lenz vector, a less well known conserved quantity for the 3D \(1/r\) potentials that can be used to solve the Kepler problem.
%
\section{Motivation: The Kepler problem.}
%
We can plug away at the Lagrangian in cylindrical coordinates and find eventually
%
\begin{equation}\label{eqn:RungeLenz:10}
\int_{\phi_0}^\phi d\phi = \int_{r_0}^r \frac{M}{m r^2} \frac{dr}{\sqrt{\frac{2}{M}
\lr{ E - U + \frac{M^2}{2 m r^2} }
 }}
\end{equation}
%
but this can be messy to solve, where we get elliptic integrals or worse, depending on the potential.
%
For the special case of the 3D problem where the potential has a \(1/r\) form, this is what Prof. Poppitz called ``super-integrable''.  With \(2N - 1 = 5\) conserved quantities to be found, we have got one more.  Here the form of that last conserved quantity is given, called the Runge-Lenz vector, and we verify that it is conserved.
%
\section{Runge-Lenz vector.}
%
Given a potential
%
\begin{equation}\label{eqn:RungeLenz:30}
U = -\frac{\alpha}{r}
\end{equation}
%
and a Lagrangian
%
\begin{equation}\label{eqn:RungeLenz:50}
\begin{aligned}
\LL &= \frac{m \rdot^2}{2} + \inv{2} \frac{M_z^2}{m r^2} - U \\
M_z &= m r^2 \dot{\phi}^2
\end{aligned}
\end{equation}
%
and writing the angular momentum as
\begin{equation}\label{eqn:RungeLenz:70}
\BM = m \Br \cross \Bv %= m r^2 \dot{\phi} \zcap
\end{equation}
%
the Runge-Lenz vector
%
\begin{equation}\label{eqn:RungeLenz:90}
\BA = \Bv \cross \BM - \alpha \rcap,
\end{equation}
%
is a conserved quantity.
%
\subsection{Verify the conservation assumption.}
%
Let us show that the conservation assumption is correct
%
\begin{equation}\label{eqn:RungeLenz:110}
\ddt{} \left( \Bv \cross \BM \right)
=
\ddt{ \Bv} \cross \BM
+ \Bv \cross \cancel{\ddt{\BM }}
\end{equation}
%
Here, we note that angular momentum conservation is really \(d\BM/dt = 0\), so we are left with only the acceleration term, which we can rewrite in terms of the Euler-Lagrange equation
%
\begin{equation}\label{eqn:RungeLenz:430}
\begin{aligned}
\ddt{} \left( \Bv \cross \BM \right)
&=
-\inv{m} \spacegrad U \cross M \\
&=
-\inv{m} \PD{r}{U} \rcap \cross M \\
&=
-\inv{m} \PD{r}{U} \rcap \cross (m \Br \cross \Bv) \\
&=
- \PD{r}{U} \rcap \cross (\Br \cross \Bv)
\end{aligned}
\end{equation}
%
We can compute the double cross product
%
\begin{equation}\label{eqn:RungeLenz:450}
\begin{aligned}
(\Ba \cross (\Bb \cross \Bc) )_i
&=
a_m b_r c_s \epsilon_{r s t} \epsilon_{m t i} \\
&=
a_m b_r c_s \delta^{[rs]}_{i m} \\
&=
a_m b_i c_m
-a_m b_m c_i
\end{aligned}
\end{equation}
%
For
%
\begin{equation}\label{eqn:RungeLenz:130}
\Ba \cross (\Bb \cross \Bc) = (\Ba \cdot \Bc) \Bb -(\Ba \cdot \Bb) \Bc
\end{equation}
%
Plugging this we have
%
\begin{equation}\label{eqn:RungeLenz:470}
\begin{aligned}
\ddt{} \left( \Bv \cross \BM \right)
&=
\PD{r}{U} \
\left(
(\rcap \cdot \Br) \Bv
-(\rcap \cdot \Bv) \Br
\right) \\
&=
\left( \frac{\alpha}{r^2} \right)
\left(
r \Bv
-\inv{r}(\Br \cdot \Bv) \Br
\right) \\
&=
\alpha
\left(
\frac{\Bv}{r}
-\frac{(\Br \cdot \Bv) \Br }{r^3}
\right) .
\end{aligned}
\end{equation}
%
Now let us look at the other term.  We will need the derivative of \(\rcap\)
%
\begin{equation}\label{eqn:RungeLenz:490}
\begin{aligned}
\ddt{\rcap}
&=
\ddt{} \frac{\Br}{r} \\
&=
\frac{\Bv}{r} + \Br \ddt{\inv{r}} \\
&=
\frac{\Bv}{r} - \frac{\Br}{r^2} \ddt{r} \\
&=
\frac{\Bv}{r} - \frac{\Br}{r^2} \ddt{ \sqrt{\Br \cdot \Br}} \\
&=
\frac{\Bv}{r} - \frac{\Br}{r^2} \frac{\Bv \cdot \Br}{\sqrt{\Br^2}} \\
&=
\frac{\Bv}{r} - \frac{\Br}{r^3} \Bv \cdot \Br
\end{aligned}
\end{equation}
%
Putting all the bits together we have now verified the conservation statement
%
\begin{equation}\label{eqn:RungeLenz:150}
\ddt{} \left(
\Bv \cross \BM - \alpha \rcap
\right)
=
\alpha
\left(
\frac{\Bv}{r}
-\frac{(\Br \cdot \Bv) \Br }{r^3}
\right)
-\alpha \left(
\frac{\Bv}{r} - \frac{\Br}{r^3} \Bv \cdot \Br \right)
= 0.
\end{equation}
%
With
%
\begin{equation}\label{eqn:RungeLenz:170}
\ddt{} \left( \Bv \cross \BM - \alpha \rcap \right) = 0,
\end{equation}
%
our vector must be some constant vector.  Let us write this
%
\begin{equation}\label{eqn:RungeLenz:190}
\Bv \cross \BM - \alpha \rcap = \alpha \Be,
\end{equation}
%
so that
%
\boxedEquation{eqn:RungeLenz:210}{
\Bv \cross \BM = \alpha \left(\Be + \rcap \right).
}
%
Dotting \eqnref{eqn:RungeLenz:210} with \(\BM\) we find
%
\begin{equation}\label{eqn:RungeLenz:510}
\begin{aligned}
\alpha \BM \cdot \left(\Be + \rcap \right)
&=
\BM \cdot (\Bv \cross \BM) \\
&= 0
\end{aligned}
\end{equation}
%
With \(\rcap\) lying in the plane of the trajectory (perpendicular to \(\BM\)), we must also have \(\Be\) lying in the plane of the trajectory.
%
Now we can dot \eqnref{eqn:RungeLenz:210} with \(\Br\) to find
%
\begin{equation}\label{eqn:RungeLenz:530}
\begin{aligned}
\Br \cdot (\Bv \cross \BM)
&= \alpha \Br \cdot \left(\Be + \rcap \right) \\
&= \alpha \left( r e \cos(\phi - \phi_0) + r \right) \\
\BM \cdot (\Br \cross \Bv) &= \\
\BM \cdot \frac{\BM}{m} &= \\
\frac{\BM^2}{m} &=
\end{aligned}
\end{equation}
%
This is
%Here we have written \(\rcap = \Be_1 e^{\Be_1 \Be_2 \phi_0}\), and \(\Be = \Be_1 e^{\Be_1 \Be_2 \phi}\), where \(\Be_1\) and \(\Be_2\) are two orthonormal unit vectors in the plane of the tragectory.
%
\begin{equation}\label{eqn:RungeLenz:230}
\frac{\BM^2}{m} = \alpha r \left( 1 + e \cos(\phi - \phi_0) \right).
\end{equation}
%
This is a kind of curious implicit relationship, since \(\phi\) is also a function of \(r\).  Recall that the kinetic portion of our Lagrangian was
%
\begin{equation}\label{eqn:RungeLenz:250}
\inv{2} m
\lr{ \dot{r}^2 + r^2 \dot{\phi}^2 }
\end{equation}
%
so that our angular momentum was
%
\begin{equation}\label{eqn:RungeLenz:270}
M_\phi = \PD{\phidot}{} \left( \inv{2} m r^2 \phidot^2 \right) = m r^2 \phidot,
\end{equation}
%
with no \(\phi\) dependence in the Lagrangian we have
%
\begin{equation}\label{eqn:RungeLenz:290}
\frac{d}{dt}
\lr{ m r^2 \phidot } = 0,
\end{equation}
%
or
%
\begin{equation}\label{eqn:RungeLenz:310}
\BM = m r^2 \phidot \zcap = \text{constant}
\end{equation}
%
Our dynamics are now fully specified, even if this not completely explicit
%
\boxedEquation{eqn:RungeLenz:330}{
\begin{aligned}
r &= \frac{M^2}{m \alpha} \inv{1 + e \cos(\phi - \phi_0)} \\
\frac{d\phi}{dt} &= \frac{M}{ m r^2}.
\end{aligned}
}
%
What we can do is rearrange and separate variables
%
\begin{equation}\label{eqn:RungeLenz:350}
\inv{r^2} = \frac{m^2 \alpha^2}{M^4} (1 + e \cos(\phi - \phi_0))^2 = \frac{m}{M} \frac{d\phi}{dt},
\end{equation}
%
to find
%
\begin{equation}\label{eqn:RungeLenz:370}
t - t_0
=
\frac{M^3}{m \alpha^3}
\int_{\phi_0}^\phi d\phi
\inv{(1 + e \cos(\phi - \phi_0))^2}
=
\frac{M^3}{m \alpha^3}
\int_0^{\phi - \phi_0} du
\inv{(1 + e \cos u)^2}
\end{equation}
%
Now, at least \(\phi = \phi(t)\) is specified implicitly.

We can also use the first of these to determine the magnitude of the radial velocity
%
\begin{equation}\label{eqn:RungeLenz:550}
\begin{aligned}
\frac{dr}{dt}
&=
-\frac{M^2}{m \alpha} \inv{(1 + e \cos(\phi - \phi_0))^2} (-e \sin(\phi - \phi_0)) \frac{d\phi}{dt} \\
&=
\frac{e M^2}{m \alpha} \inv{(1 + e \cos(\phi - \phi_0))^2} \sin(\phi - \phi_0) \frac{M}{m r^2} \\
&=
\frac{e M^3}{m^2 \alpha r^2} \inv{(1 + e \cos(\phi - \phi_0))^2} \sin(\phi - \phi_0) \\
&=
\frac{e M^3}{m^2 \alpha r^2} \left( \frac{ m r \alpha }{M^2} \right)^2 \sin(\phi - \phi_0) \\
&=
\frac{e }{M } \sin(\phi - \phi_0),
\end{aligned}
\end{equation}
%
with this, we can also find the energy
%
\begin{equation}\label{eqn:RungeLenz:570}
\begin{aligned}
E
&= \rdot( m \rdot) + \phidot
\lr{ m r^2 \phidot }
 - \left( \inv{2} m \rdot^2 + \inv{2} m r^2 \phidot^2 - U \right) \\
&= \inv{2} m \rdot^2 + \inv{2} m r^2 \phidot^2 + U  \\
&= \inv{2} m \rdot^2 + \inv{2} m r^2 \phidot^2 - \frac{\alpha}{r} \\
&= \inv{2} m \frac{e^2}{M^2} \sin^2(\phi - \phi_0) + \inv{2 m r^2 } M^2 - \frac{\alpha}{r}.
\end{aligned}
\end{equation}
%
Or
%
\begin{equation}\label{eqn:RungeLenz:390}
E
= \frac{m}{2 M^2} (\Be \cross \rcap)^2 + \inv{2 m r^2 } M^2 - \frac{\alpha}{r}.
\end{equation}
%
Is this what was used in class to state the relation
%
\begin{equation}\label{eqn:RungeLenz:410}
e = \sqrt{1 + \frac{2 E M^2}{m \alpha^2}}.
\end{equation}
%
It is not obvious exactly how that is obtained, but we can go back to \eqnref{eqn:RungeLenz:330} to eliminate the \(e^2 \sin^2 \Delta \phi\) term
%
\begin{equation}\label{eqn:RungeLenz:590}
E
= \inv{2} m \frac{1}{M^2} \left( e^2 - \left( \frac{M^2}{r m \alpha} - 1\right)^2 \right) + \inv{2 m r^2 } M^2 - \frac{\alpha}{r}.
\end{equation}
%
Presumably this simplifies to the desired result (or there is other errors made in that prevent that).
