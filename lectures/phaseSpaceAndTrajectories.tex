%
% Copyright � 2012 Peeter Joot.  All Rights Reserved.
% Licenced as described in the file LICENSE under the root directory of this GIT repository.
%
%
%\chapter{PHY354H1S.  Advanced Classical Mechanics (Taught by Prof. Erich Poppitz).  Phase Space and Trajectories}
%\chapter{Phase Space and Trajectories}
\index{phase space}
\label{chap:phaseSpaceAndTrajectories}
%\blogpage{http://sites.google.com/site/peeterjoot2/math2012/phaseSpaceAndTrajectories.pdf}
%\date{Feb 29, 2012}
%
\section{Phase space and phase trajectories.}
%
The phase space and phase trajectories are the space of \(p\)'s and \(q\)'s of a mechanical system (always even dimensional, with as many \(p\)'s as \(q\)'s for N particles in 3d: 6N dimensional space).

The state of a mechanical system \(\equiv\) the point in phase space.
Time evolution \(\equiv\) a curve in phase space.

Example: 1 dim system, say a harmonic oscillator.
%
\begin{equation}\label{eqn:phaseSpaceAndTrajectories:10}
H = \frac{p^2}{2m} + \frac{1}{2} m \omega^2 q^2
\end{equation}
%
Our phase space can be illustrated as an ellipse as in \cref{fig:phaseSpaceAndTrajectories:phaseSpaceAndTrajectoriesFig1}
\imageFigure{../figures/classicalmechanics/phaseSpaceAndTrajectoriesFig1}{Harmonic oscillator phase space trajectory.}{fig:phaseSpaceAndTrajectories:phaseSpaceAndTrajectoriesFig1}{0.2}
%
where the phase space trajectories of the SHO.  The equation describing the ellipse is
%
\begin{equation}\label{eqn:phaseSpaceAndTrajectories:30}
E = \frac{p^2}{2m} + \frac{1}{2} m \omega^2 q^2,
\end{equation}
%
which we can put into standard elliptical form as
%
\begin{equation}\label{eqn:phaseSpaceAndTrajectories:50}
1 = \left( \frac{p}{\sqrt{2 m E}}\right)^2 + \left(\sqrt{\frac{m}{2 E}} \omega\right) q^2
\end{equation}
%
\subsection{Applications of \texorpdfstring{\(H\).}{H}}
%
\begin{itemize}
\item Classical stat mech.
\item transition into QM via Poisson brackets.
\item mathematical theorems about phase space ``flow''.
\item perturbation theory.
\end{itemize}
%
\subsection{Poisson brackets.}
\index{Poisson bracket}
%
Poisson brackets arises very naturally if one asks about the time evolution of a function \(f(p, q, t)\) on phase space.
%
\begin{equation}\label{eqn:phaseSpaceAndTrajectories:530}
\begin{aligned}
\ddt{} f(p_i, q_i, t)
&=
\sum_i
 \PD{p_i}{f} \PD{t}{p_i}
+ \PD{q_i}{f} \PD{t}{q_i}
+ \PD{t}{f} \\
&=
\sum_i
- \PD{p_i}{f} \PD{q_i}{H}
+ \PD{q_i}{f} \PD{p_i}{H}
+ \PD{t}{f}
\end{aligned}
\end{equation}
%
Define the commutator of \(H\) and \(f\) as
%
\begin{equation}\label{eqn:phaseSpaceAndTrajectories:70}
\antisymmetric{H}{f} =
\sum_i
\PD{p_i}{H}
\PD{q_i}{f}
-
\PD{q_i}{H}
\PD{p_i}{f}
\end{equation}
%
This is the Poisson bracket of \(H(p,q,t)\) with \(f(p,q,t)\), defined for arbitrary functions on phase space.

Note that other conventions for sign exist (apparently in Landau and Lifshitz uses the opposite).

So we have
%
\begin{equation}\label{eqn:phaseSpaceAndTrajectories:90}
\ddt{} f(p_i, q_i, t) = \antisymmetric{H}{f} + \PD{t}{f}.
\end{equation}
%
Corollaries:

If \(f\) has no explicit time dependence \(\PDi{t}{f} = 0\) and if \(\antisymmetric{H}{f} = 0\), then \(f\) is an integral of motion.

In QM conserved quantities are the ones that commute with the Hamiltonian operator.

To see the analogy better, recall def of Poisson bracket
\begin{equation}\label{eqn:phaseSpaceAndTrajectories:110}
\antisymmetric{f}{g} =
\sum_i
\PD{p_i}{f}
\PD{q_i}{g}
-
\PD{q_i}{f}
\PD{p_i}{g}
\end{equation}
%
Properties of Poisson bracket
%
\begin{itemize}
\item antisymmetric
%
\begin{equation}\label{eqn:phaseSpaceAndTrajectories:130}
\antisymmetric{f}{g} = -\antisymmetric{g}{f}.
\end{equation}
%
\item linear
%
\begin{equation}\label{eqn:phaseSpaceAndTrajectories:150}
\begin{aligned}
\antisymmetric{a f + b h}{g} &= a \antisymmetric{f}{g} + b\antisymmetric{h}{g} \\
\antisymmetric{g}{a f + b h} &= a \antisymmetric{g}{f} + b\antisymmetric{g}{h}.
\end{aligned}
\end{equation}
\end{itemize}
%
\subsubsection{Example.  Compute \texorpdfstring{\(p\), \(q\).}{p, q} commutators}
%
\begin{equation}\label{eqn:phaseSpaceAndTrajectories:550}
\begin{aligned}
\antisymmetric{p_i}{p_j}
&=
\sum_k
\PD{p_k}{p_i}
\cancel{
\PD{q_k}{p_j}
}
-
\cancel{
\PD{q_k}{p_i}
}
\PD{p_k}{p_j} \\
&= 0
\end{aligned}
\end{equation}
%
So
%
\begin{equation}\label{eqn:phaseSpaceAndTrajectories:170}
\antisymmetric{p_i}{p_j} = 0
\end{equation}
%
Similarly \(\antisymmetric{q_i}{q_j} = 0\).

How about
%
\begin{equation}\label{eqn:phaseSpaceAndTrajectories:570}
\begin{aligned}
\antisymmetric{q_i}{p_j}
&=
\sum_k
\cancel{
\PD{p_k}{q_i}
}
\cancel{
\PD{q_k}{p_j}
}
-
\PD{q_k}{q_i}
\PD{p_k}{p_j} \\
&=
-
\sum_k
\delta_{ik}
\delta_{jk} \\
&=
-\delta_{ij}
\end{aligned}
\end{equation}
%
So
%
\begin{equation}\label{eqn:phaseSpaceAndTrajectories:190}
\antisymmetric{q_i}{p_j} = -\delta_{ij}.
\end{equation}
%
This provides a systematic (axiomatic) way to ``quantize'' a classical mechanics system, where we make replacements
%
\begin{equation}\label{eqn:phaseSpaceAndTrajectories:210}
\begin{aligned}
q_i &\rightarrow \hat{q}_i \\
p_i &\rightarrow \hat{p}_i,
\end{aligned}
\end{equation}
%
and
%
\begin{equation}\label{eqn:phaseSpaceAndTrajectories:230}
\begin{aligned}
\antisymmetric{q_i}{p_j} = -\delta_{ij} &\rightarrow
\antisymmetric{q_i}{p_j} = i \Hbar \delta_{ij} \\
H(p, q, t) &\rightarrow \hat{H}(\hat{p}, \hat{q}, t).
\end{aligned}
\end{equation}
%
So
%
\begin{equation}\label{eqn:phaseSpaceAndTrajectories:250}
\frac{\antisymmetric{\hat{q}_i}{\hat{p}_j}}{-i \Hbar } = - \delta_{ij}
\end{equation}
%
Our quantization of time evolution is therefore
%
\begin{equation}\label{eqn:phaseSpaceAndTrajectories:270}
\begin{aligned}
\ddt{} \hat{q}_i &= \inv{-i\Hbar} \antisymmetric{\hat{H}}{\hat{q}_i} \\
\ddt{} \hat{p}_i &= \inv{-i\Hbar} \antisymmetric{\hat{H}}{\hat{p}_i}.
\end{aligned}
\end{equation}
%
These are the Heisenberg equations of motion in QM.
%
\subsubsection{Conserved quantities.}
%
For conserved quantities \(f\), functions of \(p\)'s \(q\)'s, we have
%
\begin{equation}\label{eqn:phaseSpaceAndTrajectories:290}
\antisymmetric{f}{H} = 0
\end{equation}
%
Considering the components \(M_i\), where
%
\begin{equation}\label{eqn:phaseSpaceAndTrajectories:310}
\BM = \Br \cross \Bp,
\end{equation}
%
We can show \eqnref{eqn:phaseSpaceAndTrajectories:410} that our Poisson brackets obey
%
\begin{equation}\label{eqn:phaseSpaceAndTrajectories:330}
\begin{aligned}
\antisymmetric{M_x}{M_y} &= -M_z \\
\antisymmetric{M_y}{M_z} &= -M_x \\
\antisymmetric{M_z}{M_x} &= -M_y
\end{aligned}
\end{equation}
%
(Prof Poppitz was not sure if he had the sign of this right for the sign convention he happened to be using for Poisson brackets in this lecture, but it appears he had it right).

These are the analogue of the momentum commutator relationships from QM right here in classical mechanics.

Considering the symmetries that lead to this conservation relationship, it is actually possible to show that rotations in 4D space lead to these symmetries and the conservation of the Runge-Lenz vector.
%
\section{Adiabatic changes in phase space and conserved quantities.}
%
In \cref{fig:phaseSpaceAndTrajectories:phaseSpaceAndTrajectoriesFig2} where we have
%
\imageFigure{../figures/classicalmechanics/phaseSpaceAndTrajectoriesFig2}{Variable length pendulum.}{fig:phaseSpaceAndTrajectories:phaseSpaceAndTrajectoriesFig2}{0.2}
%
\begin{equation}\label{eqn:phaseSpaceAndTrajectories:350}
T = \frac{2 \pi}{\omega(t)} = \sqrt{\frac{l(t)}{g}}.
\end{equation}
%
Imagine that we change the length \(l(t)\) very \textunderline{slowly} so that
%
\begin{equation}\label{eqn:phaseSpaceAndTrajectories:370}
T \inv{l} \ddt{l} \ll 1
\end{equation}
%
where \(T\) is the period of oscillation.  This is what is called an adiabatic change, where the change of \(\omega\) is small over a period.
%
It turns out that if this rate of change is slow, then there is actually an invariant, and
%
\begin{equation}\label{eqn:phaseSpaceAndTrajectories:390}
\frac{E}{\omega},
\end{equation}
%
is the so-called ``adiabatic invariant''.  There is an important application to this (and some relations to QM).  Imagine that we have a particle bounded by two walls, where the walls are moved very slowly as in \cref{fig:phaseSpaceAndTrajectories:phaseSpaceAndTrajectoriesFig3}
\imageFigure{../figures/classicalmechanics/phaseSpaceAndTrajectoriesFig3}{Particle constrained by slowly moving walls.}{fig:phaseSpaceAndTrajectories:phaseSpaceAndTrajectoriesFig3}{0.2}

This can be used to derive the adiabatic equation for an ideal gas (also using the equipartition theorem).
%
\section{Appendix I.  Poisson brackets of angular momentum.}
%
Let us verify the angular momentum relations of \eqnref{eqn:phaseSpaceAndTrajectories:330} above (summation over \(k\) implied):
%
\begin{equation}\label{eqn:phaseSpaceAndTrajectories:590}
\begin{aligned}
\antisymmetric{M_i}{M_j}
&=
\PD{p_k}{M_i}
\PD{x_k}{M_j}
-
\PD{x_k}{M_i}
\PD{p_k}{M_j} \\
&=
\epsilon_{a b i}
\epsilon_{r s j}
\PD{p_k}{x_a p_b}
\PD{x_k}{x_r p_s}
-
\epsilon_{a b i}
\epsilon_{r s j}
\PD{x_k}{x_a p_b}
\PD{p_k}{x_r p_s} \\
&=
\epsilon_{a b i}
\epsilon_{r s j}
x_a \PD{p_k}{p_b}
p_s \PD{x_k}{x_r}
-
\epsilon_{a b i}
\epsilon_{r s j}
p_b \PD{x_k}{x_a}
x_r \PD{p_k}{p_s} \\
&=
\epsilon_{a b i}
\epsilon_{r s j}
x_a \delta_{k b}
p_s \delta_{k r}
-
\epsilon_{a b i}
\epsilon_{r s j}
p_b \delta_{k a}
x_r \delta_{s k} \\
&=
\epsilon_{a b i}
\epsilon_{r s j}
x_a
p_s \delta_{b r}
-
\epsilon_{a b i}
\epsilon_{r s j}
p_b
x_r \delta_{a s} \\
&=
\epsilon_{a r i}
\epsilon_{r s j}
x_a
p_s
-
\epsilon_{s b i}
\epsilon_{r s j}
p_b
x_r \\
&=
-
\delta_{a i}^{[s j]}
x_a
p_s
-
\delta_{b i}^{[j r]}
p_b
x_r \\
&=
-
\left(
\delta_{a s}\delta_{i j}
-\delta_{a j}\delta_{i s}
\right)
x_a
p_s
-
\left(
\delta_{b j}\delta_{i r}
-\delta_{b r}\delta_{i j}
\right)
p_b
x_r \\
&=
-
\delta_{a s}\delta_{i j}
x_a
p_s
+\delta_{a j}\delta_{i s}
x_a
p_s
-
\delta_{b j}\delta_{i r}
p_b
x_r
+\delta_{b r}\delta_{i j}
p_b
x_r
\\
&=
-
\cancel{
x_s
p_s
\delta_{i j}
}
+
x_j
p_i
-
p_j
x_i
+
\cancel{
p_b
x_b
\delta_{i j}
}
\\
\end{aligned}
\end{equation}
%
So, as claimed, if \(i \ne j \ne k\) we have
%
\begin{equation}\label{eqn:phaseSpaceAndTrajectories:410}
\antisymmetric{M_i}{M_j} = -M_k.
\end{equation}
%
\section{Appendix II. EOM for the variable length pendulum.}
%
Since we have referred to a variable length pendulum above, let us recall what form the EOM for this system take.  With cylindrical coordinates as in \cref{fig:phaseSpaceAndTrajectories:phaseSpaceAndTrajectoriesFig4}, and a spring constant \(\omega_0^2 = k/m\) our Lagrangian is
%
\imageFigure{../figures/classicalmechanics/phaseSpaceAndTrajectoriesFig4}{phaseSpaceAndTrajectoriesFig4.}{fig:phaseSpaceAndTrajectories:phaseSpaceAndTrajectoriesFig4}{0.2}
%
\begin{equation}\label{eqn:phaseSpaceAndTrajectories:430}
\LL = \inv{2} m \left( \dot{r}^2 + r^2 \thetadot^2 \right) - \inv{2} m \omega_0^2 r^2 - m g r( 1 - \cos\theta)
\end{equation}
%
The EOM follows immediately
%
\begin{equation}\label{eqn:phaseSpaceAndTrajectories:450}
\begin{aligned}
P_\theta &= \PD{\thetadot}{\LL} = m r^2 \thetadot \\
P_r &= \PD{\rdot}{\LL} = m \rdot \\
\ddt{P_\theta} &= \PD{\theta}{\LL} = -m g r \sin\theta \\
\ddt{P_r} &= \PD{r}{\LL} = m r \thetadot^2 - m \omega_0^2 r - m g (1 - \cos\theta)
\end{aligned}
\end{equation}
%
Or
%
\begin{equation}\label{eqn:phaseSpaceAndTrajectories:470}
\begin{aligned}
\ddt{} \left( r^2 \thetadot \right) &= - g r \sin\theta \\
\ddt{} \left( \rdot \right) &= r \left( \thetadot^2 - \omega_0^2 \right) - g (1 - \cos\theta)
\end{aligned}
\end{equation}
%
Even in the small angle limit this is not a terribly friendly looking system
%
\begin{equation}\label{eqn:phaseSpaceAndTrajectories:490}
\begin{aligned}
r \ddot{\theta} + 2 \thetadot \rdot + g \theta &= 0 \\
\ddot{r} - r \thetadot^2 + r \omega_0^2 &= 0.
%\ddt{} \left( r^2 \thetadot \right) &= - g r \theta \\
%\ddt{} \left( \rdot \right) &= r \left( \thetadot^2 - \omega_0^2 \right).
\end{aligned}
\end{equation}
%
However, in the first equation of this system
%
\begin{equation}\label{eqn:phaseSpaceAndTrajectories:510}
\ddot{\theta} + 2 \thetadot \frac{\rdot}{r} + \inv{r} g \theta = 0,
\end{equation}
%
we do see the \(\rdot/r\) dependence mentioned in class, and see how this being small will still result in something that approximately has the form of a SHO.
