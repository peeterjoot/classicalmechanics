%
% Copyright � 2012 Peeter Joot.  All Rights Reserved.
% Licenced as described in the file LICENSE under the root directory of this GIT repository.
%
%
%:%s/\\Brtilde/\\tilde{\\mathbf{r}}
%:%s/\\BRtilde/\\tilde{\\mathbf{R}}
%:%s/\\Brhotilde/\\tilde{\\boldsymbol{\\rho}}
%
%\chapter{Rigid body motion}
\index{rigid body}
%\chapter{Rigid body motion.  (notes from an audited phy354 classical mechanics course (Taught by Prof. Erich Poppitz))}
\label{chap:rigidBodyMotion}
%\blogpage{http://sites.google.com/site/peeterjoot2/math2012/rigidBodyMotion.pdf}
%\date{Mar 7, 2012}
%
\section{Rigid body motion.}
%
\subsection{Setup.}
%
We will consider either rigid bodies as in the connected by sticks \cref{fig:rigidBodyMotion:rigidBodyMotionFig1} or a body consisting of a continuous mass as in \cref{fig:rigidBodyMotion:rigidBodyMotionFig2}
%
\imageFigure{../figures/classicalmechanics/rigidBodyMotionFig1}{Rigid body of point masses.}{fig:rigidBodyMotion:rigidBodyMotionFig1}{0.2}
%
\imageFigure{../figures/classicalmechanics/rigidBodyMotionFig2}{Rigid solid body of continuous mass.}{fig:rigidBodyMotion:rigidBodyMotionFig2}{0.2}
%
In the first figure our mass is made of discrete particles
%
\begin{equation}\label{eqn:rigidBodyMotion:20}
M = \sum m_i
\end{equation}
%
whereas in the second figure with mass density \(\rho(\Br)\) and a volume element \(d^3\Br\), our total mass is
%
\begin{equation}\label{eqn:rigidBodyMotion:40}
M = \int_V \rho(\Br) d^3\Br
\end{equation}
%
\subsection{Degrees of freedom.}
%
How many numbers do we need to describe fixed body motion.  Consider \cref{fig:rigidBodyMotion:rigidBodyMotionFig3}
\imageFigure{../figures/classicalmechanics/rigidBodyMotionFig3}{Body local coordinate system with vector to a fixed point in the body.}{fig:rigidBodyMotion:rigidBodyMotionFig3}{0.2}

We will need to use six different numbers to describe the motion of a rigid body.  We need three for the position of the body \(\BR_{CM}\) as a whole.  We also need three degrees of freedom (in general) for the motion of the body at that point in space (how our local coordinate system at the body move at that point), describing the change of the orientation of the body as a function of time.

Note that the angle \(\phi\) has not been included in any of the pictures because it is too messy with all the rest.  Picture something like \cref{fig:rigidBodyMotion:rigidBodyMotionFig5}
\imageFigure{../figures/classicalmechanics/rigidBodyMotionFig5}{Rotation angle and normal in the body.}{fig:rigidBodyMotion:rigidBodyMotionFig5}{0.2}

Let us express the position of the body in terms of that body's center of mass
%
\begin{equation}\label{eqn:rigidBodyMotion:60}
\BR_{CM} = \frac{\sum_i m_i \Br_i}{\sum_j m_j},
\end{equation}
%
or for continuous masses
%
\begin{equation}\label{eqn:rigidBodyMotion:80}
\BR_{CM} = \frac{\int_V d^3 \Br' \Br' \rho(\Br')}{ \int d^3 \Br'' \rho(\Br'')}.
\end{equation}
%
We consider the motion of point \(\BP\), an arbitrary point in the body as in \cref{fig:rigidBodyMotion:rigidBodyMotionFig4}, whos motion consists of
%
\begin{enumerate}
\item displacement of the CM \(\BR_{CM}\)
\item rotation of \(\Br\) around some axis \(\ncap\) going through CM on some angle \(\phi\). (here \(\ncap\) is a unit vector).
\end{enumerate}
%
\imageFigure{../figures/classicalmechanics/rigidBodyMotionFig4}{A point in the body relative to the center of mass.}{fig:rigidBodyMotion:rigidBodyMotionFig4}{0.2}
%
From the picture we have
\begin{equation}\label{eqn:rigidBodyMotion:100}
\Brho = \BR_{CM} + \Br
\end{equation}
%
\begin{equation}\label{eqn:rigidBodyMotion:120}
d\Brho = d\BR_{CM} + d\Bphi \cross \Br
\end{equation}
%
where
%
\begin{equation}\label{eqn:rigidBodyMotion:140}
d\Bphi = \ncap d\phi.
\end{equation}
%
Dividing by \(dt\) we have
%
\begin{equation}\label{eqn:rigidBodyMotion:160}
\ddt{\Brho} = \ddt{\BR_{CM}} + \ddt{\Bphi} \cross \Br.
\end{equation}
%
The total velocity of this point in the body is then
%
\begin{equation}\label{eqn:rigidBodyMotion:180}
\Bv = \BV_{CM} = \BOmega_{CM} \cross \Br
\end{equation}
%
where
%
\begin{equation}\label{eqn:rigidBodyMotion:200}
\BOmega_{CM} = \ddt{\Bphi} = \ddt{(\ncap \phi)} = \text{angular velocity of the body}.
\end{equation}
%
%
This circular motion is illustrated in \cref{fig:rigidBodyMotion:rigidBodyMotionFig6}
%
\imageFigure{../figures/classicalmechanics/rigidBodyMotionFig6}{circular motion.}{fig:rigidBodyMotion:rigidBodyMotionFig6}{0.2}
%
Note that \(\Bv\) is the velocity of the particle with respect to the unprimed system.

We will spend a lot of time figuring out how to express \(\BOmega_{CM}\).

Now let us consider a second point as in \cref{fig:rigidBodyMotion:rigidBodyMotionFig7}
\imageFigure{../figures/classicalmechanics/rigidBodyMotionFig7}{Two points in a rigid body.}{fig:rigidBodyMotion:rigidBodyMotionFig7}{0.2}
%
\begin{equation}\label{eqn:rigidBodyMotion:220}
\begin{aligned}
\Brho &= \BR + \Br \\
\Brho &= \tilde{\mathbf{R}} + \tilde{\mathbf{r}} \\
\tilde{\mathbf{r}} &= \Br + \Ba.
\end{aligned}
\end{equation}
%
we have
%
\begin{equation}\label{eqn:rigidBodyMotion:520}
\begin{aligned}
\Bv_p
&=
\ddt{\Brho} \\
&= \ddt{\Br} + \ddt{\Bphi} \cross \Br \\
&= \ddt{\Br} + \ddt{\Bphi} \cross (\tilde{\mathbf{r}} - \Ba) \\
&= \ddt{\Br} - \ddt{\Bphi} \cross \Ba + \ddt{\Bphi} \cross \tilde{\mathbf{r}}.
\end{aligned}
\end{equation}
%
\boxedEquation{eqn:rigidBodyMotion:240}{
\Bv_p = \BV_{CM} -
\BOmega_{CM} \cross \Ba
+\BOmega_{CM} \cross \tilde{\mathbf{r}}
}
%
Have another way that we can use to express the position of the point
%
% notation switch midstream tilde{r} -> tilde{r_p}
\begin{equation}\label{eqn:rigidBodyMotion:260}
\ddt{\Brho}
= \ddt{\tilde{\mathbf{R}}} + \ddt{\tilde{\boldsymbol{\rho}}} \cross \tilde{\mathbf{r}}_p
\end{equation}
%
or
%
\begin{equation}\label{eqn:rigidBodyMotion:280}
\Bv_p = \BV_A + \BOmega_A \cross \tilde{\mathbf{r}}_p.
\end{equation}
%
Equating with above, and noting that this holds for all \(\tilde{\mathbf{r}}_p\), and noting that if \(\tilde{\mathbf{r}}_p = 0\)
%
\begin{equation}\label{eqn:rigidBodyMotion:300}
\BV_A = \BV_{CM} - \BOmega_{CM} \cross \Ba
\end{equation}
%
hence
%
\begin{equation}\label{eqn:rigidBodyMotion:320}
\BOmega_{CM} \cross \tilde{\mathbf{r}}_p = \BOmega_A \cross \tilde{\mathbf{r}}_p
\end{equation}
%
or
%
\begin{equation}\label{eqn:rigidBodyMotion:340}
\BOmega_{CM} = \BOmega_A.
\end{equation}
%
The moral of the story is that the angular velocity \(\BOmega\) is a characteristic of the system.  It does not matter if it is calculated with respect to the center of mass or not.

See some examples in the notes.
%
\section{Kinetic energy.}
%
For all \(P\) in the body we have
%
\begin{equation}\label{eqn:rigidBodyMotion:360}
\Bv_p = \BV_A + \BOmega \cross \Br_p
\end{equation}
%
here \(\BV_A\) is an arbitrary fixed point in the body as in \cref{fig:rigidBodyMotion:rigidBodyMotionFig8}
%
\imageFigure{../figures/classicalmechanics/rigidBodyMotionFig8}{Kinetic energy setup relative to point \(A\) in the body.}{fig:rigidBodyMotion:rigidBodyMotionFig8}{0.2}
%
The kinetic energy is
%
\begin{equation}\label{eqn:rigidBodyMotion:540}
\begin{aligned}
T
&= \sum_a \inv{2} m_a \Brho_a \\
&= \sum_a \inv{2} \Bv_a^2 \\
&=
\sum_a \inv{2} \left(
\BV_A + \BOmega \cross \Br_a
\right)^2 \\
&=
\sum_a \inv{2} \left( \BV_A^2 +
2 \BV_A \cdot (\BOmega \cross \Br_a)
+ (\BOmega \cross \Br_a)^2
\right).
\end{aligned}
\end{equation}
%
We see that if we take \(A\) to be the center of mass then our cross term
%
\begin{equation}\label{eqn:rigidBodyMotion:560}
\begin{aligned}
\sum_a m_a \BV_A \cdot (\BOmega \cross \Br_a)
&=
\BV_A \cdot
\lr{ \BOmega \cross \sum_a m_a \Br_a } \\
&=
\BV_A \cdot (\BOmega \cross \BR_{CM} ).
\end{aligned}
\end{equation}
%
which vanishes.  With
%
\begin{equation}\label{eqn:rigidBodyMotion:380}
\mu = \sum_a m_a
\end{equation}
%
we have
%
\begin{equation}\label{eqn:rigidBodyMotion:400}
T = \inv{2} \mu \BV_{CM}^2+ \inv{2} \sum_a
(\BOmega \cross \Br_a) \cdot (\BOmega \cross \Br_a)
\end{equation}
%
with
%
\begin{equation}\label{eqn:rigidBodyMotion:420}
(\BA \cross \BB) \cdot (\BC \cross \BD) = (\BA \cdot \BC) (\BB \cdot \BD) -(\BA \cdot \BD) (\BB \cdot \BC),
\end{equation}
%
or
\begin{equation}\label{eqn:rigidBodyMotion:420b}
(\BA \cross \BB) \cdot (\BA \cross \BB) = \BA^2 \BB^2 - (\BA \cdot \BB)^2.
\end{equation}
%
Forgetting about the \(\mu\) dependent term for now we have
%
\begin{equation}\label{eqn:rigidBodyMotion:440}
T = \inv{2} \sum_a m_a \left( \BOmega^2 \Br_a^2 - (\BOmega \cdot \Br_a)^2
\right)
\end{equation}
%
Expanding this out with
%
\begin{equation}\label{eqn:rigidBodyMotion:460}
\Br_a = ( r_{a_1} r_{a_2}, r_{a_3} ) = \{ r_{a_i} \}
\end{equation}
%
and
\begin{equation}\label{eqn:rigidBodyMotion:480}
\BOmega = ( \Omega_{1} \Omega_{2}, \Omega_{3} ) = \{ \Omega_{i} \}
\end{equation}
%
we have
%
\begin{equation}\label{eqn:rigidBodyMotion:500}
T = \inv{2} \sum_a m_a \left( \Omega_k \Omega_k r_{a_j} r_{a_j} - (\Omega_k r_{a_k})^2
\right)
\end{equation}
