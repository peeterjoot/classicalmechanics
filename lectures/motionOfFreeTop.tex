%
% Copyright � 2013 Peeter Joot.  All Rights Reserved.
% Licenced as described in the file LICENSE under the root directory of this GIT repository.
%
% pick one:
%\input{../latex/assignment.tex}
\input{../latex/blogpost.tex}
\renewcommand{\basename}{motionOfFreeTop}
\renewcommand{\dirname}{notes/classicalmechanics/}
%\newcommand{\dateintitle}{}
\newcommand{\keywords}{Symmetric free top, Classical Mechanics, PHY354H1S, PHY354H1, PHY354}
%
\input{../latex/peeter_prologue_print2.tex}
%
\beginArtNoToc
%
\generatetitle{PHY354H1S Advanced Classical Mechanics.  Motion of a symmetric free top.  Taught by Prof.\ Erich Poppitz}
\index{symmetric free top}
%\chapter{PHY354H1S Advanced Classical Mechanics.  Motion of a symmetric free top.  Taught by Prof.\ Erich Poppitz}
%\label{chap:\basename}
%\section{Motivation}
%\section{Guts}
%
%\section{Disclaimer.}
%
%Peeter's lecture notes from auditing this class.  May not be entirely coherent.
%
\section{Lecture notes.}
%
FIXME: F1.
%
\begin{equation}\label{eqn:motionOfFreeTop:20}
\BM = (0, 0, M).
\end{equation}
%
In the body frame
%
\begin{equation}\label{eqn:motionOfFreeTop:40}
\BM = M ( \sin\theta \sin\phi, \sin\theta \cos\phi, \cos\theta ) = (M_1, M_2, M_3).
\end{equation}
%
Recall that
\begin{equation}\label{eqn:motionOfFreeTop:60}
\begin{aligned}
M_1 &= I (\thetadot \cos\psi + \phidot \sin\theta \sin\psi) \\
M_2 &= I (-\thetadot \sin\psi + \phidot \sin\theta \cos\psi) \\
M_3 &= I (\psidot + \phidot \cos\theta).
\end{aligned}
\end{equation}
%
Our EOM for (what restriction?) are
%
\begin{equation}\label{eqn:motionOfFreeTop:80}
\begin{aligned}
M \cos\theta &= I_3 ( \psidot + \phidot \cos\theta ) \leftarrow ( M_3 = I_3 \Omega_3) \\
M \sin\theta \cos\psi &= I ( -\thetadot \sin\psi + \phidot \sin\theta \cos\psi)  \leftarrow (M_2 = I \Omega_2) \\
\cdots
\end{aligned}
\end{equation}
%
switched to paper.
%
\section{Some words on the motion of a rigid body.}
%
We need to know how the motion of the center of mass changes
%
\begin{equation}\label{eqn:motionOfFreeTop:100}
\frac{d}{dt} \BP_{\mathrm{CM}} = \BF.
\end{equation}
%
where \(\BF\) is the external force applied to the body.

FIXME: G1.

We can also write
%
\begin{equation}\label{eqn:motionOfFreeTop:120}
\BF = \sum_a \Bf_a,
\end{equation}
%
where \(\Bf_a\) is the external force on the \(a\)-th particle.

We can also write a torque equation
%
\begin{equation}\label{eqn:motionOfFreeTop:140}
\frac{d}{dt} \BM = \BK.
\end{equation}
%
where \(\BK\) is the torque of the external forces acting on the body.
%
\begin{equation}\label{eqn:motionOfFreeTop:160}
M_O = \sum_a \Brho_a \cross \Bp_a.
\end{equation}
%
\begin{equation}\label{eqn:motionOfFreeTop:180}
\ddt{\BM_\theta} = \sum_a \Bv_a \cross \Bp_a + \sum_a \Brho_a \cross \Bp_a.
\end{equation}
%
\begin{dmath}\label{eqn:motionOfFreeTop:n}
\ddt{\BM_\theta}
= \sum_a \Brho_a \cross \Bp_a
= \sum_a \BR_{\mathrm{CM}} \cross \Bf_a + \sum_a \Br_a \cross \Bf_a
= \BR_{\mathrm{CM}} \cross \BR + \sum_a \Bk_a.
\end{dmath}
%
where \(\Bk_a\) is the torque on the \(a\)-th particle with respect to the torque of external forces with respect to the CM.
%
So
%
\begin{equation}\label{eqn:motionOfFreeTop:200}
\ddt{\BM_\theta} = \BR_{\mathrm{CM}} \cross \BF + \BK.
\end{equation}
%
or
%
\begin{equation}\label{eqn:motionOfFreeTop:220}
\ddt{\BM_\theta} = \BK.
\end{equation}
%
FIXME: why?
%
\begin{equation}\label{eqn:motionOfFreeTop:240}
\ddt{\BM_\theta} = \sum_a \Br_a \cross \Bf_a.
\end{equation}
%
with
\begin{equation}\label{eqn:motionOfFreeTop:260}
\Bf_a = m_a \Bg.
\end{equation}
%
we have
%
\begin{equation}\label{eqn:motionOfFreeTop:280}
\ddt{\BM_\theta} = \sum_a m_a \Br_a \cross \Bg.
\end{equation}
%
board erased fast.  Is this where things were left?
%
% this is to produce the sites.google url and version info and so forth (for blog posts)
%\vcsinfo
%\EndArticle
%\EndNoBibArticle
