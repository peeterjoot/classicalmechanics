%
% Copyright � 2013 Peeter Joot.  All Rights Reserved.
% Licenced as described in the file LICENSE under the root directory of this GIT repository.
%
% pick one:
%\input{../assignment.tex}
%\input{../blogpost.tex}
%\renewcommand{\basename}{parallelAxesTheorem}
%\renewcommand{\dirname}{notes/classicalmechanics/}
%%\newcommand{\dateintitle}{}
%\newcommand{\keywords}{Parallel axis theorem, Classical Mechanics, PHY354H1S, PHY354H1, PHY354}
%
%\input{../peeter_prologue_print2.tex}
%
%%\usepackage[english]{cleveref}
%
%\beginArtNoToc
%
%\chapter{PHY354H1S Advanced Classical Mechanics.  Lecture: Parallel axis theorem.  Taught by Prof.\ Erich Poppitz}
%\chapter{Parallel axis theorem}
\index{parallel axis theorem}
%\generatetitle{PHY354H1S Advanced Classical Mechanics.  Lecture: Parallel axis theorem.  Taught by Prof.\ Erich Poppitz}
\label{chap:parallelAxisTheorem}

%\section{Disclaimer}
%
%Peeter's lecture notes from auditing this class.  May not be entirely coherent.
%
%\section{Guts}

We can express the kinetic energy as

\begin{dmath}\label{eqn:parallelAxisTheorem:10}
T = \inv{2} \sum_{i, j = 1}^3 \Omega_i I_{i j} \Omega_j
\end{dmath}

where

\begin{dmath}\label{eqn:parallelAxisTheorem:30}
I_{i j} = \sum_a m_a \left( \delta_{i j} \Br_a^2 - r_{a_i} r_{a_j} \right)
\end{dmath}

Here \(a\) is a sum over all particles in the body.

If the body is continuous and \(\rho(\Br)\) is the mass density then the mass inside is

\begin{dmath}\label{eqn:parallelAxisTheorem:50}
m = \int d^3 \Br \rho(\Br),
\end{dmath}

where we integrate over a volume element as in \cref{fig:parallelAxisTheorem:parallelAxisTheoremFig1}.

\imageFigure{../figures/classicalmechanics/parallelAxisTheoremFig1}{Volume element for continuous mass distribution}{fig:parallelAxisTheorem:parallelAxisTheoremFig1}{0.3}

For this continuous case we have

\begin{dmath}\label{eqn:parallelAxisTheorem:70}
I_{i j}
= \int_V d^3 r \rho(\Br) \left( \delta_{i j} \Br^2 - r_{i} r_{j} \right)
\end{dmath}

Another property of \(I_{i j}\) is the \textunderline{parallel axis theorem} (or as it is known in Europe and perhaps elsewhere, as the ``Steiner theorem'').

Let's consider a change of origin as in \cref{fig:parallelAxisTheorem:parallelAxisTheoremFig2}.

\imageFigure{../figures/classicalmechanics/parallelAxisTheoremFig2}{Shift of origin}{fig:parallelAxisTheorem:parallelAxisTheoremFig2}{0.3}

We write

\begin{dmath}\label{eqn:parallelAxisTheorem:90}
\Br_a = \Br_a' + \Bb,
\end{dmath}

and \(I_{i j}'\) for the inertia tensor with respect to \(O'\).  Write

\begin{dmath}\label{eqn:parallelAxisTheorem:110}
\Br_{a_i} = \Br_{a_i}' + \Bb_i
\end{dmath}

or

\begin{dmath}\label{eqn:parallelAxisTheorem:130}
\Br_{a_i}' = \Br_{a_i} - \Bb_i,
\end{dmath}

so that

\begin{dmath}\label{eqn:parallelAxisTheorem:150}
I_{i j}' =
\sum_a m_a \left(
\delta_{ij} \Br_a^2 - r_{a_i}' r_{b_j}'
\right)
=
\sum_a m_a \left(
\delta_{ij} (\Br_a - \Bb)^2 - (r_{a_i} - b_i) (r_{b_j} - b_j)
\right)
=
\sum_a m_a \left(
\delta_{ij} (
r_{a_k}
r_{a_k}
- \Bb^2
- 2 r_{a} b
)
- r_{a_i} r_{b_j}
- b_i b_j
+ r_{a_i} b_j
+ r_{b_j} b_i
\right),
\end{dmath}

but, by definition of center of mass, we have

\begin{dmath}\label{eqn:parallelAxisTheorem:170}
\sum_a m_a r_{a_i}' = 0,
\end{dmath}

so

\begin{dmath}\label{eqn:parallelAxisTheorem:190}
I_{i j}' =
\sum_a m_a \left(
\delta_{i j} \Br_a^2 - r_{a_i} r_{a_j} - \cdots
\right)
= I_{i j}
- 2 \cancel{ \left( \sum_a m_a \Br_a \cdot \Bb \delta_{i j} \right) }
+ \mu \left( \delta_{i j} \Bb^2 - b_i b_j \right)
\end{dmath}

This is

\begin{dmath}\label{eqn:parallelAxisTheorem:210}
I_{i j}'
= I_{i j}^{\mathrm{CM}}
+ \mu \left( \delta_{i j} \Bb^2 - b_i b_j \right)
\end{dmath}

\paragraph{Some examples}

Infinite cylinder rolling on a plane, with no slipping and no dissipation (heat?) as in \cref{fig:parallelAxisTheorem:parallelAxisTheoremFig3}.

\imageFigure{../figures/classicalmechanics/parallelAxisTheoremFig3}{Infinite rolling cylinder on plane}{fig:parallelAxisTheorem:parallelAxisTheoremFig3}{0.3}

Take the mass as uniform and set up coordinates as in \cref{fig:parallelAxisTheorem:parallelAxisTheoremFig4}.

\imageFigure{../figures/classicalmechanics/parallelAxisTheoremFig4}{Coordinates for infinite cylinder}{fig:parallelAxisTheorem:parallelAxisTheoremFig4}{0.3}

No slip means on revolution, the center of mass moves \(2 \pi R\).  We have one degree of freedom: \(\phi\).

\begin{equation}\label{eqn:parallelAxisTheorem:230}
\Abs{\Omega} = \phidot = \ddt{\phi}
\end{equation}

This is the angular velocity.

\begin{equation}\label{eqn:parallelAxisTheorem:250}
\frac{\Delta \phi}{\Delta x} = \frac{2 \pi}{ 2 \pi R}
\end{equation}

so

\begin{equation}\label{eqn:parallelAxisTheorem:270}
\Delta x = R \Delta \phi
\end{equation}

The kinetic energy is

\begin{dmath}\label{eqn:parallelAxisTheorem:290}
T
= \inv{2} \mu V_{\mathrm{CM}}^2 + \inv{2} \Omega_3^2 I_{33}
= \inv{2} \mu V_{\mathrm{CM}}^2 + \inv{2} \Omega^2 I
\end{dmath}

\begin{dmath}\label{eqn:parallelAxisTheorem:310}
V_{\mathrm{CM}} = \frac{\Delta x}{\Delta t} = R \frac{\Delta \phi}{\Delta t} = R \Omega = R \phidot
\end{dmath}

so

\begin{dmath}\label{eqn:parallelAxisTheorem:330}
T = \inv{2} \mu R^2 \phidot^2 + \inv{2} \phidot^2 I.
\end{dmath}

(can calculate \(I\) : See notes or derive).

Now suppose the CM is displaced as in \cref{fig:parallelAxisTheorem:parallelAxisTheoremFig5}.

\imageFigure{../figures/classicalmechanics/parallelAxisTheoremFig5}{Displaced CM for infinite cylinder}{fig:parallelAxisTheorem:parallelAxisTheoremFig5}{0.3}

Perhaps a hollow tube with a blob attached as in \cref{fig:parallelAxisTheorem:parallelAxisTheoremFig6}, where the torque is now due to gravity.

\imageFigure{../figures/classicalmechanics/parallelAxisTheoremFig6}{Hollow tube with blob}{fig:parallelAxisTheorem:parallelAxisTheoremFig6}{0.3}

This can have more interesting motion.  Example: Oscillation.  This is a typical test question, where calculation of the frequency of oscillation is requested.  Such a question would probably be posed with the geometry of \cref{fig:parallelAxisTheorem:parallelAxisTheoremFig7}.

\imageFigure{../figures/classicalmechanics/parallelAxisTheoremFig7}{Hollow tube with cylindrical blob}{fig:parallelAxisTheorem:parallelAxisTheoremFig7}{0.3}

Recall for a general body as in \cref{fig:parallelAxisTheorem:parallelAxisTheoremFig8}.

\imageFigure{../figures/classicalmechanics/parallelAxisTheoremFig8}{general body coordinates}{fig:parallelAxisTheorem:parallelAxisTheoremFig8}{0.3}

Write

\begin{dmath}\label{eqn:parallelAxisTheorem:350}
\Br = \Ba + \Br'
\end{dmath}

and

\begin{dmath}\label{eqn:parallelAxisTheorem:371}
\Bv
= \BV_{\mathrm{CM}}
+ \BOmega \cross \Br
\end{dmath}

or
\begin{equation}\label{eqn:parallelAxisTheorem:370}
\Bv
= \BV_{\mathrm{CM}} + \BOmega \cross \Ba
+ \BOmega \cross \Br'.
\end{equation}

Here \(\BV_{\mathrm{CM}}\) is the velocity of the origin \(A\).

If \(\BV_{\mathrm{CM}}\) and \(\BOmega\) are perpendicular always there always exists \(\Ba\) such that \(A\) is at rest.

Another example is a cone on plane or rod as in \cref{fig:parallelAxisTheorem:parallelAxisTheoremFig9}.

\imageFigure{../figures/classicalmechanics/parallelAxisTheoremFig9}{Cone on rod}{fig:parallelAxisTheorem:parallelAxisTheoremFig9}{0.3}

(this is another typical test question).

For cylinder that point is the contact between plane and cylinder.  This is called the momentary axis of rotation: \cref{fig:parallelAxisTheorem:parallelAxisTheoremFig10}.  Using this is a very useful trick.

\imageFigure{../figures/classicalmechanics/parallelAxisTheoremFig10}{Momentary axes of rotation}{fig:parallelAxisTheorem:parallelAxisTheoremFig10}{0.3}

\paragraph{Aside}

more interesting is the cone viewed from above as in \cref{fig:parallelAxisTheorem:parallelAxisTheoremFig11}.

\imageFigure{../figures/classicalmechanics/parallelAxisTheoremFig11}{Cone from above}{fig:parallelAxisTheorem:parallelAxisTheoremFig11}{0.3}

Coordinates for this problem as in \cref{fig:parallelAxisTheorem:parallelAxisTheoremFig12}.

\imageFigure{../figures/classicalmechanics/parallelAxisTheoremFig12}{Momentary axes of rotation for cone on stick}{fig:parallelAxisTheorem:parallelAxisTheoremFig12}{0.3}

Using \eqnref{eqn:parallelAxisTheorem:370} we have

\begin{dmath}\label{eqn:parallelAxisTheorem:390}
V_{\mathrm{CM}} = \BOmega \cross \Bb = \phidot \zcap \cross \Bb
\end{dmath}

where this followed from

\begin{dmath}\label{eqn:parallelAxisTheorem:410}
\Bv = \BOmega \cross \Br'
\end{dmath}

here \(\Br'\) is the vector from axes of momentary rotation to point.

Our kinetic energy is

\begin{dmath}\label{eqn:parallelAxisTheorem:430}
T = \inv{2} \mu V_{\mathrm{CM}}^2 + \frac{I}{2} \phidot^2
\end{dmath}

and our coordinates are \cref{fig:parallelAxisTheorem:parallelAxisTheoremFig13}.

\imageFigure{../figures/classicalmechanics/parallelAxisTheoremFig13}{Coordinates}{fig:parallelAxisTheorem:parallelAxisTheoremFig13}{0.3}

\begin{dmath}\label{eqn:parallelAxisTheorem:450}
\BV_{\mathrm{CM}} = \BOmega \cross \Bb
\end{dmath}

\begin{dmath}\label{eqn:parallelAxisTheorem:470}
\Abs{\BV_{\mathrm{CM}}}
= \Abs{\phidot} \Abs{\Bb}
= \phidot \Abs{\Bb} \times \mbox{moving unit vector in x y plane}
= \phidot \sqrt{ \Ba^2 + \BR^2 + 2 \Ba \cdot \BR }
= \phidot \sqrt{ a^2 + R^2 + 2 a R \cos( \pi - \phi) }
\end{dmath}

For

\begin{dmath}\label{eqn:parallelAxisTheorem:490}
T
= \frac{\mu}{2} \phidot^2
\left(
a^2 + R^2 + 2 a R \cos( \pi - \phi)
\right)
+ \frac{I}{2} \phidot^2
=
\inv{2} \phidot^2
\left(
\mu
\left(
a^2 + R^2 + 2 a R \cos( \pi - \phi)
\right)
+ I
\right),
\end{dmath}

and

\begin{dmath}\label{eqn:parallelAxisTheorem:510}
\LL = T - \mu g
\mathLabelBox{
( R - a \cos\phi )
}{Height of CM above plane}
\end{dmath}

This gravity portion accounts for the torque producing interesting effects.

% this is to produce the sites.google url and version info and so forth (for blog posts)
%\vcsinfo
%\EndArticle
%\EndNoBibArticle
