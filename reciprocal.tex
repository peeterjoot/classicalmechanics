%
% Copyright � 2020 Peeter Joot.  All Rights Reserved.
% Licenced as described in the file LICENSE under the root directory of this GIT repository.
%
%{
\input{../latex/blogpost.tex}
\renewcommand{\basename}{reciprocal}
%\renewcommand{\dirname}{notes/phy1520/}
\renewcommand{\dirname}{notes/ece1228-electromagnetic-theory/}
%\newcommand{\dateintitle}{}
%\newcommand{\keywords}{}

\input{../latex/peeter_prologue_print2.tex}

\usepackage{amsthm}
\usepackage{peeters_layout_exercise}
\usepackage{peeters_braket}
\usepackage{peeters_figures}
\usepackage{siunitx}
\usepackage{verbatim}
%\usepackage{mhchem} % \ce{}
%\usepackage{macros_bm} % \bcM
%\usepackage{macros_qed} % \qedmarker
%\usepackage{txfonts} % \ointclockwise

\beginArtNoToc

\generatetitle{Reciprocal frame and curvilinear coordinates.}
%\chapter{Reciprocal frame uniqueness}
%\label{chap:reciprocal}

\section{Motivation.}
When studying curvilinear coordinates, reciprocal frame vectors are found to be the gradients of the parameters generating the tangent space at the point of evaluation.  We can also construct reciprocal frame vectors algebraically, but have a uniqueness doing so unless additional constraints are imposed.
In particular, the coordinates of a vector can be computed with respect to either the tangent space basis, or it's reciprocal basis, so we must impose the constraint that the reciprocal frame is restricted to the span of the tangent space basis.  Failing to impose such a constraint leads to uniqueness problems, and almost certainly means that the algebraic construction does not match the gradient construction.

In these notes, I'll walk through all the ideas involved.  The focus will be on the Dirac's algebra of special relativity, known as STA (Space Time Algebra) in geometric algebra parlance.

\paragraph{On notation.}
In Euclidean space we use bold face reciprocal frame vectors \( \Bx^i \cdot \Bx_j = {\delta^i}_j \), which nicely distinguishes them from the generalized coordinates \( x_i, x^j \) associated with the basis or the reciprocal frame, that is
\begin{equation}\label{eqn:reciprocal:640}
\Bx = x^i \Bx_i = x_j \Bx^j.
\end{equation}
On the other hand, it is conventional to use non-bold face for both the four-vectors and their coordinates in STA, such as the following standard basis decomposition
\begin{equation}\label{eqn:reciprocal:660}
x = x^\mu \gamma_\mu = x_\mu \gamma^\mu.
\end{equation}
If we use non-bold face \( x^\mu, x_\nu \) for the coordinates with respect to a specified frame, then we cannot also use non-bold face for the curvilinear basis vectors.
%One way to solve this notational problem would be to choose some new symbol, say \( f^\mu, f_\nu \), for our curvilinear basis vectors.
%This seems like a grossly arbitrary choice.  Symbols that are generally available without some additional or conflicting meaning can be hard to come by, so I'll opt to
To resolve this notational ambiguity, I've chosen to use bold face \( \Bx^\mu, \Bx_\nu \) symbols as the curvilinear basis elements in this relativistic context, as we do for Euclidean spaces.
\section{Basis and coordinates.}
\makedefinition{Standard Dirac basis.}{dfn:reciprocal:720}{
The Dirac basis elements are \(\setlr{ \gamma_0, \gamma_1, \gamma_2, \gamma_3 } \), satisfying
\begin{equation*}
\gamma_0^2 = 1 = -\gamma_k^2, \quad \forall k = 1,2,3,
\end{equation*}
and
\begin{equation}\label{eqn:reciprocal:740}
\gamma_\mu \cdot \gamma_\nu = 0, \quad \forall \mu \ne \nu.
\end{equation}
} % definition
A conventional way of summarizing these orthogonality relationships is \( \gamma_\mu \cdot \gamma_\nu = \eta_{\mu\nu} \), where
\( \eta_{\mu\nu} \) are the elements of the metric \( G = \diag(+,-,-,-) \).
\makedefinition{Reciprocal basis for the standard Dirac basis.}{dfn:reciprocal:760}{
We define a reciprocal basis \( \setlr{ \gamma^0, \gamma^1, \gamma^2, \gamma^3} \) satisfying \( \gamma^\mu \cdot \gamma_\nu = {\delta^\mu}_\nu, \forall \mu,\nu \in 0,1,2,3 \).
} % definition
\maketheorem{Reciprocal basis uniqueness.}{thm:reciprocal:780}{
This reciprocal basis is unique, and for our choice of metric has the values
\begin{equation*}
\gamma^0 = \gamma_0, \quad \gamma^k = -\gamma_k, \quad \forall k = 1,2,3.
\end{equation*}
} % theorem
Proof is left to the reader.
\makedefinition{Coordinates.}{dfn:reciprocal:800}{
We define the coordinates of a vector with respect to the standard basis as \( x^\mu \) satisfying
\begin{equation*}
x = x^\mu \gamma_\mu,
\end{equation*}
and define the coordinates of a vector with respect to the reciprocal basis as \( x_\mu \) satisfying
\begin{equation*}
x = x_\mu \gamma^\mu,
\end{equation*}
} % definition
\maketheorem{Coordinates.}{thm:reciprocal:820}{
Given the definitions above, we may compute the coordinates of a vector, simply by dotting with the basis elements
\begin{equation*}
x^\mu = x \cdot \gamma^\mu,
\end{equation*}
and
\begin{equation*}
x_\mu = x \cdot \gamma_\mu,
\end{equation*}
} % theorem
\begin{proof}
This follows by straightforward computation
\begin{dmath}\label{eqn:reciprocal:840}
x \cdot \gamma^\mu
=
\lr{ x^\nu \gamma_\nu } \cdot \gamma^\mu
=
x^\nu \lr{ \gamma_\nu \cdot \gamma^\mu }
=
x^\nu {\delta_\nu}^\mu
=
x^\mu,
\end{dmath}
and
\begin{dmath}\label{eqn:reciprocal:860}
x \cdot \gamma_\mu
=
\lr{ x_\nu \gamma^\nu } \cdot \gamma_\mu
=
x_\nu \lr{ \gamma^\nu \cdot \gamma_\mu }
=
x_\nu {\delta^\nu}_\mu
=
x_\mu.
\end{dmath}
\end{proof}
\section{Derivative operators.}
We'd like to determine the form of the (spacetime) gradient operator.  The gradient can be defined in terms of coordinates directly, but we choose an implicit definition, in terms of the directional derivative.
\makedefinition{Directional derivative and gradient.}{dfn:reciprocal:880}{
Let \( F = F(x) \) be a four-vector parameterized multivector.  The directional derivative of \( F \) with respect to the (four-vector) direction \( a \) is denoted
\begin{equation*}
\lr{ a \cdot \grad } F = \lim_{\epsilon \rightarrow 0} \frac{ F(x + \epsilon a) - F(x) }{ \epsilon },
\end{equation*}
where \( \grad \) is called the space time gradient.
} % definition
\maketheorem{Gradient.}{thm:reciprocal:900}{
The standard basis representation of the gradient is
\begin{equation*}
\grad = \gamma^\mu \partial_\mu,
\end{equation*}
where
\begin{equation*}
\partial_\mu = \PD{x^\mu}{}.
\end{equation*}
} % theorem
\begin{proof}
The Dirac gradient pops naturually out of the coordinate representation of the directional derivative, as we can see by
expanding \( F(x + \epsilon a) \) in Taylor series
\begin{dmath}\label{eqn:reciprocal:900}
F(x + \epsilon a)
= F(x) + \epsilon \frac{dF(x + \epsilon a)}{d\epsilon} + O(\epsilon^2)
= F(x) + \epsilon \PD{\lr{x^\mu + \epsilon a^\mu}}{F} \PD{\epsilon}{\lr{x^\mu + \epsilon a^\mu}}
= F(x) + \epsilon \PD{\lr{x^\mu + \epsilon a^\mu}}{F} a^\mu.
\end{dmath}
The directional derivative is
\begin{dmath}\label{eqn:reciprocal:920}
\lim_{\epsilon \rightarrow 0}
\frac{F(x + \epsilon a) - F(x)}{\epsilon}
=
\lim_{\epsilon \rightarrow 0}\,
a^\mu
\PD{\lr{x^\mu + \epsilon a^\mu}}{F}
=
a^\mu
\PD{x^\mu}{F}
=
\lr{a^\nu \gamma_\nu} \cdot \gamma^\mu \PD{x^\mu}{F}
=
a \cdot \lr{ \gamma^\mu \partial_\mu } F.
\end{dmath}
\end{proof}
\section{Curvilinear bases.}
Curvilinear bases are the foundation of the fundamental theorem of multivector calculus.  This form of integral calculus is defined over parameterized surfaces (called manifolds) that satisfy some specific non-degeneracy and continuity requirements.

A parameterized vector \( x(u,v, \cdots w) \) can be thought of as tracing out a hypersurface (curve, surface, volume, ...), where the dimension of the hypersurface depends on the number of parameters.
At each point, a bases can be constructed from the differentials of the parameterized vector.  Such a basis is called the tangent space to the surface at the point in question.  Our curvilinear bases will be related to these differentials.  We will also be interested in a dual basis that is restricted to the span of the tangent space.  This dual basis will be called the reciprocal frame, and line the basis of the tangent space itself, also varies from point to point on the surface.

One and two parameter spaces are illustrated in \cref{fig:tangentSpaceSurface:tangentSpaceSurfaceFig1}.
%\cref{fig:oneParameterDifferentialArrows:oneParameterDifferentialArrowsFig1}.
%\imageFigure{../figures/GAelectrodynamics/bw/oneParameterDifferentialArrowsFig1}{One parameter curve, and some tangent vectors.}{fig:oneParameterDifferentialArrows:oneParameterDifferentialArrowsFig1}{0.2}
%\cref{fig:tangentSpaceSurface:tangentSpaceSurfaceFig1}.
%\imageFigure{../figures/GAelectrodynamics/bw/tangentSpaceSurfaceFig1}{Two parameter curve, with some tangent planes.}{fig:tangentSpaceSurface:tangentSpaceSurfaceFig1}{0.2}
\imageTwoFigures
{../figures/GAelectrodynamics/bw/oneParameterDifferentialArrowsFig1}
{../figures/GAelectrodynamics/bw/tangentSpaceSurfaceFig1}{One and two parameter curves, with illustration of tangent spaces.}
{fig:tangentSpaceSurface:tangentSpaceSurfaceFig1}
{scale=0.6}

The tangent space basis at a specific point of a two parameter
surface, \( x(u^0, u^1) \),
is illustrated in
\cref{fig:twoParameterDifferentialCov:twoParameterDifferentialCovFig1}.  The differential directions that span the tangent space are
\begin{equation}\label{eqn:reciprocal:1040}
\begin{aligned}
d\Bx_0 &= \PD{u^0}{x} du^0 \\
d\Bx_1 &= \PD{u^1}{x} du^1,
\end{aligned}
\end{equation}
and the tangent space itself is \( \Span\setlr{ d\Bx_0, d\Bx_1 } \).  We may form an oriented surface area element \( d\Bx_0 \wedge d\Bx_1 \) over this surface.
\imageFigure{../figures/GAelectrodynamics/twoParameterDifferentialCovFig1}{Two parameter surface.}{fig:twoParameterDifferentialCov:twoParameterDifferentialCovFig1}{0.3}
Tangent spaces associated with 3 or more parameters cannot be easily visualized in three dimensions, but the idea generalizes algebraically without trouble.
\makedefinition{Tangent basis and space.}{dfn:reciprocal:100}{
Given a parameterization \( x = x(u^0, \cdots, u^N) \), where \( N < 4 \), the span of the vectors
\begin{equation*}
\Bx_\mu = \PD{u^\mu}{x},
\end{equation*}
is called the tangent space for the hypersurface associated with the parameterization, and it's basis is
\( \setlr{ \Bx_\mu } \).
%This subspace is called a manifold.
} % definition
\makedefinition{Hypervolume elements.}{dfn:reciprocal:1060}{
The differentials that lie along each of the tangent basis directions
\begin{equation*}
d\Bx_\mu = \Bx_\mu du^\mu, \qquad \mbox{(no sum)},
\end{equation*}
we form line, surface, volume, and hypervolume elements by repeated wedge products
\begin{equation*}
\begin{aligned}
d^1\Bx &= d\Bx_0 = \Bx_0 du^0 \\
d^2\Bx &= d\Bx_0 \wedge d\Bx_1 = \lr{\Bx_0 \wedge \Bx_1} du^0 du^1 \\
d^3\Bx &= d\Bx_0 \wedge d\Bx_1 \wedge d\Bx_2 = \lr{\Bx_0 \wedge \Bx_1 \wedge \Bx_2 } du^0 du^1 du^2 \\
d^4\Bx &= d\Bx_0 \wedge d\Bx_1 \wedge d\Bx_2 \wedge d\Bx_3 = \lr{\Bx_0 \wedge \Bx_1 \wedge \Bx_2 \wedge \Bx_3 } du^0 du^1 du^2 du^3.
\end{aligned}
\end{equation*}
%will be of interest when formulating the fundamental theorem of geometric algebra.
} % definition
For such hypervolume elements to be meaningful, we impose a non-degeneracy constraint, and require that the
%In particular, the space must be non-degenerate, which means that the
volume elements above are non-zero (the Jacobian's of the partials are not zero.)  This non-zero constraint is required to ensure that the orientation of the volume elements do not change sign (as in differential forms, we do not take the absolute value of any Jacobians.)
For non-Euclidean spaces, we must also require that these volume elements are invertible, since that does not generally follow.
The parameterized hypersurfaces that we integrate over, when augmented with these additional constraints, are referred to as manifolds.

It's worth considering a couple examples of degenerate parameterizations.  FIXME...
\chapter{foo}
\maketheorem{Gradient, curvilinear representation.}{thm:reciprocal:940}{
Given a non-degenerate spacetime parameterization \( x = x(u^0, u^1, u^2, u^3) \), the gradient with respect to the parameters \( u^\mu \) is
\begin{equation*}
\grad = \lr{ \grad u^\mu } \PD{u^\mu}{}.
\end{equation*}
} % theorem
\begin{proof}
This follows by application of the chain rule.
\begin{dmath}\label{eqn:reciprocal:960}
\grad F
=
\partial^\alpha \PD{x^\alpha}{F}
=
\partial^\alpha
\PD{x^\alpha}{u^\mu}
\PD{u^\mu}{F}
=
\lr{ \grad u^\mu } \PD{u^\mu}{F}.
\end{dmath}
\end{proof}
\makedefinition{Curvilinear basis.}{dfn:reciprocal:980}{
We define
\begin{dmath}\label{eqn:reciprocal:280}
\Bx^\mu = \grad u^\mu,
\end{dmath}
and
\begin{dmath}\label{eqn:reciprocal:300}
\Bx_\mu = \PD{u^\mu}{x}.
\end{dmath}
} % definition
\section{REWRITE MARKER II.}

Incidentally, we now have enough prerequiste baggage that we can express the volume element for the manifold, which is just
\( d^N x = d\Bx_0 \wedge \cdots \wedge d\Bx_{N-1} \).
We'll use this when we get to geometric calculus over spacetime, but to start with, we want to relate these differential directions to our curvilinear gradient.
Rather miraculously, these \( \Bx_\mu\)'s are exactly the reciprocal frame vectors that we need for the gradient.
\maketheorem{Reciprocal frame.}{thm:reciprocal:120}{
The vectors \( \Bx^\mu = \grad u^\mu \), and \( \Bx_\mu = \PDi{u^\mu}{x} \) satisfy the reciprocal relationship
\begin{equation*}
   \Bx^\mu \cdot \Bx_\nu = {\delta^\mu}_\nu.
\end{equation*}
} % theorem
\begin{proof}
\begin{dmath}\label{eqn:reciprocal:1020}
   \Bx^\mu \cdot \Bx_\nu
   =
   \grad u^\mu \cdot
   \PD{u^\nu}{x}
   =
   \lr{
      \gamma^\alpha \PD{x^\alpha}{u^\mu}
   }
   \cdot
   \lr{
      \PD{u^\nu}{x^\beta} \gamma_\beta
   }
   =
   {\delta^\alpha}_\beta \PD{x^\alpha}{u^\mu}
      \PD{u^\nu}{x^\beta}
   =
   \PD{x^\alpha}{u^\mu} \PD{u^\nu}{x^\alpha}
   =
   \PD{u^\nu}{u^\mu}
   =
   {\delta^\mu}_\nu
.
\end{dmath}
\end{proof}

%%%Exploring that idea, I realized that this is not sufficient to define the reciprocal frame, as it is not unique.
%%%I think that this must also be supplemented by a requirement that the span of both sets are identical.
%%%
%%%\section{Recap.}
%%%Evaluating a multivector directional derivative, in terms of components motivates the definition of the space time gradient as
%%%where \( \partial_\mu = \PDi{x^\mu}{} \).
%%%From this one can show that these pairs of vectors satisfy \cref{eqn:reciprocal:200}.
%%%However, the gradients of \cref{eqn:reciprocal:280} can be really hard to compute.  Thankfully, we can avoid that computation, and compute these algebraically.
%%%When the set of parameters is complete, this can be done unambigously, but if we have an incomplete set of parameters (i.e. \( x = x(u^0, \cdots, u^{N-1}) \), where \( N < 4 \), then we have parameterized a subspace and not the whole space.
%%%In this case, things are not as obvious.
%%%
%%%\section{Guts.}
%%%\maketheorem{Satisfying the reciprocal frame requirement.}{thm:reciprocal:140}{
%%%Given a set of \( N \) vectors \( \setlr{\Bx_0, \cdots \Bx_{N-1}} \), let \( [\Bx_\mu] \) and \( [\Bx^\nu] \) be column matrices with the coordinates of these vectors and their reciprocals, with respect to the
%%%standard basis \( \setlr{\gamma_0, \gamma_1, \gamma_2, \gamma_3 } \).
%%%Let
%%%\begin{equation*}
%%%A =
%%%\begin{bmatrix}
%%%   [\Bx_0] & \cdots & [\Bx_{N-1}]
%%%\end{bmatrix}
%%%,\qquad
%%%X =
%%%\begin{bmatrix}
%%%   [\Bx^0] & \cdots & [\Bx^{N-1}]
%%%\end{bmatrix}.
%%%\end{equation*}
%%%The coordinates of the reciprocal frame vectors can be found by solving
%%%\begin{equation*}
%%%   A^\T G X = 1,
%%%\end{equation*}
%%%where \( G = \diag(1,-1,-1,-1) \) and the RHS \( 1 \) is an \( N \times N \) identity matrix.
%%%} % theorem
%%%In general, if \( N < 4 \), this is not a square system of equations.
%%%% No:!
%%%%%%We should be able to solve it analytically using Moore-Penrose inversion, but that is not neccessarily as simple as
%%%%%%\begin{equation*}
%%%%%%B = \lr{ G A A^\T G }^{-1} G A,
%%%%%%\end{equation*}
%%%%%%since \( G A A^T G \) may be singular.
%%%%%%%or (probably better), use row reduction, or general software linear solvers such as Mathematica's LinearSolve.
%%%% => \( G A A^T G \) will be singular if N < 4.
%%%
%%%In the special case that \( X \) contains zero rows, then they can be removed (with corresponding adjustments to \( G, B \)), and the reduced equation, which may be square, can be solved instead.  Such a reduced set of equations may provide the reciprocal frame vectors that satisfy the gradient identity
%%%\cref{eqn:reciprocal:260}, but that really needs to be shown.
%%%
%%%First, let's just satisfy ourselves that we have stated the matrix equivalent of the duality relationship correctly.
%%%\makeproblem{Reciprocal frame computation.}{problem:reciprocal:140}{
%%%Prove \cref{thm:reciprocal:140}.
%%%} % problem
%%%\makeanswer{problem:reciprocal:140}{
%%%\begin{proof}
%%%Let \( \Bx_\mu = {a_\mu}^\alpha \gamma_\alpha, \Bx^\nu = b^{\nu\beta} \gamma_\beta \), so that
%%%\begin{dmath}\label{eqn:reciprocal:140}
%%%   A =
%%%\begin{bmatrix}
%%%   {a_\nu}^\mu
%%%\end{bmatrix},
%%%\end{dmath}
%%%and
%%%\begin{dmath}\label{eqn:reciprocal:160}
%%%   X =
%%%\begin{bmatrix}
%%%   b^{\nu\mu}
%%%\end{bmatrix},
%%%\end{dmath}
%%%where \( \mu \in [0,3]\) are the row indexes and \( \nu \in [0,N-1]\) are the column indexes.
%%%The reciprocal frame satisfies \( \Bx_\mu \cdot \Bx^\nu = {\delta_\mu}^\nu \), which has the coordinate representation of
%%%\begin{dmath}\label{eqn:reciprocal:180}
%%%\Bx_\mu \cdot \Bx^\nu
%%%=
%%%\lr{
%%%   {a_\mu}^\alpha \gamma_\alpha
%%%}
%%%\cdot
%%%\lr{
%%%   b^{\nu\beta} \gamma_\beta
%%%}
%%%=
%%%   {a_\mu}^\alpha
%%%   \eta_{\alpha\beta}
%%%   b^{\nu\beta}
%%%=
%%%{[A^\T G B]_\mu}^\nu,
%%%\end{dmath}
%%%where \( \mu \) is the row index and \( \nu \) is the column index.
%%%\end{proof}
%%%} % answer
%%%Having done that mechanical task, let's consider some examples.
%%%\makeproblem{Two parameter basis.}{problem:reciprocal:320}{
%%%Given \( \Bx_0 = \gamma_0 + \gamma_1 \), and \( \Bx_1 = \gamma_0 - \gamma_1 \), compute a reciprocal frame.
%%%%These are the curvilinear frame for a parameterization such as
%%%%\begin{equation*}
%%%%x = u^0 \Bx_0 + u^1 \Bx_1,
%%%%\end{equation*}
%%%% (or anything else for which this is a first order approximation of.)
%%%} % problem
%%%\makeanswer{problem:reciprocal:320}{
%%%We can form
%%%\begin{dmath}\label{eqn:reciprocal:320}
%%%A =
%%%\begin{bmatrix}
%%%	1 & 1 \\
%%%	1 & -1 \\
%%%	0 & 0\\
%%%	0 & 0
%%%\end{bmatrix},
%%%\end{dmath}
%%%so that the complete system of equations for the coordinates of the reciprocal frame vectors is
%%%\begin{dmath}\label{eqn:reciprocal:340}
%%%\begin{bmatrix}
%%%	1 & 1 \\
%%%	1 & -1 \\
%%%	0 & 0\\
%%%	0 & 0
%%%\end{bmatrix}
%%%\begin{bmatrix}
%%%	1 & 0 & 0 & 0 \\
%%%	0 & -1 & 0 & 0 \\
%%%	0 & 0 & -1 & 0 \\
%%%	0 & 0 & 0 & -1 \\
%%%\end{bmatrix}
%%%X =
%%%\begin{bmatrix}
%%%	1 & 0 \\
%%%	0 & 1
%%%\end{bmatrix}.
%%%\end{dmath}
%%%Dropping all the coordinates that aren't involved in the parameterization, we can solve the simpler reduced system instead
%%%\begin{dmath}\label{eqn:reciprocal:360}
%%%\begin{bmatrix}
%%%	1 & 1 \\
%%%	1 & -1 \\
%%%\end{bmatrix}
%%%\begin{bmatrix}
%%%	1 & 0 \\
%%%	0 & -1 \\
%%%\end{bmatrix}
%%%X' =
%%%\begin{bmatrix}
%%%	1 & 0 \\
%%%	0 & 1
%%%\end{bmatrix},
%%%\end{dmath}
%%%or
%%%\begin{dmath}\label{eqn:reciprocal:380}
%%%\begin{bmatrix}
%%%	1 & -1 \\
%%%	1 & 1 \\
%%%\end{bmatrix}
%%%X' = 1.
%%%\end{dmath}
%%%This is now nicely invertable, yielding
%%%\begin{dmath}\label{eqn:reciprocal:400}
%%%	X' = \inv{2}
%%%\begin{bmatrix}
%%%	1 & 1 \\
%%%	-1 & 1 \\
%%%\end{bmatrix},
%%%\end{dmath}
%%%or
%%%\begin{equation}\label{eqn:reciprocal:420}
%%%	\Bx^0 = \inv{2} \lr{ \gamma_0 - \gamma_1 }, \qquad
%%%	\Bx^1 = \inv{2} \lr{ \gamma_0 + \gamma_1 }.
%%%\end{equation}
%%%It is straightforward to show that these satisfy the desired duality relations, that is
%%%\begin{equation}\label{eqn:reciprocal:440}
%%%	\Bx^0 \cdot \Bx_0 = \inv{2} \lr{ \gamma_0^2 - \gamma_1^2 } = 1 = \Bx^1 \cdot \Bx_1,
%%%\end{equation}
%%%\begin{equation}\label{eqn:reciprocal:460}
%%%	\Bx^0 \cdot \Bx_1 = \inv{2} \lr{ \gamma_0^2 + \gamma_1^2 } = 0 = \Bx^1 \cdot \Bx_0.
%%%\end{equation}
%%%
%%%Observe, however, that dropping the two rows of the matrix equation, really means that we are free to add anything from \( \Span \setlr{ \gamma_2, \gamma_3 } \) to our \( \Bx^\mu \) vectors, and they will still satisfy those relations.
%%%} % answer
%%%\makeproblem{A less trivial example.}{problem:reciprocal:480}{
%%%Now solve the matrix equations for the reciprocal frame that result from the basis vectors
%%%\begin{dmath}\label{eqn:reciprocal:480}
%%%	\Bx_0 = \gamma_0 + \gamma_1 + \gamma_2 + \gamma_3,
%%%\end{dmath}
%%%\begin{dmath}\label{eqn:reciprocal:500}
%%%	\Bx_1 = \gamma_0 - \gamma_1 + \gamma_2 - \gamma_3.
%%%\end{dmath}
%%%Find at least two sets of solutions to these equations.
%%%} % problem
%%%\makeanswer{problem:reciprocal:480}{
%%%Presuming that these are curvilinear coordinates at some point, the tangent space at that point is in the span of \( \Span \setlr{ \gamma_0 + \gamma_2, \gamma_1 + \gamma_3 } \).
%%%We really desire a solution that falls within that span, but a simple row reduction gives us a vastly different answer.  The equations that constrain the reciprocal frame are
%%%\begin{dmath}\label{eqn:reciprocal:520}
%%%	{
%%%	\begin{bmatrix}
%%%	1 & 1 \\
%%%	1 & -1 \\
%%%	1 & 1 \\
%%%	1 & -1
%%%	\end{bmatrix}
%%%	}^\T G X =
%%%\begin{bmatrix}
%%%	1 & 0 \\
%%%	0 & 1 \\
%%%\end{bmatrix},
%%%\end{dmath}
%%%or
%%%\begin{dmath}\label{eqn:reciprocal:540}
%%%	\begin{bmatrix}
%%%		1 & -1 & -1 & -1 \\
%%%		1 & 1 & -1 & 1 \\
%%%	\end{bmatrix}
%%%	X = 1.
%%%\end{dmath}
%%%Row reduction yields
%%%\begin{dmath}\label{eqn:reciprocal:620}
%%%\begin{bmatrix}
%%%1 & 0 & -1 & 0 & \frac{1}{2} & \frac{1}{2} \\
%%%0 & 1 & 0 & 1 & -\frac{1}{2} & \frac{1}{2} \\
%%%\end{bmatrix},
%%%\end{dmath}
%%%from which we can easily read off two sets of solutions
%%%\begin{subequations}
%%%\label{eqn:reciprocal:560}
%%%\begin{equation}\label{eqn:reciprocal:580}
%%%	\Bx^0 = (1/2)\lr{ \gamma_0 - \gamma_1 }, \qquad
%%%	\Bx^1 = (1/2)\lr{ \gamma_0 + \gamma_1 },
%%%\end{equation}
%%%\begin{equation}\label{eqn:reciprocal:600}
%%%	\Bx^0 = (-1/2)\lr{ \gamma_2 + \gamma_3 }, \qquad
%%%	\Bx^1 = (1/2)\lr{ \gamma_2 - \gamma_3 }.
%%%\end{equation}
%%%\end{subequations}
%%%However, neither of these is satifactory, as they clearly are not contained within the desired span.
%%%Using Mathematica's PseudoInverse gives a more satisfactory result, namely
%%%\begin{equation}\label{eqn:reciprocal:680}
%%%\Bx^0 = (1/4)\lr{ \gamma_0 - \gamma_1 - \gamma_2 - \gamma_3 }, \qquad
%%%\Bx^1 = (1/4)\lr{ \gamma_0 + \gamma_1 - \gamma_2 + \gamma_3 }.
%%%\end{equation}
%%%One can quickly check that this also satisfies the reciprocal relations.  Even better, this solution is in the span of the original vectors.
%%%} % answer
%%%Blundering through a numerical example above, we found that the PseudoInverse gave the best results.  However, the usual way that the reciprocal frame is expressed is as follows.
%%%%\makedefinition{Reciprocal relations.}{dfn:reciprocal:700}{
%%%%} % definition
%}
\EndArticle
%\EndNoBibArticle
