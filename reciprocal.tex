%
% Copyright � 2020 Peeter Joot.  All Rights Reserved.
% Licenced as described in the file LICENSE under the root directory of this GIT repository.
%
%{
\input{../latex/blogpost.tex}
\renewcommand{\basename}{reciprocal}
%\renewcommand{\dirname}{notes/phy1520/}
\renewcommand{\dirname}{notes/ece1228-electromagnetic-theory/}
%\newcommand{\dateintitle}{}
%\newcommand{\keywords}{}

\input{../latex/peeter_prologue_print2.tex}

\usepackage{amsthm}
\usepackage{peeters_layout_exercise}
\usepackage{peeters_braket}
\usepackage{peeters_figures}
\usepackage{siunitx}
\usepackage{verbatim}
%\usepackage{mhchem} % \ce{}
%\usepackage{macros_bm} % \bcM
%\usepackage{macros_qed} % \qedmarker
%\usepackage{txfonts} % \ointclockwise

\beginArtNoToc

\generatetitle{Reciprocal frame uniqueness}
%\chapter{Reciprocal frame uniqueness}
%\label{chap:reciprocal}

\section{Motivation.}
I was pondering the computation of reciprocal frame vectors, and initially thought that the typical duality construction could be avoided, using a simpler matrix solution instead.
The reciprocal basis \( \setlr{ \Bx^\mu} \) is related to the curvilinear basis \( \setlr{ \Bx_\mu} \) by
\begin{dmath}\label{eqn:reciprocal:200}
\Bx^\mu \cdot \Bx_\nu = {\delta^\mu}_\nu.
\end{dmath}
Exploring that idea, I realized that this is not sufficient to define the reciprocal frame, as it is not unique.
I think that this must also be supplemented by a requirement that the span of both sets are identical.

In these notes, I am assuming that we are using the Dirac algebra (expressed using the geometric algebra Space Time Algebra (STA) notation.)
\section{Recap.}
Evaluating a multivector directional derivative, in terms of components motivates the definition of the space time gradient as
\begin{dmath}\label{eqn:reciprocal:220}
	\grad = \gamma^\mu \partial_\mu,
\end{dmath}
where \( \partial_\mu = \PDi{x^\mu}{} \).  If we introduce a parameterization for the space
\begin{dmath}\label{eqn:reciprocal:240}
	x = x(u^1, u^2, u^3, u^4),
\end{dmath}
then application of the chain rule can be used to show that the gradient takes the form
\begin{dmath}\label{eqn:reciprocal:260}
	\grad = \lr{ \grad u^\mu } \PD{u^\mu}{}
\end{dmath}
We define
\begin{dmath}\label{eqn:reciprocal:280}
	\Bx^\mu = \grad u^\mu,
\end{dmath}
and
\begin{dmath}\label{eqn:reciprocal:300}
	\Bx_\mu = \PD{u^\mu}{x}.
\end{dmath}
From this one can show that these pairs of vectors satisfy \cref{eqn:reciprocal:200}.
However, the gradients of \cref{eqn:reciprocal:280} can be really hard to compute.  Thankfully, we can avoid that computation, and compute these algebraically.
When the set of parameters is complete, this can be done unambigously, but if we have an incomplete set of parameters (i.e. \( x = x(u^1, \cdots, u^N) \), where \( N < 4 \), then we have parameterized a subspace and not the whole space.
In this case, things are not as obvious.

\section{Guts.}
\maketheorem{Satisfying the reciprocal frame requirement.}{thm:reciprocal:140}{
Given a set of \( N \) vectors \( \setlr{\Bx_0, \cdots \Bx_{N-1}} \), let \( [\Bx_\mu] \) and \( [\Bx^\nu] \) be column matrices with the coordinates of these vectors and their reciprocals, with respect to the
standard basis \( \setlr{\gamma_0, \gamma_1, \gamma_2, \gamma_3 } \).
Let
\begin{equation*}
A =
\begin{bmatrix}
   [\Bx_0] & \cdots & [\Bx_{N-1}]
\end{bmatrix}
,\qquad
X =
\begin{bmatrix}
   [\Bx^0] & \cdots & [\Bx^{N-1}]
\end{bmatrix}.
\end{equation*}
The coordinates of the reciprocal frame vectors can be found by solving
\begin{equation*}
   A^\T G X = 1,
\end{equation*}
where \( G = \diag(1,-1,-1,-1) \) and the RHS \( 1 \) is an \( N \times N \) identity matrix.
} % theorem
In general, if \( N < 4 \), this is not a square system of equations.  We can solve it analytically using Moore-Penrose inversion
\begin{equation*}
B = \lr{ G A A^\T G }^{-1} G A,
\end{equation*}
or (probably better), use row reduction, or general software linear solvers such as Mathematica's LinearSolve.

In the special case that \( X \) contains zero rows, then they can be removed (with corresponding adjustments to \( G, B \)), and the reduced equation, which may be square, can be solved instead.  Such a reduced set of equations may provide the reciprocal frame vectors that satisfy the gradient identity
\cref{eqn:reciprocal:260}, but that really needs to be shown.

First, let's just satisfy ourselves that we have stated the matrix equivalent of the duality relationship correctly.
\makeproblem{Reciprocal frame computation.}{problem:reciprocal:140}{
Prove \cref{thm:reciprocal:140}.
} % problem
\makeanswer{problem:reciprocal:140}{
\begin{proof}
Let \( \Bx_\mu = {a_\mu}^\alpha \gamma_\alpha, \Bx^\nu = b^{\nu\beta} \gamma_\beta \), so that
\begin{dmath}\label{eqn:reciprocal:140}
   A =
\begin{bmatrix}
   {a_\nu}^\mu
\end{bmatrix},
\end{dmath}
and
\begin{dmath}\label{eqn:reciprocal:160}
   X =
\begin{bmatrix}
   b^{\nu\mu}
\end{bmatrix},
\end{dmath}
where \( \mu \in [0,3]\) are the row indexes and \( \nu \in [0,N-1]\) are the column indexes.
The reciprocal frame satisfies \( \Bx_\mu \cdot \Bx^\nu = {\delta_\mu}^\nu \), which has the coordinate representation of
\begin{dmath}\label{eqn:reciprocal:180}
\Bx_\mu \cdot \Bx^\nu
=
\lr{
   {a_\mu}^\alpha \gamma_\alpha
}
\cdot
\lr{
   b^{\nu\beta} \gamma_\beta
}
=
   {a_\mu}^\alpha
   \eta_{\alpha\beta}
   b^{\nu\beta}
=
{[A^\T G B]_\mu}^\nu,
\end{dmath}
where \( \mu \) is the row index and \( \nu \) is the column index.
\end{proof}
} % answer
Having done that mechanical task, let's consider some examples.
\makeproblem{Two parameter basis.}{problem:reciprocal:320}{
Given \( \Bx_0 = \gamma_0 + \gamma_1 \), and \( \Bx_1 = \gamma_0 - \gamma_1 \), compute a reciprocal frame.
%These are the curvilinear frame for a parameterization such as
%\begin{equation*}
%x = u^0 \Bx_0 + u^1 \Bx_1,
%\end{equation*}
% (or anything else for which this is a first order approximation of.)
} % problem
\makeanswer{problem:reciprocal:320}{
We can form
\begin{dmath}\label{eqn:reciprocal:320}
A =
\begin{bmatrix}
	1 & 1 \\
	1 & -1 \\
	0 & 0\\
	0 & 0
\end{bmatrix},
\end{dmath}
so that the complete system of equations for the coordinates of the reciprocal frame vectors is
\begin{dmath}\label{eqn:reciprocal:340}
\begin{bmatrix}
	1 & 1 \\
	1 & -1 \\
	0 & 0\\
	0 & 0
\end{bmatrix}
\begin{bmatrix}
	1 & 0 & 0 & 0 \\
	0 & -1 & 0 & 0 \\
	0 & 0 & -1 & 0 \\
	0 & 0 & 0 & -1 \\
\end{bmatrix}
X =
\begin{bmatrix}
	1 & 0 \\
	0 & 1
\end{bmatrix}.
\end{dmath}
Dropping all the coordinates that aren't involved in the parameterization, we can solve the simpler reduced system instead
\begin{dmath}\label{eqn:reciprocal:360}
\begin{bmatrix}
	1 & 1 \\
	1 & -1 \\
\end{bmatrix}
\begin{bmatrix}
	1 & 0 \\
	0 & -1 \\
\end{bmatrix}
X' =
\begin{bmatrix}
	1 & 0 \\
	0 & 1
\end{bmatrix},
\end{dmath}
or
\begin{dmath}\label{eqn:reciprocal:380}
\begin{bmatrix}
	1 & -1 \\
	1 & 1 \\
\end{bmatrix}
X' = 1.
\end{dmath}
This is now nicely invertable, yielding
\begin{dmath}\label{eqn:reciprocal:400}
	X' = \inv{2}
\begin{bmatrix}
	1 & 1 \\
	-1 & 1 \\
\end{bmatrix},
\end{dmath}
or
\begin{equation}\label{eqn:reciprocal:420}
	\Bx^0 = \inv{2} \lr{ \gamma_0 - \gamma_1 }, \qquad
	\Bx^1 = \inv{2} \lr{ \gamma_0 + \gamma_1 }.
\end{equation}
It is straightforward to show that these satisfy the desired duality relations, that is
\begin{equation}\label{eqn:reciprocal:440}
	\Bx^0 \cdot \Bx_0 = \inv{2} \lr{ \gamma_0^2 - \gamma_1^2 } = 1 = \Bx^1 \cdot \Bx_1,
\end{equation}
\begin{equation}\label{eqn:reciprocal:460}
	\Bx^0 \cdot \Bx_1 = \inv{2} \lr{ \gamma_0^2 + \gamma_1^2 } = 0 = \Bx^1 \cdot \Bx_0.
\end{equation}

Observe, however, that dropping the two rows of the matrix equation, really means that we are free to add anything from \( \Span \setlr{ \gamma_2, \gamma_3 } \) to our \( \Bx^\mu \) vectors, and they will still satisfy those relations.
} % answer
\makeproblem{A less trivial example.}{problem:reciprocal:480}{
Now solve the matrix equations for the reciprocal frame that result from the basis vectors
\begin{dmath}\label{eqn:reciprocal:480}
	\Bx_0 = \gamma_0 + \gamma_1 + \gamma_2 + \gamma_3,
\end{dmath}
\begin{dmath}\label{eqn:reciprocal:500}
	\Bx_1 = \gamma_0 - \gamma_1 + \gamma_2 - \gamma_3.
\end{dmath}
Find at least two sets of solutions to these equations.
} % problem
\makeanswer{problem:reciprocal:480}{
Presuming that these are curvilinear coordinates at some point, the tangent space at that point is in the span of \( \Span \setlr{ \gamma_0 + \gamma_2, \gamma_1 + \gamma_3 } \).
We really desire a solution that falls within that span, but a simple row reduction gives us a vastly different answer.  The equations that constrain the reciprocal frame are
\begin{dmath}\label{eqn:reciprocal:520}
	{
	\begin{bmatrix}
	1 & 1 \\
	1 & -1 \\
	1 & 1 \\
	1 & -1
	\end{bmatrix}
	}^\T G X =
\begin{bmatrix}
	1 & 0 \\
	0 & 1 \\
\end{bmatrix},
\end{dmath}
or
\begin{dmath}\label{eqn:reciprocal:540}
	\begin{bmatrix}
		1 & -1 & -1 & -1 \\
		1 & 1 & -1 & 1 \\
	\end{bmatrix}
	X = 1.
\end{dmath}
Row reduction yields
\begin{dmath}\label{eqn:reciprocal:620}
\begin{bmatrix}
1 & 0 & -1 & 0 & \frac{1}{2} & \frac{1}{2} \\
0 & 1 & 0 & 1 & -\frac{1}{2} & \frac{1}{2} \\
\end{bmatrix},
\end{dmath}
from which we can easily read off two sets of solutions
\begin{subequations}
\label{eqn:reciprocal:560}
\begin{equation}\label{eqn:reciprocal:580}
	\Bx^0 = (1/2)\lr{ \gamma_0 - \gamma_1 }, \qquad
	\Bx^1 = (1/2)\lr{ \gamma_0 + \gamma_1 },
\end{equation}
\begin{equation}\label{eqn:reciprocal:600}
	\Bx^0 = (-1/2)\lr{ \gamma_2 + \gamma_3 }, \qquad
	\Bx^1 = (1/2)\lr{ \gamma_2 - \gamma_3 }.
\end{equation}
\end{subequations}
However, neither of these is satifactory, as they clearly are not contained within the desired span.
} % answer
%}
\EndArticle
%\EndNoBibArticle
