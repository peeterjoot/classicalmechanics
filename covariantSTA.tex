%
% Copyright © 2020 Peeter Joot.  All Rights Reserved.
% Licenced as described in the file LICENSE under the root directory of this GIT repository.
%
%{
\section{In this chapter.}
This chapter will cover
\begin{itemize}
%\item a lighting review of the geometric algebra STA (Space Time Algebra),
%\item relations between Dirac matrix algebra and STA,
\item derivation of the relativistic form of the Euler-Lagrange equations from the covariant form of the action,
\item relationship of the STA form of the Euler-Lagrange equations to their tensor equivalents,
\item derivation of the Lorentz force equation from the STA Lorentz force Lagrangian,
\item relationship of the STA Lorentz force equation to its equivalent in the tensor formalism,
\item relationship of the STA Lorentz force equation to the traditional vector form.
\end{itemize}
Note that some of the prerequisite ideas and auxiliary details are presented as problems with solutions, all of which the reader is encouraged to try before looking at the solutions.
%
\section{Euler-Lagrange equations.}
I'll start at ground zero, with the derivation of the relativistic form of the Euler-Lagrange equations from the action.
A relativistic action for a single particle system has the form
\begin{dmath}\label{eqn:lorentzForceCovariant:20}
S = \int d\tau L(x, \dot{x}),
\end{dmath}
where
\( x \) is the spacetime coordinate, \( \dot{x} = dx/d\tau \) is the four-velocity,
and \( \tau \) is proper time.
%\( A \) is the four-potential,

\maketheorem{Relativistic Euler-Lagrange equations.}{thm:lorentzForceCovariant:40}{
Let \( x \rightarrow x + \delta x \) be any variation of the Lagrangian four-vector coordinates, where
\( \delta x = 0 \) at the boundaries of the action integral.  The variation of the action is
\begin{equation*}
\delta S = \int d\tau \delta x \cdot \delta L(x, \dot{x}),
\end{equation*}
where
\begin{equation*}
\delta L = \grad L - \frac{d}{d\tau} (\grad_v L),
\end{equation*}
where \( \grad = \gamma^\mu \partial_\mu \) (per \cref{dfn:lorentzForceCovariant:71}), and where we construct a similar velocity-gradient with respect to the proper-time derivatives of the coordinates \( \grad_v = \gamma^\mu \partial/\partial \dot{x}^\mu \).

The action is extremized when \( \delta S = 0 \), or when \( \delta L = 0 \).  This latter condition is called the Euler-Lagrange equations.
} % theorem
\begin{proof}
Let \( \epsilon = \delta x \), and expand the
Lagrangian in Taylor series to first order
\begin{dmath}\label{eqn:lorentzForceCovariant:60}
S \rightarrow
S + \delta S
= \int d\tau L( x + \epsilon, \dot{x} + \dot{\epsilon})
=
\int d\tau \lr{
   L(x, \dot{x}) + \epsilon \cdot \grad L + \dot{\epsilon} \cdot \grad_v L
}.
\end{dmath}
Subtracting off \( S \) and integrating by parts, leaves
\begin{dmath}\label{eqn:lorentzForceCovariant:80}
\delta S =
\int d\tau \epsilon \cdot \lr{
   \grad L - \frac{d}{d\tau} \grad_v L
}
+
\int d\tau \frac{d}{d\tau} (\grad_v L ) \cdot \epsilon.
\end{dmath}
The boundary integral
\begin{equation}\label{eqn:lorentzForceCovariant:100}
\int d\tau \frac{d}{d\tau} (\grad_v L ) \cdot \epsilon
=
\evalbar{(\grad_v L ) \cdot \epsilon}{\Delta \tau} = 0,
\end{equation}
is zero since the variation \( \epsilon \) is required to vanish on the boundaries.  So, if \( \delta S = 0 \), we must have
\begin{dmath}\label{eqn:lorentzForceCovariant:120}
0 =
\int d\tau \epsilon \cdot \lr{
   \grad L - \frac{d}{d\tau} \grad_v L
},
\end{dmath}
for all variations \( \epsilon \).  Clearly, this requires that
\begin{equation}\label{eqn:lorentzForceCovariant:140}
\delta L = \grad L - \frac{d}{d\tau} (\grad_v L) = 0,
\end{equation}
or
\begin{equation}\label{eqn:lorentzForceCovariant:145}
\grad L = \frac{d}{d\tau} (\grad_v L),
\end{equation}
which is the coordinate free statement of the Euler-Lagrange equations.
\end{proof}
\makeproblem{Coordinate form of the Euler-Lagrange equations.}{problem:lorentzForceCovariant:1}{
Working in coordinates, use the action argument show that the Euler-Lagrange equations have the form
\begin{equation*}
   \PD{x^\mu}{L} = \frac{d}{d\tau} \PD{\dot{x}^\mu}{L}
\end{equation*}
Observe that this is identical to the statement of \cref{thm:lorentzForceCovariant:40} after contraction with \( \gamma^\mu \).
} % problem
\makeanswer{problem:lorentzForceCovariant:1}{
%\begin{answerproof}
In terms of coordinates, the first order Taylor expansion of the action is
\begin{dmath}\label{eqn:lorentzForceCovariant:180}
S \rightarrow
S + \delta S
= \int d\tau L( x^\alpha + \epsilon^\alpha, \dot{x}^\alpha + \dot{\epsilon}^\alpha)
=
\int d\tau \lr{
L(x^\alpha, \dot{x}^\alpha) + \epsilon^\mu \PD{x^\mu}{L} + \dot{\epsilon}^\mu \PD{\dot{x}^\mu}{L}
}.
\end{dmath}
As before, we integrate by parts to separate out a pure boundary term
\begin{dmath}\label{eqn:lorentzForceCovariant:200}
\delta S =
\int d\tau \epsilon^\mu
\lr{
   \PD{x^\mu}{L} - \frac{d}{d\tau} \PD{\dot{x}^\mu}{L}
}
+
\int d\tau \frac{d}{d\tau} \lr{
   \epsilon^\mu \PD{\dot{x}^\mu}{L}
}.
\end{dmath}
The boundary term is killed since \( \epsilon^\mu = 0 \) at the end points of the action integral.  We conclude that extremization of the action (\( \delta S = 0 \), for all \( \epsilon^\mu \)) requires
\begin{dmath}\label{eqn:lorentzForceCovariant:220}
   \PD{x^\mu}{L} - \frac{d}{d\tau} \PD{\dot{x}^\mu}{L} = 0.
\end{dmath}
%\end{answerproof}
} % answer
%\subsection{Solutions.}
%\shipoutAnswer
%
\section{Lorentz force equation.}
\maketheorem{Lorentz force.}{thm:lorentzForceCovariant:2}{
The relativistic Lagrangian for a charged particle is
\begin{equation*}
L = \inv{2} m v^2 + q A \cdot v/c.
\end{equation*}
Application of the Euler-Lagrange equations to this Lagrangian yields the Lorentz-force equation
\begin{equation*}
\frac{dp}{d\tau} = q F \cdot v/c,
\end{equation*}
where \( p = m v \) is the proper momentum, \( F \) is the Faraday bivector \( F = \grad \wedge A \), and \( c \) is the speed of light.
} % theorem
\begin{proof}
To make life easier, let's take advantage of the linearity of the Lagrangian, and break it into the free particle Lagrangian \( L_0 = (1/2) m v^2 \) and a potential term \( L_1 = q A \cdot v/c \).  For the free particle case we have
\begin{dmath}\label{eqn:lorentzForceCovariant:240}
\delta L_0
= \grad L_0 - \frac{d}{d\tau} (\grad_v L_0)
=           - \frac{d}{d\tau} (m v)
= - \frac{dp}{d\tau}.
\end{dmath}
For the potential contribution we have
\begin{dmath}\label{eqn:lorentzForceCovariant:260}
\delta L_1
= \grad L_1 - \frac{d}{d\tau} (\grad_v L_1)
= \frac{q}{c} \lr{ \grad (A \cdot v) - \frac{d}{d\tau} \lr{ \grad_v (A \cdot v)} }
= \frac{q}{c} \lr{ \grad (A \cdot v) - \frac{dA}{d\tau} }.
\end{dmath}
The proper time derivative can be evaluated using the chain rule
\begin{equation}\label{eqn:lorentzForceCovariant:280}
\frac{dA}{d\tau}
=
\frac{\partial x^\mu}{\partial \tau} \partial_\mu A
= (v \cdot \grad) A.
\end{equation}
Putting all the pieces back together we have
\begin{dmath}\label{eqn:lorentzForceCovariant:300}
0 =
\delta L
=
-\frac{dp}{d\tau} + \frac{q}{c} \lr{ \grad (A \cdot v) -  (v \cdot \grad) A }
=
-\frac{dp}{d\tau} + \frac{q}{c} \lr{ \grad \wedge A } \cdot v.
\end{dmath}
\end{proof}
\makeproblem{Gradient of a squared position vector.}{problem:lorentzForceCovariant:13}{
Show that
\begin{equation*}
   \grad (a \cdot x) = a,
\end{equation*}
and
\begin{equation*}
   \grad x^2 = 2 x.
\end{equation*}
It should be clear that the same ideas can be used for the velocity gradient, where we obtain \( \grad_v (v^2) = 2 v \), and \( \grad_v (A \cdot v) = A \), as used in the derivation above.
} % problem
\makeanswer{problem:lorentzForceCovariant:13}{
%\begin{answerproof}
The first identity follows easily by expansion in coordinates
\begin{dmath}\label{eqn:lorentzForceCovariant:320}
\grad (a \cdot x)
=
\gamma^\mu \partial_\mu a_\alpha x^\alpha
=
\gamma^\mu a_\alpha \delta_\mu^\alpha
=
\gamma^\mu a_\mu
=
a.
\end{dmath}
The second identity follows by linearity of the gradient
\begin{dmath}\label{eqn:lorentzForceCovariant:340}
\grad x^2
=
\grad (x \cdot x)
=
\evalbar{\lr{\grad (x \cdot a)}}{a = x}
+
\evalbar{\lr{\grad (b \cdot x)}}{b = x}
=
\evalbar{a}{a = x}
+
\evalbar{b}{b = x}
=
2x.
\end{dmath}
%\end{answerproof}
} % answer

It is desirable to put this relativistic Lorentz force equation into the usual vector and tensor forms for comparison.
% The tensor form is pretty easy to extract.
\maketheorem{Tensor form of the Lorentz force equation.}{thm:lorentzForceCovariant:360}{
The tensor form of the Lorentz force equation is
\begin{equation*}
\frac{dp^\mu}{d\tau} = \frac{q}{c} F^{\mu\nu} v_\nu,
\end{equation*}
where the antisymmetric Faraday tensor is defined as \( F^{\mu\nu} = \partial^\mu A^\nu - \partial^\nu A^\mu \).
} % theorem
\begin{proof}
We have only to dot both sides with \( \gamma^\mu \).  On the left we have
\begin{dmath}\label{eqn:lorentzForceCovariant:380}
\gamma^\mu \cdot \frac{dp}{d\tau}
=
\frac{dp^\mu}{d\tau}.
\end{dmath}
On the right, we have
\begin{dmath}\label{eqn:lorentzForceCovariant:400}
\gamma^\mu \cdot \lr{ \frac{q}{c} F \cdot v }
=
 \frac{q}{c}  (( \grad \wedge A ) \cdot v ) \cdot \gamma^\mu
=
 \frac{q}{c}  ( \grad ( A \cdot v ) - (v \cdot \grad) A ) \cdot \gamma^\mu
=
 \frac{q}{c}  \lr{ (\partial^\mu A^\nu) v_\nu - v_\nu \partial^\nu A^\mu }
=
 \frac{q}{c} F^{\mu\nu} v_\nu.
\end{dmath}
\end{proof}
\makeproblem{Tensor expansion of \(F\).}{problem:lorentzForceCovariant:37}{
An alternate way to demonstrate \cref{thm:lorentzForceCovariant:360} is to first expand \( F = \grad \wedge A \) in terms of coordinates, an expansion that can be expressed in terms of a second rank tensor antisymmetric tensor \( F^{\mu\nu} \).  Find that expansion, and re-evaluate the dot products of \cref{eqn:lorentzForceCovariant:400} using that.
} % problem
\makeanswer{problem:lorentzForceCovariant:37}{
\begin{dmath}\label{eqn:lorentzForceCovariant:900}
F =
\grad \wedge A
=
\lr{ \gamma_\mu \partial^\mu } \wedge \lr{ \gamma_\nu A^\nu }
=
\lr{ \gamma_\mu \wedge \gamma_\nu } \partial^\mu A^\nu.
\end{dmath}
To this we can use the usual tensor trick (add self to self, change indexes, and divide by two), to give
\begin{dmath}\label{eqn:lorentzForceCovariant:920}
F =
\inv{2} \lr{
\lr{ \gamma_\mu \wedge \gamma_\nu } \partial^\mu A^\nu
+
\lr{ \gamma_\nu \wedge \gamma_\mu } \partial^\nu A^\mu
}
=
\inv{2}
\lr{ \gamma_\mu \wedge \gamma_\nu } \lr{
\partial^\mu A^\nu
-
\partial^\nu A^\mu
},
\end{dmath}
which is just
\begin{dmath}\label{eqn:lorentzForceCovariant:940}
F =
\inv{2} \lr{ \gamma_\mu \wedge \gamma_\nu } F^{\mu\nu}.
\end{dmath}
Now, let's expand \( (F \cdot v) \cdot \gamma^\mu \) to compare to the earlier expansion in terms of \( \grad \) and \( A \).
\begin{dmath}\label{eqn:lorentzForceCovariant:960}
(F \cdot v) \cdot \gamma^\mu
=
\inv{2}
F^{\alpha\nu}
\lr{ \lr{ \gamma_\alpha \wedge \gamma_\nu } \cdot \lr{ \gamma^\beta v_\beta } } \cdot \gamma^\mu
=
\inv{2}
F^{\alpha\nu} v_\beta
\lr{
   {\delta_\nu}^\beta {\gamma_\alpha}^\mu
-
   {\delta_\alpha}^\beta {\gamma_\nu}^\mu
}
=
\inv{2}
\lr{
   F^{\mu\beta} v_\beta
   -
   F^{\beta\mu} v_\beta
}
=
F^{\mu\nu} v_\nu.
\end{dmath}
This alternate expansion illustrates some of the connectivity between the geometric algebra approach and the traditional tensor formalism.
} % answer
\makeproblem{Lorentz force direct tensor derivation.}{problem:lorentzForceCovariant:420}{
Instead of using the geometric algebra form of the Lorentz force equation as a stepping stone, we may derive the tensor form from the Lagrangian directly, provided the Lagrangian is put into tensor form
\begin{equation*}
L = \inv{2} m v^\mu v_\mu + q A^\mu v_\mu /c.
\end{equation*}
Evaluate the Euler-Lagrange equations in coordinate form and compare to \cref{thm:lorentzForceCovariant:360}.
} % problem
%
\makeanswer{problem:lorentzForceCovariant:420}{
%\begin{answerproof}
Let \( \delta_\mu L = \gamma_\mu \cdot \delta L \), so that we can write the
Euler-Lagrange equations as
\begin{equation}\label{eqn:lorentzForceCovariant:460}
0 = \delta_\mu L = \PD{x^\mu}{L} - \frac{d}{d\tau} \PD{\dot{x}^\mu}{L}.
\end{equation}
Operating on the kinetic term of the Lagrangian, we have
\begin{dmath}\label{eqn:lorentzForceCovariant:480}
\delta_\mu L_0 = - \frac{d}{d\tau} m v_\mu.
\end{dmath}
For the potential term
\begin{dmath}\label{eqn:lorentzForceCovariant:500}
\delta_\mu L_1
=
\frac{q}{c} \lr{
v_\nu \PD{x^\mu}{A^\nu} - \frac{d}{d\tau} A_\mu
}
=
\frac{q}{c} \lr{
v_\nu \PD{x^\mu}{A^\nu} - \frac{dx_\alpha}{d\tau} \PD{x_\alpha}{ A_\mu }
}
=
\frac{q}{c} v^\nu \lr{
\partial_\mu A_\nu - \partial_\nu A_\mu
}
=
\frac{q}{c} v^\nu F_{\mu\nu}.
\end{dmath}
Putting the pieces together gives
\begin{dmath}\label{eqn:lorentzForceCovariant:520}
\frac{d}{d\tau} (m v_\mu) = \frac{q}{c} v^\nu F_{\mu\nu},
\end{dmath}
which is identical\footnote{Some minor index raising and lowering gymnastics are required.} to the tensor form that we found by expanding the geometric algebra form of Maxwell's equation in coordinates.
%\end{answerproof}
} % answer
\maketheorem{Vector Lorentz force equation.}{thm:lorentzForceCovariant:540}{
Relative to a fixed observer's frame, the Lorentz force equation of \cref{thm:lorentzForceCovariant:2} splits into a spatial rate of change of momentum, and (timelike component) rate of change of energy, as follows
\begin{equation*}
\begin{aligned}
   \ddt{(\gamma m \Bv)} &= q \lr{ \BE + \Bv \cross \BB } \\
   \ddt{(\gamma m c^2)} &= q \Bv \cdot \BE,
\end{aligned}
\end{equation*}
where \( F = \BE + I c \BB \), \( \gamma = 1/\sqrt{1 - \Bv^2/c^2 }\).
%, and \( \Bv = \sum_{k = 1}^3 \ifrac{dx^k}{dt} \Be_k \), \( \BE =
} % theorem
\begin{proof}
The first step is to eliminate the proper time dependencies in the Lorentz force equation.  Consider first the coordinate representation of an arbitrary position four-vector \( x \)
\begin{dmath}\label{eqn:lorentzForceCovariant:1140}
x = c t \gamma_0 + x^k \gamma_k.
\end{dmath}
The corresponding four-vector velocity is
\begin{equation}\label{eqn:lorentzForceCovariant:1160}
v = \ddtau{x} = c \ddtau{t} \gamma_0 + \ddtau{t} \ddt{x^k} \gamma_k.
\end{equation}
By construction, \( v^2 = c^2 \) is a Lorentz invariant quantity (this is one of the relativistic postulates), so the LHS of \cref{eqn:lorentzForceCovariant:1160} must have the same square.  That is
\begin{dmath}\label{eqn:lorentzForceCovariant:1240}
c^2 = \lr{ \ddtau{t} }^2 \lr{ c^2 - \Bv^2 },
\end{dmath}
where \( \Bv = v \wedge \gamma_0 \).  This shows that we may make the identification
\begin{equation}\label{eqn:lorentzForceCovariant:1260}
\gamma = \ddtau{t} = \inv{1 - \Bv^2/c^2 },
\end{equation}
and
\begin{equation}\label{eqn:lorentzForceCovariant:1280}
\ddtau{} = \ddtau{t} \ddt{} = \gamma \ddt{}.
\end{equation}
We may now factor the four-velocity \( v \) into its spacetime split
\begin{dmath}\label{eqn:lorentzForceCovariant:1300}
v = \gamma \lr{ c + \Bv } \gamma_0.
\end{dmath}
In particular the LHS of the Lorentz force equation can be rewritten as
\begin{dmath}\label{eqn:lorentzForceCovariant:1320}
\ddtau{p} = \gamma \ddt{}\lr{ \gamma \lr{ c + \Bv } } \gamma_0,
\end{dmath}
and the RHS of the Lorentz force equation can be rewritten as
\begin{dmath}\label{eqn:lorentzForceCovariant:1340}
\frac{q}{c} F \cdot v
=
\frac{\gamma q}{c} F \cdot \lr{ (c + \Bv) \gamma_0 }.
\end{dmath}
Equating timelike and spacelike components leaves us
\begin{subequations}
\label{eqn:lorentzForceCovariant:1360}
\begin{equation}\label{eqn:lorentzForceCovariant:1380}
\ddt{ (m \gamma c) } = \frac{q}{c} \lr{ F \cdot \lr{ (c + \Bv) \gamma_0 } } \cdot \gamma_0,
\end{equation}
\begin{equation}\label{eqn:lorentzForceCovariant:1400}
\ddt{ (m \gamma \Bv) } = \frac{q}{c} \lr{ F \cdot \lr{ (c + \Bv) \gamma_0 } } \wedge \gamma_0,
\end{equation}
\end{subequations}
Evaluating these products requires some care, but is an essentially manual process.  The reader is encouraged to do so once, but the end result may also be obtained easily using software (see lorentzForce.nb in \citep{gapauli}).  One finds
\begin{subequations}
\label{eqn:lorentzForceCovariant:1420}
\begin{dmath}\label{eqn:lorentzForceCovariant:1440}
F = \BE + I c \BB
=
    E^1 \gamma_{10} 
+   E^2 \gamma_{20} 
+   E^3 \gamma_{30} 
- c B^1 \gamma_{23} 
- c B^2 \gamma_{31} 
- c B^3 \gamma_{12},
\end{dmath}
\begin{dmath}\label{eqn:lorentzForceCovariant:1460}
\frac{q}{c} \lr{ F \cdot \lr{ (c + \Bv) \gamma_0 } } \cdot \gamma_0
= \frac{q}{c} \BE \cdot \Bv,
\end{dmath}
\begin{dmath}\label{eqn:lorentzForceCovariant:1480}
\frac{q}{c} \lr{ F \cdot \lr{ (c + \Bv) \gamma_0 } } \wedge \gamma_0
= q \lr{ \BE + \Bv \cross \BB }.
\end{dmath}
\end{subequations}
\end{proof}
\makeproblem{Algebraic spacetime split of the Lorentz force equation.}{problem:lorentzForceCovariant:69}{
Derive the results of \cref{eqn:lorentzForceCovariant:1420} algebraically.
} % problem
\makeanswer{problem:lorentzForceCovariant:69}{
First calculate the field velocity product in terms of electric and magnetic components.  In this new frame of reference write the proper velocity of the charged particle as
\(v = \gamma_\mu \xdot^\mu\)
%
\begin{equation}\label{eqn:lorentzForce:100}
\begin{aligned}
F \cdot v
&= (\BE + I c \BB) \cdot v \\
&= (E^i \gamma_{i0} - \epsilon_{ijk}c B^k \gamma_{ij}) \cdot \gamma_\mu \xdot^\mu \\
&=
  E^i \xdot^0 \gamma_{i0} \cdot \gamma_0
+ E^i \xdot^j \gamma_{i0} \cdot \gamma_j
- \epsilon_{ijk} c B^k \xdot^m \gamma_{ij} \cdot \gamma_m .
\end{aligned}
\end{equation}
%
We apply a \(\gamma_0\) wedge to determine this observer dependent expression of the force.
%
\begin{equation}\label{eqn:lorentzForce:120}
\begin{aligned}
\gamma^{-1} (F \cdot v) \wedge \gamma_0
&=
\left(
  E^i \xdot^0 (\gamma_{i0} \cdot \gamma_0)
+ E^i \xdot^j (\gamma_{i0} \cdot \gamma_j)
- \epsilon_{ijk} c B^k \xdot^m \gamma_{ij} \cdot \gamma_m \right) \wedge \gamma_0 \\
&= E^i \xdot^0 \gamma_{i0} - \epsilon_{ijk} c B^k \xdot^m (\gamma_i)^2 ( \gamma_{i} \delta_{jm} -\gamma_{j} \delta_{im} ) \wedge \gamma_0 \\
&= \left( E^i \xdot^0 \gamma_{i0} + \epsilon_{ijk} c B^k \left( \xdot^j \gamma_{i0} - \xdot^i \gamma_{j0} \right) \right),
\end{aligned}
\end{equation}
where \(\gamma = dt/d\tau\).
%
This wedge application has discarded the timelike components of the force equation with respect to this observer rest frame.
Introduce the basis
\(\{\Be_i = \gamma_i \wedge \gamma_0\}\) for this observers' Euclidean space.  These spacetime bivectors square to unity, and thus behave in every respect like
Euclidean space vector basis vectors.  Writing \(\BE = E^i \Be_i\), \(\BB = B^i \Be_i\), and \(\Bv = \Be_i dx^i/dt\) we have
%
\begin{equation}\label{eqn:lorentzForce:140}
\gamma^{-1} (F \cdot v) \wedge \gamma_0
= 
c \left( \BE + \epsilon_{ijk} B^k \left( \frac{dx^j}{dt} \Be_{i} - \frac{dx^i}{dt} \Be_{j} \right) \right) .
\end{equation}
%
This inner antisymmetric sum is just the cross product.  This can be observed by expanding the determinant
%
\begin{equation}\label{eqn:lorentzForce:160}
\begin{aligned}
\Ba \cross \Bb &=
\begin{vmatrix}
\Be_1 & \Be_2 & \Be_3 \\
a_1 & a_2 & a_3 \\
b_1 & b_2 & b_3 \\
\end{vmatrix} \\
&=
  \Be_1 (a_2 b_3 - a_3 b_2)
+ \Be_2 (a_3 b_1 - a_1 b_3)
+ \Be_3 (a_1 b_2 - a_2 b_1) \\
&=
  \Be_i a_j b_k \epsilon_{ijk}.
\end{aligned}
\end{equation}
%
This leaves
\begin{equation}\label{eqn:lorForce:FdotVwedge}
q (F \cdot v/c) \wedge \gamma_0
= \gamma q \left( \BE + \Bv \cross \BB \right).
\end{equation}
%
Plugging back into \cref{eqn:lorentzForceCovariant:1400} gives
%Next expand the left hand side acceleration term in coordinates, and wedge with \(\gamma_0\)
%%
%\begin{equation}\label{eqn:lorentzForce:180}
%\begin{aligned}
%\pdot \wedge \gamma_0
%&= \left(\gamma_\mu \frac{d m \xdot^\mu}{dt}\frac{dt}{d\tau} \right) \wedge \gamma_0 \\
%&= \gamma_{i0} \frac{d m \xdot^i}{dt} \gamma,
%\end{aligned}
%\end{equation}
%%
%Equating with \eqnref{eqn:lorForce:FdotVwedge}, and cancelling the \(\gamma\) factors, we are left with
%%
\begin{equation}\label{eqn:lorentzForce:200}
\frac{d}{dt}\left( m \gamma \Bv \right) = q \left( \BE + \Bv \cross \BB \right).
\end{equation}
%TODO: do the \cref{eqn:lorentzForceCovariant:1380} expansion algebraically too.
} % answer
\makeproblem{Spacetime split of the Lorentz force tensor equation.}{problem:lorentzForceCovariant:68}{
Show that \cref{thm:lorentzForceCovariant:540} also follows from the tensor form of the Lorentz force equation (\cref{thm:lorentzForceCovariant:360}) provided we identify
\begin{subequations}
\label{eqn:lorentzForceCovariant:1540}
\begin{equation}\label{eqn:lorentzForceCovariant:1500}
F^{k0} = E^k,
\end{equation}
and
\begin{equation}\label{eqn:lorentzForceCovariant:1520}
F^{rs} = -\epsilon^{rst} B^t.
\end{equation}
\end{subequations}

Also verify that the identification \cref{eqn:lorentzForceCovariant:1540}
is consistent with the geometric algebra Faraday bivector \( F = \BE + I c \BB \), and the associated coordinate expansion of the field \( F = (1/2) (\gamma_\mu \wedge \gamma_\nu) F^{\mu\nu} \).
} % problem
%\makeanswer{problem:lorentzForceCovariant:68}{
% TODO.  Have a matrix exposition of F^{\mu\nu} in other places in these notes.  Merge that with this, or potentially move both to the STA chapter?
%} % answer
%}
