%
% Copyright © 2012 Peeter Joot.  All Rights Reserved.
% Licenced as described in the file LICENSE under the root directory of this GIT repository.
%
\makeoproblem{Angular momentum in spherical and cylindrical coordinates.}{classicalMechanicsPs2:problem:5}{'12 phy356 ps2.5}
{
%{2012 PHY354 Problem set 2, problem 5}
\makesubproblem{Find \(M_x, M_y, M_z, \BM^2\) in spherical coordinates \((r, \theta, \phi)\).}{classicalMechanicsPs2:problem:5:1}
\makesubproblem{Find \(M_x, M_y, M_z, \BM^2\) in cylindrical coordinates \((r, \phi, z)\).}{classicalMechanicsPs2:problem:5:2}
%
%Everybody should do this once in their lives (and should be able to do at any point thereafter).
%My SQL-guru, but non-physisist girlfriend, was amused by the additional comment in the statement of the problem ``\textunderline{Everybody} should do this once in their lives (and should be able to do at any point thereafter).
} % makeproblem
%
\makeanswer{classicalMechanicsPs2:problem:5}{
%
\makesubanswer{Spherical coordinates.}{classicalMechanicsPs2:problem:5:1}
%
In Cartesian coordinates our angular momentum is
%
\begin{equation}\label{eqn:classicalMechanicsPs2:950}
\begin{aligned}
\BM
&= \Br \cross (m \Bv) \\
&=
m (y v_z - z v_y) \xcap
+m (z v_x - x v_z) \ycap
+m (x v_y - y v_x) \zcap.
\end{aligned}
\end{equation}
%
Substituting \(x,y,z\) is easy since we have
%
\begin{dmath}\label{eqn:classicalMechanicsPs2:10}
\begin{bmatrix}
x \\
y \\
z
\end{bmatrix}
=
r
\begin{bmatrix}
\sin\theta \cos\phi \\
\sin\theta \sin\phi \\
\cos\theta
\end{bmatrix},
\end{dmath}
%
but the \(\Bv\) substitution requires more work.  We have
%
\begin{equation}\label{eqn:classicalMechanicsPs2:970}
\begin{aligned}
\Bv
&= \frac{d \Br}{dt} \\
&= \frac{d}{dt} (r \rcap) \\
&= \rdot \rcap + r \ddt{\rcap}.
\end{aligned}
\end{equation}
%
\begin{equation}\label{eqn:classicalMechanicsPs2:990}
\begin{aligned}
\ddt{\rcap}
&=
\ddt{}
\begin{bmatrix}
\sin\theta \cos\phi \\
\sin\theta \sin\phi \\
\cos\theta
\end{bmatrix} \\
&=
\begin{bmatrix}
\cos\theta \cos\phi \thetadot - \sin\theta \sin\phi \phidot \\
\cos\theta \sin\phi \thetadot + \sin\theta \cos\phi \phidot \\
-\sin\theta \thetadot
\end{bmatrix}.
\end{aligned}
\end{equation}
%
So we have
%
\begin{subequations}
\label{eqn:classicalMechanicsPs2:30a}
\begin{dmath}\label{eqn:classicalMechanicsPs2:30}
\Bv =
\begin{bmatrix}
\rdot \sin\theta \cos\phi + r \cos\theta \cos\phi \thetadot - r \sin\theta \sin\phi \phidot \\
\rdot \sin\theta \sin\phi + r \cos\theta \sin\phi \thetadot + r \sin\theta \cos\phi \phidot \\
\rdot \cos\theta -r \sin\theta \thetadot
\end{bmatrix},
\end{dmath}
%
\begin{equation}\label{eqn:classicalMechanicsPs2:1010}
\frac{\BM}{mr} =
%m r
\begin{bmatrix}
\sin\theta \sin\phi v_z - \cos\theta v_y \\
\cos\theta v_x - \sin\theta \cos\phi v_z \\
\sin\theta \cos\phi v_y - \sin\theta \sin\phi v_x
\end{bmatrix}.
\end{equation}
\end{subequations}
Expanding this is a bit of a mess, but it eventually simplifies.  We start with
\begin{equation}\label{eqn:classicalMechanicsPs2:1010a}
\begin{bmatrix}
S_\theta S_\phi (\rdot C_\theta -r S_\theta \thetadot) - C_\theta (\rdot S_\theta S_\phi + r C_\theta S_\phi \thetadot + r S_\theta C_\phi \phidot) \\
C_\theta (\rdot S_\theta C_\phi + r C_\theta C_\phi \thetadot - r S_\theta S_\phi \phidot) - S_\theta C_\phi (\rdot C_\theta -r S_\theta \thetadot) \\
S_\theta C_\phi (\rdot S_\theta S_\phi + r C_\theta S_\phi \thetadot + r S_\theta C_\phi \phidot) - S_\theta S_\phi (\rdot S_\theta C_\phi + r C_\theta C_\phi \thetadot - r S_\theta S_\phi \phidot)
\end{bmatrix},
\end{equation}
then
\begin{equation}\label{eqn:classicalMechanicsPs2:1010b}
\begin{bsmallmatrix}
\cancel{\rdot C_\theta S_\theta S_\phi }
- r \thetadot S_\theta^2 S_\phi
- \cancel{\rdot S_\theta C_\theta S_\phi }
- r \thetadot C_\theta^2 S_\phi
- r \phidot S_\theta C_\theta C_\phi
\\
\cancel{\rdot S_\theta C_\theta C_\phi }
+ r \thetadot C_\theta^2 C_\phi
- r \phidot S_\theta C_\theta S_\phi
-\cancel{\rdot C_\theta S_\theta C_\phi }
+r \thetadot S_\theta^2 C_\phi
\\
\cancel{\rdot S_\theta^2 S_\phi C_\phi }
+ \cancel{r \thetadot C_\theta S_\theta C_\phi S_\phi }
+ r \phidot S_\theta^2 C_\phi^2
- \cancel{\rdot S_\theta^2 S_\phi C_\phi }
- \cancel{r \thetadot C_\theta S_\theta S_\phi C_\phi }
+ r \phidot S_\theta^2 S_\phi^2
\end{bsmallmatrix},
\end{equation}
and finally
\begin{equation}\label{eqn:classicalMechanicsPs2:1010c}
\begin{bmatrix}
- r \thetadot S_\phi
- r \phidot S_\theta C_\theta C_\phi
\\
+ r \thetadot C_\phi
- r \phidot S_\theta C_\theta S_\phi
\\
+ r \phidot S_\theta^2
\end{bmatrix}.
\end{equation}
%
In matrix form, we have (and can read off \(M_x, M_y, M_z\))
%
\boxedEquation{eqn:classicalMechanicsPs2:50}{
\BM =
\inv{2} m r^2
\begin{bmatrix}
-  2 \sin\phi & - \sin(2\theta) \cos\phi \\
  2 \cos\phi & - \sin(2\theta) \sin\phi \\
0 & 1 - \cos(2\theta)
\end{bmatrix}
\begin{bmatrix}
\thetadot \\
\phidot
\end{bmatrix}.
}
%
We have also been asked to find \(\BM^2\) and can write this as a quadratic form
%
\begin{equation}\label{eqn:classicalMechanicsPs2:1030}
\begin{aligned}
\BM^2
&=
\inv{4} m^2 r^4
\begin{bmatrix}
\thetadot & \phidot
\end{bmatrix}
\begin{bmatrix}
-  2 \sin\phi  & 2 \cos\phi  & 0 \\
- \sin(2\theta) \cos\phi & - \sin(2\theta) \sin\phi  & 1 - \cos(2\theta)
\end{bmatrix} \times \\
&\quad
\begin{bmatrix}
-  2 \sin\phi & - \sin(2\theta) \cos\phi \\
  2 \cos\phi & - \sin(2\theta) \sin\phi \\
0 & 1 - \cos(2\theta)
\end{bmatrix}
\begin{bmatrix}
\thetadot \\
\phidot
\end{bmatrix}  \\
&=
\inv{4} m^2 r^4
\begin{bmatrix}
\thetadot & \phidot
\end{bmatrix}
\begin{bmatrix}
4 & 0 \\
0 & 2(1 - \cos(2\theta))
\end{bmatrix}
\begin{bmatrix}
\thetadot \\
\phidot
\end{bmatrix}.
\end{aligned}
\end{equation}
%
This simplifies surprisingly, leaving only
%
\boxedEquation{eqn:classicalMechanicsPs2:70}{
\BM^2
=
m^2 r^4 \lr{  \thetadot^2 + \sin^2\theta \phidot^2  }.
}
%
\makesubanswer{Spherical coordinates - a smarter way.}{classicalMechanicsPs2:problem:5:1}
%
Observe that we have no \(\rdot\) factors in the angular momentum.  This makes sense when we consider that the total angular momentum is
%
\begin{dmath}\label{eqn:classicalMechanicsPs2:190}
\BM = m r \rcap \cross \Bv,
\end{dmath}
%
so the \(\rdot \rcap\) term of the velocity is necessarily killed.  Let us do that simplification first.  We want our velocity completely specified in a \(\{\rcap, \thetacap, \phicap\}\) basis, and note that our basis vectors are
%
\begin{equation}\label{eqn:classicalMechanicsPs2:210}
\begin{aligned}
\rcap &=
\begin{bmatrix}
\sin\theta \cos\phi \\
\sin\theta \sin\phi \\
\cos\theta
\end{bmatrix} \\
\thetacap &=
\begin{bmatrix}
\cos\theta \cos\phi \\
\cos\theta \sin\phi \\
-\sin\theta
\end{bmatrix} \\
\phicap &=
\begin{bmatrix}
-\sin\phi \\
\cos\phi \\
0
\end{bmatrix} .
\end{aligned}
\end{equation}
%
We wish to rewrite
%
\begin{dmath}\label{eqn:classicalMechanicsPs2:230}
\ddt{\rcap} =
\begin{bmatrix}
\cos\theta \cos\phi & - \sin\theta \sin\phi \\
\cos\theta \sin\phi & \sin\theta \cos\phi \\
-\sin\theta & 0
\end{bmatrix}
\begin{bmatrix}
\thetadot \\
\phidot
\end{bmatrix},
\end{dmath}
%
in terms of these spherical unit vectors and find
%
\begin{equation}\label{eqn:classicalMechanicsPs2:250}
\begin{aligned}
\frac{d\rcap}{dt} \cdot \rcap &= \rcap^\T \ddt{\rcap} = 0  \\
\frac{d\rcap}{dt} \cdot \thetacap &= \thetacap^\T \ddt{\rcap} = \thetadot \\
\frac{d\rcap}{dt} \cdot \phicap &= \phicap^\T \ddt{\rcap} = \phidot \sin\theta.
\end{aligned}
\end{equation}
%
So our velocity is
%
\begin{dmath}\label{eqn:classicalMechanicsPs2:270}
\Bv = \rdot \rcap + r
\lr{ \thetadot \thetacap + \phidot \sin\theta \phicap }.
\end{dmath}
%
As an aside, now that we know the Euler-Lagrange methods, we could also compute this velocity from the spherical free particle Lagrangian by writing out the canonical momentum in vector form.  We have
%
\begin{dmath}\label{eqn:classicalMechanicsPs2:290}
\LL = \inv{2} m
\lr{  \rdot^2 + r^2 \thetadot^2 + r^2 \phidot^2 \sin^2 \theta }.
\end{dmath}
%
We expect our canonical momentum in vector form to be
%
\begin{equation}\label{eqn:classicalMechanicsPs2:1050}
\begin{aligned}
\BP &=
\PD{\rdot}{\LL} \rcap
+\PD{\thetadot}{\LL} \frac{\thetacap}{r}
+\PD{\phidot}{\LL} \frac{\phicap}{r \sin\theta} \\
&=
m \rdot \rcap
+ m r^2 \thetadot \frac{\thetacap}{r}
+ m r^2 \sin^2\theta \phidot \frac{\phicap}{r \sin\theta} \\
&=
m
\left(
\rdot \rcap + r \thetadot \thetacap + r \sin\theta \phidot \phicap
\right) \\
&= m \Bv.
\end{aligned}
\end{equation}
%
This is consistent with \eqnref{eqn:classicalMechanicsPs2:270} calculated hard way, and is a nice verification that the canonical momentum matches the expectation of being nothing more than how to express the momentum in different coordinate systems.  Returning to the angular momentum calculation we have
%
\begin{equation}\label{eqn:classicalMechanicsPs2:1070}
\begin{aligned}
\rcap \cross \Bv
&=
r \rcap \cross \lr{ \thetadot \thetacap + \phidot \sin\theta \phicap } \\
&=
r \left( \thetadot \phicap - \phidot \sin\theta \thetacap \right).
\end{aligned}
\end{equation}
%
Our total angular momentum in vector form is
%
\begin{dmath}\label{eqn:classicalMechanicsPs2:310}
\BM = m r^2
\left( \thetadot \phicap - \phidot \sin\theta \thetacap \right).
\end{dmath}
%
Now, should we wish to extract coordinates with respect to \(x,y,z\) we just have to write our vectors \(\phicap\) and \(\thetacap\) in the \(\{\Be_1, \Be_2, \Be_3\}\) basis and have
%
\begin{equation}\label{eqn:classicalMechanicsPs2:1090}
\begin{aligned}
\BM
&=
m r^2
\begin{bmatrix}
\phicap & -\sin\theta \thetacap
\end{bmatrix}
\begin{bmatrix}
\thetadot \\
\phidot
\end{bmatrix} \\
&=
m r^2
\begin{bmatrix}
-\sin\phi & -\sin\theta(\cos\theta \cos\phi) \\
\cos\phi & -\sin\theta(\cos\theta \sin\phi) \\
0 & \sin^2\theta
\end{bmatrix}
\begin{bmatrix}
\thetadot \\
\phidot
\end{bmatrix} .
\end{aligned}
\end{equation}
%
This matches \eqnref{eqn:classicalMechanicsPs2:50}, but all the messy trig is isolated to the calculation of \(\Bv\) in the spherical polar basis.
%
\makesubanswer{Cylindrical coordinates.}{classicalMechanicsPs2:problem:5:2}
%
This one should be easier.  To start our position vector is
%
\begin{equation}\label{eqn:classicalMechanicsPs2:90}
\Br =
\begin{bmatrix}
\rho \cos\phi \\
\rho \sin\phi \\
z
\end{bmatrix}
= \rho \rhocap + z \zcap.
\end{equation}
%
Our velocity is
%
\begin{equation}\label{eqn:classicalMechanicsPs2:1110}
\begin{aligned}
\Bv
&= \rhodot \rhocap + \rho \ddt{\rhocap} + \zdot \zcap \\
&= \rhodot \rhocap + \rho \ddt{}
\lr{ \Be_1 e^{i\phi} }
 + \zdot \zcap \\
&= \rhodot \rhocap + \rho \phidot \Be_2 e^{i\phi} + \zdot \zcap \\
&= \rhodot \rhocap + \rho \phidot \phicap + \zdot \zcap .
\end{aligned}
\end{equation}
%
Here, I have used the Clifford algebra representation of \(\rhocap\) with the plane bivector \(i = \Be_1 \Be_2\).  In coordinates we have
%
\begin{dmath}\label{eqn:classicalMechanicsPs2:110}
\phicap = \Be_2
\lr{ \cos\phi + \Be_1 \Be_2 \sin\phi }
 = -\Be_1 \sin\phi + \Be_2 \cos\phi,
\end{dmath}
%
so our velocity in matrix form is
\begin{dmath}\label{eqn:classicalMechanicsPs2:130}
\Bv = \rhodot
\begin{bmatrix}
\cos\phi \\
\sin\phi \\
0
\end{bmatrix}
+
\rho \phidot
\begin{bmatrix}
-\sin\phi \\
\cos\phi \\
0
\end{bmatrix}
+
\zdot
\begin{bmatrix}
0 \\
0 \\
1
\end{bmatrix}
=
\begin{bmatrix}
\rhodot \cos\phi - \rho \phidot \sin\phi \\
\rhodot \sin\phi + \rho \phidot \cos\phi \\
\zdot
\end{bmatrix}.
\end{dmath}
%
For our angular momentum we get
%
\begin{equation}\label{eqn:classicalMechanicsPs2:1130}
\begin{aligned}
\BM
&= \Br \cross (m \Bv) \\
&=
m
\begin{bmatrix}
\rho\sin\phi \zdot - z \lr{ \rhodot \sin\phi + \rho \phidot \cos\phi } \\
z \lr{ \rhodot \cos\phi - \rho \phidot \sin\phi } - \rho \cos\phi \zdot \\
\rho \cos\phi \lr{ \cancel{\rhodot \sin\phi} + \rho \phidot \cos\phi }
 - \rho\sin\phi \lr{ \cancel{\rhodot \cos\phi} - \rho \phidot \sin\phi }
\end{bmatrix} .
\end{aligned}
\end{equation}
%
We can now read off \(M_x, M_y, M_z\) by inspection
%
\begin{dmath}\label{eqn:classicalMechanicsPs2:150}
\BM =
m
\begin{bmatrix}
(\rho \zdot
- z \rhodot )\sin\phi
- z \rho \phidot \cos\phi
\\
( z \rhodot
- \rho \zdot ) \cos\phi
- z \rho \phidot \sin\phi
\\
\rho^2 \phidot
\end{bmatrix}.
\end{dmath}
%
We also want the (squared) magnitude, which is
%
\begin{dmath}\label{eqn:classicalMechanicsPs2:170}
\BM^2
%=
%m^2 \left(
%(\rho \zdot - z \rhodot )^2
%+ (z \rho \phidot)^2
%+ \rho^4 \phidot^2
%\right)
=
m^2 \left(
(\rho \zdot - z \rhodot )^2
+ \rho^2 \phidot^2 ( z^2 + \rho^2 )
%+ (z \rho \phidot)^2
%+ \rho^4 \phidot^2
\right).
\end{dmath}
}
%
%\vcsinfo
%\EndArticle
%\EndNoBibArticle
