%
% Copyright � 2012 Peeter Joot.  All Rights Reserved.
% Licenced as described in the file LICENSE under the root directory of this GIT repository.
%
% pick one:
%\input{../assignment.tex}
%
%\input{../blogpost.tex}
%\renewcommand{\basename}{landauMechanicsCh1P3}
%\newcommand{\dirname}{notes/phy354/}
%\newcommand{\keywords}{Landau,mechanics,Lagrangian}
%\input{../peeter_prologue_print2.tex}
%
\beginArtNoToc
%
%\generatetitle{Typo in Landau Mechanics problem?}
%\generatetitle{Some worked Landau pendulum problems}
\label{chap:landauMechanicsCh1P3}
%\section{Motivation}
%
%\section{Guts}
%
\makeproblem{Pendulum with support moving in circle.}{landauMechanics:ch1:pr3a}{
Attempting a mechanics problem from Landau I get a different answer.  I wrote up my solution to see if I can spot either where I went wrong, or demonstrate the error, and then posted it to
\href{https://www.physicsforums.com/threads/typo-in-landau-mechanics-pendulum-problem.620775/}{physicsforums}
.  I wasn't wrong, but the text wasn't either.  The complete result is given below, where the problem (\S 1 problem 3a) of \citep{landau1976mechanics} is to calculate the Lagrangian of a
\href{https://goo.gl/IjqeO}{pendulum where the point of support is moving in a circle (figure and full text for problem in this Google books reference)}
 }
\makeanswer{landauMechanics:ch1:pr3a}{
The coordinates of the mass are
%
\begin{dmath}\label{eqn:landauMechanicsCh1P3:10}
p = a e^{i \gamma t} + i l e^{i\phi},
\end{dmath}
%
or in coordinates
%
\begin{dmath}\label{eqn:landauMechanicsCh1P3:30}
p = (a \cos\gamma t + l \sin\phi, -a \sin\gamma t + l \cos\phi).
\end{dmath}
%
The velocity is
%
\begin{dmath}\label{eqn:landauMechanicsCh1P3:50}
\pdot = (-a \gamma \sin\gamma t + l \phidot \cos\phi, -a \gamma \cos\gamma t - l \phidot \sin\phi),
\end{dmath}
%
and in the square
\begin{dmath}\label{eqn:landauMechanicsCh1P3:70}
\pdot^2 =
a^2 \gamma^2 + l^2 \phidot^2 - 2 a \gamma \phidot \sin\gamma t \cos\phi + 2 a \gamma l \phidot \cos \gamma t \sin\phi
=
a^2 \gamma^2 + l^2 \phidot^2 + 2 a \gamma l \phidot \sin (\gamma t - \phi).
\end{dmath}
%
For the potential our height above the minimum is
%
\begin{dmath}\label{eqn:landauMechanicsCh1P3:90}
h = 2a + l - a (1 -\cos\gamma t) - l \cos\phi = a ( 1 + \cos\gamma t) + l (1 - \cos\phi).
\end{dmath}
%
In the potential the total derivative \(\cos\gamma t\) can be dropped, as can all the constant terms, leaving
%
\begin{dmath}\label{eqn:landauMechanicsCh1P3:110}
U = - m g l \cos\phi,
\end{dmath}
%
so by the above the Lagrangian should be (after also dropping the constant term \(m a^2 \gamma^2/2\)
\begin{dmath}\label{eqn:landauMechanicsCh1P3:130}
\LL =
\inv{2} m \left( l^2 \phidot^2 + 2 a \gamma l \phidot \sin (\gamma t - \phi) \right) + m g l \cos\phi.
\end{dmath}
%
This is almost the stated value in the text
\begin{dmath}\label{eqn:landauMechanicsCh1P3:150}
\LL =
\inv{2} m \left( l^2 \phidot^2 + 2 a \gamma^2 l \sin (\gamma t - \phi) \right) + m g l \cos\phi.
\end{dmath}
%
We have what appears to be an innocent looking typo (text putting in a \(\gamma\) instead of a \(\phidot\)), but the subsequent text also didn't make sense.  That referred to the omission of the total derivative \(m l a \gamma \cos( \phi - \gamma t)\), which isn't even a term that I have in my result.

In the physicsforums response it was cleverly pointed out by Dickfore that \eqnref{eqn:landauMechanicsCh1P3:130} can be recast into a total derivative
%
\begin{equation}\label{eqn:landauMechanicsCh1P3:170}
\begin{aligned}
m &a l \gamma \phidot \sin (\gamma t - \phi) \\
&=
m a l \gamma ( \phidot - \gamma ) \sin (\gamma t - \phi)
+m a l \gamma^2 \sin (\gamma t - \phi) \\
&=
\ddt{}\left(
m a l \gamma \cos (\gamma t - \phi)
\right)
+m a l \gamma^2 \sin (\gamma t - \phi),
\end{aligned}
\end{equation}
%
which resolves the conundrum!
}
%
\makeproblem{Pendulum with support moving in line.}{landauMechanics:ch1:pr3b}{
This problem like the last, but with the point of suspension moving in a horizontal line \(x = a \cos\gamma t\).
}
\makeanswer{landauMechanics:ch1:pr3b}{
%
Our mass point has coordinates
%
\begin{dmath}\label{eqn:landauMechanicsCh1P3:190}
p = a \cos\gamma t + l i e^{-i\phi}
  = a \cos \gamma t + l i ( \cos \phi - i \sin \phi )
  = ( a \cos \gamma t + l \sin \phi, l \cos \phi ),
\end{dmath}
%
so that the velocity is
\begin{dmath}\label{eqn:landauMechanicsCh1P3:210}
\pdot
  = ( -a \gamma \sin \gamma t + l \phidot \cos \phi, -l \phidot \sin \phi ).
\end{dmath}
%
Our squared velocity is
%
\begin{dmath}\label{eqn:landauMechanicsCh1P3:230}
\pdot^2
= a^2 \gamma^2 \sin^2 \gamma t + l^2 \phidot^2 - 2 a \gamma l \phidot \sin\gamma t \cos \phi
= \inv{2} a^2 \gamma^2 \ddt{}\left( t - \inv{2 \gamma} \sin 2 \gamma t \right) + l^2 \phidot^2 - a \gamma l \phidot ( \sin( \gamma t + \phi) + \sin(\gamma t - \phi)).
\end{dmath}
%
In the last term, we can reduce the sum of sines, finding a total derivative term and a remainder as in the previous problem.  That is
%
\begin{dmath}\label{eqn:landauMechanicsCh1P3:250}
\phidot (\sin( \gamma t + \phi) + \sin(\gamma t - \phi))
=
(\phidot + \gamma)\sin(\gamma t + \phi) - \gamma \sin(\gamma t + \phi)
+
(\phidot - \gamma)\sin(\gamma t - \phi) + \gamma \sin(\gamma t - \phi)
=
\ddt{} \left( -\cos(\gamma t + \phi) + \cos(\gamma t - \phi) \right)
+ \gamma ( \sin(\gamma t - \phi) - \sin(\gamma t + \phi) )
=
\ddt{} \left( -\cos(\gamma t + \phi) + \cos(\gamma t - \phi) \right)
- 2 \gamma \cos \gamma t \sin\phi.
\end{dmath}
%
Putting all the pieces together and dropping the total derivatives we have the stated solution
%
\begin{dmath}\label{eqn:landauMechanicsCh1P3:270}
\LL = \inv{2} m \left( l^2 \phidot^2 + 2 a \gamma^2 l \cos \gamma t \sin\phi \right) + m g l \cos\phi.
\end{dmath}
}
%
\makeproblem{Pendulum with support moving in vertical line.}{landauMechanics:ch1:pr3c}{
As above, but with the support point moving up and down as \(a \cos\gamma t\).
}
\makeanswer{landauMechanics:ch1:pr3c}{
Our mass point is
%
\begin{dmath}\label{eqn:landauMechanicsCh1P3:290}
p = a \cos \gamma t + l e^{i \phi}.
\end{dmath}
%
with velocity
%
\begin{dmath}\label{eqn:landauMechanicsCh1P3:310}
\pdot
= -a \gamma \sin \gamma t + l i \phidot e^{i \phi}
= (-a \gamma \sin\gamma t - l \phidot \sin\phi, l \phidot \cos\phi).
\end{dmath}
%
In the square this is
%
\begin{dmath}\label{eqn:landauMechanicsCh1P3:330}
\Abs{\pdot}^2 =
a^2 \gamma^2 \sin^2 \gamma t + l^2 \phidot^2 \sin^2 \phi + 2 a l \gamma \phidot \sin\gamma t \sin\phi.
\end{dmath}
%
Having done the simplification in the last problem in a complicated way, let's try it, knowing what our answer is
%
\begin{dmath}\label{eqn:landauMechanicsCh1P3:350}
\phidot \sin\gamma t \sin\phi
=
\phidot \sin\gamma t \sin\phi
- \gamma \cos \gamma t \cos\phi
+ \gamma \cos \gamma t \cos\phi
=
\sin\gamma t \ddt{}\left( -\cos\phi \right)
+ \left( \ddt{}\left( -\sin \gamma t \right) \right) \cos\phi
+ \gamma \cos \gamma t \cos\phi
=
\gamma \cos \gamma t \cos\phi
- \ddt{} \left(
\sin \gamma t \cos\phi
\right).
\end{dmath}
%
With the height of the particle above the lowest point given by
%
\begin{dmath}\label{eqn:landauMechanicsCh1P3:370}
h = a + l - a \cos \gamma t - l \cos\phi,
\end{dmath}
%
we can write the Lagrangian immediately (dropping all the total derivative terms)
%
\begin{dmath}\label{eqn:landauMechanicsCh1P3:390}
\LL =
\inv{2} m \left( l^2 \phidot^2 \sin^2 \phi + 2 a l \gamma^2 \cos \gamma t \cos\phi \right)
+ m g l \cos\phi.
\end{dmath}
}
%
\EndArticle
