%
% Copyright © 2012 Peeter Joot.  All Rights Reserved.
% Licenced as described in the file LICENSE under the root directory of this GIT repository.
%
\makeoproblem{Pendulum with support moving in line.}{landauMechanics:ch1:pr3b}{\citep{landau1976mechanics} p1.3b.}{
This problem like the last, but with the point of suspension moving in a horizontal line \(x = a \cos\gamma t\).
}
\makeanswer{landauMechanics:ch1:pr3b}{
%
Our mass point has coordinates
%
\begin{dmath}\label{eqn:landauMechanicsCh1P3:190}
p = a \cos\gamma t + l i e^{-i\phi}
  = a \cos \gamma t + l i ( \cos \phi - i \sin \phi )
  = ( a \cos \gamma t + l \sin \phi, l \cos \phi ),
\end{dmath}
%
so that the velocity is
\begin{dmath}\label{eqn:landauMechanicsCh1P3:210}
\pdot
  = ( -a \gamma \sin \gamma t + l \phidot \cos \phi, -l \phidot \sin \phi ).
\end{dmath}
%
Our squared velocity is
%
\begin{dmath}\label{eqn:landauMechanicsCh1P3:230}
\pdot^2
= a^2 \gamma^2 \sin^2 \gamma t + l^2 \phidot^2 - 2 a \gamma l \phidot \sin\gamma t \cos \phi
= \inv{2} a^2 \gamma^2 \ddt{}\left( t - \inv{2 \gamma} \sin 2 \gamma t \right) + l^2 \phidot^2 - a \gamma l \phidot ( \sin( \gamma t + \phi) + \sin(\gamma t - \phi)).
\end{dmath}
%
In the last term, we can reduce the sum of sines, finding a total derivative term and a remainder as in the previous problem.  That is
%
\begin{dmath}\label{eqn:landauMechanicsCh1P3:250}
\phidot (\sin( \gamma t + \phi) + \sin(\gamma t - \phi))
=
(\phidot + \gamma)\sin(\gamma t + \phi) - \gamma \sin(\gamma t + \phi)
+
(\phidot - \gamma)\sin(\gamma t - \phi) + \gamma \sin(\gamma t - \phi)
=
\ddt{} \left( -\cos(\gamma t + \phi) + \cos(\gamma t - \phi) \right)
+ \gamma ( \sin(\gamma t - \phi) - \sin(\gamma t + \phi) )
=
\ddt{} \left( -\cos(\gamma t + \phi) + \cos(\gamma t - \phi) \right)
- 2 \gamma \cos \gamma t \sin\phi.
\end{dmath}
%
Putting all the pieces together and dropping the total derivatives we have the stated solution
%
\begin{dmath}\label{eqn:landauMechanicsCh1P3:270}
\Lq = \inv{2} m \left( l^2 \phidot^2 + 2 a \gamma^2 l \cos \gamma t \sin\phi \right) + m g l \cos\phi.
\end{dmath}
}
%
