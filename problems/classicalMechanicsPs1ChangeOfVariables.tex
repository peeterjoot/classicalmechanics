%
% Copyright � 2012 Peeter Joot.  All Rights Reserved.
% Licenced as described in the file LICENSE under the root directory of this GIT repository.
%
%
\label{chap:classicalMechanicsPs1}
%\blogpage{http://sites.google.com/site/peeterjoot2/math2012/classicalMechanicsPs1.pdf}
%\date{Jan 24, 2012}
%
%\section{Disclaimer}
%
%Ungraded solutions to posted problem set 1 (I am auditing half the lectures for this course and will not be submitting any solutions for grading).
%
\makeoproblem{Change of coordinates.}{problem:classicalMechanicsPs1:3}{'12 phy356 ps1.3}{
%
Consider a Lagrangian \(\LL({q},{\qdot}) \equiv \LL(q_1,\cdots, q_N, \qdot_1,\cdots \qdot_N)\). Now change the coordinates to some
new ones, e.g. let \(q_i = q_i(x_1,\cdots, x_N),i = 1 \cdots N\), or in short \(q_i = q_i({x})\). This defines a new
Lagrangian:
%
\begin{equation}\label{eqn:classicalMechanicsPs1:480}
\tilde{\LL}({x},{\xdot}) = \LL(q_1({x}),\cdots q_N({x}),\ddt{} q_1({x}),\cdots \ddt{} q_N({x})),
\end{equation}
%
which is now a function of \(x_i\) and \(\xdot_i\). Show that the Euler-Lagrange equations for \(\LL({q},{\qdot})\):
%
\begin{equation}\label{eqn:classicalMechanicsPs1:500}
\frac{\partial \LL({q},{\qdot})}{\partial q_i} =
\ddt{}
\frac{\partial \LL({q},{\qdot})}{\partial \qdot_i},
\end{equation}
%
imply that the Euler-Lagrange equations for \(\tilde{\LL}({x},{\xdot})\) hold (provided the change of variables \(q \rightarrow x\) is nonsingular):
\begin{equation}\label{eqn:classicalMechanicsPs1:520}
\frac{\partial \tilde{\LL}({x},{\xdot})}{\partial x_i} =
\ddt{}
\frac{\partial \tilde{\LL}({x},{\xdot})}{\partial \xdot_i}.
\end{equation}
%
The moral is that the action formalism is very convenient: one can write the Lagrangian in any set of coordinates; the Euler-Lagrange equations for the new coordinates can then be obtained by using the Lagrangian expressed in these coordinates.

Hint: Solving this problem only requires repeated use of the chain rule.
} % end problem
%
\makeanswer{problem:classicalMechanicsPs1:3}{
%
Here we want to show that after a change of variables, provided such a transformation is non-singular, the Euler-Lagrange equations are still valid.

Let us write
%
\begin{equation}\label{eqn:classicalMechanicsPs1:320}
r_i = r_i(q_1, q_2, \cdots q_N).
\end{equation}
%
Our ``velocity'' variables in terms of the original parametrization \(q_i\) are
%
\begin{equation}\label{eqn:classicalMechanicsPs1:340}
\dot{r}_j = \frac{dr_j}{dt} = \PD{q_i}{r_j} \PD{t}{q_i} = \qdot_i \PD{q_i}{r_j},
\end{equation}
%
so we have
%
\begin{equation}\label{eqn:classicalMechanicsPs1:360}
\PD{\qdot_i}{\dot{r}_j} = \PD{q_i}{r_j}.
\end{equation}
%
Computing the LHS of the Euler Lagrange equation we find
%
\begin{equation}\label{eqn:classicalMechanicsPs1:380}
\PD{q_i}{\LL} =
\PD{r_j}{\LL} \PD{q_i}{r_j}
+\PD{\rdot_j}{\LL} \PD{q_i}{\rdot_j}.
\end{equation}
%
For our RHS we start with
%
\begin{equation}\label{eqn:classicalMechanicsPs1:400}
\PD{\qdot_i}{\LL}
=
\PD{r_j}{\LL} \PD{\qdot_i}{r_j}
+\PD{\rdot_j}{\LL} \PD{\qdot_i}{\rdot_j}
=
\PD{r_j}{\LL} \PD{\qdot_i}{r_j}
+\PD{\rdot_j}{\LL} \PD{q_i}{r_j},
\end{equation}
%
but \(\PDi{\qdot_i}{r_j} = 0\), so this is just
%
\begin{equation}\label{eqn:classicalMechanicsPs1:420}
\PD{\qdot_i}{\LL}
=
\PD{r_j}{\LL} \PD{\qdot_i}{r_j}
+\PD{\rdot_j}{\LL} \PD{\qdot_i}{\rdot_j}
=
\PD{\rdot_j}{\LL} \PD{q_i}{r_j}.
\end{equation}
%
The Euler-Lagrange equations become
%
\begin{equation}\label{eqn:classicalMechanicsPs1:660}
\begin{aligned}
0 &=
\PD{r_j}{\LL} \PD{q_i}{r_j}
+\PD{\rdot_j}{\LL} \PD{q_i}{\rdot_j}
-
\ddt{} \left(
\PD{\rdot_j}{\LL} \PD{q_i}{r_j}
\right) \\
&=
  \PD{r_j}{\LL} \PD{q_i}{r_j}
+ \cancel{\PD{\rdot_j}{\LL} \PD{q_i}{\rdot_j}}
- \left( \ddt{} \PD{\rdot_j}{\LL} \right) \PD{q_i}{r_j}
- \cancel{\PD{\rdot_j}{\LL} \ddt{} \PD{q_i}{r_j} }
\\
&=
\left( \PD{r_j}{\LL}
-\ddt{} \PD{\rdot_j}{\LL}
\right) \PD{q_i}{r_j}.
\end{aligned}
\end{equation}
%
Since we have an assumption that the transformation is non-singular, we have for all \(j\)
%
\begin{equation}\label{eqn:classicalMechanicsPs1:440}
\PD{q_i}{r_j} \ne 0,
\end{equation}
%
so we have the Euler-Lagrange equations for the new abstract coordinates as well
%
\begin{equation}\label{eqn:classicalMechanicsPs1:460}
0 = \PD{r_j}{\LL} -\ddt{} \PD{\rdot_j}{\LL}.
\end{equation}
} % end answer
