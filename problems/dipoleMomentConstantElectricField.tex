%
% Copyright � 2012 Peeter Joot.  All Rights Reserved.
% Licenced as described in the file LICENSE under the root directory of this GIT repository.
%
% pick one:
%\input{../assignment.tex}
%\input{../blogpost.tex}
%\renewcommand{\basename}{dipoleMomentConstantElectricField}
%\renewcommand{\dirname}{notes/phy354/}
%%\newcommand{\dateintitle}{}
%%\newcommand{\keywords}{}
%
%\input{../peeter_prologue_print2.tex}
%
%\beginArtNoToc
%
%\generatetitle{Dipole Moment induced by a constant electric field}
%\chapter{Dipole Moment induced by a constant electric field}
\index{dipole moment}
\label{chap:dipoleMomentConstantElectricField}
%\section{Motivation}
%\section{Guts}
%
\makeproblem{Dipole Moment induced by a constant electric field.}{pr:dipoleMomentConstantElectricField:1}{
%
In \citep{jackson2000equilibrium} it is stated that the force per unit angle on a dipole system as illustrated in \cref{fig:dipoleMomentConstantElectricField:dipoleMomentConstantElectricFieldFig1} is
%
\begin{equation}\label{eqn:dipoleMomentConstantElectricField:20}
F_\theta = -p \calE \sin\theta,
\end{equation}
%
where \(\Bp = q \Br\).  The text was also referring to torques, and it wasn't clear to me if the result was the torque or the force.  Derive the result to resolve any doubt (in retrospect dimensional analysis would also have worked).
%
\imageFigure{../figures/classicalmechanics/dipoleMomentConstantElectricFieldFig1}{Dipole moment coordinates.}{fig:dipoleMomentConstantElectricField:dipoleMomentConstantElectricFieldFig1}{0.3}
%
} % makeproblem
%
\makeanswer{pr:dipoleMomentConstantElectricField:1}{
%
Let's put the electric field in the \(\xcap\) direction (\(\theta = 0\)), so that the potential acting on charge \(i\) is given implicitly by
%
\begin{equation}\label{eqn:dipoleMomentConstantElectricField:200}
\BF_i = q_i \calE \xcap = -\grad \phi_i = -\xcap \ddx{\phi_i}.
\end{equation}
%
or
%
\begin{equation}\label{eqn:dipoleMomentConstantElectricField:40}
\phi_i = -q_i (x_i - x_0).
\end{equation}
%
Our positions, and velocities are
%
\begin{subequations}
\begin{equation}\label{eqn:dipoleMomentConstantElectricField:60}
\Br_{1,2} = \pm \frac{r}{2} \xcap e^{\xcap \ycap \theta}.
\end{equation}
\begin{equation}\label{eqn:dipoleMomentConstantElectricField:80}
\ddt{\Br_{1,2}} = \pm \frac{r}{2} \thetadot \ycap e^{\xcap \ycap \theta}.
\end{equation}
\end{subequations}
%
Our kinetic energy is
%
\begin{equation}\label{eqn:dipoleMomentConstantElectricField:100}
\begin{aligned}
T
&= \inv{2} \sum_i m_i \left( \ddt{\Br_i} \right)^2 \\
&= \inv{2} \sum_i m_i \left( \frac{r}{2} \right)^2 \thetadot^2 \\
&= \inv{2} (m_1 + m_2) \left( \frac{r}{2} \right)^2 \thetadot^2.
\end{aligned}
\end{equation}
%
For our potential energies we require the \(x\) component of the position vectors, which are
%
\begin{equation}\label{eqn:dipoleMomentConstantElectricField:120}
\begin{aligned}
x_i
&= \Br_i \cdot \xcap \\
&= \pm \gpgradezero{ \frac{r}{2} \xcap e^{\xcap \ycap \theta} \xcap} \\
&= \pm \frac{r}{2} \cos\theta.
\end{aligned}
\end{equation}
%
Our potentials are
%
\begin{subequations}
\begin{equation}\label{eqn:dipoleMomentConstantElectricField:140}
\phi_1 = -q_1 \calE \frac{r}{2} \cos\theta + \phi_0.
\end{equation}
\begin{equation}\label{eqn:dipoleMomentConstantElectricField:160}
\phi_2 = q_2 \calE \frac{r}{2} \cos\theta + \phi_0.
\end{equation}
\end{subequations}
%
Our system Lagrangian, after dropping the constant reference potential that doesn't effect the dynamics is
%
\begin{equation}\label{eqn:dipoleMomentConstantElectricField:180}
\Lq
=
\inv{2} (m_1 + m_2) \left( \frac{r}{2} \right)^2 \thetadot^2
+q_1 \calE \frac{r}{2} \cos\theta
-q_2 \calE \frac{r}{2} \cos\theta.
\end{equation}
%
For this problem we had two equal masses and equal magnitude charges \(m = m_1 = m_2\) and \(q = q_1 = -q_2\)
%
\begin{equation}\label{eqn:dipoleMomentConstantElectricField:220}
\Lq
=
\inv{4} m r^2 \thetadot^2 + q r \calE \cos\theta.
\end{equation}
%
\begin{equation}\label{eqn:dipoleMomentConstantElectricField:240}
p_\theta = \PD{\thetadot}{\Lq} = \inv{2} m r^2 \thetadot.
\end{equation}
\begin{equation}\label{eqn:dipoleMomentConstantElectricField:260}
\begin{aligned}
\PD{\theta}{\Lq} = -q r \calE \sin\theta
&= \ddt{p_\theta} \\
&= \inv{2} m r^2 \ddot{\theta}.
\end{aligned}
\end{equation}
%
Putting these together, with \(p = q r\), we have the result stated in the text
%
\begin{equation}\label{eqn:dipoleMomentConstantElectricField:280}
F_\theta = \ddt{p_\theta} = -p \calE \sin\theta.
\end{equation}
%
} % makeanswer
%
% this is to produce the sites.google url and version info and so forth (for blog posts)
%\vcsinfo
%\EndArticle
