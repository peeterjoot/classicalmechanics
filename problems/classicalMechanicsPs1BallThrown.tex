%
% Copyright © 2012 Peeter Joot.  All Rights Reserved.
% Licenced as described in the file LICENSE under the root directory of this GIT repository.
%
\makeoproblem{Find trajectory using action.}{problem:classicalMechanicsPs1:2}{'12 phy356 ps1.2.}{
For a ball thrown upward, guess a solution for the height \(y\) of the form \(y(t) = a_2 t^2+ a_1 t + a_0\).  Assuming that \(y(0) = y(T) = 0\), this quickly becomes \(y(t) = a_2(t^2- T t)\). Calculate the action (to do that, you need to first write the Lagrangian, of course) between \(t = 0\) and \(t = T\), and show that it is minimized when \(a_2= -g/2\).
} % end problem
%
\makeanswer{problem:classicalMechanicsPs1:2}{
%
We are told to guess at a solution
%
\begin{equation}\label{eqn:classicalMechanicsPs1:180}
y = a_2 t^2 + a_1 t + a_0,
\end{equation}
%
for the height of a particle thrown up into the air.  With initial condition \(y(0) = 0\) we have
%
\begin{equation}\label{eqn:classicalMechanicsPs1:200}
a_0 = 0,
\end{equation}
%
and with a final condition of \(y(T) = 0\) we also have
%
\begin{equation}\label{eqn:classicalMechanicsPs1:620}
\begin{aligned}
0
&=
a_2 T^2 + a_1 T \\
&= T( a_2 T + a_1 ),
\end{aligned}
\end{equation}
%
so have
%
\begin{equation}\label{eqn:classicalMechanicsPs1:220}
\begin{aligned}
y(t) &= a_2 t^2 - a_2 T t = a_2
\lr{ t^2 - T t } \\
\dot{y}(t) &=
a_2 (2 t - T ).
\end{aligned}
\end{equation}
%
So our Lagrangian is
%
\begin{equation}\label{eqn:classicalMechanicsPs1:240}
\LL =
\inv{2} m a_2^2
\lr{ 2 t - T }^2 - m g a_2 \lr{ t^2 - T t },
\end{equation}
%
and our action is
%
\begin{equation}\label{eqn:classicalMechanicsPs1:260}
S = \int_0^T dt
\left(
\inv{2} m a_2^2 \lr{ 2 t - T  }^2
- m g a_2 \lr{ t^2 - T t }
\right).
\end{equation}
%
To minimize this action with respect to \(a_2\) we take the derivative
%
\begin{equation}\label{eqn:classicalMechanicsPs1:280}
\PD{a_2}{S} = \int_0^T
\left(
m a_2 \lr{ 2 t - T  }^2
- m g \lr{ t^2 - T t }
\right).
\end{equation}
%
Integrating we have
%
\begin{equation}\label{eqn:classicalMechanicsPs1:640}
\begin{aligned}
0 &= \PD{a_2}{S} \\
&={\left.
\left(
\inv{6} m a_2
\lr{ 2 t - T  }^3
- m g \left(\inv{3}t^3 - \inv{2}T t^2 \right)
\right)\right\vert}_0^T \\
&=
\inv{6} m a_2 T^3 - m g \left(\inv{3}T^3 - \inv{2}T^3 \right)
-
\inv{6} m a_2 (- T )^3 \\
&=
m T^3 \left( \inv{3} a_2 - g \left( \inv{3} - \inv{2} \right) \right) \\
&=
\inv{3} m T^3 \left( a_2 - g \left( 1 - \frac{3}{2} \right) \right) .
\end{aligned}
\end{equation}
%
or
%
\begin{equation}\label{eqn:classicalMechanicsPs1:300}
a_2 + g/2 = 0,
\end{equation}
%
which is the result we are required to show.
} % end answer
