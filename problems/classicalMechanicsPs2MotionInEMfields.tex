%
% Copyright © 2012 Peeter Joot.  All Rights Reserved.
% Licenced as described in the file LICENSE under the root directory of this GIT repository.
%
\makeoproblem{Symmetries and conservation (E.M.)}{classicalMechanicsPs2:problem:1}{'12 phy356 ps2.1}
{
%{2012 PHY354 Problem set 2, problem 1}
%
Let us continue studying the Lagrangian of Problem 1 of Homework 1, namely, its symmetries, and the relevant conserved quantities. To this end, we will consider various cases of external scalar and vector potentials.
%
\makesubproblem{Consider first the case of time-independent \(\BA\) and \(\phi\). Find the expression for the conserved energy, \(\calE\), of the particle.}{classicalMechanicsPs2:problem:1:1}
\makesubproblem{For external \(\BA\) and \(\phi\) dependent on time, find \(d \calE/dt\).}{classicalMechanicsPs2:problem:1:2}
\makesubproblem{Let now \(\BA\) and \(\phi\) be spatially homogeneous, i.e. \(\Bx\)-independent. Find the conserved momentum. Is it equal to the usual \(m \Bv\)?}{classicalMechanicsPs2:problem:1:3}
\makesubproblem{Consider motion in the field of an electrostatic source (creating an external static \(\phi(\Bx)\)). Find the angular momentum of the particle. Is it conserved for all \(\phi(\Bx)\)?}{classicalMechanicsPs2:problem:1:4}
} % makeproblem
%
\makeanswer{classicalMechanicsPs2:problem:1}{
\makesubanswer{Conserved energy.}{classicalMechanicsPs2:problem:1:1}
%
Recall the argument for energy conservation, the result of considering time dependence of the Lagrangian.  We have
%
\begin{dmath}\label{eqn:classicalMechanicsPs2:370}
\ddt{} \Lq(q_i, \qdot_i, t)
=
\PD{q_i}{\Lq} \PD{t}{q_i}
+\PD{\qdot_i}{\Lq} \PD{t}{\qdot_i}
\PD{t}{\Lq}
=
\left( \ddt{} \PD{\qdot_i}{\Lq} \right) \PD{t}{q_i}
+ \PD{\qdot_i}{\Lq} \PD{t}{\qdot_i}
+\PD{t}{\Lq}
=
\ddt{} \left( \PD{\qdot_i}{\Lq} \PD{t}{q_i} \right)
+ \PD{t}{\Lq}.
\end{dmath}
%
Rearranging we have the conservation equation
%
\begin{dmath}\label{eqn:classicalMechanicsPs2:390}
\ddt{} \left( \PD{\qdot_i}{\Lq} \qdot_i -\Lq \right) + \PD{t}{\Lq} = 0.
\end{dmath}
%
We define the energy as
%
\boxedEquation{eqn:classicalMechanicsPs2:410}{
\calE = \PD{\qdot_i}{\Lq} \qdot_i -\Lq,
}
%
so that the when the Lagrangian is independent of time \(\calE\) is conserved, and in general
%
\begin{equation}\label{eqn:classicalMechanicsPs2:430}
\ddt{\calE} = - \PD{t}{\Lq}.
\end{equation}
%
Application to this problem where our Lagrangian is
%
\begin{dmath}\label{eqn:classicalMechanicsPs2:450}
\Lq = \inv{2} m \Bv^2 + q \Bv \cdot \BA - q \phi,
\end{dmath}
%
we have
\begin{dmath}\label{eqn:classicalMechanicsPs2:470}
\PD{\Bv}{\Lq} = m \Bv + q \BA.
\end{dmath}
%
so the energy is
%
\begin{dmath}\label{eqn:classicalMechanicsPs2:490}
\calE =
\left( m \Bv + \cancel{q \BA} \right) \cdot \Bv -
\left( \inv{2} m \Bv^2 + \cancel{q \Bv \cdot \BA} - q \phi \right)
=
\inv{2} m \Bv^2 + q \phi,
\end{dmath}
%
with an end result of
\boxedEquation{eqn:classicalMechanicsPs2:491}{
\calE =
\inv{2} m \Bv^2 + q \phi.
}
%
\makesubanswer{Find \(d\calE/dt\).}{classicalMechanicsPs2:problem:1:2}
%
\paragraph{With direct computation.}
There are two ways we can try this.  One is with direct computation of the derivative from \eqnref{eqn:classicalMechanicsPs2:490}
%
\begin{dmath}\label{eqn:classicalMechanicsPs2:1150}
\ddt{\calE}
=
\Bv \cdot (m \Ba) + q \ddt{\phi}
=
\Bv \cdot \left( q \BE + q \Bv \cross \BB \right) + q \left( \PD{t}{\phi} + \Bv \cdot \spacegrad \phi \right)
=
q \Bv \cdot (\BE + \spacegrad \phi) + q \cancel{\Bv \cdot \left(\Bv \cross \BB \right)} + q \PD{t}{\phi}
=
q \Bv \cdot \left(-\cancel{\spacegrad \phi} - \PD{t}{\BA} + \cancel{\spacegrad \phi} \right) + q \PD{t}{\phi}.
\end{dmath}
%
So our end result is
%
\boxedEquation{eqn:classicalMechanicsPs2:510}{
\ddt{\calE}
=
-q \Bv \cdot \PD{t}{\BA} + q \PD{t}{\phi}.
}
%
\paragraph{Using the Lagrangian time partial.}
%
Doing it explicitly as above is the hard way.  We can do it from the conservation identity \eqnref{eqn:classicalMechanicsPs2:430} instead
%
\begin{dmath}\label{eqn:classicalMechanicsPs2:590}
\ddt{\calE}
= -\PD{t}{\Lq}
=
-\PD{t}{} \left(
\inv{2} m \Bv^2 + q \Bv \cdot \BA - q \phi
\right)
=
-q \Bv \cdot \PD{t}{\BA} + q \PD{t}{\phi},
\end{dmath}
%
as before.
%
\paragraph{Aside: Why not the ``expected'' \(q \Bv \cdot \BE\) result?}
%
From the relativistic treatment I expected
%
\begin{dmath}\label{eqn:classicalMechanicsPs2:530}
\ddt{\calE} \questionEquals q \Bv \cdot \BE,
\end{dmath}
%
but that's not what we got.  With \(\calE = m \Bv^2/2 + q \phi\), it appears that we get a similar result considering just the Kinetic portion of the energy
%
\begin{dmath}\label{eqn:classicalMechanicsPs2:550}
\inv{2} m \Bv^2 = \calE - q \phi.
\end{dmath}
%
Computing the derivative from above we have
%
\begin{dmath}\label{eqn:classicalMechanicsPs2:1170}
\ddt{}\left(
\inv{2} m \Bv^2 \right)
=
-q \Bv \cdot \PD{t}{\BA} + q \PD{t}{\phi}
- q \ddt{\phi}
=
-q \Bv \cdot \PD{t}{\BA} + \cancel{q \PD{t}{\phi}}
- \cancel{q \PD{t}{\phi}} - q \Bv \cdot \spacegrad \phi
=
q \Bv \cdot \left(
-\spacegrad \phi
-\PD{t}{\BA} \right),
\end{dmath}
%
or
\begin{dmath}\label{eqn:classicalMechanicsPs2:570}
\ddt{} \left(
\inv{2} m \Bv^2 \right) = \ddt{} \left( \calE - q \phi \right) = q \Bv \cdot \BE.
\end{dmath}
%
Looking back to what we did in the relativistic treatment, I see that my confusion was due to the fact that we actually computed
%
\begin{dmath}\label{eqn:classicalMechanicsPs2:530b}
\ddt{\calE_{\mathrm{kin}}} = q \Bv \cdot \BE,
\end{dmath}
%
where \(\calE_{\mathrm{kin}} = \gamma m c^2\).  To first order, removing an additive constant, we have \(\gamma m c^2 \approx m \Bv^2/2\), so everything checks out.
%
\makesubanswer{Conserved momentum.}{classicalMechanicsPs2:problem:1:3}
%
The conserved momentum followed from a Noether's argument where we compute
%
\begin{dmath}\label{eqn:classicalMechanicsPs2:610}
\frac{d\Lq'}{d\epsilon}
=
\PD{q_i}{\Lq'} \PD{\epsilon}{q_i}
+\PD{\qdot_i'}{\Lq'} \PD{\epsilon}{\qdot_i'}
=
\left( \ddt{} \PD{\qdot_i}{\Lq'} \right) \PD{\epsilon}{q_i}
+\PD{\qdot_i'}{\Lq'} \PD{\epsilon}{\qdot_i'}
=
\ddt{} \left( \PD{\qdot_i'}{\Lq'} \PD{\epsilon}{q_i'} \right),
\end{dmath}
%
where it has been assumed that a perturbed Lagrangian
\begin{equation}\label{eqn:classicalMechanicsPs2:610a}
\Lq'(\epsilon) = \Lq(q_i'(\epsilon), \qdot_i'(\epsilon), t),
\end{equation}
also satisfies the Euler Lagrange equations using the transformed coordinates.
With the coordinates transformed by a shift along some constant direction \(\Ba\) as in
%
\begin{dmath}\label{eqn:classicalMechanicsPs2:630}
\Bx' = \Bx + \epsilon \Ba,
\end{dmath}
%
we have \(\PDi{\epsilon}{\Bx'} = \Ba\), so \eqnref{eqn:classicalMechanicsPs2:610} takes the form
%
\begin{dmath}\label{eqn:classicalMechanicsPs2:650}
\frac{d\Lq'}{d\epsilon} = \ddt{} \left( \PD{\xdot_i}{\Lq'} a_i \right).
\end{dmath}
%
Our shifted Lagrangian for spatially homogeneous potentials \(\phi' = \phi\) and \(\BA' = \BA\) is
%
\begin{dmath}\label{eqn:classicalMechanicsPs2:670}
\Lq' = \inv{2} m {\Bv'}^2 + q \Bv' \cdot \BA - q \phi = \Lq,
\end{dmath}
%
but \(\Bv' = \Bv\), so we've just got our canonical momentum \(\BM = \PDi{\xdot_i}{\Lq}\) within the time derivative, and must have for all \(\Ba\)
%
\begin{dmath}\label{eqn:classicalMechanicsPs2:690}
\ddt{ \BM } \cdot \Ba = 0.
\end{dmath}
%
The conserved momentum is then just
%
\begin{dmath}\label{eqn:classicalMechanicsPs2:710}
\BM = m \Bv + q \BA.
\end{dmath}
%
\makesubanswer{Conserved angular momentum.}{classicalMechanicsPs2:problem:1:4}
%
Does the conserved angular momentum take the same from as \(\Bx \cross \BM\) as we had in a non-velocity dependent Lagrangian?  We can check using the same Noether arguments using the following coordinate transformation
%
\begin{dmath}\label{eqn:classicalMechanicsPs2:730}
\Bx' = e^{-\epsilon j/2} \Bx e^{\epsilon j/2},
\end{dmath}
%
where \(j = \ucap \wedge \vcap\) is the geometric product of two perpendicular unit vectors, and \(\epsilon\) is the magnitude of the rotation.  This gives us
%
\begin{dmath}\label{eqn:classicalMechanicsPs2:750}
\frac{d\Bx'}{d\epsilon} =
-\frac{j}{2} e^{-\epsilon j/2} \Bx e^{\epsilon j/2}
+ e^{-\epsilon j/2} \Bx e^{\epsilon j/2} \frac{j}{2}
=
\inv{2} \left( \Bx' j - j \Bx' \right)
= \Bx' \cdot j.
\end{dmath}
%
The Noether conservation statement is then
%
\begin{dmath}\label{eqn:classicalMechanicsPs2:770}
\frac{d\Lq'}{d\epsilon} = \ddt{} \left( \PD{\xdot_i}{\Lq'} \Be_i \cdot (\Bx' \cdot j) \right).
\end{dmath}
%
With a static scalar potential \(\phi(\Bx)\) is our Lagrangian rotation invariant?  We have
%
\begin{dmath}\label{eqn:classicalMechanicsPs2:790}
\Lq'
=
\inv{2} {\Bv'}^2 - q \phi(\Bx')
=
\inv{2} {\Bv}^2 - q \phi(\Bx').
\end{dmath}
%
With zero vector potential, our kinetic term is invariant since the squared velocity is invariant, but we require \(\phi(\Bx') = \phi(\Bx)\) for total Lagrangian invariance.  We have that if \(\phi(\Bx) = \phi(\Abs{\Bx})\).
%
Evaluating the conservation identity \eqnref{eqn:classicalMechanicsPs2:770} at \(\epsilon = 0\) we have
%
\begin{dmath}\label{eqn:classicalMechanicsPs2:810}
0
= \ddt{} \left( \BM \cdot (\Bx \cdot j) \right).
\end{dmath}
%
We are used to seeing this in dual form using the cross product
%
\begin{dmath}\label{eqn:classicalMechanicsPs2:830}
\BM \cdot (\Bx \cdot j)
=
\gpgradezero{
\BM (\Bx \cdot j)
}
=
\inv{2} \gpgradezero{
\BM \Bx j - \BM j \Bx
}
=
\inv{2} \gpgradezero{
\BM \Bx j - \Bx \BM j
}
=
\inv{2} \gpgradezero{
\BM \wedge \Bx - \Bx \wedge \BM
}
\cdot j
=
\inv{2} \gpgradezero{
\left( \BM \wedge \Bx - \Bx \wedge \BM \right) \cdot j
}
=
(\BM \wedge \Bx) \cdot j
=
I (\BM \cross \Bx) \cdot j.
\end{dmath}
%
We are left with
%
\begin{dmath}\label{eqn:classicalMechanicsPs2:910}
0 = I \frac{d}{dt} (\Bx \cross \BM) \cdot j,
\end{dmath}
%
but since \(j\) can be arbitrarily oriented, we have a requirement that
%
\begin{dmath}\label{eqn:classicalMechanicsPs2:930}
0 = \frac{d}{dt} (\Bx \cross \BM).
\end{dmath}
%
This verifies that the our angular momentum is conserved, provided that \(\phi(\Bx) = \phi(\Abs{\Bx})\), and \(\BA = 0\).  With \(\BA = 0\), so that \(\BM = m \Bv + q \BA = m \Bv\) this is just
%
\begin{dmath}\label{eqn:classicalMechanicsPs2:870}
\Bx \cross \BM = m \Bx \cross \Bv.
\end{dmath}
%
Note that the dependency on geometric algebra in the Noether's argument above can probably be eliminated by utilizing a rotational transformation of the form
%
\begin{dmath}\label{eqn:classicalMechanicsPs2:871}
\Bx' = \Bx + \ncap \cross \Bx.
\end{dmath}
%
I'd guess (or perhaps recall if I attended that class), that this was the approach used.
%
%\paragraph{Computing the conserved quantity without use of Geometric Algebra.}
%
%Consider the expansion of our infinisimal rotation, projecting our vector \(\Bx\) into components lying in the plane of rotation \((\Bx \cdot j) j\) and perpendicular to that plane.  We have
%
%\begin{dmath}\label{eqn:classicalMechanicsPs2:890}
%\Bx'
%= e^{-j \epsilon/2} \Bx e^{j \epsilon/2}
%= e^{-j \epsilon/2} ((\Bx \cdot j) j + (\Bx \wedge j) j) e^{j \epsilon/2}
%= e^{-j \epsilon/2} ((\Bx \cdot j) e^{j \epsilon/2} j
%= ((\Bx \cdot j) e^{j \epsilon} j
%\approx ((\Bx \cdot j) j \epsilon
%=
%\inv{2} (\Bx j - j \Bx) j \epsilon
%=
%-I^2 \inv{2} (\Bx j - j \Bx) j \epsilon
%=
%\inv{2} (\Bx (-Ij) - (-Ij) \Bx) (I j) \epsilon
%=
%( \Bx \wedge (-Ij) ) j \epsilon.
%\end{dmath}
%
%FIXME: messed up.  try on paper first.  One of the mistakes is that j^2 = -1, I've got +1 above.
}
%
