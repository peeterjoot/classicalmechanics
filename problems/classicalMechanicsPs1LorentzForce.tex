%
% Copyright © 2012 Peeter Joot.  All Rights Reserved.
% Licenced as described in the file LICENSE under the root directory of this GIT repository.
%
\makeoproblem{Lorentz force Lagrangian.}{problem:classicalMechanicsPs1:1}{'12 phy356 ps1.1}{
\makesubproblem{For the non-covariant electrodynamic Lorentz force Lagrangian.}{problem:classicalMechanicsPs1:1.1}{
%
\begin{equation}\label{eqn:classicalMechanicsPs1:40}
\LL = \inv{2} m \Bv^2 + q \Bv \cdot \BA - q \phi,
\end{equation}
%
derive the Lorentz force equation
%
\begin{equation}\label{eqn:classicalMechanicsPs1:20}
\begin{aligned}
\BF &= q (\BE + \Bv \cross \BB) \\
\BE &= -\spacegrad \phi - \PD{t}{\BA} \\
\BB &= \spacegrad \cross \BA.
\end{aligned}
\end{equation}
} % end 1.1
%
\makesubproblem{With a gauge transformation of the form:}{problem:classicalMechanicsPs1:1.2}{
%
\begin{equation}\label{eqn:classicalMechanicsPs1:120}
\begin{aligned}
\phi &\rightarrow \phi + \PD{t}{\chi} \\
\BA &\rightarrow \BA - \spacegrad \chi,
\end{aligned}
\end{equation}
%
show that the Lagrangian is invariant.
} % end 1.2
} % end problem
%
\makeanswer{problem:classicalMechanicsPs1:1}{
%
\makesubanswer{Evaluate the Euler-Lagrange equations.}{problem:classicalMechanicsPs1:1.1}
%
In coordinates, employing summation convention, this Lagrangian is
%
\begin{equation}\label{eqn:classicalMechanicsPs1:60}
\LL = \inv{2} m \xdot_j \xdot_j + q \xdot_j A_j - q \phi.
\end{equation}
%
Taking derivatives
%
\begin{equation}\label{eqn:classicalMechanicsPs1:80}
\PD{\xdot_i}{\LL} = m \xdot_i + q A_i,
\end{equation}
%
\begin{equation}\label{eqn:classicalMechanicsPs1:540}
\begin{aligned}
\frac{d}{dt} \PD{\xdot_i}{\LL}
&=
m \ddot{x}_i
+ q \PD{t}{A_i}
+ q \PD{x_j}{A_i} \frac{dx_j}{dt} \\
&=
m \ddot{x}_i
+ q \PD{t}{A_i}
+ q \PD{x_j}{A_i} \xdot_j.
\end{aligned}
\end{equation}
%
This must equal
%
\begin{equation}\label{eqn:classicalMechanicsPs1:100}
\PD{x_i}{\LL} = q \xdot_j \PD{x_i}{A_j} - q \PD{x_i}{\phi},
\end{equation}
%
so we have
%
\begin{equation}\label{eqn:classicalMechanicsPs1:560}
\begin{aligned}
m \ddot{x}_i
&=
-q \PD{t}{A_i}
- q \PD{x_j}{A_i} \xdot_j
+q \xdot_j \PD{x_i}{A_j} - q \PD{x_i}{\phi} \\
&=
-q \left( \PD{t}{A_i} - \PD{x_i}{\phi} \right)
+q v_j \left( \PD{x_i}{A_j} - \PD{x_j}{A_i} \right).
\end{aligned}
\end{equation}
%
The first term is just \(E_i\).  If we expand out \((\Bv \cross \BB)_i\) we see that matches
%
\begin{equation}\label{eqn:classicalMechanicsPs1:580}
\begin{aligned}
(\Bv \cross \BB)_i
&=
v_a B_b \epsilon_{abi} \\
&=
v_a \partial_r A_s \epsilon_{rsb} \epsilon_{abi} \\
&=
v_a \partial_r A_s \delta_{rs}^{[ia]} \\
&=
v_a (\partial_i A_a - \partial_a A_i).
\end{aligned}
\end{equation}
%
A \(a \rightarrow j\) substitution, and comparison of this with the Euler-Lagrange result above completes the exercise.
%
\makesubanswer{Gauge invariance.}{problem:classicalMechanicsPs1:1.2}
%
We really only have to show that
%
\begin{equation}\label{eqn:classicalMechanicsPs1:140}
\Bv \cdot \BA - \phi,
\end{equation}
%
is invariant.  Making the transformation we have
%
\begin{equation}\label{eqn:classicalMechanicsPs1:600}
\begin{aligned}
\Bv \cdot \BA - \phi
&\rightarrow
v_j \left(A_j - \partial_j \chi \right) - \left(\phi + \PD{t}{\chi} \right) \\
&=
v_j A_j - \phi - v_j \partial_j \chi - \PD{t}{\chi} \\
&=
\Bv \cdot \BA - \phi
- \left( \frac{d x_j}{dt} \PD{x_j}{\chi} + \PD{t}{\chi} \right) \\
&=
\Bv \cdot \BA - \phi
- \frac{d \chi(\Bx, t)}{dt}.
\end{aligned}
\end{equation}
%
We see then that the Lagrangian transforms as
%
\begin{equation}\label{eqn:classicalMechanicsPs1:160}
\LL \rightarrow \LL + \frac{d}{dt}\left( -q \chi \right),
\end{equation}
%
and differs only by a total derivative.  With the lemma from the lecture, we see that this gauge transformation does not have any effect on the end result of applying the Euler-Lagrange equations.
} % end answer
