%
% Copyright � 2012 Peeter Joot.  All Rights Reserved.
% Licenced as described in the file LICENSE under the root directory of this GIT repository.
%
% pick one:
%\input{../assignment.tex}
%
%\input{../blogpost.tex}
%\renewcommand{\basename}{landauMechanicsCh1P3}
%\newcommand{\dirname}{notes/phy354/}
%\newcommand{\keywords}{Landau,mechanics,Lagrangian}
%\input{../peeter_prologue_print2.tex}
%
\beginArtNoToc
%
%\generatetitle{Typo in Landau Mechanics problem?}
%\generatetitle{Some worked Landau pendulum problems}
\label{chap:landauMechanicsCh1P3}
%\section{Motivation}
%
%\section{Guts}
%
\makeoproblem{Pendulum with support moving in circle.}{landauMechanics:ch1:pr3a}{\citep{landau1976mechanics} p1.3.}{
Attempting a mechanics problem from Landau I get a different answer.  I wrote up my solution to see if I can spot either where I went wrong, or demonstrate the error, and then posted it to
\href{https://www.physicsforums.com/threads/typo-in-landau-mechanics-pendulum-problem.620775/}{physicsforums}
.  I wasn't wrong, but the text wasn't either.  The complete result is given below, where the problem (\S 1 problem 3a) of \citep{landau1976mechanics} is to calculate the Lagrangian of a
\href{https://goo.gl/IjqeO}{pendulum where the point of support is moving in a circle (figure and full text for problem in this Google books reference)}
 }
\makeanswer{landauMechanics:ch1:pr3a}{
The coordinates of the mass are
%
\begin{dmath}\label{eqn:landauMechanicsCh1P3:10}
p = a e^{i \gamma t} + i l e^{i\phi},
\end{dmath}
%
or in coordinates
%
\begin{dmath}\label{eqn:landauMechanicsCh1P3:30}
p = (a \cos\gamma t + l \sin\phi, -a \sin\gamma t + l \cos\phi).
\end{dmath}
%
The velocity is
%
\begin{dmath}\label{eqn:landauMechanicsCh1P3:50}
\pdot = (-a \gamma \sin\gamma t + l \phidot \cos\phi, -a \gamma \cos\gamma t - l \phidot \sin\phi),
\end{dmath}
%
and in the square
\begin{dmath}\label{eqn:landauMechanicsCh1P3:70}
\pdot^2 =
a^2 \gamma^2 + l^2 \phidot^2 - 2 a \gamma \phidot \sin\gamma t \cos\phi + 2 a \gamma l \phidot \cos \gamma t \sin\phi
=
a^2 \gamma^2 + l^2 \phidot^2 + 2 a \gamma l \phidot \sin (\gamma t - \phi).
\end{dmath}
%
For the potential our height above the minimum is
%
\begin{dmath}\label{eqn:landauMechanicsCh1P3:90}
h = 2a + l - a (1 -\cos\gamma t) - l \cos\phi = a ( 1 + \cos\gamma t) + l (1 - \cos\phi).
\end{dmath}
%
In the potential the total derivative \(\cos\gamma t\) can be dropped, as can all the constant terms, leaving
%
\begin{dmath}\label{eqn:landauMechanicsCh1P3:110}
U = - m g l \cos\phi,
\end{dmath}
%
so by the above the Lagrangian should be (after also dropping the constant term \(m a^2 \gamma^2/2\)
\begin{dmath}\label{eqn:landauMechanicsCh1P3:130}
\Lq =
\inv{2} m \left( l^2 \phidot^2 + 2 a \gamma l \phidot \sin (\gamma t - \phi) \right) + m g l \cos\phi.
\end{dmath}
%
This is almost the stated value in the text
\begin{dmath}\label{eqn:landauMechanicsCh1P3:150}
\Lq =
\inv{2} m \left( l^2 \phidot^2 + 2 a \gamma^2 l \sin (\gamma t - \phi) \right) + m g l \cos\phi.
\end{dmath}
%
We have what appears to be an innocent looking typo (text putting in a \(\gamma\) instead of a \(\phidot\)), but the subsequent text also didn't make sense.  That referred to the omission of the total derivative \(m l a \gamma \cos( \phi - \gamma t)\), which isn't even a term that I have in my result.

In the physicsforums response it was cleverly pointed out by Dickfore that \eqnref{eqn:landauMechanicsCh1P3:130} can be recast into a total derivative
%
\begin{equation}\label{eqn:landauMechanicsCh1P3:170}
\begin{aligned}
m &a l \gamma \phidot \sin (\gamma t - \phi) \\
&=
m a l \gamma ( \phidot - \gamma ) \sin (\gamma t - \phi)
+m a l \gamma^2 \sin (\gamma t - \phi) \\
&=
\ddt{}\left(
m a l \gamma \cos (\gamma t - \phi)
\right)
+m a l \gamma^2 \sin (\gamma t - \phi),
\end{aligned}
\end{equation}
%
which resolves the conundrum!
}
%
