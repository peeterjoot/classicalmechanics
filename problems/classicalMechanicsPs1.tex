%
% Copyright � 2012 Peeter Joot.  All Rights Reserved.
% Licenced as described in the file LICENSE under the root directory of this GIT repository.
%
%
\label{chap:classicalMechanicsPs1}
%\blogpage{http://sites.google.com/site/peeterjoot2/math2012/classicalMechanicsPs1.pdf}
%\date{Jan 24, 2012}
%
%\section{Disclaimer}
%
%Ungraded solutions to posted problem set 1 (I am auditing half the lectures for this course and will not be submitting any solutions for grading).
%
\makeproblem{Lorentz force Lagrangian.}{problem:classicalMechanicsPs1:1}{
\makesubproblem{For the non-covariant electrodynamic Lorentz force Lagrangian.}{problem:classicalMechanicsPs1:1.1}{
%
\begin{equation}\label{eqn:classicalMechanicsPs1:40}
\LL = \inv{2} m \Bv^2 + q \Bv \cdot \BA - q \phi,
\end{equation}
%
derive the Lorentz force equation
%
\begin{equation}\label{eqn:classicalMechanicsPs1:20}
\begin{aligned}
\BF &= q (\BE + \Bv \cross \BB) \\
\BE &= -\spacegrad \phi - \PD{t}{\BA} \\
\BB &= \spacegrad \cross \BA.
\end{aligned}
\end{equation}
} % end 1.1
%
\makesubproblem{With a gauge transformation of the form.}{problem:classicalMechanicsPs1:1.2}{
%
\begin{equation}\label{eqn:classicalMechanicsPs1:120}
\begin{aligned}
\phi &\rightarrow \phi + \PD{t}{\chi} \\
\BA &\rightarrow \BA - \spacegrad \chi,
\end{aligned}
\end{equation}
%
show that the Lagrangian is invariant.
} % end 1.2
} % end problem
%
\makeanswer{problem:classicalMechanicsPs1:1}{
%
\makesubanswer{Evaluate the Euler-Lagrange equations.}{problem:classicalMechanicsPs1:1.1}
%
In coordinates, employing summation convention, this Lagrangian is
%
\begin{equation}\label{eqn:classicalMechanicsPs1:60}
\LL = \inv{2} m \xdot_j \xdot_j + q \xdot_j A_j - q \phi.
\end{equation}
%
Taking derivatives
%
\begin{equation}\label{eqn:classicalMechanicsPs1:80}
\PD{\xdot_i}{\LL} = m \xdot_i + q A_i,
\end{equation}
%
\begin{equation}\label{eqn:classicalMechanicsPs1:540}
\begin{aligned}
\frac{d}{dt} \PD{\xdot_i}{\LL}
&=
m \ddot{x}_i
+ q \PD{t}{A_i}
+ q \PD{x_j}{A_i} \frac{dx_j}{dt} \\
&=
m \ddot{x}_i
+ q \PD{t}{A_i}
+ q \PD{x_j}{A_i} \xdot_j
\end{aligned}
\end{equation}
%
This must equal
%
\begin{equation}\label{eqn:classicalMechanicsPs1:100}
\PD{x_i}{\LL} = q \xdot_j \PD{x_i}{A_j} - q \PD{x_i}{\phi},
\end{equation}
%
So we have
%
\begin{equation}\label{eqn:classicalMechanicsPs1:560}
\begin{aligned}
m \ddot{x}_i
&=
-q \PD{t}{A_i}
- q \PD{x_j}{A_i} \xdot_j
+q \xdot_j \PD{x_i}{A_j} - q \PD{x_i}{\phi} \\
&=
-q \left( \PD{t}{A_i} - \PD{x_i}{\phi} \right)
+q v_j \left( \PD{x_i}{A_j} - \PD{x_j}{A_i} \right)
\end{aligned}
\end{equation}
%
The first term is just \(E_i\).  If we expand out \((\Bv \cross \BB)_i\) we see that matches
%
\begin{equation}\label{eqn:classicalMechanicsPs1:580}
\begin{aligned}
(\Bv \cross \BB)_i
&=
v_a B_b \epsilon_{abi} \\
&=
v_a \partial_r A_s \epsilon_{rsb} \epsilon_{abi} \\
&=
v_a \partial_r A_s \delta_{rs}^{[ia]} \\
&=
v_a (\partial_i A_a - \partial_a A_i).
\end{aligned}
\end{equation}
%
A \(a \rightarrow j\) substitution, and comparison of this with the Euler-Lagrange result above completes the exercise.
%
\makesubanswer{Gauge invariance.}{problem:classicalMechanicsPs1:1.2}
%
We really only have to show that
%
\begin{equation}\label{eqn:classicalMechanicsPs1:140}
\Bv \cdot \BA - \phi
\end{equation}
%
is invariant.  Making the transformation we have
%
\begin{equation}\label{eqn:classicalMechanicsPs1:600}
\begin{aligned}
\Bv \cdot \BA - \phi
&\rightarrow
v_j \left(A_j - \partial_j \chi \right) - \left(\phi + \PD{t}{\chi} \right) \\
&=
v_j A_j - \phi - v_j \partial_j \chi - \PD{t}{\chi} \\
&=
\Bv \cdot \BA - \phi
- \left( \frac{d x_j}{dt} \PD{x_j}{\chi} + \PD{t}{\chi} \right) \\
&=
\Bv \cdot \BA - \phi
- \frac{d \chi(\Bx, t)}{dt}.
\end{aligned}
\end{equation}
%
We see then that the Lagrangian transforms as
%
\begin{equation}\label{eqn:classicalMechanicsPs1:160}
\LL \rightarrow \LL + \frac{d}{dt}\left( -q \chi \right),
\end{equation}
%
and differs only by a total derivative.  With the lemma from the lecture, we see that this gauge transformation does not have any effect on the end result of applying the Euler-Lagrange equations.
} % end answer
%
\makeproblem{Finding trajectory through explicit minimization of the action.}{problem:classicalMechanicsPs1:2}{
For a ball thrown upward, guess a solution for the height \(y\) of the form \(y(t) = a_2 t^2+ a_1 t + a_0\).  Assuming that \(y(0) = y(T) = 0\), this quickly becomes \(y(t) = a_2(t^2- T t)\). Calculate the action (to do that, you need to first write the Lagrangian, of course) between \(t = 0\) and \(t = T\), and show that it is minimized when \(a_2= -g/2\).
} % end problem
%
\makeanswer{problem:classicalMechanicsPs1:2}{
%
We are told to guess at a solution
%
\begin{equation}\label{eqn:classicalMechanicsPs1:180}
y = a_2 t^2 + a_1 t + a_0,
\end{equation}
%
for the height of a particle thrown up into the air.  With initial condition \(y(0) = 0\) we have
%
\begin{equation}\label{eqn:classicalMechanicsPs1:200}
a_0 = 0,
\end{equation}
%
and with a final condition of \(y(T) = 0\) we also have
%
\begin{equation}\label{eqn:classicalMechanicsPs1:620}
\begin{aligned}
0
&=
a_2 T^2 + a_1 T \\
&= T( a_2 T + a_1 ),
\end{aligned}
\end{equation}
%
so have
%
\begin{equation}\label{eqn:classicalMechanicsPs1:220}
\begin{aligned}
y(t) &= a_2 t^2 - a_2 T t = a_2
\lr{ t^2 - T t } \\
\dot{y}(t) &=
a_2 (2 t - T )
\end{aligned}
\end{equation}
%
So our Lagrangian is
%
\begin{equation}\label{eqn:classicalMechanicsPs1:240}
\LL =
\inv{2} m a_2^2
\lr{ 2 t - T }^2 - m g a_2 \lr{ t^2 - T t }
\end{equation}
%
and our action is
%
\begin{equation}\label{eqn:classicalMechanicsPs1:260}
S = \int_0^T dt
\left(
\inv{2} m a_2^2 \lr{ 2 t - T  }^2
- m g a_2 \lr{ t^2 - T t }
\right).
\end{equation}
%
To minimize this action with respect to \(a_2\) we take the derivative
%
\begin{equation}\label{eqn:classicalMechanicsPs1:280}
\PD{a_2}{S} = \int_0^T
\left(
m a_2 \lr{ 2 t - T  }^2
- m g \lr{ t^2 - T t }
\right).
\end{equation}
%
Integrating we have
%
\begin{equation}\label{eqn:classicalMechanicsPs1:640}
\begin{aligned}
0 &= \PD{a_2}{S} \\
&={\left.
\left(
\inv{6} m a_2
\lr{ 2 t - T  }^3
- m g \left(\inv{3}t^3 - \inv{2}T t^2 \right)
\right)\right\vert}_0^T \\
&=
\inv{6} m a_2 T^3 - m g \left(\inv{3}T^3 - \inv{2}T^3 \right)
-
\inv{6} m a_2 (- T )^3 \\
&=
m T^3 \left( \inv{3} a_2 - g \left( \inv{3} - \inv{2} \right) \right) \\
&=
\inv{3} m T^3 \left( a_2 - g \left( 1 - \frac{3}{2} \right) \right) .
\end{aligned}
\end{equation}
%
or
%
\begin{equation}\label{eqn:classicalMechanicsPs1:300}
a_2 + g/2 = 0,
\end{equation}
%
which is the result we are required to show.
} % end answer
%
\makeproblem{Coordinate changes and Euler-Lagrange equations.}{problem:classicalMechanicsPs1:3}{
%
Consider a Lagrangian \(\LL({q},{\qdot}) \equiv \LL(q_1,\cdots, q_N, \qdot_1,\cdots \qdot_N)\). Now change the coordinates to some
new ones, e.g. let \(q_i = q_i(x_1,\cdots, x_N),i = 1 \cdots N\), or in short \(q_i = q_i({x})\). This defines a new
Lagrangian:
%
\begin{equation}\label{eqn:classicalMechanicsPs1:480}
\tilde{\LL}({x},{\xdot}) = \LL(q_1({x}),\cdots q_N({x}),\ddt{} q_1({x}),\cdots \ddt{} q_N({x}))
\end{equation}
%
which is now a function of \(x_i\) and \(\xdot_i\). Show that the Euler-Lagrange equations for \(\LL({q},{\qdot})\):
%
\begin{equation}\label{eqn:classicalMechanicsPs1:500}
\frac{\partial \LL({q},{\qdot})}{\partial q_i} =
\ddt{}
\frac{\partial \LL({q},{\qdot})}{\partial \qdot_i}
\end{equation}
%
imply that the Euler-Lagrange equations for \(\tilde{\LL}({x},{\xdot})\) hold (provided the change of variables \(q \rightarrow x\) is nonsingular):
\begin{equation}\label{eqn:classicalMechanicsPs1:520}
\frac{\partial \tilde{\LL}({x},{\xdot})}{\partial x_i} =
\ddt{}
\frac{\partial \tilde{\LL}({x},{\xdot})}{\partial \xdot_i}
\end{equation}
%
The moral is that the action formalism is very convenient: one can write the Lagrangian in any set of coordinates; the Euler-Lagrange equations for the new coordinates can then be obtained by using the Lagrangian expressed in these coordinates.

Hint: Solving this problem only requires repeated use of the chain rule.
} % end problem
%
\makeanswer{problem:classicalMechanicsPs1:3}{
%
Here we want to show that after a change of variables, provided such a transformation is non-singular, the Euler-Lagrange equations are still valid.

Let us write
%
\begin{equation}\label{eqn:classicalMechanicsPs1:320}
r_i = r_i(q_1, q_2, \cdots q_N).
\end{equation}
%
Our ``velocity'' variables in terms of the original parametrization \(q_i\) are
%
\begin{equation}\label{eqn:classicalMechanicsPs1:340}
\dot{r}_j = \frac{dr_j}{dt} = \PD{q_i}{r_j} \PD{t}{q_i} = \qdot_i \PD{q_i}{r_j},
\end{equation}
%
so we have
%
\begin{equation}\label{eqn:classicalMechanicsPs1:360}
\PD{\qdot_i}{\dot{r}_j} = \PD{q_i}{r_j}.
\end{equation}
%
Computing the LHS of the Euler Lagrange equation we find
%
\begin{equation}\label{eqn:classicalMechanicsPs1:380}
\PD{q_i}{\LL} =
\PD{r_j}{\LL} \PD{q_i}{r_j}
+\PD{\rdot_j}{\LL} \PD{q_i}{\rdot_j}.
\end{equation}
%
For our RHS we start with
%
\begin{equation}\label{eqn:classicalMechanicsPs1:400}
\PD{\qdot_i}{\LL}
=
\PD{r_j}{\LL} \PD{\qdot_i}{r_j}
+\PD{\rdot_j}{\LL} \PD{\qdot_i}{\rdot_j}
=
\PD{r_j}{\LL} \PD{\qdot_i}{r_j}
+\PD{\rdot_j}{\LL} \PD{q_i}{r_j},
\end{equation}
%
but \(\PDi{\qdot_i}{r_j} = 0\), so this is just
%
\begin{equation}\label{eqn:classicalMechanicsPs1:420}
\PD{\qdot_i}{\LL}
=
\PD{r_j}{\LL} \PD{\qdot_i}{r_j}
+\PD{\rdot_j}{\LL} \PD{\qdot_i}{\rdot_j}
=
\PD{\rdot_j}{\LL} \PD{q_i}{r_j}.
\end{equation}
%
The Euler-Lagrange equations become
%
\begin{equation}\label{eqn:classicalMechanicsPs1:660}
\begin{aligned}
0 &=
\PD{r_j}{\LL} \PD{q_i}{r_j}
+\PD{\rdot_j}{\LL} \PD{q_i}{\rdot_j}
-
\ddt{} \left(
\PD{\rdot_j}{\LL} \PD{q_i}{r_j}
\right) \\
&=
  \PD{r_j}{\LL} \PD{q_i}{r_j}
+ \cancel{\PD{\rdot_j}{\LL} \PD{q_i}{\rdot_j}}
- \left( \ddt{} \PD{\rdot_j}{\LL} \right) \PD{q_i}{r_j}
- \cancel{\PD{\rdot_j}{\LL} \ddt{} \PD{q_i}{r_j} }
\\
&=
\left( \PD{r_j}{\LL}
-\ddt{} \PD{\rdot_j}{\LL}
\right) \PD{q_i}{r_j}
\end{aligned}
\end{equation}
%
Since we have an assumption that the transformation is non-singular, we have for all \(j\)
%
\begin{equation}\label{eqn:classicalMechanicsPs1:440}
\PD{q_i}{r_j} \ne 0,
\end{equation}
%
so we have the Euler-Lagrange equations for the new abstract coordinates as well
%
\begin{equation}\label{eqn:classicalMechanicsPs1:460}
0 = \PD{r_j}{\LL} -\ddt{} \PD{\rdot_j}{\LL}.
\end{equation}
} % end answer
