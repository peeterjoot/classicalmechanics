%
% Copyright © 2012 Peeter Joot.  All Rights Reserved.
% Licenced as described in the file LICENSE under the root directory of this GIT repository.
%
\makeoproblem{Pendulum with support moving in vertical line.}{landauMechanics:ch1:pr3c}{\citep{landau1976mechanics} p1.3c.}{
As above, but with the support point moving up and down as \(a \cos\gamma t\).
}
\makeanswer{landauMechanics:ch1:pr3c}{
Our mass point is
%
\begin{dmath}\label{eqn:landauMechanicsCh1P3:290}
p = a \cos \gamma t + l e^{i \phi}.
\end{dmath}
%
with velocity
%
\begin{dmath}\label{eqn:landauMechanicsCh1P3:310}
\pdot
= -a \gamma \sin \gamma t + l i \phidot e^{i \phi}
= (-a \gamma \sin\gamma t - l \phidot \sin\phi, l \phidot \cos\phi).
\end{dmath}
%
In the square this is
%
\begin{dmath}\label{eqn:landauMechanicsCh1P3:330}
\Abs{\pdot}^2 =
a^2 \gamma^2 \sin^2 \gamma t + l^2 \phidot^2 \sin^2 \phi + 2 a l \gamma \phidot \sin\gamma t \sin\phi.
\end{dmath}
%
Having done the simplification in the last problem in a complicated way, let's try it, knowing what our answer is
%
\begin{dmath}\label{eqn:landauMechanicsCh1P3:350}
\phidot \sin\gamma t \sin\phi
=
\phidot \sin\gamma t \sin\phi
- \gamma \cos \gamma t \cos\phi
+ \gamma \cos \gamma t \cos\phi
=
\sin\gamma t \ddt{}\left( -\cos\phi \right)
+ \left( \ddt{}\left( -\sin \gamma t \right) \right) \cos\phi
+ \gamma \cos \gamma t \cos\phi
=
\gamma \cos \gamma t \cos\phi
- \ddt{} \left(
\sin \gamma t \cos\phi
\right).
\end{dmath}
%
With the height of the particle above the lowest point given by
%
\begin{dmath}\label{eqn:landauMechanicsCh1P3:370}
h = a + l - a \cos \gamma t - l \cos\phi,
\end{dmath}
%
we can write the Lagrangian immediately (dropping all the total derivative terms)
%
\begin{dmath}\label{eqn:landauMechanicsCh1P3:390}
\LL =
\inv{2} m \left( l^2 \phidot^2 \sin^2 \phi + 2 a l \gamma^2 \cos \gamma t \cos\phi \right)
+ m g l \cos\phi.
\end{dmath}
}
%
\EndArticle
