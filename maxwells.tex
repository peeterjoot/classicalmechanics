%
% Copyright � 2020 Peeter Joot.  All Rights Reserved.
% Licenced as described in the file LICENSE under the root directory of this GIT repository.
%
%{
\input{../latex/blogpost.tex}
\renewcommand{\basename}{maxwells}
%\renewcommand{\dirname}{notes/phy1520/}
\renewcommand{\dirname}{notes/ece1228-electromagnetic-theory/}
%\newcommand{\dateintitle}{}
%\newcommand{\keywords}{}

\input{../latex/peeter_prologue_print2.tex}

\PassOptionsToPackage{answerdelayed}{exercise}

% proof:
\usepackage{amsthm}
\usepackage{macros_cal} % \LL
\usepackage{peeters_layout_exercise}
\usepackage{peeters_braket}
\usepackage{peeters_figures}
\usepackage{siunitx}
\usepackage{verbatim}
%\usepackage{mhchem} % \ce{}
%\usepackage{macros_bm} % \bcM
%\usepackage{macros_qed} % \qedmarker
%\usepackage{txfonts} % \ointclockwise

\beginArtNoToc

\generatetitle{Maxwell's equation using geometric algebra Lagrangian.}
%\chapter{XXX}
%\label{chap:maxwells}
\section{Motivation.}
In my classical mechanics notes, I've got computations of Maxwell's equation (singular in it's geometric algebra form) from a Lagrangian in various ways (using a tensor, scalar and multivector Lagrangians), but all of these seem more complex than they should be.  Here's a new derivation, starting with the action principle for field variables.
\section{Field action.}
\maketheorem{Relativistic Euler-Lagrange field equations.}{thm:maxwells:40}{
Let \( \phi \rightarrow \phi + \delta \phi \) be any variation of the field, such that the variation
\( \delta \phi = 0 \) vanishes at the boundaries of the action integral
\begin{equation*}
S = \int d^4 x \LL(\phi, \partial_\nu \phi).
\end{equation*}
The extreme value of the action is found when the Euler-Lagrange equations
\begin{equation*}
0 = \PD{\phi}{\LL} - \partial_\nu \PD{(\partial_\nu \phi)}{\LL},
\end{equation*}
are satisfied.  For a Lagrangian with multiple field variables, there will be one such equation for each field.
} % theorem
\begin{proof}
To ease the visual burden, designate the variation of the field by \( \delta \phi = \epsilon \), and perform a first order expansion of the Lagrangian
\begin{dmath}\label{eqn:maxwells:20}
\LL \rightarrow
\LL(\phi + \epsilon, \partial_\nu (\phi + \epsilon))
=
\LL(\phi, \partial_\nu \phi)
+
\PD{\phi}{\LL} \epsilon + 
\PD{\partial_\nu \phi}{\LL} \partial_\nu \epsilon.
\end{dmath}
The variation of the Lagrangian is 
\begin{dmath}\label{eqn:maxwells:40}
\delta \LL =
\PD{\phi}{\LL} \epsilon + 
\PD{\partial_\nu \phi}{\LL} \partial_\nu \epsilon
=
\PD{\phi}{\LL} \epsilon + 
\partial_\nu \lr{ \PD{\partial_\nu \phi}{\LL} \epsilon }
-
\epsilon \partial_\nu \PD{\partial_\nu \phi}{\LL},
\end{dmath}
which we may plug into the action integral to find
\begin{dmath}\label{eqn:maxwells:60}
   \delta S =
   \int d^4 x \epsilon \lr{ 
   \PD{\phi}{\LL} 
   -
   \partial_\nu \PD{\partial_\nu \phi}{\LL} 
}
+
   \int d^4 x 
\partial_\nu \lr{ \PD{\partial_\nu \phi}{\LL} \epsilon }.
\end{dmath}
The last integral can be evaluated along the \( dx^\nu \) direction, leaving
\begin{dmath}\label{eqn:maxwells:80}
   \int d^3 x 
   \evalbar{ \PD{\partial_\nu \phi}{\LL} \epsilon }{\Delta x^\nu},
\end{dmath}
where \( d^3 x = dx^\alpha dx^\beta dx^\gamma \) is the product of differentials that does not include \( dx^\nu \).  By construction, \( \epsilon \) vanishes on the boundary of the action integral, showing that we extremize the action when
\begin{dmath}\label{eqn:maxwells:100}
0 = \delta S =
   \int d^4 x \epsilon \lr{ 
   \PD{\phi}{\LL} 
   -
   \partial_\nu \PD{\partial_\nu \phi}{\LL} 
}.
\end{dmath}
The proof is complete after noting that this must hold for all variations of the field \( \epsilon \) requiring
\begin{dmath}\label{eqn:maxwells:120}
0  = 
   \PD{\phi}{\LL} 
   -
   \partial_\nu \PD{\partial_\nu \phi}{\LL} .
\end{dmath}
\end{proof}
%}
%\EndArticle
\EndNoBibArticle
