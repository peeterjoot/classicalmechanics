%
% Copyright � 2020 Peeter Joot.  All Rights Reserved.
% Licenced as described in the file LICENSE under the root directory of this GIT repository.
%
%{
\input{../latex/blogpost.tex}
\renewcommand{\basename}{maxwells}
%\renewcommand{\dirname}{notes/phy1520/}
\renewcommand{\dirname}{notes/ece1228-electromagnetic-theory/}
%\newcommand{\dateintitle}{}
%\newcommand{\keywords}{}

\input{../latex/peeter_prologue_print2.tex}

\PassOptionsToPackage{answerdelayed}{exercise}

% proof:
\usepackage{amsthm}
\usepackage{macros_cal} % \LL
\usepackage{peeters_layout_exercise}
\usepackage{peeters_braket}
\usepackage{peeters_figures}
\usepackage{siunitx}
\usepackage{verbatim}
%\usepackage{mhchem} % \ce{}
%\usepackage{macros_bm} % \bcM
%\usepackage{macros_qed} % \qedmarker
%\usepackage{txfonts} % \ointclockwise

\beginArtNoToc

\generatetitle{Maxwell's equation using geometric algebra Lagrangian.}
%\chapter{XXX}
%\label{chap:maxwells}
\section{Motivation.}
In my classical mechanics notes, I've got computations of Maxwell's equation (singular in it's geometric algebra form) from a Lagrangian in various ways (using a tensor, scalar and multivector Lagrangians), but all of these seem more convoluted than they should be.  Here's a new derivation, starting with the action principle for field variables.
\section{Field action.}
\maketheorem{Relativistic Euler-Lagrange field equations.}{thm:maxwells:40}{
Let \( \phi \rightarrow \phi + \delta \phi \) be any variation of the field, such that the variation
\( \delta \phi = 0 \) vanishes at the boundaries of the action integral
\begin{equation*}
S = \int d^4 x \LL(\phi, \partial_\nu \phi).
\end{equation*}
The extreme value of the action is found when the Euler-Lagrange equations
\begin{equation*}
0 = \PD{\phi}{\LL} - \partial_\nu \PD{(\partial_\nu \phi)}{\LL},
\end{equation*}
are satisfied.  For a Lagrangian with multiple field variables, there will be one such equation for each field.
} % theorem
\begin{proof}
To ease the visual burden, designate the variation of the field by \( \delta \phi = \epsilon \), and perform a first order expansion of the varied Lagrangian
\begin{dmath}\label{eqn:maxwells:20}
\LL \rightarrow
\LL(\phi + \epsilon, \partial_\nu (\phi + \epsilon))
=
\LL(\phi, \partial_\nu \phi)
+
\PD{\phi}{\LL} \epsilon +
\PD{(\partial_\nu \phi)}{\LL} \partial_\nu \epsilon.
\end{dmath}
The variation of the Lagrangian is
\begin{dmath}\label{eqn:maxwells:40}
\delta \LL =
\PD{\phi}{\LL} \epsilon +
\PD{(\partial_\nu \phi)}{\LL} \partial_\nu \epsilon
=
\PD{\phi}{\LL} \epsilon +
\partial_\nu \lr{ \PD{(\partial_\nu \phi)}{\LL} \epsilon }
-
\epsilon \partial_\nu \PD{(\partial_\nu \phi)}{\LL},
\end{dmath}
which we may plug into the action integral to find
\begin{dmath}\label{eqn:maxwells:60}
   \delta S =
   \int d^4 x \epsilon \lr{
   \PD{\phi}{\LL}
   -
   \partial_\nu \PD{(\partial_\nu \phi)}{\LL}
}
+
   \int d^4 x
\partial_\nu \lr{ \PD{(\partial_\nu \phi)}{\LL} \epsilon }.
\end{dmath}
The last integral can be evaluated along the \( dx^\nu \) direction, leaving
\begin{dmath}\label{eqn:maxwells:80}
   \int d^3 x
   \evalbar{ \PD{(\partial_\nu \phi)}{\LL} \epsilon }{\Delta x^\nu},
\end{dmath}
where \( d^3 x = dx^\alpha dx^\beta dx^\gamma \) is the product of differentials that does not include \( dx^\nu \).  By construction, \( \epsilon \) vanishes on the boundary of the action integral so \cref{eqn:maxwells:80} is zero.  The action takes its extreme value when
\begin{dmath}\label{eqn:maxwells:100}
0 = \delta S
=
\int d^4 x \epsilon \lr{
   \PD{\phi}{\LL}
   -
   \partial_\nu \PD{(\partial_\nu \phi)}{\LL}
}.
\end{dmath}
The proof is complete after noting that this must hold for all variations of the field \( \epsilon \), which means that we must have
\begin{dmath}\label{eqn:maxwells:120}
0  =
   \PD{\phi}{\LL}
   -
   \partial_\nu \PD{(\partial_\nu \phi)}{\LL}.
\end{dmath}
\end{proof}
%}
Armed with the Euler-Lagrange equations, we can apply them to the Maxwell's equation Lagrangian, which we will claim has the following form.
\maketheorem{Maxwell's equation Lagrangian.}{thm:maxwells:140}{
Application of the Euler-Lagrange equations to the Lagrangian
\begin{equation*}
\LL = - \frac{\epsilon_0 c}{2} F \cdot F + J \cdot A,
\end{equation*}
where \( F = \grad \wedge A \), yields the vector portion of Maxwell's equation
\begin{equation*}
\grad \cdot F = \inv{\epsilon_0 c} J,
\end{equation*}
which implies
\begin{equation*}
\grad F = \inv{\epsilon_0 c} J.
\end{equation*}
This is Maxwell's equation.
} % definition
\begin{proof}
We wish to apply all of the Euler-Lagrange equations simulaneously (i.e. for each of the \(A_\mu\) field variables), and cast it into four-vector form
\begin{dmath}\label{eqn:maxwells:140}
0 = \gamma_\nu \lr{ \PD{A_\nu}{} - \partial_\mu \PD{(\partial_\mu A_\nu)}{} } \LL.
\end{dmath}
Since our Lagrangian splits nicely into kinetic and interaction terms, this gives us
\begin{dmath}\label{eqn:maxwells:160}
   0 = \gamma_\nu \lr{ \PD{A_\nu}{(A \cdot J)} + \frac{\epsilon_0 c}{2} \partial_\mu \PD{(\partial_\mu A_\nu)}{ (F \cdot F)} }.
\end{dmath}
The interaction term above is just
\begin{equation}\label{eqn:maxwells:180}
   \gamma_\nu \PD{A_\nu}{(A \cdot J)}
=
\gamma_\nu \PD{A_\nu}{(A_\mu J^\mu)}
=
\gamma_\nu J^\nu
=
J.
\end{equation}
The kinetic term takes a bit more work.  Let's start with evaluating
\begin{dmath}\label{eqn:maxwells:200}
   \PD{(\partial_\mu A_\nu)}{ (F \cdot F)}
=
\PD{(\partial_\mu A_\nu)}{ F } \cdot F
+
F \cdot \PD{(\partial_\mu A_\nu)}{ F }
=
2 \PD{(\partial_\mu A_\nu)}{ F } \cdot F
=
2 \PD{(\partial_\mu A_\nu)}{ \partial_\alpha A_\beta } \lr{ \gamma^\alpha \wedge \gamma^\beta } \cdot F
=
2 \lr{ \gamma^\mu \wedge \gamma^\nu } \cdot F.
\end{dmath}
We hit this with the mu-partial and expand as a scalar selection to find
\begin{dmath}\label{eqn:maxwells:220}
   \partial_\mu \PD{(\partial_\mu A_\nu)}{ (F \cdot F)}
=
2 \lr{ \partial_\mu \gamma^\mu \wedge \gamma^\nu } \cdot F
=
- 2 (\gamma^\nu \wedge \grad) \cdot F
=
- 2 \gpgradezero{ (\gamma^\nu \wedge \grad) F }
=
- 2 \gpgradezero{ \gamma^\nu \grad F - \cancel{ \gamma^\nu \cdot \grad F } }
=
- 2 \gamma^\nu \cdot \lr{ \grad \cdot F }.
\end{dmath}
Putting all the pieces together yields
\begin{equation}\label{eqn:maxwells:240}
0
= J - \epsilon_0 c \gamma_\nu \lr{ \gamma^\nu \cdot \lr{ \grad \cdot F } }
= J - \epsilon_0 c \lr{ \grad \cdot F },
\end{equation}
but
\begin{dmath}\label{eqn:maxwells:260}
\grad \cdot F
=
\grad F - \grad \wedge F
=
\grad F - \grad \wedge (\grad \wedge A)
=
\grad F,
\end{dmath}
so the multivector field equations for this Lagrangian are
\begin{dmath}\label{eqn:maxwells:280}
\grad F = \inv{\epsilon_0 c} J,
\end{dmath}
as claimed.
\end{proof}
%\EndArticle
\EndNoBibArticle
