%
% Copyright � 2020 Peeter Joot.  All Rights Reserved.
% Licenced as described in the file LICENSE under the root directory of this GIT repository.
%
%{
\input{../latex/blogpost.tex}
\renewcommand{\basename}{relativisticSurface}
%\renewcommand{\dirname}{notes/phy1520/}
\renewcommand{\dirname}{notes/ece1228-electromagnetic-theory/}
%\newcommand{\dateintitle}{}
%\newcommand{\keywords}{}

\input{../latex/peeter_prologue_print2.tex}

\usepackage{peeters_layout_exercise}
\usepackage{peeters_braket}
\usepackage{peeters_figures}
\usepackage{siunitx}
\usepackage{verbatim}
%\usepackage{mhchem} % \ce{}
%\usepackage{macros_bm} % \bcM
%\usepackage{macros_qed} % \qedmarker
%\usepackage{txfonts} % \ointclockwise

\beginArtNoToc

\generatetitle{Relativistic multivector surface integrals.}
%\chapter{Relativistic multivector surface integrals.}
%\label{chap:relativisticSurface}

We've now covered line integrals and the fundamental theorem for line integrals, so it's now time to move on to surface integrals.
\makedefinition{Surface integral.}{dfn:relativisticSurface:20}{
Given a two variable parameterization \( x = x(u,v) \), we write \( d^2\Bx = \Bx_u \wedge \Bx_v du dv \), and call
\begin{equation*}
\int F d^2\Bx\, G,
\end{equation*}
a \emph{surface integral}, where \( F,G \) are arbitrary multivector functions.
} % definition
Like our multivector line integral, this is intrinsically multivector valued, with a product of \( F \) with arbitrary grades, a bivector \( d^2 \Bx \), and \( G \), also potentially with arbitrary grades.  Let's consider an example.
\makeproblem{Surface area integral example.}{problem:relativisticSurface:10}{
Given the hyperbolic surface parameterization \( x(\rho,\alpha) = \rho \gamma_0 e^{-\vcap \alpha} \), where \( \vcap = \gamma_{20} \) evaluate the indefinite integral
\begin{dmath}\label{eqn:relativisticSurface:40}
\int \gamma_1 e^{\gamma_{21}\alpha} d^2 \Bx\, \gamma_2.
\end{dmath}
} % problem
\makeanswer{problem:relativisticSurface:10}{
We have \( \Bx_\rho = \gamma_0 e^{-\vcap \alpha} \) and \( \Bx_\alpha = \rho\gamma_{2} e^{-\vcap \alpha} \), so
\begin{dmath}\label{eqn:relativisticSurface:60}
d^2 \Bx 
=
(\Bx_\rho \wedge \Bx_\alpha) d\rho d\alpha
= 
\gpgradetwo{
\gamma_{0} e^{-\vcap \alpha} \rho\gamma_{2} e^{-\vcap \alpha} 
}
d\rho d\alpha
=
\rho \gamma_{02} d\rho d\alpha,
\end{dmath}
so the integral is
\begin{dmath}\label{eqn:relativisticSurface:80}
\int \rho \gamma_1 e^{\gamma_{21}\alpha} \gamma_{022} d\rho d\alpha
=
-\inv{2} \rho^2 \int \gamma_1 e^{\gamma_{21}\alpha} \gamma_{0} d\alpha
=
\frac{\gamma_{01}}{2} \rho^2 \int e^{\gamma_{21}\alpha} d\alpha
=
\frac{\gamma_{01}}{2} \rho^2 \gamma^{12} e^{\gamma_{21}\alpha}
=
\frac{\rho^2 \gamma_{20}}{2} e^{\gamma_{21}\alpha}.
\end{dmath}
Because \( F \) and \( G \) were both vectors, the resulting integral could only have been a multivector with grades 0,2,4.
As it happens, there were no scalar nor pseudoscalar grades in the end result, and we ended up with the spacetime plane between \( \gamma_0 \), and \( \gamma_2 e^{\gamma_{21}\alpha} \), which are rotations of \(\gamma_2\) in the x,y plane.  This is illustrated in \cref{fig:gammazeroWedgedWithXYplaneRotation:gammazeroWedgedWithXYplaneRotationFig1} (omitting scale and sign factors, and leaving the orientation of the bivector unspecified.)
\imageFigure{../figures/classicalmechanics/gammazeroWedgedWithXYplaneRotationFig1}{Spacetime plane.}{fig:gammazeroWedgedWithXYplaneRotation:gammazeroWedgedWithXYplaneRotationFig1}{0.3}
} % answer

%}
%\EndArticle
\EndNoBibArticle
