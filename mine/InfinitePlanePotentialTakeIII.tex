%
% Copyright � 2012 Peeter Joot.  All Rights Reserved.
% Licenced as described in the file LICENSE under the root directory of this GIT repository.
%
%
%\chapter{Potential for an infinitesimal width infinite plane.  Take III}
\index{infinitesimal width potential}
\label{chap:InfinitePlanePotentialTakeIII}
%\blogpage{http://sites.google.com/site/peeterjoot2/math2012/InfinitePlanePotentialTakeIII.pdf}
%\date{Feb 24, 2012}
%
\section{Document generation experiment}
%
%\href{https://raw.github.com/peeterjoot/physicsplay/master/notes/phy354/problemSetIIproblem4IntegralsEvaluatingInfinitePlanePotentialTakeIII.cdf}{
See \nbref{psIIp4InfPlanePotTakeIII.cdf} for the notebook that generated this document.
%This little document was generated as an experiment using 
%and some post processing in latex.

The File menu save as latex produced latex that could not be compiled, but mouse selected, copy-as latex worked out fairly well.

Post processing done included:
%
\begin{itemize}
\item Adding in latex prologue.
\item Stripping out the text boxes.
\item Adding in equation environments.
\item Latex generation for math output in inline text sections was uniformly poor.
\end{itemize}
%
\section{Guts}
%
I had like to attempt again to evaluate the potential for infinite plane distribution.  The general form of our potential takes the form
%
\begin{equation}\label{eqn:InfinitePlanePotentialTakeIII:10}
\phi(\Bx) = G \rho \int \inv{\Abs{\Bx - \Bx'}} dV'
\end{equation}
%
We want to evaluate this with cylindrical coordinates \((r', \theta', z')\), for a width \(\epsilon\), and radius \(r\), at distance \(z\) from the plane.
%
%
\begin{equation}\label{eqn:InfinitePlanePotentialTakeIII:30}
\phi (z, \epsilon , r)= 2 \pi  G \sigma  \frac{1}{\epsilon }\int _{r' = 0}^r\int _{z' = 0}^{\epsilon }\frac{r'}{\sqrt{\left(z-z'\right)^2+\left(r'\right)^2}}dz' dr''
\end{equation}
%
With the assumption that we will take the limits \(\epsilon \rightarrow 0\), and \(r \rightarrow \infty\).  With \(r^2 = c/\epsilon\), this does not converge.  How about with \(r = c/\epsilon\)?
%
Performing the r' integration (with \(r^2 = c/\epsilon\)) we find
%
\begin{equation}\label{eqn:InfinitePlanePotentialTakeIII:50}
\phi (z, \epsilon )= 2 \pi  G \sigma  \frac{1}{\epsilon }\int_{z' = 0}^{\epsilon } \left(\sqrt{\frac{c^2}{\epsilon ^2}+(z-z')^2}-\sqrt{(z-z')^2}\right) \, dz'
\end{equation}
%
Attempting to let \textit{Mathematica} evaluate this takes a long time.  Long enough that I aborted the attempt to evaluate it.

Instead, first evaluating the z' integral we have
%
\begin{equation}\label{eqn:InfinitePlanePotentialTakeIII:70}
\phi (z, \epsilon , r)=\frac{2 \pi  G \sigma }{\epsilon }
\int _{r' = 0}^{c/\epsilon }\left(\ln \left(\sqrt{\left(r'\right)^2+z^2}+z\right)-\ln \left(\sqrt{\left(r'\right)^2+(z-\epsilon )^2}+z-\epsilon \right)\right)
dr'
\end{equation}
%
This second integral can then be evaluated in reasonable time:
%
\begin{equation}\label{eqn:InfinitePlanePotentialTakeIII:90}
\begin{aligned}
\phi (z, \epsilon )
=
\frac{2 \pi  G \sigma }{\epsilon ^2}
\Biggl(
&c \ln \left(\frac{\sqrt{\frac{c^2}{\epsilon ^2}+z^2}+z}{\sqrt{\frac{c^2}{\epsilon ^2}+(z-\epsilon )^2}+z-\epsilon }\right)
+ \epsilon z \ln \left(\frac{(z-\epsilon ) \left(\sqrt{c^2+z^2 \epsilon ^2}+c\right)}{z}\right) \\
&+ \epsilon (\epsilon -z) \ln \left(\sqrt{c^2+\epsilon ^2 (z-\epsilon )^2}+c\right)
- \epsilon^2  \ln (\epsilon  (z-\epsilon ))
\Biggr).
\end{aligned}
\end{equation}
%
Grouping the log terms we have
%
\begin{equation}\label{eqn:InfinitePlanePotentialTakeIII:90b}
\begin{aligned}
\phi (z, \epsilon )
=
2 \pi  G \sigma  \Biggl(
&\frac{c}{\epsilon ^2}\ln \left(\frac{\sqrt{c^2+z^2 \epsilon ^2}+z \epsilon }{\sqrt{c^2+\epsilon ^2(z-\epsilon )^2}+\epsilon (z-\epsilon )}\right)
+ \frac{z}{\epsilon } \ln \left(\frac{(z-\epsilon ) \left(\sqrt{c^2+z^2 \epsilon ^2}+c\right)}{z\left(\sqrt{c^2+\epsilon ^2 (z-\epsilon )^2}+c\right)}\right) \\
&+ \ln \left(\frac{\sqrt{c^2+\epsilon ^2 (z-\epsilon )^2}+c}{\epsilon  (z-\epsilon )}\right)
\Biggr).
\end{aligned}
\end{equation}
%
Does this have a limit as \(\epsilon \rightarrow 0\)?  No, the last term is clearly divergent for \(c \neq 0\).
