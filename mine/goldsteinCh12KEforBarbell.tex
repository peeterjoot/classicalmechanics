%
% Copyright � 2012 Peeter Joot.  All Rights Reserved.
% Licenced as described in the file LICENSE under the root directory of this GIT repository.
%
%
%\label{chap:goldsteinCh12}
%
%Now have the so often cited \citep{goldstein1951cm} book to study (an ancient version from the 50's).  Here is an attempt at a few of the problems.  Some problems were tackled but omitted here since they overlapped with those written up in \bookchapcite{PJTongMf1}{phy354} before getting this book.
%
\makeoproblem{Kinetic energy for barbell shaped object.}{gold:ch1:pr6}{\citep{goldstein1951cm} 1.6}{
Two points of mass \(m\) are joined by a rigid weightless rod of length \(l\), the center of which is constrained to move on a circle of radius \(a\).  Set up the kinetic energy in generalized coordinates.
}
\makeanswer{gold:ch1:pr6}{
%
Barbell shape, equal masses.  center of rod between masses constrained to circular motion.
%
Assuming motion in a plane, the equation for the center of the rod is:
%
\begin{equation}\label{eqn:goldsteinCh12:1000}
c = a e^{i\theta},
\end{equation}
%
and the two mass points positions are:
\begin{equation}\label{eqn:goldsteinCh12:20}
\begin{aligned}
q_1 &= c + (l/2) e^{i\alpha} \\
q_2 &= c - (l/2) e^{i\alpha}.
\end{aligned}
\end{equation}
%
taking derivatives:
\begin{equation}\label{eqn:goldsteinCh12:40}
\begin{aligned}
\qdot_1 &= a i \dottheta e^{i\theta} + (l/2) i \dotalpha e^{i\alpha} \\
\qdot_2 &= a i \dottheta e^{i\theta} - (l/2) i \dotalpha e^{i\alpha} .
\end{aligned}
\end{equation}
%
and squared magnitudes:
%
\begin{equation}\label{eqn:goldsteinCh12:60}
\begin{aligned}
\qdot_{\pm}
&= \Abs{a \dottheta \pm (l/2) \dotalpha e^{i(\alpha - \theta)}}^2 \\
&= \left(a \dottheta   \pm   \inv{2} l \dotalpha \cos(\alpha - \theta)\right)^2 + \left(\inv{2} l \dotalpha \sin(\alpha - \theta)\right)^2.
\end{aligned}
\end{equation}
%
Summing the kinetic terms yields
%
\begin{equation}\label{eqn:goldsteinCh12:1020}
K = m \left(a \dottheta \right)^2 + m \left(\inv{2} l \dotalpha\right)^2.
\end{equation}
%
Summing the potential energies, presuming that the motion is vertical, we have:
%
\begin{equation}\label{eqn:goldsteinCh12:1040}
V = m g (l/2) \cos\theta - m g (l/2) \cos \theta,
\end{equation}
%
so the Lagrangian is just the Kinetic energy.

Taking derivatives to get the EOMs we have:
%
\begin{equation}\label{eqn:goldsteinCh12:80}
\begin{aligned}
\lr{ m a^2 \dottheta }' &= 0 \\
\left(\inv{4} m l^2 \dotalpha \right)' &= 0.
\end{aligned}
\end{equation}
%
This is surprising seeming.  Is this correct?
}
%
