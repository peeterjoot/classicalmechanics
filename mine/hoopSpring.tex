%
% Copyright � 2012 Peeter Joot.  All Rights Reserved.
% Licenced as described in the file LICENSE under the root directory of this GIT repository.
%
%
%\chapter{Hoop and spring oscillator problem}
\index{hoop and spring}
\label{chap:hoopSpring}
%\blogpage{http://sites.google.com/site/peeterjoot/math2010/hoopSpring.pdf}
%\date{June 19, 2010}
%
\section{Motivation}
%
Nolan was attempting to setup and solve the equations for the following system \cref{fig:hoopSpring}
%
\imageFigure{../figures/classicalmechanics/hoopSpring}{Coupled hoop and spring system}{fig:hoopSpring}{0.4}
%
One mass is connected between two springs to a bar.  That bar moves up and down as forced by the motion of the other mass along a immovable hoop.  While Nolan did not include any gravitational force in his potential terms (ie: system lying on a table perhaps) it does not take much more to include that, and I will do so.  I also include the distance \(L\) to the center of the hoop, which I believe required.
%
\section{Guts}
%
The Lagrangian can be written by inspection.  Writing \(x = x_1\), and \(x_2 = R \sin\theta\), we have
%
\begin{dmath}\label{eqn:hoopSpring:1}
\LL =
\inv{2} m_1 \dot{x}^2
+ \inv{2} m_2 R^2 \dot{\theta}^2
- \inv{2} k_1 x^2
- \inv{2} k_2 ( L + R \sin\theta - x )^2
- m_1 g x
- m_2 g ( L + R \sin\theta).
\end{dmath}
%
Evaluation of the Euler-Lagrange equations gives
%
\begin{subequations}
\label{eqn:hoopSpring:2}
\begin{align}
m_1 \ddot{x} &= - k_1 x + k_2 ( L + R \sin\theta - x ) - m_1 g \\
m_2 R^2 \ddot{\theta} &= - k_2 ( L + R \sin\theta - x ) R \cos\theta - m_2 g R \cos\theta,
\end{align}
\end{subequations}
%
or
%
\begin{subequations}
\label{eqn:hoopSpring:3}
\begin{align}
\ddot{x} &=
%- \frac{k_1}{m_1} x + \frac{k_2}{m_1} ( L + R \sin\theta - x ) - g \\
-x \frac{k_1 + k_2}{m_1}
+ \frac{k_2 R \sin\theta}{m_1}
- g + \frac{k_2 L }{m_1} \\
\ddot{\theta} &=
%1/(m_2 R^2) (- k_2 ( L + R \sin\theta - x ) R \cos\theta - m_2 g R \cos\theta)
- \inv{R}\left( \frac{k_2}{m_2} \left( L + R \sin\theta - x \right) +g \right) \cos\theta
.
\end{align}
\end{subequations}
%
Just like any other coupled pendulum system, this one is non-linear.  There is no obvious way to solve this in closed form, but we could determine a solution in the neighborhood of a point \((x, \theta) = (x_0, \theta_0)\).  Let us switch our dynamical variables to ones that express the deviation from the initial point \(\delta x = x - x_0\), and \(\delta \theta = \theta - \theta_0\), with \(u = (\delta x)'\), and \(v = (\delta \theta)'\).  Our system then takes the form
%
\begin{subequations}
\label{eqn:hoopSpring:4}
\begin{align}
u' &= f(x,\theta) =
-x \frac{k_1 + k_2}{m_1}
+ \frac{k_2 R \sin\theta}{m_1}
- g + \frac{k_2 L }{m_1} \\
v' &= g(x,\theta) =
- \inv{R}\left( \frac{k_2}{m_2} \left( L + R \sin\theta - x \right) +g \right) \cos\theta \\
(\delta x)' &= u \\
(\delta \theta)' &= v
.
\end{align}
\end{subequations}
%
We can use a first order Taylor approximation of the form \(f(x, \theta) = f(x_0, \theta_0) + f_x(x_0,\theta_0) (\delta x) + f_\theta(x_0,\theta_0) (\delta \theta)\).  So, to first order, our system has the approximation
%
\begin{subequations}
\label{eqn:hoopSpring:5}
\begin{align}
u' &=
-x_0 \frac{k_1 + k_2}{m_1}
+ \frac{k_2 R \sin\theta_0}{m_1} \\
&\quad - g + \frac{k_2 L }{m_1}
-(\delta x) \frac{k_1 + k_2}{m_1}
+ \frac{k_2 R \cos\theta_0}{m_1} (\delta \theta)
\\
v' &=
- \inv{R}\left( \frac{k_2}{m_2} \left( L + R \sin\theta_0 - x_0 \right) +g \right) \cos\theta_0
+ \frac{k_2 \cos\theta_0}{m_2 R} (\delta x) \\
&\quad - \inv{R}\left( \frac{k_2}{m_2} \left( \left( L - x_0 \right) \sin\theta_0 + R \right) + g \sin\theta_0 \right) (\delta \theta)
\\
(\delta x)' &= u \\
(\delta \theta)' &= v
.
\end{align}
\end{subequations}
%
This would be tidier in matrix form with \(\Bx = (u, v, \delta x, \delta \theta)\)
%
\begin{align}
\label{eqn:hoopSpring:6}
\Bx' &=
\begin{bmatrix}
-x_0 \frac{k_1 + k_2}{m_1}
+ \frac{k_2 R \sin\theta_0}{m_1}
- g + \frac{k_2 L }{m_1} \\
- \inv{R}\left( \frac{k_2}{m_2} \left( L + R \sin\theta_0 - x_0 \right) +g \right) \cos\theta_0 \\
0 \\
0
\end{bmatrix} \\
&\quad +
\begin{bmatrix}
0 & 0 &
-\frac{k_1 + k_2}{m_1} & \frac{k_2 R \cos\theta_0}{m_1} \\
0 & 0 &
\frac{k_2 \cos\theta_0}{m_2 R} &
- \inv{R}\left( \frac{k_2}{m_2} \left( \left( L - x_0 \right) \sin\theta_0 + R \right) + g \sin\theta_0 \right) \\
1 & 0 & 0 & 0 \\
0 & 1 & 0 & 0 \\
\end{bmatrix}\Bx
.
\end{align}
%
This reduces the problem to the solutions of first order equations of the form
%
\begin{align}\label{eqn:hoopSpring:7}
\Bx' = \Ba
+ \begin{bmatrix}
0 & A \\
I & 0
\end{bmatrix}\Bx = \Ba + \BB \Bx
\end{align}
%
where \(\Ba\), and \(A\) are constant matrices.  Such a matrix equation has the solution
%
\begin{align}\label{eqn:hoopSpring:8}
\Bx = e^{B t} \Bx_0 + (e^{Bt} - I) B^{-1} \Ba,
\end{align}
%
but the zeros in \(B\) should allow the exponential and inverse to be calculated with less work.  That inverse is readily verified to be
%
\begin{align}\label{eqn:hoopSpring:9}
B^{-1} =
\begin{bmatrix}
0 & I \\
A^{-1} & 0
\end{bmatrix}.
\end{align}
%
It is also not hard to show that
%
\begin{subequations}
\label{eqn:hoopSpring:10}
\begin{align}
B^{2n} =
\begin{bmatrix}
A^n & 0 \\
0 & A^n
\end{bmatrix} \\
B^{2n+1} =
\begin{bmatrix}
0 & A^{n+1} \\
A^n & 0
\end{bmatrix}.
\end{align}
\end{subequations}
%
Together this allows for the power series expansion
%
\begin{align}\label{eqn:hoopSpring:11}
e^{Bt} &=
\begin{bmatrix}
\cosh(t \sqrt{A}) & \sinh(t \sqrt{A}) \\
\sinh(t \sqrt{A}) \inv{\sqrt{A}} & \cosh(t \sqrt{A})
\end{bmatrix}.
\end{align}
%
All of the remaining sub matrix expansions should be straightforward to calculate provided the eigenvalues and vectors of \(A\) are calculated.  Specifically, suppose that we have
%
\begin{align}\label{eqn:hoopSpring:12}
A = U
\begin{bmatrix}
\lambda_1 & 0 \\
0 & \lambda_2
\end{bmatrix}
U^{-1}.
\end{align}
%
Then all the perhaps non-obvious functions of matrices expand to just
\begin{subequations}
\label{eqn:hoopSpring:13}
\begin{align}
A^{-1} &= U
\begin{bmatrix}
\lambda_1^{-1} & 0 \\
0 & \lambda_2^{-1}
\end{bmatrix}
U^{-1} \\
\sqrt{A} &= U
\begin{bmatrix}
\sqrt{\lambda_1} & 0 \\
0 & \sqrt{\lambda_2}
\end{bmatrix}
U^{-1} \\
\cosh(t \sqrt{A}) &= U
\begin{bmatrix}
\cosh( t \sqrt{\lambda_1} ) & 0 \\
0 & \cosh( t \sqrt{\lambda_2} )
\end{bmatrix}
U^{-1} \\
\sinh(t \sqrt{A}) &= U
\begin{bmatrix}
\sinh( t \sqrt{\lambda_1} ) & 0 \\
0 & \sinh( t \sqrt{\lambda_2} )
\end{bmatrix}
U^{-1} \\
\sinh(t \sqrt{A}) \inv{\sqrt{A}} &= U
\begin{bmatrix}
\sinh( t \sqrt{\lambda_1} )/\sqrt{\lambda_1} & 0 \\
0 & \sinh( t \sqrt{\lambda_2} )/\sqrt{\lambda_2}
\end{bmatrix}
U^{-1}.
\end{align}
\end{subequations}
%
An interesting question would be how are the eigenvalues and eigenvectors changed with each small change to the initial position \(\Bx_0\) in phase space.  Can these be related to each other?
