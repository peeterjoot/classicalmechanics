%
% Copyright � 2012 Peeter Joot.  All Rights Reserved.
% Licenced as described in the file LICENSE under the root directory of this GIT repository.
%
%
%\chapter{Compare some wave equation's and their Lagrangians}
\index{wave equation}
\label{chap:waveLagrangian}
%\date{ Dec 02, 2008.  waveLagrangian.tex }
%
\section{Motivation}
%
Compare the Lagrangians for the classical wave equation of a vibrating string/film with the wave equation Lagrangian for electromagnetism and Quantum mechanics.

Observe the similarities and differences, and come back to this later after grasping some of the concepts of Field energy and momentum (energy in vibration and electromagnetism and momentum in quantum mechanics).  Do the ideas of field momentum carry in quantum have equivalents in electromagnetism?
%
\section{Vibrating object equations}
%
\subsection{One dimensional wave equation}
%
\citep{goldstein1951cm} does a nice derivation of the one dimensional wave
equation Lagrangian, using a limiting argument applied to an infinite
sequence of connected masses on springs.
%
\begin{equation}\label{eqn:wave_lagrangian:oneDimensionalWaveLagrangian}
\LL = \inv{2} \left(\mu \left(\PD{t}{\eta}\right)^2 - Y \left(\PD{x}{\eta}\right)^2 \right)
\end{equation}
%
here \(\eta\) was the displacement from the equilibrium position, \(\mu\) is the mass line density and \(Y\) is Young's modulus.
%
Taking derivatives confirms that this is the correct form.  The Euler-Lagrange
equations for this equation are:
%
\begin{equation}\label{eqn:waveLagrangian:20}
\begin{aligned}
\PD{\eta}{\LL} &= \PD{t}{} \PD{\PD{t}{\eta}}{\LL} +\PD{x}{} \PD{\PD{x}{\eta}}{\LL} \\
0 &= \PD{t}{} \mu \PD{t}{\eta} -\PD{x}{} Y \PD{x}{\eta} \\
\end{aligned}
\end{equation}
%
Which has the expected form
%
\begin{equation}\label{eqn:waveLagrangian:40}
\begin{aligned}
\mu \PDsq{t}{\eta} - Y \PDsq{x}{\eta} &= 0 \\
\end{aligned}
\end{equation}
%
\subsection{Higher dimension wave equation}
%
For a string or film or other wavy material with more degrees of freedom than a string with back and forth motion one can guess the Lagrangian from \eqnref{eqn:wave_lagrangian:oneDimensionalWaveLagrangian}.
%
\begin{equation}\label{eqn:wave_lagrangian:moreDimensionalWaveLagrangian}
\LL = \inv{2} \left(\mu \left(\PD{t}{\eta}\right)^2 - Y \sum_i \left(\PD{x^i}{\eta}\right)^2 \right)
\end{equation}
%
Calculating the Euler-Lagrange equations gives
\begin{equation}\label{eqn:waveLagrangian:60}
\begin{aligned}
\PD{\eta}{\LL} &= \PD{t}{} \PD{\PD{t}{\eta}}{\LL} +\sum_i \PD{x^i}{} \PD{\PD{x^i}{\eta}}{\LL} \\
0 &= \PD{t}{} \mu \PD{t}{\eta} - \sum_i \PD{x^i}{} Y \PD{x^i}{\eta} \\
\end{aligned}
\end{equation}
%
Which also has the expected form
%
\begin{equation}\label{eqn:waveLagrangian:80}
\begin{aligned}
\mu \PDsq{t}{\eta} - Y \sum_i \PDsq{x^i}{\eta} &= 0 \\
\end{aligned}
\end{equation}
%
\section{Electrodynamics wave equation}
%
From \eqnref{eqn:wave_lagrangian:moreDimensionalWaveLagrangian} one can guess the Lagrangian for the electrodynamic potential wave equations.  Maxwell's equation in potential form are:
%
\begin{equation}\label{eqn:wave_lagrangian:maxwellPotential}
\begin{aligned}
\grad^2 A &= J/\epsilon_0 c
\end{aligned}
\end{equation}
%
Which has the following split into four scalar equations
\begin{equation}\label{eqn:waveLagrangian:100}
\begin{aligned}
\grad^2 A^\mu \gamma_\mu &= J^\mu \gamma_\mu/\epsilon_0 c \\
\grad^2 A^\mu &= J^\mu /\epsilon_0 c
\end{aligned}
\end{equation}
%
For the \(A^\mu\) coordinate try the Lagrangian
%
\begin{equation}\label{eqn:waveLagrangian:120}
\begin{aligned}
\LL
&= \sum_\nu \inv{2} (\gamma^\nu)^2 \left( \PD{x^\nu}{A^\mu} \right)^2 + J^\nu A^\nu / \epsilon_0 c \\
&= \sum_\nu \inv{2} (\gamma^\nu)^2 \left(\partial_{\nu}{A^\mu}\right)^2 + J^\nu A^\nu / \epsilon_0 c \\
\end{aligned}
\end{equation}
%
With evaluation of the Euler-Lagrange equations we have
%
\begin{equation}\label{eqn:waveLagrangian:140}
\begin{aligned}
\PD{A^\mu}{\LL} &= \sum_\alpha \partial_{\alpha} \PD{(\partial{\alpha}{A^\mu})}{\LL} \\
\implies \\
J^\mu/\epsilon_0 c
&= \sum \partial_\alpha (\gamma^\alpha)^2 \partial_{\alpha}{A^\mu} \\
&= \partial^\alpha \partial_{\alpha}{A^\mu} \\
&= \grad^2 A^\mu \\
\end{aligned}
\end{equation}
%
Which recovers Maxwell's equation.  Having done that the Lagrangian can be tidied slightly introducing the spacetime gradient:
%
\begin{equation}\label{eqn:wave_lagrangian:potentialLagrangianWithGrad}
\begin{aligned}
\LL &= \inv{2} \left( \grad {A^\alpha} \right)^2 + J^\alpha A^\alpha / \epsilon_0 c
\end{aligned}
\end{equation}
%
\subsection{Comparing with complex (bivector) version of Maxwell Lagrangian}
Previously, in \bookchapcite{PJMaxwellLagrangian}{phy354} and \bookchapcite{PJMaxwellLagrangian}{phy354}, Maxwell's equation
%
\begin{equation}\label{eqn:wave_lagrangian:maxwell}
\begin{aligned}
\grad (\grad \wedge A) &= J/ \epsilon_0 c
\end{aligned}
\end{equation}
%
was seen as the result of evaluating the Lagrangian
%
\begin{equation}\label{eqn:wave_lagrangian:maxlag}
\begin{aligned}
\LL &= -\frac{\epsilon_0 c}{2} (\grad \wedge A)^2 + J \cdot A
\end{aligned}
\end{equation}
%
\Eqnref{eqn:wave_lagrangian:maxwell} with the gauge condition \(\grad \cdot A = 0\)
is where we get the potential form \eqnref{eqn:wave_lagrangian:maxwellPotential} from.
%
For comparison it should be possible to reconcile this with
\eqnref{eqn:wave_lagrangian:potentialLagrangianWithGrad}.  We can multiply by \((\gamma_\alpha)^2\), which is \((\pm 1)\) dependent on \(\alpha\), as well as multiply by \(\epsilon_0 c\)
%
\begin{equation}\label{eqn:waveLagrangian:160}
\begin{aligned}
\LL &= \frac{\epsilon_0 c}{2} \grad {A^\alpha} \grad {A_\alpha} + J^\alpha A_\alpha \\
\end{aligned}
\end{equation}
%
No sum need be implied here, but since the field variables are independent we can sum them without changing the field equations.  So, instead of having four independent Lagrangians, we are now left with a (sums now implied) single density that can be evaluated for each of the potential coordinate variables:
%
\begin{equation}\label{eqn:waveLagrangian:180}
\begin{aligned}
\LL &= \frac{\epsilon_0 c}{2} \grad {A^\alpha} \grad {A_\alpha} + J \cdot A \\
\end{aligned}
\end{equation}
%
This is looking more like \eqnref{eqn:wave_lagrangian:maxlag} now.  It is expected that the gauge condition can be used to complete the reconciliation.  However, I have had trouble actually doing this, despite the fact that both Lagrangians appear to correctly
lead to equivalent results.

Also notable perhaps is a comparison to the four potential Lagrangian in Goldstein:
%
\begin{equation}\label{eqn:waveLagrangian:200}
\LL =
-\inv{16\pi} \sum_{\mu,\nu} \left(
\PD{x_\nu}{A_\mu} - \PD{x_\mu}{A_\nu}
\right)^2
-\inv{8\pi}
\sum_\mu
\left(
\PD{x_\mu}{A_\mu}
\right)^2
+
\sum_\mu \frac{j_\mu A_\mu}{c}
\end{equation}
%
This one is considerably more complex looking, and
it should be possible to see how exactly this is related to the wave
equation guessed by comparison to the vibrating string.
%
\section{Quantum Mechanics}
%
\subsection{Non-relativistic case}
%
The non-relativistic Lagrangian given by Goldstein (problem 11.3) is
%
\begin{equation}\label{eqn:waveLagrangian:220}
\LL = \frac{\Hbar^2}{2m}
(\spacegrad \psi) \cdot (\spacegrad \psi^\conj) + V \psi \psi^\conj + {i \Hbar} \left( \psi \partial_t \psi^\conj - \psi^\conj \partial_t \psi \right)
\end{equation}
%
Again we see the square of the spatial gradient so we expect a (spatial) Laplacian
in the field equation, which one has:
%
\begin{equation}\label{eqn:waveLagrangian:240}
\left( \frac{-\Hbar^2}{2m} \spacegrad^2 + V \right) \psi = i \Hbar \PD{t}{\psi}
\end{equation}
%
\subsection{Relativistic case. Klein-Gordon}
%
The Klein-Gordon Lagrangian is
\begin{equation}\label{eqn:waveLagrangian:260}
\begin{aligned}
\LL
&= -(\grad \psi) \cdot (\grad \psi^\conj) + \frac{m^2 c^2}{\Hbar^2} \psi \psi^\conj \\
\end{aligned}
\end{equation}
%
from which we can recover the Klein-Gordon scalar wave equation which applies to a
specific subset of quantum phenomena (what exactly?)
%
\begin{equation}\label{eqn:waveLagrangian:280}
\begin{aligned}
\left(\frac{\Hbar^2}{2m} \grad^2 + \inv{2} m c^2\right) \psi = 0 \\
\end{aligned}
\end{equation}
%
\subsection{Dirac wave equation}
%
The Dirac wave equation, for vector wave function \(\psi\) can be formally
obtained by
taking vector roots of the scalar operators in the Klein-Gordon equation
to yield:
%
\begin{equation}\label{eqn:waveLagrangian:300}
i \Hbar \grad \psi = \pm m c \psi
\end{equation}
%
The Lagrangian for this field equation is
%
\begin{equation}\label{eqn:waveLagrangian:320}
\LL = mc \overbar{\psi}\psi - {\inv{2}i\Hbar}(\overbar{\psi}\gamma^\mu (\partial_\mu\psi) - (\partial_\mu\overbar{\psi})\gamma^\mu \psi)
\end{equation}
%
Where \(\overbar{\psi} = \gamma_0 \tilde{\psi}\), and \(\tilde{\psi}\) is the reversed field spinor.
%
\section{Summary comparison of all the second order wave equations}
%
\begin{itemize}
%
\item Vibration wave equation.
%
\begin{equation}\label{eqn:waveLagrangian:340}
\begin{aligned}
\LL &= \mu \left( \PD{t}{\eta} \right)^2 - Y \left( \spacegrad \eta \right)^2 \\
0 &= \mu \PDsq{t}{\eta} - Y \spacegrad^2 {\eta}
\end{aligned}
\end{equation}
%
\item Maxwell wave equation.
%
\begin{equation}\label{eqn:waveLagrangian:360}
\begin{aligned}
\LL &= \inv{2} \left( \grad {A^\alpha} \right)^2 + J^\alpha A^\alpha / \epsilon_0 c \\
\grad^2 A^\alpha &= J^\alpha/\epsilon_0 c
\end{aligned}
\end{equation}
%
\item Schr\"{o}dinger non-relativistic wave equation.
%
\begin{equation}\label{eqn:waveLagrangian:380}
\begin{aligned}
\LL &= \frac{\Hbar^2}{2m}
(\spacegrad \psi) \cdot (\spacegrad \psi^\conj) + V \psi \psi^\conj + {i \Hbar} \left( \psi \partial_t \psi^\conj - \psi^\conj \partial_t \psi \right) \\
\left( \frac{-\Hbar}{2m} \spacegrad^2 + V \right) \psi &= \Hbar i \PD{t}{\psi}
\end{aligned}
\end{equation}
%
\item Klein-Gordon wave equation.
%
\begin{equation}\label{eqn:waveLagrangian:400}
\begin{aligned}
\LL &= -(\grad \psi) \cdot (\grad \psi^\conj) + \frac{m^2 c^2}{\Hbar^2} \psi \psi^\conj \\
-\grad^2 \psi &= \frac{m^2 c^2}{\Hbar^2} \psi
\end{aligned}
\end{equation}
%
\end{itemize}
