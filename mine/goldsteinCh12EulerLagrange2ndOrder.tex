%
% Copyright © 2012 Peeter Joot.  All Rights Reserved.
% Licenced as described in the file LICENSE under the root directory of this GIT repository.
%
\makeoproblem{Euler-Lagrange equations for second order systems.}{gold:ch2:pr3}{\citep{goldstein1951cm} 2.4}{
For \(f = f( y, \ydot, \yddot, x )\), find the equations for extreme values of
%
\begin{equation*}
I = \int_a^b f dx.
\end{equation*}
}
%
\makeanswer{gold:ch2:pr3}{
%
Here we want \(y\) and \(\ydot\) fixed at the end points.  Following the first derivative derivation write the
functions in terms of the desired extremum functions plus a set of arbitrary functions:
%
\begin{equation}\label{eqn:goldsteinCh12:860}
\begin{aligned}
y( x, \alpha ) &= y( x, 0 ) + \alpha n(x) \\
\ydot( x, \alpha ) &= \ydot( x, 0 ) + \alpha m(x).
\end{aligned}
\end{equation}
%
Here we specify that these arbitrary variational functions vanish at the endpoints:
%
\begin{equation}\label{eqn:goldsteinCh12:1440}
n(a) = n(b) = m(a) = m(b) = 0.
\end{equation}
%
The functions \(y(x, 0)\), and \(\ydot(x, 0)\) are the functions we are looking for as solutions to the min/max problem.
%
Calculating derivatives we have:
%
\begin{equation}\label{eqn:goldsteinCh12:1460}
\frac{dI}{d\alpha} =
\int \left(
\PD{y}{f} \PD{\alpha}{y}
+\PD{\ydot}{f} \PD{\alpha}{\ydot}
+\PD{\yddot}{f} \PD{\alpha}{\yddot}
\right) d x.
\end{equation}
%
Assuming sufficient continuity look at the second term where we have:
%
\begin{equation}\label{eqn:goldsteinCh12:880}
\begin{aligned}
\PD{\alpha}{\ydot}
&= \PD{\alpha}{} \PD{x}{y} \\
&= \PD{x}{} \PD{\alpha}{y} \\
&= \PD{x}{} n(x) \\
&= \frac{d}{ d x} n(x) \\
&= \frac{d}{ d x} \PD{\alpha}{y} .
\end{aligned}
\end{equation}
%
Similarly for the third term we have:
%
\begin{equation}\label{eqn:goldsteinCh12:1480}
\PD{\alpha}{\ydot} = \frac{d}{ d x} \PD{\alpha}{\ydot},
\end{equation}
%
\begin{equation}\label{eqn:goldsteinCh12:1500}
\frac{dI}{d\alpha} =
\int \PD{y}{f} \PD{\alpha}{y} d x +
\mathLabelBox{\PD{\ydot}{f} \frac{d}{ d x} \PD{\alpha}{y}}{\(u v' = (u v)' - u' v \)}
d x
+\PD{\yddot}{f} \frac{d}{ d x} \PD{\alpha}{\ydot} d x.
\end{equation}
%
Now integrating by parts:
\begin{equation}\label{eqn:goldsteinCh12:900}
\begin{aligned}
\frac{dI}{d\alpha}
&=
 \int \PD{y}{f} \PD{\alpha}{y} d x
+\int \PD{\ydot}{f} \frac{d}{ d x} \PD{\alpha}{y} d x
+\int \PD{\yddot}{f} \frac{d}{ d x} \PD{\alpha}{\ydot} d x \\
\frac{dI}{d\alpha} &=
 \int \PD{y}{f} \PD{\alpha}{y} d x
+
%\Bigl(
\PD{\ydot}{f}
\mathLabelBox{
\evalrange{
\PD{\alpha}{y}
}{a}{b}
}{\(n(x)\)}
%\Bigr)_a^b
- \int \PD{\alpha}{y} \frac{d}{ d x} \PD{\ydot}{f} d x \\
&\quad +
%\Bigl(
\PD{\yddot}{f}
\mathLabelBox{
\evalrange{
\PD{\alpha}{\ydot}
}{a}{b}
}{\(m(x)\)}
%\Bigr)_a^b
-\int \PD{\alpha}{\ydot} \frac{d}{ d x} \PD{\yddot}{f} d x.
\end{aligned}
\end{equation}
%
Because \(m(a) = m(b) = n(a) = n(b)\) the non-integral terms are all zero, leaving:
%
\begin{equation}\label{eqn:goldsteinCh12:920}
\frac{dI}{d\alpha} =
  \int \PD{y}{f} \PD{\alpha}{y} d x
- \int \PD{\alpha}{y} \frac{d}{ d x} \PD{\ydot}{f} d x
- \int \PD{\alpha}{\ydot} \frac{d}{ d x} \PD{\yddot}{f} d x.
\end{equation}
%
Now consider just this last integral, which we can again integrate by parts:
\begin{equation}\label{eqn:goldsteinCh12:940}
\begin{aligned}
\int \PD{\alpha}{\ydot} \frac{d}{ d x} \PD{\yddot}{f} d x
&= \int
\mathLabelBox{\frac{d}{dx} \PD{\alpha}{y}}{\(u'\)}
\mathLabelBox{\frac{d}{ d x} \PD{\yddot}{f}}{\(v\)}
d x \\
&=
\mathLabelBox{\PD{\alpha}{y}}{\(n(x)\)}
{\frac{d}{ d x} \PD{\yddot}{f}}
{\Bigg\vert}_a^b
-\int \PD{\alpha}{y} \frac{d}{dx} {\frac{d}{ d x} \PD{\yddot}{f}} d x \\
&=
-\int \PD{\alpha}{y} \frac{d^2}{dx^2} \PD{\yddot}{f} d x .
\end{aligned}
\end{equation}
%
This gives:
\begin{equation}\label{eqn:goldsteinCh12:960}
\begin{aligned}
\frac{dI}{d\alpha} &=
  \int \PD{y}{f} \PD{\alpha}{y} d x
- \int \PD{\alpha}{y} \frac{d}{ d x} \PD{\ydot}{f} d x
+ \int \PD{\alpha}{y} \frac{d^2}{dx^2} \PD{\yddot}{f} d x \\
\frac{dI}{d\alpha}
&= \int d x \PD{\alpha}{y} \left( \PD{y}{f} - \frac{d}{ d x} \PD{\ydot}{f} + \frac{d^2}{dx^2} \PD{\yddot}{f} \right) \\
&= \int d x n(x) \left( \PD{y}{f} - \frac{d}{ d x} \PD{\ydot}{f} + \frac{d^2}{dx^2} \PD{\yddot}{f} \right).
\end{aligned}
\end{equation}
%
So, if we want this derivative to equal zero for any \(n(x)\) we require the inner quantity to by zero:
%
\begin{equation}
\PD{y}{f} - \frac{d}{ d x} \PD{\ydot}{f} + \frac{d^2}{dx^2} \PD{\yddot}{f} = 0.
\end{equation}
%
Question.  Goldstein writes this in total differential form instead of a derivative:
%
\begin{equation}\label{eqn:goldsteinCh12:980}
\begin{aligned}
dI &= \frac{dI}{d\alpha} d\alpha \\
&= \int d x \left(\PD{\alpha}{y} d \alpha\right) \left( \PD{y}{f} - \frac{d}{ d x} \PD{\ydot}{f} + \frac{d^2}{dx^2} \PD{\yddot}{f} \right) .
\end{aligned}
\end{equation}
%
and then calls this quantity \(\PD{\alpha}{y} d \alpha = \delta y\), the variation of \(y\).  There must be a mathematical subtlety which motivates this
but it is not clear to me what that is.  Since the variational calculus texts go a different route, with norms on functional spaces and so forth, perhaps
understanding that motivation is not worthwhile.  In the end, the conclusion is the same, namely that the inner expression must equal zero for the extremum
condition.
}
