%
% Copyright � 2012 Peeter Joot.  All Rights Reserved.
% Licenced as described in the file LICENSE under the root directory of this GIT repository.
%
%
%\chapter{Potential and Kinetic Energy}
\index{potential energy}
\index{kinetic energy}
%\label{chap:pe}
\label{chap:potentialEnergy}
%\author{Peeter Joot \quad peeterjoot@protonmail.com}
%\date{ July 28, 2008.  pe.tex }
%
\section{Potential and Kinetic Energy.}
%
Attempting some Lagrangian calculation problems I found I got all the signs of my potential energy terms wrong.  Here is a quick step back to basics to clarify for myself what the definition of potential energy is, and thus implicitly determine the correct signs.

Starting with kinetic energy, expressed in vector form:
%
\begin{equation}\label{eqn:potentialEnergy:220}
K
= \inv{2} m \Br' \cdot \Br'
= \inv{2} \Bp \cdot \Br',
\end{equation}
%
one can calculate the rate of change of that energy:
%
\begin{equation}\label{eqn:potentialEnergy:20}
\begin{aligned}
\frac{dK }{dt}
&= \inv{2} \left(\Bp' \cdot \Br' + \Bp \cdot \Br''\right) \\
&= \inv{2} \left(\Bp' \cdot \Br' + \Br' \cdot \Bp'\right) \\
&= \Bp' \cdot \Br'.
\end{aligned}
\end{equation}
%
Note that the mass has been assumed constant above.

Integrating this time rate of change of kinetic energy produces a force line integral:
%
\begin{equation}\label{eqn:potentialEnergy:40}
\begin{aligned}
K_2 - K_1
&= \int_{t1}^{t2} \frac{dK}{dt} dt \\
&= \int_{t1}^{t2} \Bp' \cdot \Br' dt \\
&= \int_{t1}^{t2} \Bp' \cdot \frac{d\Br'}{dt} dt \\
&= \int_{\Br_1}^{\Br_2} \BF \cdot d\Br.
\end{aligned}
\end{equation}
%
For the path integral to depend on only the end points or the corresponding end times requires a conservative force that can be expressed as a gradient.
Let us say that \(\BF = \grad f\), then integrating:
%
\begin{equation}\label{eqn:potentialEnergy:60}
\begin{aligned}
K_2 - K_1
&= \int_{\Br_1}^{\Br_2} \BF \cdot d\Br \\
&= \int_{\Br_1}^{\Br_2} \grad f \cdot d\Br \\
&= {\text{limit}}_{\epsilon \rightarrow 0} \int_{\Br_1}^{\Br_1 + \epsilon\rcap}
   \left(\rcap \frac{f(\Br + \epsilon \rcap)}{\epsilon}\right)
      \cdot d\Br \\
&= \text{handwaving} \\
&= f(\Br_2) - f(\Br_1).
\end{aligned}
\end{equation}
%
Assembling the quantities for times \(1\), and \(2\), we have
%
\begin{equation}
K_2 -f(\Br_2) = K_1 - f(\Br_1) = \text{constant}.
\end{equation}
%
This constant is what we give the name Energy.  The quantities \(-f(\Br_i)\) we label potential energy \(V_i\), and finally write the total energy as the sum of the kinetic and potential energies for a particle at a point in time and space:
%
\begin{equation}
K_2 + V_2 = K_1 + V_1 = E,
\end{equation}
\begin{equation}
\BF = -\grad V.
\end{equation}
%
\subsection{Work with a specific example.  Newtonian gravitational force.}
%
Take the gravitational force:
%
\begin{equation}
F = -\frac{GmM}{r^2} \rcap.
\end{equation}
%
The rate of change of kinetic energy with respect to such a force (FIXME: think though signs ... with or against?), is:
%
\begin{equation}\label{eqn:potentialEnergy:80}
\begin{aligned}
\frac{dK}{dt}
&= \Bp' \cdot \Br' \\
&= -\frac{GmM}{r^2} \rcap \cdot \frac{d\Br}{dt} \\
&= -\frac{GmM}{r^3} \Br \cdot \frac{d\Br}{dt}.
\end{aligned}
\end{equation}
%
The vector dot products above can be eliminated with the standard trick:
%
\begin{equation}\label{eqn:potentialEnergy:100}
\begin{aligned}
\frac{dr^2}{dt}
&= \frac{\Br \cdot \Br}{dt} \\
&= 2 \frac{d\Br}{dt} \cdot \Br.
\end{aligned}
\end{equation}
%
Thus,
\begin{equation}\label{eqn:potentialEnergy:120}
\begin{aligned}
\frac{dK}{dt}
&= -\frac{GmM }{2r^3} \frac{dr^2}{dt} \\
&= -\frac{GmM }{r^2} \frac{dr}{dt} \\
&= \frac{d}{dt} \left( \frac{GmM }{r} \right).
\end{aligned}
\end{equation}
%
This can be integrated to find the kinetic energy difference associated with a change of position in a gravitational field:
%
\begin{equation}\label{eqn:potentialEnergy:140}
\begin{aligned}
K_2 - K_1
&= \int_{t_1}^{t_2} \frac{d}{dt} \left( \frac{GmM }{r} \right) dt \\
&= GmM \left( \inv{r_2} - \inv{r_1} \right).
\end{aligned}
\end{equation}
%
Rearranging
%
\begin{equation}\label{eqn:potentialEnergy:160}
K_2 - \frac{GmM}{r_2} = K_1 - \frac{GmM}{r_1} = E.
\end{equation}
%
Taking gradients of this negative term:
%
\begin{equation}\label{eqn:potentialEnergy:180}
\begin{aligned}
\grad \left( - \frac{GmM}{r} \right)
&= \rcap \frac{\partial}{\partial r} \left( - \frac{GmM}{r} \right) \\
&= \rcap \frac{GmM}{r^2},
\end{aligned}
\end{equation}
%
returns the negation of the original force, so if we write \(V = -GmM/r\), it implies the force is:
%
\begin{equation}
\BF = -\grad V.
\end{equation}
%
By this example we see how one arrives at the negative sign convention for the potential energy.  Our
Lagrangian in a gravitational field is thus:
%
\begin{equation}
L = \inv{2} m \Bv^2 + \frac{GmM}{r}.
\end{equation}
%
Now, we have seen strictly positive terms \(mgh\) in the Lagrangian in the Tong and Goldstein examples.  We can account for this by
Taylor expanding this potential in the vicinity of the surface \(R\) of the Earth:
%
\begin{equation}\label{eqn:potentialEnergy:200}
\begin{aligned}
\frac{GmM}{r}
&= \frac{GmM}{R + h} \\
&= \frac{GmM}{R(1 + h/R)} \\
&\approx \frac{GmM}{R} (1 - h/R).
\end{aligned}
\end{equation}
%
The Lagrangian is thus:
%
\begin{equation}\label{eqn:potentialEnergy:240}
L \approx \inv{2} m \Bv^2 + \frac{GmM}{R} -\frac{GmM}{R^2} h.
\end{equation}
%
but the constant term will not change the EOM, so can be dropped from the Lagrangian, and with \(g=\frac{GM}{R^2}\) we have:
%
\begin{equation}\label{eqn:pe:measuredup}
L' = \inv{2} m \Bv^2 - g m h.
\end{equation}
%
Here the potential term of the Lagrangian is negative, but in the Goldstein and Tong examples the reference point is up, and the height is measured down
from that point.  Put another way, if the total energy is
%
\begin{equation}\label{eqn:potentialEnergy:260}
E = V_0.
\end{equation}
%
when the mass is unmoving in the air, and then drops gaining Kinetic energy, an unchanged total energy means that potential energy must be counted as lost, in proportion to the distance fallen:
%
\begin{equation}\label{eqn:potentialEnergy:280}
E = V_0 = K_1 + V_1 = \inv{2} m \Bv^2 - m g h.
\end{equation}
%
So, one can write
%
\begin{equation}\label{eqn:potentialEnergy:300}
V = -m g h,
\end{equation}
%
and
\begin{equation}\label{eqn:pe:measureddown}
L' = \inv{2} m \Bv^2 + g m h.
\end{equation}
%
BUT.  Here the height \(h\) is the distance fallen from the reference point, compared to \eqnref{eqn:pe:measuredup}, where \(h\) was the distance measured up from the surface of the Earth (or other convenient local point where the gravitational field can be linearly approximated)!

Care must be taken here because it is all too easy to get the signs wrong blindly plugging into the equations without considering where they come from and how exactly they are defined.
