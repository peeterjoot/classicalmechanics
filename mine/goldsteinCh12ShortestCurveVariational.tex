%
% Copyright © 2012 Peeter Joot.  All Rights Reserved.
% Licenced as described in the file LICENSE under the root directory of this GIT repository.
%
\makeoproblem{Shortest curve variational problem.}{gold:ch2:pr1}{\citep{goldstein1951cm} 2.1}{
Prove that the shortest length curve between two points in space is a straight line.
}
%
\makeanswer{gold:ch2:pr1}{
In a first attempt of this I used:
%
\begin{equation}\label{eqn:goldsteinCh12:1180}
ds = \sqrt{ 1 + (dy/dx)^2 + (dz/dx)^2 } dx.
\end{equation}
%
Application of the Euler-Lagrange equations does show that one ends up with a linear relation between the y and z coordinates, but no mention of x.  Rather than write that up, consider instead a parametrization of the coordinates:
%
\begin{equation}\label{eqn:goldsteinCh12:640}
\begin{aligned}
x &= x_1(\lambda) \\
y &= x_2(\lambda) \\
z &= x_3(\lambda),
\end{aligned}
\end{equation}
%
in terms of this arbitrary parametrization we have a segment length of:
%
\begin{equation}\label{eqn:goldsteinCh12:1200}
ds = \sqrt{ \sum \left(\frac{d x_i}{d\lambda}\right)^2 } d \lambda = f\left(x_i\right) d\lambda.
\end{equation}
%
Application of the Euler-Lagrange equation to \(f\) we have:
%
\begin{equation}\label{eqn:goldsteinCh12:660}
\begin{aligned}
\PD{x_i}{f}
&= 0 \\
&= \frac{d}{d\lambda} \PD{\xdot_i}{} \sqrt{ \sum {\xdot_j}^2 } \\
&= \frac{d}{d\lambda} \frac{ \xdot_i }{\sqrt{ \sum {\xdot_j}^2 }}.
\end{aligned}
\end{equation}
%
Therefore each of these quotients can be equated to a constant:
%
\begin{equation}\label{eqn:goldsteinCh12:680}
\begin{aligned}
\frac{ \xdot_i }{\sqrt{ \sum {\xdot_j}^2 }} &= {c_i}^{-2} \\
{c_i}^2 \xdot_i^2 &= \sum {\xdot_j}^2 \\
({c_i}^2 -1)\xdot_i^2 &= \sum_{j \ne i} {\xdot_j}^2 \\
(1 - {c_i}^2)\xdot_i^2 + \sum_{j \ne i} {\xdot_j}^2 &= 0.
\end{aligned}
\end{equation}
%
This last form shows explicitly that not all of these squared derivative terms can be linearly independent.  In particular, we have a
zero determinant:
%
\begin{equation}\label{eqn:goldsteinCh12:1220}
0 =
\begin{vmatrix}
1 - c_1^2   & 1            & 1         & 1 & \hdots \\
1           & 1 - c_2^2    & 1         & 1 & \vdots \\
1           & 1            & 1 - c_3^2 & 1 & \\
            &              &           & \ddots & \\
            &              &           &        & 1 - {c_n}^2
\end{vmatrix}.
\end{equation}
%
Now, expanding this for a couple specific cases is not too hard.  For \(n=2\) we have:
%
\begin{equation}\label{eqn:goldsteinCh12:700}
\begin{aligned}
0 &= (1 - c_1^2)(1-c_2^2) - 1 \\
c_1^2 + c_2^2 &= c_1^2 c_2^2 \\
c_1^2 &= \frac{c_2^2}{ c_2^2 - 1 } \\
c_2^2 - 1 &= \frac{c_2^2}{ c_1^2 }.
\end{aligned}
\end{equation}
%
This can be substituted back into one our \(c_2^2\) equation:
%
\begin{equation}\label{eqn:goldsteinCh12:720}
\begin{aligned}
({c_2}^2 -1)\xdot_2^2 &= {\xdot_1}^2 \\
\frac{c_2^2}{ c_1^2 } \xdot_2^2 &= {\xdot_1}^2 \\
\pm \frac{c_2}{ c_1 } \xdot_2 &= {\xdot_1} \\
\pm \frac{c_2}{ c_1 } x_2 &= x_1 + \kappa .
\end{aligned}
\end{equation}
%
This is precisely the straight line that was desired, but we have setup for proving that consideration of all path variations from two points
in \R{N} space has the shortest distance when that path is a straight line.
%
Despite the general setup, I am going to chicken out and show this only for the \R{3} case.  In that case our determinant expands to:
%
\begin{equation}\label{eqn:goldsteinCh12:1240}
c_1^2 + c_2^2 + c_3^2 = c_1^2 c_2^2 c_3^2.
\end{equation}
%
Since not all of the \(\xdot_i^2\) can be linearly independent, one can be eliminated:
%
\begin{equation}\label{eqn:goldsteinCh12:740}
\begin{aligned}
(1 - c_1^2) \xdot_1^2 + \xdot_2^2 + \xdot_3^2 &= 0 \\
(1 - c_2^2) \xdot_2^2 + \xdot_3^2 + \xdot_1^2 &= 0 \\
(1 - c_3^2) \xdot_3^2 + \xdot_1^2 + \xdot_2^2 &= 0.
\end{aligned}
\end{equation}
%
Let us pick \(\xdot_1^2\) to eliminate, and subst 2 into 3:
%
\begin{equation}\label{eqn:goldsteinCh12:760}
\begin{aligned}
%(1 - c_1^2) (-(1 - c_2^2) \xdot_2^2 - \xdot_3^2) + \xdot_2^2 + \xdot_3^2 &= 0 \\
(1 - c_3^2) \xdot_3^2 + (-(1 - c_2^2) \xdot_2^2 - \xdot_3^2) + \xdot_2^2 &= 0
\implies \\
%\xdot_2^2 ( 1 - (1 - c_1^2)(1 - c_2^2) ) + \xdot_3^2 ( 1 - (1 - c_1^2) ) &= 0 \\
- c_3^2 \xdot_3^2 + c_2^2 \xdot_2 &= 0 \\
\pm c_3 \xdot_3 &= c_2 \xdot_2 .
\end{aligned}
\end{equation}
%
%Which is, once again a straight line:
%
%\begin{equation*}
%\pm c_3 x_3 = c_2 x_2 + \kappa.
%\end{equation*}
%
Since these equations are symmetric, we can do this for all, with the result:
\begin{equation}\label{eqn:goldsteinCh12:780}
\begin{aligned}
\pm c_3 \xdot_3 &= c_2 \xdot_2 \\
\pm c_3 \xdot_3 &= c_1 \xdot_1 \\
\pm c_2 \xdot_2 &= c_1 \xdot_1 .
\end{aligned}
\end{equation}
%
Since the \(c_i\) constants are arbitrary, then we can for example pick the negative sign for \(\pm c_2\), and the positive for the rest, then add all of these, and scale by two:
%
\begin{equation}\label{eqn:goldsteinCh12:1260}
c_3 \xdot_3 - c_2 \xdot_2 = c_1 \xdot_1,
\end{equation}
%
and integrating:
%
\begin{equation}\label{eqn:goldsteinCh12:1280}
c_3 x_3 - c_2 x_2 = c_1 x_1 + \kappa.
\end{equation}
%
Again, we have the general equation of a line, subject to the desired constraints on the end points.  In the end we did not need to
evaluate the determinant after all, as done in the \R{2} case.
}
%
