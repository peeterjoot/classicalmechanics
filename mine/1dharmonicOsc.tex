%
% Copyright � 2012 Peeter Joot.  All Rights Reserved.
% Licenced as described in the file LICENSE under the root directory of this GIT repository.
%
%
%\chapter{1D forced harmonic oscillator.  Quick solution of non-homogeneous problem}
\index{forced harmonic oscillator}
\label{chap:1dharmonicOsc}
%\blogpage{http://sites.google.com/site/peeterjoot/math2010/1dharmonicOsc.pdf}
%\date{Feb 19, 2010}
%
\section{Motivation}
%
In \citep{brown1954feynman} equation (25) we have a forced harmonic oscillator equation
%
\begin{equation}\label{eqn:1dharmonicOsc:1}
m \ddot{x} + m \omega^2 x = \gamma(t).
\end{equation}
%
The solution of this equation is provided, but for fun lets derive it.
%
\section{Guts}
%
Writing
%
\begin{equation}\label{eqn:1dharmonicOsc:2}
\omega u = \dot{x},
\end{equation}
%
we can rewrite the second order equation as a first order linear system
%
\begin{equation}\label{eqn:1dharmonicOsc:3}
\begin{aligned}
\dot{u} + \omega x &= \gamma(t)/m \omega \\
\dot{x} - \omega u &= 0,
\end{aligned}
\end{equation}
%
Or, with \(X = (u, x)\), in matrix form
%
\begin{equation}\label{eqn:1dharmonicOsc:4}
\begin{aligned}
\dot{X} + \omega
\begin{bmatrix}
0 & 1 \\
-1 & 0
\end{bmatrix}
X
&=
\begin{bmatrix}
\gamma(t)/m \omega \\
0
\end{bmatrix}.
\end{aligned}
\end{equation}
%
The two by two matrix has the same properties as the complex imaginary, squaring to the identity matrix, so the equation to solve is now of the form
%
\begin{equation}\label{eqn:1dharmonicOsc:5}
\begin{aligned}
\dot{X} + \omega i X &= \Gamma.
\end{aligned}
\end{equation}
%
The homogeneous part of the solution is just the matrix
%
\begin{equation}\label{eqn:1dharmonicOsc:28}
\begin{aligned}
X
&= e^{-i \omega t} A \\
&=
\left(
\cos(\omega t)
\begin{bmatrix}
1 & 0 \\
0 & 1
\end{bmatrix}
-
\sin(\omega t)
\begin{bmatrix}
0 & 1 \\
-1 & 0
\end{bmatrix}
\right) A,
\end{aligned}
\end{equation}
%
where \(A\) is a two by one column matrix of constants.  Assuming for the specific solution \(X = e^{-i \omega t} A(t)\), and substituting we have
%
\begin{equation}\label{eqn:1dharmonicOsc:6}
e^{-i \omega t} \dot{A} = \Gamma(t).
\end{equation}
%
This integrates directly, fixing the unknown column vector function \(A(t)\)
%
\begin{equation}\label{eqn:1dharmonicOsc:7}
A(t) = A(0) + \int_0^t e^{i \omega \tau} \Gamma(\tau).
\end{equation}
%
Thus the non-homogeneous solution takes the form
%
\begin{equation}\label{eqn:1dharmonicOsc:8}
X = e^{-i \omega t} A(0) + \int_0^t e^{i \omega (\tau - t)} \Gamma(\tau).
\end{equation}
%
Note that \(A(0) = (\dot{x}_0/\omega, x_0)\).  Multiplying this out, and discarding all but the second row of the matrix product gives \(x(t)\), and Feynman's equation (26) follows directly.
