%
% Copyright � 2012 Peeter Joot.  All Rights Reserved.
% Licenced as described in the file LICENSE under the root directory of this GIT repository.
%
%
%\chapter{Attempts at calculating potential distribution for infinite homogeneous plane}
\index{potential!infinite homogeneous plane}
\label{chap:infiniteSheetPotentials}
%\blogpage{http://sites.google.com/site/peeterjoot2/math2012/infiniteSheetPotentials.pdf}
%\date{Feb 19, 2012}
%
\section{Motivation.}
%
Part of a classical mechanics problem set was to look at what portions of momentum and angular momentum are conserved for various fields.  Since this was also a previous midterm question, I am expecting that some intuition was expected to be used determine the form of the Lagrangians, with not much effort on finding the precise form of the potentials.  Here I try to calculate one such potential explicitly.
%
%Can the potential of 4.2 (infinite homogeneous cylinder) actually be calculated?  I get infinities if I let the length of the cylinder become unbounded.
%
Oddly, I can explicitly calculate the potential for the infinite homogeneous plane if I start with the force and then calculate the potential, but if I start with the potential the integral also diverges?  That does not make any sense, so I am wondering if I have miscalculated.  I have tried a few different ways below, but can not get a non-divergent result if I start from the integral definition of the potential instead of deriving the potential from the force after adding up all the directional force contributions (where I find of course that all the component but the one perpendicular to the plane cancel out).
%
%For both of these my expectation is that you do not actually care about the exact form of the Lagrangian, and just want us to consider potentials that are dependent on the distance from the plane and the radial distance from the center of the cylinder, but I felt like playing and hit trouble doing the calculations determining the precise form of the Lagrangian.
%For the two and three point particle portions of the problems, is your assumption that these particles are fixed in position, or do you really want us to find the components of momentum and angular momentum conservation for the three and four body problems?
%
\section{Forces and potential for an infinite homogeneous plane.}
%
\subsection{Calculating the potential from the force.}
%
For the plane, with \(z\) as the distance from the plane, and \((r,\theta)\) coordinates in the plane, as illustrated in \cref{fig:infiniteSheetPotentials:infiniteSheetPotentialsFig1}.  The gravitational force from an element of mass on the plane is
\imageFigure{../figures/classicalmechanics/infiniteSheetPotentialsFig1}{Coordinate choice for interaction with infinite plane mass distribution.}{fig:infiniteSheetPotentials:infiniteSheetPotentialsFig1}{0.2}
%
\begin{equation}\label{eqn:infiniteSheetPotentials:590}
\begin{aligned}
d\BF
&= -G \sigma m r dr d\theta \frac{z \zcap - r \rcap}{\Abs{z \zcap - r \rcap}^3 } \\
&= -G \sigma m r dr d\theta \frac{z \zcap - r (\Be_1 \cos\theta + \Be_2 \sin\theta) }{(z^2 + r^2)^{3/2} } \\
\end{aligned}
\end{equation}
%
%Here I have used a Clifford Algebra representation of the unit vector \(\rcap\) with \(i = \Be_1 \Be_2\).  Should a more conventional physics notation be desired we can consider this the element \(\Bsigma \cdot d\BF\), and make the substitutions
%
%%\begin{align*}
%%1 &\rightarrow \sigma_0 =
%%\begin{bmatrix}
%%1 & 0 \\
%%0 & 1
%%\end{bmatrix}
%%\\
%\zcap &\rightarrow \Bsigma \cdot \Be_3 = \PauliZ
%\\
%\rcap &\rightarrow \Bsigma \cdot \Be_1 e^{ (\Bsigma \cdot \Be_1) (\Bsigma \cdot \Be_2) \theta} = \PauliX \cos\theta + \PauliY \sin\theta.
%\end{align*}
%
Integrating for the total force on the test mass, noting that the sinusoidal terms vanish when integrated over a \([0, 2 \pi]\) interval, we have
%
\begin{equation}\label{eqn:infiniteSheetPotentials:10}
\BF = - 2 \pi G \sigma m z \zcap \int_0^\infty r dr \inv{(z^2 + r^2)^{3/2} },
\end{equation}
%
a substitution \(r = z \tan \alpha\) gives us
%
\begin{equation}\label{eqn:infiniteSheetPotentials:610}
\begin{aligned}
\BF
&= - 2 \pi G \sigma m z \zcap \int_0^{\pi/2} z \tan \alpha z \sec^2 \alpha d\alpha \inv{z^3 \sec^3 \alpha } \\
&= - 2 \pi G \sigma m \zcap \int_0^{\pi/2} \sin \alpha d\alpha \\
&= 2 \pi G \sigma m \zcap (\cos(\pi/2) - \cos(0)),
\end{aligned}
\end{equation}
%
For
%
\begin{equation}\label{eqn:infiniteSheetPotentials:30}
\BF = -2 \pi G \sigma m \zcap.
\end{equation}
%
For the Lagrangian problem we want the potential
%
\begin{equation}\label{eqn:infiniteSheetPotentials:630}
\begin{aligned}
-\spacegrad \phi &= \BF \\
-\zcap \PD{z}{\phi} &= \\
\end{aligned}
\end{equation}
%
or
%
\begin{equation}\label{eqn:infiniteSheetPotentials:50}
\phi = 2 \pi G \sigma m z.
\end{equation}
%
This is a reasonable seeming answer.  Our potential is of the form
%
\begin{equation}\label{eqn:infiniteSheetPotentials:70}
\phi = m g z,
\end{equation}
%
with
\begin{equation}\label{eqn:infiniteSheetPotentials:90}
g = 2 \pi G \sigma.
\end{equation}
%
\subsection{Calculating the potential directly.}
%
If we only want the potential, why start with the force?  We ought to be able to work with potentials directly, and write
%
\begin{equation}\label{eqn:infiniteSheetPotentials:110}
\phi(\Br) = G m \rho \int \frac{dV'}{\Abs{\Br' - \Br}}.
\end{equation}
%
It is a quick calculation to verify that is correct, and we find (provided \(\Br \ne \Br'\))
%
\begin{equation}\label{eqn:infiniteSheetPotentials:130}
\BF = - \spacegrad \phi = - G m \rho \int \frac{dV' (\Br - \Br')}{\Abs{\Br - \Br'}^3}.
\end{equation}
%
To verify the signs, it is helpful to refer to the diagram \cref{fig:infiniteSheetPotentials:infiniteSheetPotentialsFig2} which illustrates the position vectors.  Now, suppose we calculate the potential directly
%
\imageFigure{../figures/classicalmechanics/infiniteSheetPotentialsFig2}{Direction vectors for interaction with mass distribution.}{fig:infiniteSheetPotentials:infiniteSheetPotentialsFig2}{0.2}
%
\subsubsection{Naive approach, with bogus result.}
Our mass density as a function of \(Z\) is
%
\begin{equation}\label{eqn:infiniteSheetPotentials:170}
\rho(r', \theta', z') = \lambda \delta(z')
\end{equation}
%
\begin{equation}\label{eqn:infiniteSheetPotentials:650}
\begin{aligned}
\phi(z)
&= G m \sigma \int dz' \delta(z') \int d\theta' \int dr' \frac{r' }{((z-z')^2 + {r'}^2)^{1/2}} \\
&= 2 \pi G m \sigma \int_0^\infty \frac{r' dr' }{(z^2 + {r'}^2)^{1/2}} \\
&= 2 \pi G m \sigma z \int_0^\infty \frac{u du }{(1 + u^2)^{1/2}}.
\end{aligned}
\end{equation}
%
Again we can make a tangent substitution
%
\begin{equation}\label{eqn:infiniteSheetPotentials:150}
u = \tan\alpha,
\end{equation}
%
for
%
\begin{equation}\label{eqn:infiniteSheetPotentials:670}
\begin{aligned}
\phi(z)
&= 2 \pi G m \sigma z \int_0^{\pi/2} \frac{\tan \alpha \sec^2 \alpha d \alpha }{\sec\alpha} \\
&= 2 \pi G m \sigma z \int_0^{\pi/2} \tan \alpha \sec \alpha d \alpha \\
&= 2 \pi G m \sigma z \int_0^{\pi/2} \frac{\sin \alpha}{\cos^2 \alpha} d \alpha.
\end{aligned}
\end{equation}
%
This has the same functional form \(\phi = m g z\) as \eqnref{eqn:infiniteSheetPotentials:70}, except with
%
\begin{equation}\label{eqn:infiniteSheetPotentials:190}
g = 2 \pi G \sigma \int_0^{\pi/2} d\alpha \frac{\sin \alpha}{\cos^2 \alpha}.
\end{equation}
%
There is one significant and irritating difference.  The integral above does not have a unit value, but instead diverges (with \(\cos\alpha \rightarrow \infty\) as \(\alpha \rightarrow \pi/2\)).

What went wrong?  Trouble can be seen right from the beginning.  Consider the differential form above for \(u \gg 1\)
%
\begin{equation}\label{eqn:infiniteSheetPotentials:210}
\frac{u du}{\sqrt{1 + u^2}} \approx du.
\end{equation}
%
This we are integrating on \(u \in [0, \infty]\), so long before we make the trig substitutions we are in trouble.
%
\subsubsection{As the limit of a finite volume.}
%
My guess is that we have to tie the limits of the width of the plane and its diameter, decreasing the thickness in proportion to the increase in the radius.  Let us try that.

With a finite cylinder of height \(\epsilon\), radius \(R\), with a measurement of the potential directly above the cylinder at height \(z\), we have
%
\begin{equation}\label{eqn:infiniteSheetPotentials:230}
\phi(z)
= \rho G \int_0^{2\pi} d\theta' \int_0^\epsilon dz' \int_0^R r' dr' \inv{\sqrt{(z-z')^2 + {r'}^2}}.
\end{equation}
%
Performing the \(\theta'\) integration and substituting
%
\begin{equation}\label{eqn:infiniteSheetPotentials:250}
r' = (z - z') \tan\alpha,
\end{equation}
%
we have
%
\begin{equation}\label{eqn:infiniteSheetPotentials:690}
\begin{aligned}
\phi(z)
&=
2 \pi \rho G \int_0^\epsilon dz' \int_0^{\arctan(R/(z-z'))} (z - z') \tan \alpha \sec^2 \alpha d\alpha \inv{ \sec\alpha} \\
&=
2 \pi \rho G \int_0^\epsilon dz' \int_0^{\arctan(R/(z-z'))} (z - z') \frac{\sin \alpha}{\cos^2\alpha} d\alpha \\
&=
2 \pi \rho G \int_0^\epsilon dz' (z - z')\int_0^{\arctan(R/(z-z'))} \frac{-d\cos \alpha}{\cos^2\alpha} \\
&=
2 \pi \rho G \int_0^\epsilon dz' \evalrange{\inv{\cos\alpha}}{0}{\arctan(R/(z-z'))} \\
&=
2 \pi \rho G \int_0^\epsilon dz' \left( \inv{\cos\arctan(R/(z-z'))} -1 \right) \\
&=
2 \pi \rho G \int_0^\epsilon dz' \left( \sqrt{1 + \left(\frac{R}{z-z'}\right)^2} -1 \right) \\
&=
2 \pi \rho G
\left( -\epsilon +
\int_0^\epsilon dz'
\sqrt{1 + \left(\frac{R}{z-z'}\right)^2} \right)  \\
&=
2 \pi \rho G
\left( -\epsilon + R
\int_{(z - \epsilon)/R}^{z/R} dx
\sqrt{1 + \inv{x^2}}
\right)
\end{aligned}
\end{equation}
%% MATHEMATICA: FullSimplify[1/Cos[ArcTan[u]]] = Sqrt[1 + u^2]
%
For this integral we find
%\href{https://raw.github.com/peeterjoot/physicsplay/master/notes/phy354/.cdf}{this integral we find}
%
\begin{equation}\label{eqn:infiniteSheetPotentials:270}
\int dx \sqrt{1 + \inv{x^2}}
%=
%\frac{\sqrt{1+\frac{1}{x^2}} x }{\sqrt{1+x^2}}
%\left(\sqrt{1+x^2}+\ln(x)-\ln\left(1+\sqrt{1+x^2}\right)\right)
=
\frac{x}{\Abs{x}}
\left(\sqrt{1+x^2}+\ln(x)-\ln\left(1+\sqrt{1+x^2}\right)\right)
\end{equation}
%
Taking limits \(\epsilon \rightarrow 0\) and \(R \rightarrow \infty\) the terms
\(\sqrt{1 + x^2}\) at either \(x = z/R\) or \(x = (z-\epsilon)/R\) tend to 1.  This leaves us only with the \(\ln(x)\) contribution, so
%
\begin{equation}\label{eqn:infiniteSheetPotentials:290}
\phi(z) =
2 \pi \rho G
\left( -\epsilon + R \left(
\ln\left(\frac{z}{R}\right) -\ln\left(\frac{z - \epsilon}{R}\right) \right)\right)
\end{equation}
%
With \(\epsilon \rightarrow 0\) we have the structure of a differential above, but instead of expressing the derivative in the usual forward difference form
%
\begin{equation}\label{eqn:infiniteSheetPotentials:310}
\frac{df}{dx} = \frac{f(x + \epsilon) - f(x)}{\epsilon},
\end{equation}
%
we have to use a backwards difference, which is equivalent, provided the function \(f(x)\) is continuous at \(x\)
%
\begin{equation}\label{eqn:infiniteSheetPotentials:310b}
\frac{df}{dx} = \frac{f(x) - f(x -\epsilon)}{\epsilon}.
\end{equation}
%
We can then form the differential
%
\begin{equation}\label{eqn:qmTwoExamReflection:310c}
df(x) = f(x) - f(x -\epsilon) = \epsilon \frac{df}{dx}.
\end{equation}
%
We also want to express the charge density \(\rho\) in terms of surface charge density \(\sigma\), and note that these are related by \(\rho \Delta A \epsilon = \sigma \Delta A\).
%
\begin{equation}\label{eqn:infiniteSheetPotentials:710}
\begin{aligned}
\phi(z)
&=
2 \pi (\rho \epsilon) G
\left( -1 + R \frac{d}{dz} \ln(z/R) \right) \\
&=
2 \pi \sigma G
\left( -1 + \frac{d}{dz/R} \ln(z/R) \right),
\end{aligned}
\end{equation}
%
or
\begin{equation}\label{eqn:infiniteSheetPotentials:350}
\phi(z)
=
2 \pi \sigma G
\left( -1 + \frac{R}{z} \right)
\end{equation}
%
%There appears to have been a dropped factor of \(R\) here, and somehow I have a \(1/z\) instead of \(z\).  Since \(-\ln(x) = \ln(1/x)\), both of these suggest a sign error somewhere.  \textunderline{Suppose} the \(R\) factor was removed and the \(z\) was inverted.  Then we \textunderline{would} have
%
%\begin{equation}\label{eqn:infiniteSheetPotentials:330}
%\phi(z) =
%2 \pi \sigma G (z - 1),
%\end{equation}
%
%differing only be a constant from the result obtained by looking at the forces (which is allowable).  There is got to be an error above somewhere, but where?
%
Looking at this result, we have the same divergent integration result as in the first attempt, and the reason for this is clear after some reflection.  The limiting process for the radius and the thickness of the slice were allowed to complete independently.  Before taking limits we had
%
\begin{equation}\label{eqn:infiniteSheetPotentials:370}
\phi(z) = 2 \pi \sigma G
\left( -1
\inv{\epsilon} \int_0^\epsilon dz'
\sqrt{1 + \left(\frac{R}{z-z'}\right)^2} \right).
\end{equation}
%
Consider a similar, but slightly more general case, where we evaluate the limit
%
\begin{equation}\label{eqn:infiniteSheetPotentials:390}
L = \lim_{\epsilon \rightarrow 0} \inv{\epsilon} \int_a^{a + \epsilon} f(x) dx,
\end{equation}
%
where \(F'(x) = f(x)\), so that
%
\begin{equation}\label{eqn:infiniteSheetPotentials:730}
\begin{aligned}
L &=
\lim_{\epsilon \rightarrow 0} \frac{ F(a + \epsilon) - F(a)}{\epsilon} \\
&= F'(a) \\
&= f(a).
\end{aligned}
\end{equation}
%
So even without evaluating the integral we expect that we will have
%
\begin{equation}\label{eqn:infiniteSheetPotentials:750}
\begin{aligned}
\phi(z) &=
2 \pi \sigma G
\left( -1 + \evalbar{
\sqrt{1 + \left(\frac{R}{z-z'}\right)^2} }{z' = 0} \right) \\
&\rightarrow
2 \pi \sigma G
\left( -1 + \frac{R}{z} \right).
\end{aligned}
\end{equation}
%
This matches what was obtained in \eqnref{eqn:infiniteSheetPotentials:350} by brute forcing the integral with Mathematica, and then evaluating the limit the hard way.  Darn.
%
\subsubsection{As the limit of a finite volume.  Take II.}
%
Let us try once more.  We will consider a homogeneous cylindrical volume of radius \(R\), thickness \(\epsilon\) with total mass
%
\begin{equation}\label{eqn:infiniteSheetPotentials:410}
M = \rho \pi R^2 \epsilon = \sigma \pi R^2,
\end{equation}
%
so that the area density is
%
\begin{equation}\label{eqn:infiniteSheetPotentials:430}
\sigma = \rho \epsilon.
\end{equation}
%
Now we will reduce the thickness of the volume, keeping the total mass fixed, so that
%
\begin{equation}\label{eqn:infiniteSheetPotentials:450}
\pi R^2 \epsilon = \text{constant} = \pi c^2,
\end{equation}
%
or
%
\begin{equation}\label{eqn:infiniteSheetPotentials:470}
R = \frac{c}{\sqrt{\epsilon}}.
\end{equation}
%
We wish to evaluate
%
\begin{equation}\label{eqn:infiniteSheetPotentials:490}
\phi(z)
=
2 \pi \sigma G
\inv{\epsilon}
\int_0^\epsilon dz' \int_0^{c/\sqrt{\epsilon}} r' dr' \inv{\sqrt{(z-z')^2 + {r'}^2}}.
\end{equation}
%
% cancelled save with this calculation by accident... damn!
%\href{https://raw.github.com/peeterjoot/physicsplay/master/notes/phy354/problemSetIIproblem4IntegralsEvaluatingInfinitePlanePotentialTakeII.cdf}{
Performing the integrals, (first \(r'\), then \(z'\)) we find
%
\begin{dmath}\label{eqn:infiniteSheetPotentials:510}
\phi(z) = 2 \pi \sigma G \inv{2 \epsilon^2}
\Biggl(
z \left(-2 \epsilon ^2+\sqrt{\epsilon  \left(c^2+z^2 \epsilon \right)}-\sqrt{\epsilon  \left(c^2+(z-\epsilon )^2 \epsilon \right)}\right)
+\epsilon  \left(\epsilon ^2+\sqrt{\epsilon  \left(c^2+(z-\epsilon )^2 \epsilon \right)}\right)
+c^2 \left(-\ln \left(-z \epsilon +\sqrt{\epsilon  \left(c^2+z^2 \epsilon \right)}\right)+\ln \left(-(z - \epsilon) \epsilon +\sqrt{\epsilon  \left(c^2+(z-\epsilon )^2 \epsilon \right)}\right)\right)
\Biggr)
\end{dmath}
%
The \(\epsilon^3/\epsilon^2\) term is clearly killed in the limit, and we have a \(-z\) contribution from the first term.  For \(\epsilon \ne 0\) the difference of logarithms above can be written as
%
\begin{equation}\label{eqn:infiniteSheetPotentials:530}
\frac{-z+\epsilon +\sqrt{\frac{c^2+(z-\epsilon )^2 \epsilon }{\epsilon }}}{-z+\sqrt{\frac{c^2+z^2 \epsilon }{\epsilon }}}
\end{equation}
%
Suppose we could also validly argue that this tends to \(\ln(1) = 0\), and the difference of square roots could also be canceled.  Then we would be left with just
%
\begin{equation}\label{eqn:infiniteSheetPotentials:550}
\phi(z) = 2 \pi \sigma G \left(-z + \inv{2\epsilon} \sqrt{\epsilon(c^2 + (z - \epsilon)^2\epsilon} \right)
\end{equation}
%
If we also demand that \(z^2 \epsilon \gg c^2\), then we have a \(z/2\) contribution from the remaining square root and are left with
%
\begin{equation}\label{eqn:infiniteSheetPotentials:570}
\phi(z) = 2 \pi \sigma G (-z/2).
\end{equation}
%
This differs from the expected result by a factor of \(-2\), and we have had to do some very fishy root taking to even get that far.  Employing l'H\^opital's rule (or letting Mathematica attempt to evaluate the limit), we get infinities for the difference of logarithm term.

So, with a lot of cheating we get a result that is similar to the expected, but not actually a match, and even to get that we had to take the limits in an invalid way.  It looks like it is back to the drawing board, but I am not sure how to approach it.

After thinking about it a bit, perhaps the limiting process for the width needs to be explicitly accounted for using a delta function?  Perhaps a QM like treatment where we express the integral in terms of some basis and look for the resolution of identity in that resolution?
%
%Another possibility is that we can relate the limits of \(R\) and \(\epsilon\) in a different fashion?
