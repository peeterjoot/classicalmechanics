%
% Copyright © 2012 Peeter Joot.  All Rights Reserved.
% Licenced as described in the file LICENSE under the root directory of this GIT repository.
%
\makeproblem{One dimensional wave equation.}{problem:waveLagrangian:1}{
The Lagrangian for a one dimensional wave is derived in \citep{goldstein1951cm}
using a limiting argument applied to an infinite
sequence of connected masses on springs.  The result is
%
\begin{equation}\label{eqn:wave_lagrangian:oneDimensionalWaveLagrangian}
\LL = \inv{2} \left(\mu \left(\PD{t}{\eta}\right)^2 - Y \left(\PD{x}{\eta}\right)^2 \right).
\end{equation}
%
Here \(\eta\) was the displacement from the equilibrium position, \(\mu\) is the mass line density and \(Y\) is Young's modulus.

Using this Lagrangian, find the equations of the field, showing that it has the expected form.
} % problem
\makeanswer{problem:waveLagrangian:1}{
%
Taking derivatives confirms that this is the correct form.  The Euler-Lagrange
equations for this equation are:
%
\begin{equation}\label{eqn:waveLagrangian:20}
\begin{aligned}
\PD{\eta}{\LL} &= \PD{t}{} \PD{\PD{t}{\eta}}{\LL} +\PD{x}{} \PD{\PD{x}{\eta}}{\LL} \\
0 &= \PD{t}{} \mu \PD{t}{\eta} -\PD{x}{} Y \PD{x}{\eta} .
\end{aligned}
\end{equation}
%
Which has the expected form
%
\begin{equation}\label{eqn:waveLagrangian:40}
\mu \PDsq{t}{\eta} - Y \PDsq{x}{\eta} = 0.
\end{equation}
} % answer
