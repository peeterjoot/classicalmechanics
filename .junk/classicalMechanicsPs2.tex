%
% Copyright � 2012 Peeter Joot.  All Rights Reserved.
% Licenced as described in the file LICENSE under the root directory of this GIT repository.
%
%
%\input{../blogpost.tex}
%\renewcommand{\basename}{classicalMechanicsPs2}
%\renewcommand{\dirname}{notes/phy354/problems/}
%%\newcommand{\keywords}{}
%
%\input{../peeter_prologue_print2.tex}
%
%%\usepackage{lipsum}
%
%\beginArtNoToc
%
%\generatetitle{PHY354 Advanced Classical Mechanics.  Problem set 2 (2012)}
%\label{chap:classicalMechanicsPs2}
%
%\section{Disclaimer}
%
%Ungraded solutions to two of five of the problems from problem set 2 (I audited half the lectures for this course and intend to do the course problem sets if time allows after the fact.)
%
%\makeoproblem{Integrals of motion in a helix potential}{classicalMechanicsPs2:problem:2}
%{2012 PHY354 Problem set 2, problem 2}
%{
%Consider a particle moving in the external potential field (it could be gravitational or Coulomb, it is inessential
%for this problem) of an infinite homogeneous cylindrical helix. Find the conserved (linear combination of)
%components of \(\BP\) and \(\BM\).
%} % makeoproblem
%
%\makeanswer{classicalMechanicsPs2:problem:2}{
%TODO.
%
%Google shows that this is a problem found in \citep{landau1960classical}.  The argument there doesn't even rely on knowing what the form of the Lagrangian is (something I tried to calculate, and had trouble doing so).  %It would be nice to read that chapter of the text before attempting that problem, but that's not one that I have. ... ordered.
%% http://books.google.ca/books?id=e-xASAehg1sC&pg=PA21&lpg=PA21&dq=potential+for+an+infinite+cylindrical+helix&source=bl&ots=XPLMLrSh_s&sig=07oTC2n9GB_0svsV1cUXbbvoAZc&hl=en&sa=X&ei=1IfwT9SXMqmR6gGX1e3RBg&ved=0CEEQ6AEwAA#v=onepage&q=potential%20for%20an%20infinite%20cylindrical%20helix&f=false
%
%}
%
%\makeoproblem{``Hidden'' symmetries and integrals of motion}{classicalMechanicsPs2:problem:3}
%{2012 PHY354 Problem set 2, problem 3}
%{
%Consider the equation of motion of a particle of charge q moving in the field of a magnetic monopole:
%
%\begin{dmath}\label{eqn:classicalMechanicsPs2:330}
%\BB = g \frac{\Br}{r^3}, \BE = 0.
%\end{dmath}
%
%Here \(g\) is the magnetic charge, which can be found using the magnetic analogue of Gausses law: \(4 \pi Q_{\mathrm{magn}} = \int_{S^2} d\BSigma \cdot \BB = 4 \pi g\).  In this problem, do not start with a Lagrangian
%,but simply use the Lorentz force
%equation. Since we no longer deal with central forces, the angular momentum is no longer conserved, and
%the motion is no longer necessarily planar. However, it can be thought that a certain amount of angular
%momentum resides in the magnetic field, and, as first observed by Poincar\'e the total angular momentumm
%
%\begin{dmath}\label{eqn:classicalMechanicsPs2:350}
%\BD = \BM + c\frac{\Br}{r}.
%\end{dmath}
%
%is conserved (here \(\BM\) is the mechanical angular momentum \(\Br \cross \Bv\), and \(c\) is a constant).
%
%\makesubproblem{Find the value of \(c\) ensuring that \(d \BD/dt = 0\) for solutions of the equations of motion.}{classicalMechanicsPs2:problem:3:1}
%\makesubproblem{Show that the radius vector of the trajectory for a particle moving in the field of a monopole obeys \(\dot{\rcap} \cdot \BD = c\), where \(\dot{\rcap} = \Br/r\).}{classicalMechanicsPs2:problem:3:2}
%\makesubproblem{For a given value of \(\BD\), determined by the initial conditions, the relation found above restricts the
%motion of the particle on a particular surface in space. Draw this surface. What happens to this
%surface when \(g \rightarrow 0\)? Is it surprising?}{classicalMechanicsPs2:problem:3:3}
%\makesubproblem{Is energy conserved? Find the energy of the particle \(\calE\) as a function of \(\Bv\), \(\Br\), ... as appropriate to the case in hand.}{classicalMechanicsPs2:problem:3:4}
%
%Note that it is a fun and challenging problem to find \(\BA\) for the magnetic monopole, but it does not belong in this class.  Besides being a fun problem, the moral here is that finding symmetries and the related integrals of motion is not always so obvious. Usually, when integrals of motion exist, they are due to a symmetry. In this example, there is a ``hidden'' symmetry responsible for the conservation of \(\BD\).
%} % makeoproblem
%
%\makeanswer{classicalMechanicsPs2:problem:3}{
%\makesubanswer{Value of \(c\)}{classicalMechanicsPs2:problem:3:1}
%TODO.
%\makesubanswer{Trajectory radius vector}{classicalMechanicsPs2:problem:3:2}
%TODO.
%\makesubanswer{Trajectory surface}{classicalMechanicsPs2:problem:3:3}
%TODO.
%\makesubanswer{Is energy conserved}{classicalMechanicsPs2:problem:3:4}
%TODO.
%}
%
%\makeoproblem{Symmetries and conservation laws}{classicalMechanicsPs2:problem:4}
%{2012 PHY354 Problem set 2, problem 4}
%{
%What components of the momentum and angular momentum are conserved when a particle moves in the
%(say, gravitational) field of the following objects:
%
%\makesubproblem{An infinite homogeneous plane}{classicalMechanicsPs2:problem:4:1}
%\makesubproblem{An infinite homogeneous cylinder}{classicalMechanicsPs2:problem:4:2}
%\makesubproblem{Two point particles}{classicalMechanicsPs2:problem:4:3}
%\makesubproblem{An infinite homogeneous prism}{classicalMechanicsPs2:problem:4:4}
%\makesubproblem{A homogeneous cone}{classicalMechanicsPs2:problem:4:5}
%\makesubproblem{Three point particles}{classicalMechanicsPs2:problem:4:6}
%\makesubproblem{A homogeneous torus}{classicalMechanicsPs2:problem:4:7}
%
%This is a problem from last year's midterm. Problems 2. and 4. are rather similar in nature. Be sure to ``internalize'' them -- they will come handy during exams but also in your life as physicists.
%} % makeoproblem
%
%\makeanswer{classicalMechanicsPs2:problem:4}{
%\makesubanswer{An infinite homogeneous plane}{classicalMechanicsPs2:problem:4:1}
%TODO.
%\makesubanswer{An infinite homogeneous cylinder}{classicalMechanicsPs2:problem:4:2}
%TODO.
%\makesubanswer{Two point particles}{classicalMechanicsPs2:problem:4:3}
%TODO.
%\makesubanswer{An infinite homogeneous prism}{classicalMechanicsPs2:problem:4:4}
%TODO.
%\makesubanswer{A homogeneous cone}{classicalMechanicsPs2:problem:4:5}
%TODO.
%\makesubanswer{Three point particles}{classicalMechanicsPs2:problem:4:6}
%TODO.
%\makesubanswer{A homogeneous torus}{classicalMechanicsPs2:problem:4:7}
%TODO.
%}
%
