%
% Copyright � 2020 Peeter Joot.  All Rights Reserved.
% Licenced as described in the file LICENSE under the root directory of this GIT repository.
%
%{
\input{../latex/blogpost.tex}
\renewcommand{\basename}{curvilinearSTA}
%\renewcommand{\dirname}{notes/phy1520/}
\renewcommand{\dirname}{notes/ece1228-electromagnetic-theory/}
%\newcommand{\dateintitle}{}
%\newcommand{\keywords}{}

\input{../latex/peeter_prologue_print2.tex}

\PassOptionsToPackage{answerdelayed}{exercise}

% proof:
\usepackage{amsthm}
\usepackage{macros_cal} % \LL
\newcommand{\Lq}[0]{L}

\usepackage{peeters_layout_exercise}
\usepackage{peeters_braket}
\usepackage{peeters_figures}
\usepackage{siunitx}
\usepackage{verbatim}
%\usepackage{mhchem} % \ce{}
%\usepackage{macros_bm} % \bcM
%\usepackage{macros_qed} % \qedmarker
%\usepackage{txfonts} % \ointclockwise

\beginArtNoToc

\generatetitle{Space time algebra multivector calculus.}
%\chapter{XXX}
%\label{chap:curvilinearSTA}

\section{Motivation.}
We've seen how tidy the Euler-Lagrange equations are when expressed in terms of the gradient.
These equations are usually expressed in terms of generalized coordinates.
The two mechanisms ought to be connected by different representations of the gradient.
This justifies some exploration of the ideas required to express the STA (space-time) gradient in alternate coordinate systems.
Another reason for doing this exploration is to enumerate all the variations of Stokes' theorem for the one, two, three, and four dimensional subspaces that we are generated by our four-dimensional generating vector space.

Part of our problem is actually just one of notation.
In Euclidean space we use bold face reciprocal frame vectors \( \Bx^i \cdot \Bx_j = {\delta^i}_j \), which are notationally different than the generalized coordinates \( x_i, x^j \) that satisfy \( \Bx = x^i \Bx_i = x_j \Bx^j \).
Since we drop bold face for STA vectors \( x = x^\mu \gamma_\mu = x_\mu \gamma^\mu \), we need a different notation for both our generalized coordinates and frame vectors in curvilinear coordinate systems.
One way to solve this notational problem would be to choose some new symbol, say \( f^\mu, f_\mu \), which satisfies the non-bold face convention that we use for four-vectors in STA.
However, that is a pretty arbitrary choice, and symbols are at a premium.
Instead, let's break our conventions slightly, and use the same bold \( \Bx^\mu, \Bx_\nu \) symbols that we use for our curvilinear basis elements for Euclidean geometric algebras.
\section{Gradient.}
Recall that Our gradient in the standard basis is just
\begin{equation}\label{eqn:curvilinearSTA:20}
   \grad = \gamma^\mu \PD{x^\mu}{} \qquad \lr{ = \gamma^\mu \partial_\mu }.
\end{equation}
Given a subspace spanned by the parameterization \( x = x(u^1, u^2, u^3, u^4) \), it's just an application of the chain rule to find the representation of the gradient.
\maketheorem{Curvilinear representation of the gradient.}{thm:curvilinearSTA:40}{
The curvilinear representation of the gradient can be expressed as
\begin{equation*}
\grad = (\grad u^\mu) \PD{u^\mu}{}.
\end{equation*}
} % theorem
% d/dx = du/dx d/du
\begin{proof}
\begin{dmath}\label{eqn:curvilinearSTA:80}
\grad
= \gamma^\mu \PD{x^\mu}{}
= \gamma^\mu \PD{u^\nu}{x^\mu} \PD{u^\nu}{}
= (\grad u^\nu) \PD{u^\nu}{}.
\end{dmath}
\end{proof}
This is not a particularly helpful representation, as the coordinates \( u^\mu = u^\mu(x) \) are implicit functions of parameterization vector \( x \) (or it's coordinates).
This implicit nature can make these gradients really hard to compute.
Charging in to rescue us are the reciprocal frame vectors associated with the tangent space of the parameterization.
\section{Reciprocal frame.}
\section{Solutions.}
\shipoutAnswer
%}
%\EndArticle
\EndNoBibArticle
