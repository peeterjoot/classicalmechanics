%
% Copyright © 2012 Peeter Joot.  All Rights Reserved.
% Licenced as described in the file LICENSE under the root directory of this GIT repository.
%
\makeoproblem{Lorentz force Lagrangian.}{tongmf1:pr6}{\citep{TongMf1} p6.}{
%
Lorentz force Lagrangian
%
\begin{equation*}
\LL = \inv{2} m \Bv^2 -q \phi + q \Bv/c \cdot \BA,
\end{equation*}
%
%derive the Lorentz force equation, and some other stuff.
%FIXME: describe.
}
%
\makeanswer{tongmf1:pr6}{
We are showing that the Lagrangian is equivalent to the Lorentz force law.
%
When I first tried this problem I had trouble with it, and also had trouble following the text for the same in Tong's paper.  Later I did the somewhat harder problem of exactly
this, but for the covariant form of the Lorentz force law, so I thought I had come back to this and try again.
%
First step that seemed natural was to put the equation into four vector form, despite the fact that the proper time Lagrangian equations were not going to be used to
produce the equation of motion.  For just the Lorentz part of the Lagrangian we have:
%
%
\begin{equation}\label{eqn:tongMf1:600}
\begin{aligned}
\LL'
&= -\phi + \Bv/c \cdot \BA \\
&= -\phi \sum v^i/c A^j
\mathLabelBox{\sigma_i \cdot \sigma_j}{\(\inv{2}(\gamma_{i0j0} + \gamma_{j0i0}) = - \gamma_i \cdot \gamma_j\)}
\\
&= -\inv{c} \phi c + \sum v^i A^i {\gamma_i}^2 \\
&= -\inv{c} \phi c {\gamma_0}^2 + \sum v^i A^i {\gamma_i}^2 .
\end{aligned}
\end{equation}
%
Thus with \(v = c \gamma_0 + \sum v^i \gamma_i = \sum v^{\mu} \gamma_{\mu}\), and \(A = \phi \gamma_0 + \sum A^{i} \gamma_i = \sum A^{\mu} \gamma_{\mu}\), we can thus write the complete
Lagrangian as:
%
\begin{equation}\label{eqn:tongMf1:1720}
\LL = \inv{2} m \Bv^2 - q A \cdot v/c.
\end{equation}
%
As usual we recover our vector forms by wedging with the time basis vector:
%
\begin{equation}\label{eqn:tongMf1:1740}
A \wedge \gamma_0 = \sum A^i \gamma_{i0} = \sum A^i \sigma_i = \BA.
\end{equation}
%
and \(v \wedge \gamma_0 = \cdots = \Bv\).
%
Notice the sign in the potential term, which is negative, unlike the same Lagrangian in relativistic (proper) form: \(\LL = \inv{2}m v^2 + q A \cdot v/c\).  That difference is required
since the lack of the use of time as one of the generalized coordinates will change the signs of some of the results.
%
Now, this does not matter for this particular problem, but also observe that this Lagrangian is almost in its proper form.  All we have to do is add a \(-\inv{2} m c^2\) constant to it, which should not effect the equations of motion.  Doing so yields:
%
\begin{equation}\label{eqn:tongMf1:1760}
\LL = \inv{2} m (-c^2 + \Bv^2) - q A \cdot v/c = - \left( \inv{2} mv^2 + q A \cdot v/c \right).
\end{equation}
%
I did not notice that until writing this up.  So we have the same Lagrangian in both cases, which makes sense.  Whether or not one gets the traditional Lorentz force law from this
or the equivalent covariant form depends only on whether one treats time as one of the generalized coordinates or not (and if doing so, use proper time in the place of the time
derivatives when applying the Lagrange equations).  Cool.
%
Anyways, now that we have a more symmetric form of the Lagrangian, lets compute the equations of motion.
%
\begin{equation}\label{eqn:tongMf1:620}
\begin{aligned}
\PD{x^i}{\LL}
&= \frac{d}{dt} \PD{\xdot^i}{\LL} \\
&= \frac{d}{dt} \left(m v^i - q/c A \cdot \PD{\xdot^i}{v} \right) \\
&= \frac{d}{dt} \left(m v^i - q/c A \cdot \gamma_i \right) \\
&= \frac{d}{dt} \left(m v^i + q/c A^i \right) \\
&= p^i + q/c \sum \xdot^j \PD{x^j}{A^i} \\
q \PD{x^i}{A} \cdot v/c &= \\
\implies \\
\dot{p}^i
&= -q/c \left( \PD{x^i}{A} \cdot v - \sum \xdot^j \PD{x^j}{A^i} \right) \\
&= -q/c \left( \sum \PD{x^i}{A^{\mu}} v^{\nu} \gamma_{\mu} \cdot \gamma_{\nu} - \sum v^j \PD{x^j}{A^i} \right) \\
&= -q/c \left( \sum \PD{x^i}{A^0} v^0 {\gamma_0}^2 +\sum \PD{x^i}{A^j} v^j {\gamma_j}^2 - \sum v^j \PD{x^j}{A^i} \right) \\
\implies \\
\sum \sigma_i \dot{p}^i = \Bp
&= q/c \sum \sigma_i \left( -\PD{x^i}{A^0} v^0 {\gamma_0}^2 - \PD{x^i}{A^j} v^j {\gamma_j}^2 - v^j \PD{x^j}{A^i} \right) \\
&= -q \grad \phi + \sum \sigma_i v^j \left( \PD{x^i}{A^j} - \PD{x^j}{A^i} \right) .
\end{aligned}
\end{equation}
%
Now, it is not obvious by looking, but this last expression is \(\Bv \cross (\grad \cross \BA)\).  Let us verify this by going backwards:
%
\begin{equation}\label{eqn:tongMf1:640}
\begin{aligned}
\Bv \cross
\lr{ \grad \cross \BA }
&= \inv{i} \left( \Bv \wedge
\lr{  \grad \cross \BA  }
\right) \\
&= \inv{2i} \left( \Bv
\lr{  \grad \cross \BA  }
- \lr{  \grad \cross \BA  }
 \Bv \right) \\
&= \inv{2i} \left( \Bv \inv{i}
\lr{  \grad \wedge \BA  }
 - \inv{i}
\lr{  \grad \wedge \BA  }
 \Bv \right) \\
&= -\inv{2} \left( \Bv
\lr{  \grad \wedge \BA  }
 -
\lr{  \grad \wedge \BA  }
 \Bv \right) \\
&=
\lr{  \grad \wedge \BA  }
\cdot \Bv \\
&= \sum v^k \PD{x^i}{A^j} \sigma_i
\lr{ \sigma_j \cdot \sigma_k } - \sigma_j
\lr{ \sigma_i \cdot \sigma_k } \\
&= \sum v^k \PD{x^i}{A^j} \sigma_i \delta_{jk} - \sigma_j \delta_{ik} \\
&= \sum v^j \PD{x^i}{A^j} \sigma_i -v^i \PD{x^i}{A^j} \sigma_j \\
&= \sum v^j \PD{x^i}{A^j} \sigma_i -v^j \PD{x^j}{A^i} \sigma_i \\
&= \sum v^j \sigma_i \left(\PD{x^i}{A^j} - \PD{x^j}{A^i} \right) .
\end{aligned}
\end{equation}
%
Therefore the final result is our Lorentz force law, as expected:
%
\begin{equation}\label{eqn:tongMf1:1780}
\Bp = -q \grad \phi + q \Bv/c \cross (\grad \cross \BA).
\end{equation}
%
}
