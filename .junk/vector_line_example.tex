% This is obvious when stated in example form that there's not much value to a specific example.

\subsection{Examples of vector derivative line integrals.}

\makeproblem{A double sided example.}{problem:fundamentalTheoremOfGC:11}{
Given a parameterization \( x = \alpha \gamma_2 \), integrate the following, checking against the fundamental theorem
\begin{dmath}\label{eqn:fundamentalTheoremOfGC:n}
I = \int \gamma_0 e^{\gamma_{01}\alpha} d^1\Bx \lrpartial \gamma_{1} e^{\gamma_{01}\alpha}.
\end{dmath}
} % problem
\makeanswer{problem:fundamentalTheoremOfGC:11}{
The tangent vector is \( d\Bx_\alpha = \gamma_2 d\alpha \), and \( \lrpartial = \gamma^2 \PDi{\alpha}{} \), so
\begin{dmath}\label{eqn:fundamentalTheoremOfGC:n}
I = 
\int d\alpha
\gamma_0 \PD{\alpha}{e^{\gamma_{01}\alpha}} \gamma_{1} e^{\gamma_{01}\alpha}
+
\int d\alpha
\gamma_0 e^{\gamma_{01}\alpha} \gamma_{1} \PD{\alpha}{e^{\gamma_{01}\alpha}}
=
\int d\alpha 
\PD{\alpha}{}
\lr{
\gamma_0 e^{\gamma_{01}\alpha} \gamma_{1} e^{\gamma_{01}\alpha}
}
=
\gamma_0 e^{\gamma_{01}\alpha} \gamma_{1} e^{\gamma_{01}\alpha}
\end{dmath}
} % answer
