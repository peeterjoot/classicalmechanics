%
% Copyright � 2020 Peeter Joot.  All Rights Reserved.
% Licenced as described in the file LICENSE under the root directory of this GIT repository.
%
%{
\input{../latex/blogpost.tex}
\renewcommand{\basename}{maxwells}
%\renewcommand{\dirname}{notes/phy1520/}
\renewcommand{\dirname}{notes/ece1228-electromagnetic-theory/}
%\newcommand{\dateintitle}{}
%\newcommand{\keywords}{}

\input{../latex/peeter_prologue_print2.tex}

\PassOptionsToPackage{answerdelayed}{exercise}

% proof:
\usepackage{amsthm}
\usepackage{macros_cal} % \LL
\usepackage{peeters_layout_exercise}
\usepackage{peeters_braket}
\usepackage{peeters_figures}
\usepackage{siunitx}
\usepackage{verbatim}
%\usepackage{mhchem} % \ce{}
%\usepackage{macros_bm} % \bcM
%\usepackage{macros_qed} % \qedmarker
%\usepackage{txfonts} % \ointclockwise

\beginArtNoToc

\generatetitle{Maxwell's equation using geometric algebra Lagrangian.}
%\chapter{XXX}
%\label{chap:maxwells}
\section{Motivation.}
In my classical mechanics notes, I've got computations of Maxwell's equation (singular in it's geometric algebra form) from a Lagrangian in various ways (using a tensor, scalar and multivector Lagrangians), but all of these seem more convoluted than they should be.
\input{fieldLagrangianAndMaxwell.tex}
\subsection{Solutions.}
\shipoutAnswer
\EndArticle
