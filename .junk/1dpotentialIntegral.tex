%
% Copyright � 2012 Peeter Joot.  All Rights Reserved.
% Licenced as described in the file LICENSE under the root directory of this GIT repository.
%
%
%\chapter{Integrating the equation of motion for a one dimensional problem}
\index{pendulum}
\label{chap:1dpotentialIntegral}
%\blogpage{http://sites.google.com/site/peeterjoot/math2010/1dpotentialIntegral.pdf}
%\date{Jan 1, 2010}
%
\section{Motivation.}
%
While linear approximations, such as the small angle approximation for the pendulum, are often used to understand the dynamics of non-linear systems, exact solutions may be possible in some cases.  Walk through the setup for such an exact solution.
%
\section{Guts.}
%
The equation to consider solutions of has the form
%
\begin{equation}
\label{eqn:1dpotentialIntegral:1}
\frac{d}{dt} \left( m \frac{dx}{dt} \right) = -\PD{x}{U(x)}.
\end{equation}
%
We have an unpleasant mix of space and time derivatives, preventing any sort of antidifferentiation.  Assuming constant mass \(m\), and employing the chain rule a refactoring in terms of velocities is possible.
%
\begin{equation}\label{eqn:1dpotentialIntegral:45}
\begin{aligned}
\frac{d}{dt} \left( \frac{dx}{dt} \right)
&=
\frac{dx}{dt} \frac{d}{dx} \left( \frac{dx}{dt} \right)  \\
&=
\inv{2} \frac{d}{dx} \left( \frac{dx}{dt} \right)^2  .
\end{aligned}
\end{equation}
%
The one dimensional Newton's law \autoref{eqn:1dpotentialIntegral:1} now takes the form
\begin{equation}
\label{eqn:1dpotentialIntegral:2}
\frac{d}{dx} \left( \frac{dx}{dt} \right)^2 = -\frac{2}{m} \PD{x}{U(x)}.
\end{equation}
%
This can now be antidifferentiated for
%
\begin{equation}\label{eqn:1dpotentialIntegral:3}
\left( \frac{dx}{dt} \right)^2 = \frac{2}{m} (H - U(x)).
\end{equation}
%
What has now been accomplished is the removal of the second derivative.  Having done so, one can see that this was a particularly dumb approach, since \autoref{eqn:1dpotentialIntegral:3} is nothing more than the Hamiltonian for the system, something more obvious if rearranged slightly
%
\begin{equation}\label{eqn:1dpotentialIntegral:3b}
H = \inv{2} m \dot{x}^2 + U(x).
\end{equation}
%
We could have started with a physics principle instead of mechanically plugging through calculus manipulations and saved some work.  Regardless of the method used to get this far, one can now take roots and rearrange for
%
\begin{equation}\label{eqn:1dpotentialIntegral:4}
dt = \frac{dx}{\sqrt{ \frac{2}{m} (H - U(x)) } }.
\end{equation}
%
We now have a differential form implicitly relating time and position.  One can conceivably integrate this and invert to solve for position as a function of time, but substitution of a more specific potential is required to go further.
%
\begin{equation}\label{eqn:1dpotentialIntegral:5}
t(x) = t(x_0) + \int_{y=x_0}^{x} \frac{dy}{\sqrt{ \frac{2}{m} (H - U(y)) } }.
\end{equation}
