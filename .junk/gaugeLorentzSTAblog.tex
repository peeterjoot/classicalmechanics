%
% Copyright � 2020 Peeter Joot.  All Rights Reserved.
% Licenced as described in the file LICENSE under the root directory of this GIT repository.
%
%{
\input{../latex/blogpost.tex}
\renewcommand{\basename}{gaugeLorentzSTA}
%\renewcommand{\dirname}{notes/phy1520/}
\renewcommand{\dirname}{notes/ece1228-electromagnetic-theory/}
%\newcommand{\dateintitle}{}
%\newcommand{\keywords}{}

\input{../latex/peeter_prologue_print2.tex}

\usepackage{peeters_layout_exercise}
\usepackage{peeters_braket}
\usepackage{peeters_figures}
\usepackage{siunitx}
\usepackage{verbatim}
%\usepackage{mhchem} % \ce{}
%\usepackage{macros_bm} % \bcM
%\usepackage{macros_qed} % \qedmarker
%\usepackage{txfonts} % \ointclockwise
\beginArtNoToc
\generatetitle{Gauge transformation in the Lorentz force Lagrangian.}
%\chapter{XXX}
%\label{chap:gaugeLorentzSTA}
\makeproblem{Lorentz force gauge transformation.}{problem:gaugeLorentzSTA:20}{
Show that the gauge transformation \( A \rightarrow A + \grad \psi \) applied to the Lorentz force Lagrangian
\begin{equation}\label{eqn:gaugeLorentzSTA:20}
L = \inv{2} m v^2 + q A \cdot v/c,
\end{equation}
does not change the equations of motion.
} % problem
\makeanswer{problem:gaugeLorentzSTA:20}{
The gauge transformed Lagrangian is
\begin{equation}\label{eqn:gaugeLorentzSTA:40}
L = \inv{2} m v^2 + q A \cdot v/c + \frac{q v}{c} \cdot \grad \phi.
\end{equation}
We know that the Lorentz force equations are obtained from the first two terms, so need only consider the effects of the new \( \phi \) dependent term on the action.  First observe that 
\begin{equation}\label{eqn:gaugeLorentzSTA:60}
v \cdot \grad \phi
=
\frac{dx^\mu}{d\tau} \PD{x^\mu}{\phi}
=
\frac{d \phi}{d\tau}.
\end{equation}
This means that the action is transformed to 
\begin{equation}\label{eqn:gaugeLorentzSTA:80}
S 
\rightarrow S + \frac{q}{c} \int d\tau \frac{d\phi}{d\tau}
= S + \frac{q}{c} \evalbar{\phi}{\Delta \tau}.
\end{equation}
As the action is evaluated over a fixed interval, the gauge transformation only changes the action by a constant, so the equations of motion are unchanged.
} % answer
%}
%\EndArticle
\EndNoBibArticle
