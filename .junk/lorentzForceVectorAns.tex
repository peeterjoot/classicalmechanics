
First calculate the field velocity product in terms of electric and magnetic components.  In this new frame of reference write the proper velocity of the charged particle as
\(v = \gamma_\mu \xdot^\mu\)
%
\begin{equation}\label{eqn:lorentzForce:100}
\begin{aligned}
F \cdot v
&= (\BE + I c \BB) \cdot v \\
&= (E^i \gamma_{i0} - \epsilon_{ijk}c B^k \gamma_{ij}) \cdot \gamma_\mu \xdot^\mu \\
&=
  E^i \xdot^0 \gamma_{i0} \cdot \gamma_0
+ E^i \xdot^j \gamma_{i0} \cdot \gamma_j
- \epsilon_{ijk} c B^k \xdot^m \gamma_{ij} \cdot \gamma_m .
\end{aligned}
\end{equation}
%
We apply a \(\gamma_0\) wedge to determine this observer dependent expression of the force.
%
\begin{equation}\label{eqn:lorentzForce:120}
\begin{aligned}
\gamma^{-1} (F \cdot v) \wedge \gamma_0
&=
\left(
  E^i \xdot^0 (\gamma_{i0} \cdot \gamma_0)
+ E^i \xdot^j (\gamma_{i0} \cdot \gamma_j)
- \epsilon_{ijk} c B^k \xdot^m \gamma_{ij} \cdot \gamma_m \right) \wedge \gamma_0 \\
&= E^i \xdot^0 \gamma_{i0} - \epsilon_{ijk} c B^k \xdot^m (\gamma_i)^2 ( \gamma_{i} \delta_{jm} -\gamma_{j} \delta_{im} ) \wedge \gamma_0 \\
&= \left( E^i \xdot^0 \gamma_{i0} + \epsilon_{ijk} c B^k \left( \xdot^j \gamma_{i0} - \xdot^i \gamma_{j0} \right) \right),
\end{aligned}
\end{equation}
where \(\gamma = dt/d\tau\).
%
This wedge application has discarded the timelike components of the force equation with respect to this observer rest frame.
Introduce the basis
\(\{\Be_i = \gamma_i \wedge \gamma_0\}\) for this observers' Euclidean space.  These spacetime bivectors square to unity, and thus behave in every respect like
Euclidean space vector basis vectors.  Writing \(\BE = E^i \Be_i\), \(\BB = B^i \Be_i\), and \(\Bv = \Be_i dx^i/dt\) we have
%
\begin{equation}\label{eqn:lorentzForce:140}
\gamma^{-1} (F \cdot v) \wedge \gamma_0
= c \frac{dt}{d\tau} \left( \BE + \epsilon_{ijk} B^k \left( \frac{dx^j}{dt} \Be_{i} - \frac{dx^i}{dt} \Be_{j} \right) \right) .
\end{equation}
%
This inner antisymmetric sum is just the cross product.  This can be observed by expanding the determinant
%
\begin{equation}\label{eqn:lorentzForce:160}
\begin{aligned}
\Ba \cross \Bb &=
\begin{vmatrix}
\Be_1 & \Be_2 & \Be_3 \\
a_1 & a_2 & a_3 \\
b_1 & b_2 & b_3 \\
\end{vmatrix} \\
&=
  \Be_1 (a_2 b_3 - a_3 b_2)
+ \Be_2 (a_3 b_1 - a_1 b_3)
+ \Be_3 (a_1 b_2 - a_2 b_1) \\
&=
  \Be_i a_j b_k.
\end{aligned}
\end{equation}
%
This leaves
\begin{equation}\label{eqn:lorForce:FdotVwedge}
q (F \cdot v/c) \wedge \gamma_0
= \gamma q \left( \BE + \Bv \cross \BB \right).
\end{equation}
%
Next expand the left hand side acceleration term in coordinates, and wedge with \(\gamma_0\)
%
\begin{equation}\label{eqn:lorentzForce:180}
\begin{aligned}
\pdot \wedge \gamma_0
&= \left(\gamma_\mu \frac{d m \xdot^\mu}{dt}\frac{dt}{d\tau} \right) \wedge \gamma_0 \\
&= \gamma_{i0} \frac{d m \xdot^i}{dt} \gamma,
\end{aligned}
\end{equation}
%
Equating with \eqnref{eqn:lorForce:FdotVwedge}, and cancelling the \(\gamma\) factors, we are left with
%
\begin{equation}\label{eqn:lorentzForce:200}
\frac{d}{dt}\left( m \gamma \Bv \right) = q \left( \BE + \Bv \cross \BB \right).
\end{equation}
