%
% Copyright � 2012 Peeter Joot.  All Rights Reserved.
% Licenced as described in the file LICENSE under the root directory of this GIT repository.
%
%
%
%
%\chapter{Tensor derivation of Maxwell equation (non-dual part) from Lagrangian}
\index{Maxwell's equation!tensor}
\label{chap:maxwellTensorFromLagrangian}
%\date{April 20, 2009.  maxwellTensorFromLagrangian.tex}
%
\makeproblem{Tensor derivation of Maxwell's equation.}{problem:maxwellTensorFromLagrangian:1}{
Using the Lagrangian for Maxwell's equation in it's tensor form
\begin{equation}\label{eqn:maxwellTensorFromLagrangian:20}
\LL = \frac{\epsilon_0}{4} F_{\mu\nu}F^{\mu\nu} + \inv{c} J_\mu \cdot A^\mu.
\end{equation}
show that
\begin{equation}\label{eqn:maxwellTensorFromLagrangian:3}
\partial_\nu F^{\nu\mu} = \inv{\epsilon_0 c} J^\mu.
\end{equation}
} % problem
\makeanswer{problem:maxwellTensorFromLagrangian:1}{
%\section{Motivation.}
%%
%Looking through my notes for a purely tensor derivation of Maxwell's equation, and not finding one.  Have done this on
%paper a number of times, but writing it up once for reference to refer to for signs will be useful.
%%
%\section{Lagrangian.}
%%
%Notes containing derivations of Maxwell's equation
%%
%\begin{equation}\label{eqn:maxTenLag:maxwell}
%\grad F = J/\epsilon_0 c.
%\end{equation}
%%
%From the Lagrangian
%%
%\begin{equation}\label{eqn:maxTenLag:maxlag}
%\LL = -\frac{\epsilon_0}{2} (\grad \wedge A)^2 + \frac{J}{c} \cdot A.
%\end{equation}
%%
%can be found in \chapcite{PJFieldLagrangian}, and the earlier \chapcite{PJMaxwellLagrangian}.
%
%We will work from the scalar part of this Lagrangian, expressed strictly in tensor form
%%
%
%\section{Calculation.}
%
%\subsection{Preparation.}
%
In preparation, an expansion of the Faraday tensor in terms of potentials is desirable
%
\begin{equation}\label{eqn:maxwellTensorFromLagrangian:40}
\begin{aligned}
F_{\mu\nu}F^{\mu\nu}
&=
(\partial_\mu A_\nu - \partial_\nu A_\mu) (\partial^\mu A^\nu - \partial^\nu A^\mu) \\
&=
\partial_\mu A_\nu \partial^\mu A^\nu
-\partial_\mu A_\nu \partial^\nu A^\mu
- \partial_\nu A_\mu \partial^\mu A^\nu
+ \partial_\nu A_\mu \partial^\nu A^\mu \\
&=
2 (\partial_\mu A_\nu \partial^\mu A^\nu - \partial_\mu A_\nu \partial^\nu A^\mu)
.
\end{aligned}
\end{equation}
%
So we have
\begin{equation}\label{eqn:maxwellTensorFromLagrangian:60}
\LL = \frac{\epsilon_0}{2} \partial_\mu A_\nu (\partial^\mu A^\nu - \partial^\nu A^\mu)
+ \inv{c} J_\mu \cdot A^\mu.
\end{equation}
%
\paragraph{Derivatives.}
%
We want to compute
%
\begin{equation}\label{eqn:maxwellTensorFromLagrangian:80}
\PD{A_\alpha}{\LL} = \sum \partial_\beta \PD{(\partial_\beta A_\alpha)}{\LL}.
\end{equation}
%
Starting with the LHS we have
%
\begin{equation}\label{eqn:maxwellTensorFromLagrangian:100}
\PD{A_\alpha}{\LL} = \inv{c} J^\alpha,
\end{equation}
%
and for the RHS
%
\begin{equation}\label{eqn:maxwellTensorFromLagrangian:120}
\begin{aligned}
\PD{(\partial_\beta A_\alpha)}{\LL}
&=
\frac{\epsilon_0}{2}
\PD{(\partial_\beta A_\alpha)}{}
\partial_\mu A_\nu (\partial^\mu A^\nu - \partial^\nu A^\mu) \\
&=
\frac{\epsilon_0}{2}
\left(
F^{\beta\alpha}
+
\partial^\mu A^\nu
\PD{(\partial_\beta A_\alpha)}{}
(\partial_\mu A_\nu - \partial_\nu A_\mu)
\right) \\
&=
\frac{\epsilon_0}{2}
\left(
F^{\beta\alpha}
+ \partial^\beta A^\alpha
-\partial^\alpha A^\beta
\right) \\
&=
{\epsilon_0} F^{\beta\alpha} .
\end{aligned}
\end{equation}
%
Taking the \(\beta\) derivatives and combining the results for the LHS and RHS this is
%
\begin{equation}\label{eqn:maxTenLag:maxwellTensor}
\partial_\beta F^{\beta\alpha} = \inv{\epsilon_0 c} J^\alpha.
\end{equation}
A trivial index renaming yields \cref{eqn:maxwellTensorFromLagrangian:3}.
} % answer
\makeproblem{Compare to STA form.}{problem:maxwellTensorFromLagrangian:2}{
Compare \cref{eqn:maxwellTensorFromLagrangian:3} to
\begin{equation}\label{eqn:maxwellTensorFromLagrangian:140}
\grad \cdot F = J/\epsilon_0 c.
\end{equation}
} % problem
\makeanswer{problem:maxwellTensorFromLagrangian:2}{
%
%\subsection{Compare to STA form.}
%
%To verify that no sign errors have been made during the index manipulations above, this result should also match
%the STA Maxwell equation of \eqnref{eqn:maxTenLag:maxwell}, the vector part of which is
%
%
Dotting the LHS with \(\gamma^\alpha\) we have
%
\begin{equation}\label{eqn:maxwellTensorFromLagrangian:160}
\begin{aligned}
(\grad \cdot F) \cdot \gamma^\alpha
&=
((\gamma^\mu \partial_\mu) \cdot (\inv{2} F^{\beta\sigma} (\gamma^\beta \wedge \gamma_\sigma))) \cdot \gamma^\alpha \\
&=
\inv{2} \partial_\mu F^{\beta\sigma}
(\gamma^\mu \cdot (\gamma_\beta \wedge \gamma_\sigma)) \cdot \gamma^\alpha \\
&=
\inv{2}
\left( \partial_\beta F^{\beta\sigma} \gamma_\sigma -\partial_\sigma F^{\beta\sigma} \gamma_\beta \right) \cdot \gamma^\alpha \\
&=
\inv{2}
\left( \partial_\beta F^{\beta\alpha} -\partial_\sigma F^{\alpha\sigma} \right) \\
&=
\partial_\beta F^{\beta\alpha}.
\end{aligned}
\end{equation}
%
This gives us
%
\begin{equation}\label{eqn:maxwellTensorFromLagrangian:180}
\partial_\beta F^{\beta\alpha} = J^\alpha/\epsilon_0 c.
\end{equation}
%
In agreement with
% \eqnref{eqn:maxTenLag:maxwellTensor}.
\cref{eqn:maxwellTensorFromLagrangian:3}
} % answer
