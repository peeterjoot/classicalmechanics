In \citep{poisson1999ild},
the covariant Lorentz force Lagrangian is given by
%
\begin{equation}\label{eqn:lforceLag2:poissonLag}
\LL = \int A_\alpha j^\alpha d^4 x - m \int d\tau,
\end{equation}
%
...
\section{Lagrangian with Absolute Velocity.}
%
Now, with
%
\begin{equation}\label{eqn:lForceLag2:240}
d\tau = \sqrt{\frac{dx}{d\lambda}} d\lambda.
\end{equation}
%
it appears from
\eqnref{eqn:lforceLag2:poissonLag}
that we can form a different Lagrangian
%
\begin{equation}\label{eqn:lForceLag2:260}
\LL = \alpha m \Abs{v} + q A \cdot v/c.
\end{equation}
%
where \(\alpha\) is a constant to be determined.  Most of the work of evaluating the variational derivative has been done, but we need \(\grad_v \Abs{v}\), omitting dots this is
%
\begin{equation}\label{eqn:lForceLag2:280}
\begin{aligned}
\grad \Abs{x}
&= \gamma^\mu \partial_\mu \sqrt{x^\alpha x_\alpha} \\
&= \gamma^\mu \inv{2 \sqrt{x^2}} \partial_\mu (x^\alpha x_\alpha) \\
&= \gamma^\mu \inv{\sqrt{x^2}} x_\mu \\
&= \frac{x}{\Abs{x}}.
\end{aligned}
\end{equation}
%
We therefore have
%
\begin{equation}\label{eqn:lForceLag2:300}
\begin{aligned}
\grad_v \Abs{v}
&= \frac{v}{\Abs{v}} \\
&= \frac{v}{c} .
\end{aligned}
\end{equation}
%
which gives us
\begin{equation}\label{eqn:lForceLag2:320}
\alpha \frac{d(mv/c)}{d\tau} = q F \cdot v/c.
\end{equation}
%
This fixes the constant \(\alpha = c\), and we now have a new form for the Lagrangian
%
\begin{equation}\label{eqn:lForceLag2:340}
\LL = m \Abs{v} c + q A \cdot v/c.
\end{equation}
%
Observe that only after varying the Lagrangian can one make use of the \(\Abs{v} = c\) equality.
%
%TODO: What is the canonical momentum for this Lagrangian?
%
%\bibliographystyle{plainnat}
%\bibliography{myrefs}
%
%\end{document}
