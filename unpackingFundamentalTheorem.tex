%
% Copyright � 2021 Peeter Joot.  All Rights Reserved.
% Licenced as described in the file LICENSE under the root directory of this GIT repository.
%
%{
\input{../latex/blogpost.tex}
\renewcommand{\basename}{unpackingFundamentalTheorem}
%\renewcommand{\dirname}{notes/phy1520/}
\renewcommand{\dirname}{notes/ece1228-electromagnetic-theory/}
%\newcommand{\dateintitle}{}
%\newcommand{\keywords}{}

\input{../latex/peeter_prologue_print2.tex}

\usepackage{peeters_layout_exercise}
\usepackage{peeters_braket}
\usepackage{peeters_figures}
\usepackage{siunitx}
\usepackage{verbatim}
\usepackage{amsthm}
\usepackage{txfonts} % \ointclockwise

\beginArtNoToc

\generatetitle{Unpacking the fundamental theorem}
%\chapter{Unpacking the fundamental theorem}
\label{chap:unpackingFundamentalTheorem}

\section{Two dimensional vector spaces.}
Given a two dimensional generating vector space, we have two cases of the fundamental theorem
\begin{equation}\label{eqn:unpackingFundamentalTheorem:20}
\int_S F d\Bx \lrpartial G = \evalbar{F G}{\Delta S}
\end{equation}
and
\begin{equation}\label{eqn:unpackingFundamentalTheorem:40}
\int_S F d^2\Bx \lrpartial G = \ointclockwise_{\partial S} F d\Bx G.
\end{equation}
The first case is trivial.  Given a parameterizated curve \( x = x(u) \), it just states
\begin{equation}\label{eqn:unpackingFundamentalTheorem:60}
\int_{u(0)}^{u(1)} du \PD{u}{}\lr{FG} = F(u(1))G(u(1)) - F(u(0))G(u(0)),
\end{equation}
for all multivectors \( F, G\), regardless of the signature of the underlying space.

The surface integral is more interesting.  Let's first look at the area element for this surface integral, which is
\begin{equation}\label{eqn:unpackingFundamentalTheorem:80}
d^2 \Bx = d\Bx_u \wedge d \Bx_v.
\end{equation}
Geometrically, this has the area of the parallelogram spanned by \( d\Bx_u \) and \( d\Bx_v \), but weighted by the pseudoscalar of the space.  This is explored algebraically in \cref{problem:unpackingFundamentalTheorem:11} and illustrated in \cref{fig:areaElementParallelograph:areaElementParallelographFig1}.
\imageFigure{../figures/classicalmechanics/areaElementParallelographFig1}{2D vector space and area element.}{fig:areaElementParallelograph:areaElementParallelographFig1}{0.3}
\makeproblem{Expansion of 2D area bivector.}{problem:unpackingFundamentalTheorem:11}{
Let \( \setlr{e_1, e_2} \) be an orthonormal basis for a two dimensional space, with reciprocal frame \( \setlr{e^1, e^2} \).  Expand the area bivector \( d^2 \Bx \) in coordinates relating the bivector to the Jacobian and the pseuduscalar.
} % problem
\makeanswer{problem:unpackingFundamentalTheorem:11}{
With parameterization \( x = x(u,v) = x^\alpha e_\alpha = x_\alpha e^\alpha \), we have
\begin{dmath}\label{eqn:unpackingFundamentalTheorem:120}
\Bx_u \wedge \Bx_v
=
\lr{ \PD{u}{x^\alpha} e_\alpha } \wedge
\lr{ \PD{v}{x^\beta} e_\beta }
=
\PD{u}{x^\alpha}
\PD{v}{x^\beta}
e_\alpha
e_\beta
=
\PD{(u,v)}{(x^1,x^2)} e_1 e_2,
\end{dmath}
or
\begin{dmath}\label{eqn:unpackingFundamentalTheorem:160}
\Bx_u \wedge \Bx_v
=
\lr{ \PD{u}{x_\alpha} e^\alpha } \wedge
\lr{ \PD{v}{x_\beta} e^\beta }
=
\PD{u}{x_\alpha}
\PD{v}{x_\beta}
e^\alpha
e^\beta
=
\PD{(u,v)}{(x_1,x_2)} e^1 e^2.
\end{dmath}
The upper and lower index pseudoscalars are related by
\begin{equation}\label{eqn:unpackingFundamentalTheorem:180}
e^1 e^2 e_1 e_2 =
-e^1 e^2 e_2 e_1 =
-1,
\end{equation}
so with \( I = e_1 e_2 \),
\begin{equation}\label{eqn:unpackingFundamentalTheorem:200}
e^1 e^2 = -I^{-1},
\end{equation}
leaving us with
\begin{equation}\label{eqn:unpackingFundamentalTheorem:140}
d^2 \Bx
= \PD{(u,v)}{(x^1,x^2)} du dv\, I
= -\PD{(u,v)}{(x_1,x_2)} du dv\, I^{-1}.
\end{equation}
We see that the area bivector is proportional to either the upper or lower index Jabobian and to the pseudoscalar for the space.  

We may write the fundamental theorem for a 2D space as
\begin{equation}\label{eqn:unpackingFundamentalTheorem:680}
\int_S du dv \, \PD{(u,v)}{(x^1,x^2)} F I \lrgrad G = \ointclockwise_{\partial S} F d\Bx G,
\end{equation}
where we have dispensed with the partial derivative and use the gradient instead, since they are identical in two dimensions.  Of course, unless we are using \( x^1, x^2 \) as our parameterization, we still want the curvilinear representation of the gradient \( \grad = \Bx^u \PDi{u}{} + \Bx^v \PDi{v}{} \).
%There is an ambiguity to the sense (sign) of the area bivector that depends on the parameterization, and the choice of unit pseudoscalar.
} % answer
\makeproblem{Standard basis expansion of fundamental surface relation.}{problem:unpackingFundamentalTheorem:99}{
For a parameterization \( x = x^1 e_1 + x^2 e_2 \), where \( \setlr{ e_1, e_2 } \) is a standard (orthogonal) basis, expand the fundamental theorem for surface integrals for the single sided \( F = 1 \) case.  Consider functions \( G \) of each grade (scalar, vector, bivector.)
} % problem
\makeanswer{problem:unpackingFundamentalTheorem:99}{
From \cref{problem:unpackingFundamentalTheorem:11}
%\cref{eqn:unpackingFundamentalTheorem:140}
we see that the fundamential theorem takes the form
\begin{equation}\label{eqn:unpackingFundamentalTheorem:220}
\int_S dx^1 dx^2\, F I \lrgrad G = \ointclockwise_{\partial S} F d\Bx G.
\end{equation}
In a Euclidean space, the operator \( I \lrgrad \), is a \( \pi/2 \) rotation of the gradient, but has a rotated like structure in all metrics:
\begin{dmath}\label{eqn:unpackingFundamentalTheorem:240}
I \grad 
=
e_1 e_2 \lr{ e^1 \partial_1 + e^2 \partial_2 }
=
-e_2 \partial_1 + e_1 \partial_2.
\end{dmath}
\begin{itemize}
\item \( F = 1 \) and 
\( G \in \bigwedge^0 \) or 
\( G \in \bigwedge^2 \).
For \( F = 1 \) and scalar or bivector \( G \) we have
\begin{equation}\label{eqn:unpackingFundamentalTheorem:260}
\int_S dx^1 dx^2\, \lr{ -e_2 \partial_1 + e_1 \partial_2 } G = \ointclockwise_{\partial S} d\Bx G,
\end{equation}
where, for 
\( x^1 \in [x^1(0),x^1(1)] \) and 
\( x^2 \in [x^2(0),x^2(1)] \), the RHS written explicitly is
\begin{equation}\label{eqn:unpackingFundamentalTheorem:280}
\ointclockwise_{\partial S} d\Bx G
=
\int dx^1 e_1 
\lr{ G(x^1, x^2(1)) - G(x^1, x^2(0)) }
- dx^2 e_2 
\lr{ G(x^1(1),x^2) - G(x^1(0), x^2) }.
\end{equation}
This is sketched in \cref{fig:lineIntegralAroundRectangle:lineIntegralAroundRectangleFig1}.
Since a 2D bivector \( G \) can be written as \( G = I g \), where \( g \) is a scalar, we may write the pseudoscalar case as
\begin{equation}\label{eqn:unpackingFundamentalTheorem:300}
\int_S dx^1 dx^2\, \lr{ -e_2 \partial_1 + e_1 \partial_2 } g = \ointclockwise_{\partial S} d\Bx g,
\end{equation}
after right multiplying both sides with \( I^{-1} \).  Algebraically the scalar and pseudoscalar cases can be thought of as identical scalar relationships.
\item \( F = 1, G \in \bigwedge^1 \).  For 
For \( F = 1 \) and vector \( G \) the 2D fundamental theorem for surfaces can be split into scalar
\begin{equation}\label{eqn:unpackingFundamentalTheorem:320}
\int_S dx^1 dx^2\, \lr{ -e_2 \partial_1 + e_1 \partial_2 } \cdot G = \ointclockwise_{\partial S} d\Bx \cdot G,
\end{equation}
and bivector relations
\begin{equation}\label{eqn:unpackingFundamentalTheorem:340}
\int_S dx^1 dx^2\, \lr{ -e_2 \partial_1 + e_1 \partial_2 } \wedge G = \ointclockwise_{\partial S} d\Bx \wedge G.
\end{equation}
To expand \cref{eqn:unpackingFundamentalTheorem:320}, let
\begin{equation}\label{eqn:unpackingFundamentalTheorem:360}
G = g_1 e^1 + g_2 e^2,
\end{equation}
for which
\begin{dmath}\label{eqn:unpackingFundamentalTheorem:380}
\lr{ -e_2 \partial_1 + e_1 \partial_2 } \cdot G 
= 
\lr{ -e_2 \partial_1 + e_1 \partial_2 } \cdot 
\lr{ g_1 e^1 + g_2 e^2 }
=
\partial_2 g_1 - \partial_1 g_2,
\end{dmath}
and
\begin{dmath}\label{eqn:unpackingFundamentalTheorem:400}
d\Bx \cdot G
=
\lr{ dx^1 e_1 - dx^2 e_2 } \cdot \lr{ g_1 e^1 + g_2 e^2 }
=
dx^1 g_1 - dx^2 g_2,
\end{dmath}
so
\cref{eqn:unpackingFundamentalTheorem:320} expands to
\boxedEquation{eqn:unpackingFundamentalTheorem:500}{
\int_S dx^1 dx^2\, \lr{ \partial_2 g_1 - \partial_1 g_2 }
=
\int
\evalbar{dx^1 g_1}{\Delta x^2} - \evalbar{ dx^2 g_2 }{\Delta x^1}.
}
This coordinate expansion illustrates how the pseudoscalar nature of the area element results in a duality transformation, as we end up with a curl like 
operation on the LHS, despite the dot product nature of the decomposition that we used.  That can also be seen directly for vector \( G \), since
\begin{dmath}\label{eqn:unpackingFundamentalTheorem:560}
dA (I \grad) \cdot G
=
dA \gpgradezero{ I \grad G }
=
dA I \lr{ \grad \wedge G },
\end{dmath}
since the scalar selection of \( I \lr{ \grad \cdot G } \) is zero.

In the grade-2 relation \cref{eqn:unpackingFundamentalTheorem:340}, we expect a pseudoscalar cancellation on both sides, leaving a scalar (divergence-like) relationship.  This time, we use upper index coordinates for the vector \( G \), letting
\begin{equation}\label{eqn:unpackingFundamentalTheorem:440}
G = g^1 e_1 + g^2 e_2,
\end{equation}
so
\begin{dmath}\label{eqn:unpackingFundamentalTheorem:460}
\lr{ -e_2 \partial_1 + e_1 \partial_2 } \wedge G
=
\lr{ -e_2 \partial_1 + e_1 \partial_2 } \wedge G
\lr{ g^1 e_1 + g^2 e_2 }
=
e_1 e_2 \lr{ \partial_1 g^1 + \partial_2 g^2 },
\end{dmath}
and
\begin{dmath}\label{eqn:unpackingFundamentalTheorem:480}
d\Bx \wedge G
=
\lr{ dx^1 e_1 - dx^2 e_2 } \wedge 
\lr{ g^1 e_1 + g^2 e_2 }
=
e_1 e_2 \lr{ dx^1 g^2 + dx^2 g^1 }.
\end{dmath}
So \cref{eqn:unpackingFundamentalTheorem:340}, after multiplication of both sides by \( I^{-1} \), is
\boxedEquation{eqn:unpackingFundamentalTheorem:520}{
\int_S dx^1 dx^2\, 
\lr{ \partial_1 g^1 + \partial_2 g^2 }
= 
\int
\evalbar{dx^1 g^2}{\Delta x^2} + \evalbar{dx^2 g^1 }{\Delta x^1}.
}
\end{itemize}
As before, we've implicitly performed a duality transformation, and end up with a divergence operation.  That can be seen directly without coordinate expansion, by rewriting the wedge as a grade two selection, and expanding the gradient action on the vector \( G \), as follows
\begin{dmath}\label{eqn:unpackingFundamentalTheorem:580}
dA (I \grad) \wedge G
=
dA \gpgradetwo{ I \grad G }
=
dA I \lr{ \grad \cdot G },
\end{dmath}
since \( I \lr{ \grad \wedge G } \) has only a scalar component.
} % answer
\imageFigure{../figures/classicalmechanics/lineIntegralAroundRectangleFig1}{Line integral around rectangular boundary.}{fig:lineIntegralAroundRectangle:lineIntegralAroundRectangleFig1}{0.3}
\maketheorem{Green's theorem \citep{salas1990coa}.}{thm:unpackingFundamentalTheorem:600}{
Let \( S \) be a Jordan region with a piecewise-smooth boundary \( C \).
If \( P, Q \) are continuously differentiable on an open set that contains \( S \), then
\begin{equation*}
\int dx dy \lr{ \PD{y}{P} - \PD{x}{Q} } = \ointclockwise P dx + Q dy.
\end{equation*}
} % theorem
\makeproblem{Relationship to Green's theorem.}{problem:unpackingFundamentalTheorem:33}{
If the space is Euclidean, show that 
\cref{eqn:unpackingFundamentalTheorem:500} and
\cref{eqn:unpackingFundamentalTheorem:520} are both instances of Green's theorem
with suitable choices of \( P \) and \( Q \).
} % problem
\makeanswer{problem:unpackingFundamentalTheorem:33}{
I will omit the subtlties related to general regions and consider just the case of an infinitesimal square region.
\begin{proof}
Let's start with \cref{eqn:unpackingFundamentalTheorem:500}, with \( g_1 = P \) and \( g_2 = Q \), and \( x^1 = x, x^2 = y \), the RHS is
\begin{dmath}\label{eqn:unpackingFundamentalTheorem:600}
\int dx dy \lr{ \PD{y}{P} - \PD{x}{Q} }.
\end{dmath}
On the RHS we have
\begin{dmath}\label{eqn:unpackingFundamentalTheorem:620}
\int \evalbar{dx P}{\Delta y} - \evalbar{ dy Q }{\Delta x}
=
\int dx \lr{ P(x, y_1) - P(x, y_0) } - \int dy \lr{ Q(x_1, y) - Q(x_0, y) }.
\end{dmath}
This pair of integrals is plotted in \cref{fig:GreensTheoremBoundary:GreensTheoremBoundaryFig1}, from which we see that
\cref{eqn:unpackingFundamentalTheorem:620} can be expressed as the line integral, leaving us with
\begin{dmath}\label{eqn:unpackingFundamentalTheorem:640}
\int dx dy \lr{ \PD{y}{P} - \PD{x}{Q} }
=
\ointclockwise dx P + dy Q,
\end{dmath}
which is Green's theorem over the infinitesimal square integration region.

For the equivalence of \cref{eqn:unpackingFundamentalTheorem:520} to Green's theorem, let \( g^2 = P \), and \( g^1 = -Q \).  Plugging into the LHS, we find the Green's theorem integrand.  On the RHS, the integrand expands to
\begin{dmath}\label{eqn:unpackingFundamentalTheorem:660}
\evalbar{dx g^2}{\Delta y} + \evalbar{dy g^1 }{\Delta x}
=
dx \lr{ P(x,y_1) - P(x, y_0)}
+
dy \lr{ -Q(x_1, y) + Q(x_0, y)},
\end{dmath}
which is exactly what we found in \cref{eqn:unpackingFundamentalTheorem:620}.
\end{proof}
} % answer
\imageFigure{../figures/classicalmechanics/GreensTheoremBoundaryFig1}{Path for Green's theorem.}{fig:GreensTheoremBoundary:GreensTheoremBoundaryFig1}{0.3}

We may also relate multivector gradient integrals in 2D to the normal integral around the boundary of the bounding curve.  That relationship is as follows.
\maketheorem{2D gradient integrals.}{thm:unpackingFundamentalTheorem:44}{
\begin{equation*}
\begin{aligned}
\int J du dv \rgrad G &= \ointclockwise I^{-1} d\Bx G = \int J \lr{ \Bx^v du + \Bx^u dv ) G \\
\int J du dv F \lgrad &= \ointclockwise F I^{-1} d\Bx  = \int J F \lr{ \Bx^v du + \Bx^u dv ),
\end{aligned}
\end{equation*}
where \( J = \partial(x^1, x^2)/\partial(u,v) \) is the Jacobian of the parameterization \( x = x(u,v) \).
In terms of the coordinates \( x^1, x^2 \), this reduces to
\begin{equation*}
\begin{aligned}
\int dx^1 dx^2 \rgrad G &= \ointclockwise \lr{ e^2 dx^1 + e^1 dx^2 } G 
\int dx^1 dx^2 F \lgrad &= \ointclockwise F \lr{ e^2 dx^1 + e^1 dx^2 }.
\end{aligned}
\end{equation*}
We identify \( I^{-1} d\Bx \) as the outwards normal in Euclidean spaces (and is normal for any metric, but a geometrical interpretation in mixed or non-Euclidean spaces can be non-trivial.)
} % theorem

%\section{Three dimensional vector spaces.}

%}
\EndArticle
%\EndNoBibArticle
