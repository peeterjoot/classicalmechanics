%
% Copyright � 2020 Peeter Joot.  All Rights Reserved.
% Licenced as described in the file LICENSE under the root directory of this GIT repository.
%
%{
\input{../latex/blogpost.tex}
\renewcommand{\basename}{fundamentalTheoremOfGC}
%\renewcommand{\dirname}{notes/phy1520/}
\renewcommand{\dirname}{notes/ece1228-electromagnetic-theory/}
%\newcommand{\dateintitle}{}
%\newcommand{\keywords}{}

\input{../latex/peeter_prologue_print2.tex}

\usepackage{txfonts}
\usepackage{peeters_layout_exercise}
\usepackage{peeters_braket}
\usepackage{peeters_figures}
\usepackage{siunitx}
\usepackage{verbatim}
%\usepackage{mhchem} % \ce{}
%\usepackage{macros_bm} % \bcM
%\usepackage{macros_qed} % \qedmarker
%\usepackage{txfonts} % \ointclockwise

\beginArtNoToc

\generatetitle{Fundamental theorem of geometric calculus (relativistic.)}
%\chapter{Fundamental theorem of geometric calculus}
%\label{chap:fundamentalTheoremOfGC}

\section{Motivation.}
I've been slowly working my way towards a statement of the fundamental theorem of integral calculus, where the functions being integrated are elements of the Dirac algebra (space time multivectors in the geometric algebra parlance.)

This is interesting because we want to be able to do line, surface, 3-volume and 4-volume space time integrals.  We have many \R{3} integral theorems
\begin{subequations}
\label{eqn:fundamentalTheoremOfGC:20}
\begin{equation}
\int_A^B d\Bl \cdot \spacegrad f = f(B) - f(A),
\end{equation}
\begin{equation}
\int_S dA\, \ncap \cross \spacegrad f = \ointctrclockwise_{\partial S} d\Bx\, f,
\end{equation}
\begin{equation}
\int_S dA\, \ncap \cdot \lr{ \spacegrad \cross \Bf} = \ointctrclockwise_{\partial S} d\Bx \cdot \Bf,
\end{equation}
\begin{equation}
\int_S dx dy \lr{ \PD{y}{P} - \PD{x}{Q} }
=
\ointclockwise_{\partial S} P dx + Q dy,
\end{equation}
\begin{equation}
\int_V dV\, \spacegrad f = \int_{\partial V} dA\, \ncap f,
\end{equation}
\begin{equation}
\int_V dV\, \spacegrad \cross \Bf = \int_{\partial V} dA\, \ncap \cross \Bf,
\end{equation}
\begin{equation}
\int_V dV\, \spacegrad \cdot \Bf = \int_{\partial V} dA\, \ncap \cdot \Bf,
\end{equation}
\end{subequations}
and want to know how to generalize these to four dimensions and also make sure that we are handling the relativistic mixed signature correctly.  If our starting point was the mess of equations above, we'd be in trouble, since it is not obvious how these generalize.  All the theorems with unit normals have to be handled completely differently in four dimensions since we don't have a unique normal to any given spacetime plane.
What comes to our rescue is the Fundamental Theorem of Geometric Calculus (FTGC), which has the form
\begin{equation}\label{eqn:fundamentalTheoremOfGC:40}
\int F d^n \Bx\, \lrpartial G = \int F d^{n-1} \Bx\, G,
\end{equation}
where \(F,G\) are multivectors functions (i.e. sums of products of vectors.)  We've seen (\citep{aMacdonaldVAGC}, \citep{pjootGAEE}) that all the theorems above are special cases of the fundamental theorem.  Do we need any special care to state the FTGC correctly for our relativistic case?  It turns out that the answer is no!  Tangent and reciprocal frame vectors do all the heavy lifting, and we can use the fundamental theorem as is, even in our mixed signature space.  The only real change that we need to make is use spacetime gradient and vector derivative operators instead of their spatial equivalents.  Let's see how this works.
\section{Multivector integration theory.}
\makedefinition{Vector derivative.}{dfn:fundamentalTheoremOfGC:10}{
Given a spacetime manifold parameterized by \( x = x(u^0, \cdots u^{N-1}) \), with tangent vectors \( \Bx_\mu = \PDi{u^\mu}{x} \), and reciprocal vectors \( \Bx^\mu \in \Span\setlr{\Bx_\nu} \), such that \( \Bx^\mu \cdot \Bx_\nu = {\delta^\mu}_\nu \), the \emph{vector derivative} is defined as
\begin{equation*}
\partial = \sum_{\mu = 0}^{N-1} \Bx^\mu \PD{u^\mu}{}.
\end{equation*}
Observe that if this is a full parameterization of the space (\(N = 4\)), then the vector derivative is identical to the gradient.  The vector derivative is the projection of the gradient onto the tangent space at the point of evaluation.
} % definition

Next, let's define multivector line integrals.  Recall that in \R{3} we would say that for scalar functions \( f\), the integral
\begin{equation*}
\int d\Bx\, f = \int f d\Bx,
\end{equation*}
is a line integral.  Also, for vector functions \( \Bf \) we call
\begin{equation*}
\int d\Bx \cdot \Bf = \inv{2} \int d\Bx\, \Bf + \Bf d\Bx.
\end{equation*}
a line integral.
In order to generalize the idea to line integral, we will allow our multivector functions to be placed on either or both sides of the differential, as in the following.
\makedefinition{Line integral.}{dfn:fundamentalTheoremOfGC:20}{
Given a single variable parameterization \( x = x(u) \), we write \( d^1\Bx = \Bx_u du \), and call
\begin{equation*}
\int F d^1\Bx\, G,
\end{equation*}
a \emph{line integral}, where \( F,G \) are arbitrary multivector functions.
} % definition
We must be careful not to reorder any of the factors in the integrand, since the differential may not commute with either \( F \) or \( G \).  Here is a simple example.
\makeproblem{Circular parameterization.}{problem:fundamentalTheoremOfGC:10}{
Given a circular parameterization \( x(\theta) = \gamma_1 e^{-i\theta} \), where \( i = \gamma_1 \gamma_2 \), the unit bivector for the \(x,y\) plane.  Compute the line integral
\begin{dmath}\label{eqn:fundamentalTheoremOfGC:100}
\int_0^{2\pi} \Bx^\theta d^1 \Bx.
\end{dmath}
} % problem
\makeanswer{problem:fundamentalTheoremOfGC:10}{
The tangent vector for the curve is
\begin{equation}\label{eqn:fundamentalTheoremOfGC:60}
\Bx_\theta
= -\gamma_1 \gamma_1 \gamma_2 e^{-i\theta}
= \gamma_2 e^{-i\theta},
\end{equation}
with reciprocal vector \( \Bx^\theta = e^{i \theta} \gamma^2 \).  The differential element is \( d^1 \Bx = \gamma_2 e^{-i\theta} d\theta \), so the integrand is just a scalar
\begin{equation}\label{eqn:fundamentalTheoremOfGC:80}
\int_0^{2\pi} \Bx^\theta d^1 \Bx
=
\int_0^{2\pi} d\theta
= 2 \pi,
\end{equation}
the circumference of the unit circle.
} % answer

%}
\EndArticle
%\EndNoBibArticle
