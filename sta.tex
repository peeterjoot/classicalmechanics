%
% Copyright © 2020 Peeter Joot.  All Rights Reserved.
% Licenced as described in the file LICENSE under the root directory of this GIT repository.
%
%{
The STA and geometric algebra ideas used here are not complete to learn from in isolation.  The reader is referred to \citep{doran2003gap} for a more complete exposition of both STA and geometric algebra.
\subsection{Conventions.}
\makedefinition{Index conventions.}{dfn:lorentzForceCovariant:23}{
Latin indexes \( i, j, k, r, s, t, \cdots \) are used to designate values in the range \( \setlr{ 1,2,3 } \).  Greek indexes are \( \alpha, \beta, \mu, \nu, \cdots \) are used for indexes of spacetime quantities \( \setlr{0,1,2,3} \).
The Einstein convention of implied summation for mixed upper and lower Greek indexes will be used, for example
\begin{equation*}
x^\alpha x_\alpha \equiv \sum_{\alpha = 0}^3 x^\alpha x_\alpha.
\end{equation*}
} % definition
\subsection{Space Time Algebra (STA.)}
In the geometric algebra literature, the Dirac algebra of quantum field theory has been rebranded Space Time Algebra (STA).
The differences between STA and the Dirac theory that uses matrices (\( \gamma_\mu \)) are as follows
\begin{itemize}
\item STA completely omits any representation of the Dirac basis vectors \( \gamma_\mu \).  In particular, any possible matrix representation is irrelevant.
\item STA provides a rich set of fundamental operations (grade selection, generalized dot and wedge products for multivector elements, rotation and reflection operations, ...)
\item Matrix trace, and commutator and anticommutator operations are nowhere to be found in STA, as geometrically grounded equivalents are available instead.
\item The ``slashed'' quantities from Dirac theory, such as \( \pslash = \gamma_\mu p^\mu \) are nothing more than vectors in their entirety in STA (where the basis is no longer implicit, as is the case for coordinates.)
\end{itemize}
Our basis vectors have the following properties.
\makedefinition{Standard basis.}{dfn:lorentzForceCovariant:17}{
Let the four-vector standard basis be designated
\( \setlr{\gamma_0, \gamma_1, \gamma_2, \gamma_3 } \), where the basis vectors satisfy
\( \gamma_0^2 = -\gamma_i^2 = 1\), and \( \gamma_\alpha \cdot \gamma_\beta = 0, \forall \alpha \ne \beta \).
%We use a \((+,-,-,-)\) basis
} % definition
\makeproblem{Commutator properties of the STA basis.}{problem:lorentzForceCovariant:91}{
In Dirac theory, the commutator properties of the Dirac matrices is considered fundamental, namely
\begin{equation*}
\symmetric{\gamma_\mu}{\gamma_\nu} = 2 \eta_{\mu\nu}.
\end{equation*}
%\begin{equation*}
%\antisymmetric{\gamma_\mu}{\gamma_\nu} = 2 \lr{ \gamma_\mu \gamma_\nu - \eta_{\mu\nu} }.
%\end{equation*}
Show that this follows from the axiomatic assumptions of geometric algebra, and describe how the dot and wedge products are related to the anticommutator and commutator products of Dirac theory.
} % problem
\makeanswer{problem:lorentzForceCovariant:91}{
The anticommutator is defined as symmetric sum of products
\begin{equation}\label{eqn:lorentzForceCovariant:1040}
\symmetric{\gamma_\mu}{\gamma_\nu}
\equiv
\gamma_\mu \gamma_\nu
+
\gamma_\nu \gamma_\mu,
\end{equation}
but this is just twice the dot product in its geometric algebra form \( a b = (a b + ba)/2 \).   Observe that the properties of the basis vectors defined in \cref{dfn:lorentzForceCovariant:17} may be summarized as
\begin{equation}\label{eqn:lorentzForceCovariant:1060}
\gamma_\mu \cdot \gamma_\nu = \eta_{\mu\nu},
\end{equation}
where \( \eta_{\mu\nu} = \diag(+,-,-,-)
=
\begin{bsmallmatrix}
1 & 0 & 0 & 0 \\
0 & -1 & 0 & 0 \\
0 & 0 & -1 & 0 \\
0 & 0 & 0 & -1 \\
\end{bsmallmatrix}
\) is the conventional metric tensor.  This means
\begin{equation}\label{eqn:lorentzForceCovariant:1080}
\gamma_\mu \cdot \gamma_\nu = \eta_{\mu\nu} = 2 \symmetric{\gamma_\mu}{\gamma_\nu},
\end{equation}
as claimed.

Similarly, observe that the commutator, defined as the antisymmetric sum of products
\begin{equation}\label{eqn:lorentzForceCovariant:1100}
\antisymmetric{\gamma_\mu}{\gamma_\nu} \equiv
\gamma_\mu \gamma_\nu
-
\gamma_\nu \gamma_\mu,
\end{equation}
is twice the wedge product \( a \wedge b = (a b - b a)/2 \).
This provides geometric identifications for the respective anti-commutator and commutator products respectively
\begin{equation}\label{eqn:lorentzForceCovariant:1120}
\begin{aligned}
\symmetric{\gamma_\mu}{\gamma_\nu} &= 2 \gamma_\mu \cdot \gamma_\nu \\
\antisymmetric{\gamma_\mu}{\gamma_\nu} &= 2 \gamma_\mu \wedge \gamma_\nu,
\end{aligned}
\end{equation}
} % answer
\makedefinition{Pseudoscalar.}{dfn:lorentzForceCovariant:540}{
The pseudoscalar for the space is denoted \( I = \gamma_0 \gamma_1 \gamma_2 \gamma_3 \).
} % definition
\makeproblem{Pseudoscalar.}{problem:lorentzForceCovariant:99}{
% use parts.
Show that the STA pseudoscalar \( I \) defined by \cref{dfn:lorentzForceCovariant:17} satisfies
\begin{equation*}
\tilde{I} = I,
\end{equation*}
where the tilde operator designates reversion.
Also show that \( I \) has the properties of an imaginary number
\begin{equation*}
I^2 = -1.
\end{equation*}
Finally, show that, unlike the spatial pseudoscalar that commutes with all grades, \( I \) anticommutes with any vector or trivector, and commutes with any bivector.
} % problem
\makeanswer{problem:lorentzForceCovariant:99}{
Since
\( \gamma_\alpha \gamma_\beta = -\gamma_\beta \gamma_\alpha \)
for any
\( \alpha \ne \beta \)
, any permutation of the factors of \( I \) changes the sign once.  In particular
\begin{equation}\label{eqn:lorentzForceCovariant:680}
\begin{aligned}
I
&=
\gamma_0
\gamma_1
\gamma_2
\gamma_3 \\
&=
-
\gamma_1
\gamma_2
\gamma_3
\gamma_0 \\
&=
-
\gamma_2
\gamma_3
\gamma_1
\gamma_0 \\
&=
+
\gamma_3
\gamma_2
\gamma_1
\gamma_0
= \tilde{I}.
\end{aligned}
\end{equation}
Using this, we have
\begin{equation}\label{eqn:lorentzForceCovariant:700}
\begin{aligned}
I^2
&= I \tilde{I} \\
&=
(
\gamma_0
\gamma_1
\gamma_2
\gamma_3
)(
\gamma_3
\gamma_2
\gamma_1
\gamma_0
) \\
&=
\lr{\gamma_0}^2
\lr{\gamma_1}^2
\lr{\gamma_2}^2
\lr{\gamma_3}^2 \\
=
(+1)
(-1)
(-1)
(-1) \\
&= -1.
\end{aligned}
\end{equation}
To illustrate the anticommutation property with any vector basis element, consider the following two examples:
\begin{equation}\label{eqn:lorentzForceCovariant:720}
\begin{aligned}
I \gamma_0
&=
\gamma_0
\gamma_1
\gamma_2
\gamma_3
\gamma_0 \\
&=
-
\gamma_0
\gamma_0
\gamma_1
\gamma_2
\gamma_3 \\
&=
-
\gamma_0 I,
\end{aligned}
\end{equation}
\begin{equation}\label{eqn:lorentzForceCovariant:740}
\begin{aligned}
   I \gamma_2
&=
\gamma_0
\gamma_1
\gamma_2
\gamma_3
\gamma_2 \\
&=
-
\gamma_0
\gamma_1
\gamma_2
\gamma_2
\gamma_3 \\
&=
-
\gamma_2
\gamma_0
\gamma_1
\gamma_2
\gamma_3 \\
&= -\gamma_2 I.
\end{aligned}
\end{equation}
A total of three sign swaps is required to ``percolate'' any given \(\gamma_\alpha\) through the factors of \( I \), resulting in an overall sign change of \( -1 \).

For any bivector basis element \( \alpha \ne \beta \)
\begin{equation}\label{eqn:lorentzForceCovariant:760}
\begin{aligned}
   I \gamma_\alpha \gamma_\beta
   &= -\gamma_\alpha I \gamma_\beta \\
   &= +\gamma_\alpha \gamma_\beta I.
\end{aligned}
\end{equation}

Similarly
for any trivector basis element \( \alpha \ne \beta \ne \sigma \)
\begin{equation}\label{eqn:lorentzForceCovariant:780}
\begin{aligned}
   I \gamma_\alpha \gamma_\beta \gamma_\sigma
   &= -\gamma_\alpha I \gamma_\beta \gamma_\sigma \\
   &= +\gamma_\alpha \gamma_\beta I \gamma_\sigma \\
   &= -\gamma_\alpha \gamma_\beta \gamma_\sigma I.
\end{aligned}
\end{equation}
} % answer
\makedefinition{Reciprocal basis.}{dfn:lorentzForceCovariant:19}{
The reciprocal basis \( \setlr{ \gamma^0, \gamma^1, \gamma^2, \gamma^3 } \) is defined , such that the property \( \gamma^\alpha \cdot \gamma_\beta = {\delta^\alpha}_\beta \) holds.
} % definition
Observe that, \( \gamma^0 = \gamma_0 \) and \( \gamma^i = -\gamma_i \).
\maketheorem{Coordinates.}{thm:lorentzForceCovariant:800}{
Coordinates are defined in terms of dot products with the standard basis, or reciprocal basis
\begin{equation*}
\begin{aligned}
x^\alpha &= x \cdot \gamma^\alpha \\
x_\alpha &= x \cdot \gamma_\alpha,
\end{aligned}
\end{equation*}
} % theorem
\begin{proof}
Suppose that a coordinate representation of the following form is assumed
\begin{equation}\label{eqn:lorentzForceCovariant:820}
x = x^\alpha \gamma_\alpha = x_\beta \gamma^\beta.
\end{equation}
We wish to determine the representation of the \( x^\alpha \) or \( x_\beta \) coordinates in terms of \( x\) and the basis elements.  Taking the dot product with any standard basis element, we find
\begin{equation}\label{eqn:lorentzForceCovariant:840}
\begin{aligned}
x \cdot \gamma_\mu
&= (x_\beta \gamma^\beta) \cdot \gamma_\mu \\
&= x_\beta {\delta^\beta}_\mu \\
&= x_\mu,
\end{aligned}
\end{equation}
as claimed.  Similarly, dotting with a reciprocal frame vector, we find
\begin{equation}\label{eqn:lorentzForceCovariant:860}
\begin{aligned}
x \cdot \gamma^\mu
&= (x^\beta \gamma_\beta) \cdot \gamma^\mu \\
&= x^\beta {\delta_\beta}^\mu \\
&= x^\mu.
\end{aligned}
\end{equation}
\end{proof}
Observe that raising or lowering the index of a spatial index toggles the sign of a coordinate, but timelike indexes are left unchanged.
\begin{equation}\label{eqn:lorentzForceCovariant:880}
\begin{aligned}
x^0 &= x_0 \\
x^i &= -x_i \\
\end{aligned}
\end{equation}

%Note that the slash notation from Dirac theory, such as \( \pslash = \gamma_\mu p^\mu \) is nothing more than the coordinate free vector representation used in STA (i.e. vectors are really the entire ``slashed'' quantities from Dirac theory, and we omit the
\makedefinition{Spacetime gradient.}{dfn:lorentzForceCovariant:71}{
The spacetime gradient operator is
\begin{equation*}
\grad = \gamma^\mu \partial_\mu = \gamma_\nu \partial^\nu,
\end{equation*}
where
\begin{equation*}
\partial_\mu = \PD{x^\mu}{},
\end{equation*}
and
\begin{equation*}
\partial^\mu = \PD{x_\mu}{}.
\end{equation*}
} % definition
This definition of gradient is consistent with the Dirac gradient (sometimes denoted \(\partialslash\)).
%
%Next we want to establish correspondence with the traditional vector form of the Lorentz force equation.

%In conventional special relativity, pairs of interrelated timelike and spacelike quantities are often designated as tuples with scalar and vector components
%\begin{equation*}
%\begin{aligned}
%&\setlr{ c, \Bv } \\
%&\setlr{ \calE/c, \Bp } \\
%&\setlr{ \omega, \Bk c } \\
%\end{aligned}
%\end{equation*}
%In STA, this decomposition is obtained by taking dot and wedge products with the fixed observer frame vector.
\makedefinition{Timelike and spacelike components of a four-vector.}{dfn:lorentzForceCovariant:980}{
Given a four vector \( x = \gamma_\mu x^\mu \), that would be designated \( x^\mu = \setlr{ x^0, \Bx} \) in conventional special relativity, we write
\begin{equation*}
x^0 = x \cdot \gamma_0,
\end{equation*}
and
\begin{equation*}
\Bx = x \wedge \gamma_0,
\end{equation*}
or
\begin{equation*}
x = (x^0 + \Bx) \gamma_0.
\end{equation*}
} % definition
The spacetime split of a four-vector \( x \) is relative to the frame.  In the relativistic lingo, one would say that it is ``observer dependent'',
as the same operations with \( {\gamma_0}' \), the timelike basis vector for a different frame,
would yield a different set of coordinates.
%A spacetime split that utilizes the timelike standard basis vector \( \gamma_0 \) provides the view of the coordinates from a fixed observer point of view, with worldline \( c \tau \gamma_0 \).

While the dot and wedge products above provide an effective mechanism to split a four vector into a set of timelike and spacelike quantities, the spatial component of a vector has a bivector representation in STA.  Consider the following coordinate expansion of a spatial vector
\begin{equation}\label{eqn:lorentzForceCovariant:1000}
\begin{aligned}
\Bx
&= x \wedge \gamma_0 \\
&= \lr{ x^\mu \gamma_\mu } \wedge \gamma_0 \\
&= \sum_{k = 1}^3 x^k \gamma_k \gamma_0.
\end{aligned}
\end{equation}
\makedefinition{Spatial basis.}{dfn:lorentzForceCovariant:77}{
We designate \( \Be_i = \gamma_i \gamma_0 \) as the standard basis vectors for \R{3}.
} % definition
In the literature, this bivector representation of the spatial basis may be designated \( \sigma_i = \gamma_i \gamma_0 \), as these bivectors have the properties of the Pauli matrices \( \sigma_i \).
Because this book a number of purely non-relativistic applications too, the Pauli notation will not be used here.
\makeproblem{Orthonormality of the spatial basis.}{problem:lorentzForceCovariant:33}{
Show that the spatial basis \( \setlr{ \Be_1, \Be_2, \Be_3 } \), defined by \cref{dfn:lorentzForceCovariant:77}, is orthonormal.
} % problem
\makeanswer{problem:lorentzForceCovariant:33}{
\begin{equation}\label{eqn:lorentzForceCovariant:620}
\begin{aligned}
\Be_i \cdot \Be_j
&= \gpgradezero{ \gamma_i \gamma_0 \gamma_j \gamma_0 } \\
&= -\gpgradezero{ \gamma_i \gamma_j } \\
&= - \gamma_i \cdot \gamma_j.
\end{aligned}
\end{equation}
This is zero for all \( i \ne j \), and unity for any \( i = j \).
} % answer
\makeproblem{Spatial pseudoscalar.}{problem:lorentzForceCovariant:11}{
Show that the STA pseudoscalar \( I = \gamma_0 \gamma_1 \gamma_2 \gamma_3 \) equals the spatial pseudoscalar \( I = \Be_1 \Be_2 \Be_3 \).
} % problem
\makeanswer{problem:lorentzForceCovariant:11}{
The spatial pseudoscalar, expanded in terms of the STA basis vectors, is
\begin{equation}\label{eqn:lorentzForceCovariant:1020}
\begin{aligned}
I
&= \Be_1 \Be_2 \Be_3 \\
&= \lr{ \gamma_1 \gamma_0 }
  \lr{ \gamma_2 \gamma_0 }
  \lr{ \gamma_3 \gamma_0 } \\
&= \lr{ \gamma_1 \gamma_0 } \gamma_2 \lr{ \gamma_0 \gamma_3 } \gamma_0 \\
&= \lr{ -\gamma_0 \gamma_1 } \gamma_2 \lr{ -\gamma_3 \gamma_0 } \gamma_0 \\
&=       \gamma_0 \gamma_1 \gamma_2 \gamma_3 \lr{ \gamma_0 \gamma_0 } \\
&= \gamma_0 \gamma_1 \gamma_2 \gamma_3,
\end{aligned}
\end{equation}
as claimed.
} % answer
\makeproblem{Characteristics of the Pauli matrices.}{problem:lorentzForceCovariant:7}{
The Pauli matrices obey the following anticommutation relations:
\begin{equation}\label{eqn:lorentzForceCovariant:660}
\symmetric{ \sigma_a}{\sigma_b } = 2 \delta_{a b},
\end{equation}
and commutation relations:
\begin{equation}\label{eqn:lorentzForceCovariant:640}
\antisymmetric{ \sigma_a}{ \sigma_b } = 2 i \epsilon_{a b c}\,\sigma_c,
\end{equation}
Show how these relate to the geometric algebra dot and wedge products, and determine the geometric algebra representation of the imaginary \( i \) above.
} % problem
%\makeanswer{problem:lorentzForceCovariant:7}{
%TODO.
%} % answer
%\subsubsection{Solutions.}
%\shipoutAnswer
%}
