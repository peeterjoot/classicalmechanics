%
% Copyright � 2021 Peeter Joot.  All Rights Reserved.
% Licenced as described in the file LICENSE under the root directory of this GIT repository.
%
%{
\input{../latex/blogpost.tex}
\renewcommand{\basename}{volumeAndFourVolume}
%\renewcommand{\dirname}{notes/phy1520/}
\renewcommand{\dirname}{notes/ece1228-electromagnetic-theory/}
%\newcommand{\dateintitle}{}
%\newcommand{\keywords}{}

\input{../latex/peeter_prologue_print2.tex}

\usepackage{peeters_layout_exercise}
\usepackage{peeters_braket}
\usepackage{peeters_figures}
\usepackage{siunitx}
\usepackage{verbatim}
\usepackage{amsthm}

\definecolor{BlueDarker}{HTML}{0000AA}
\definecolor{RedDarker}{HTML}{AA0000}
\definecolor{PurpleDarker}{HTML}{550055}
\definecolor{OrangeDarker}{HTML}{AA5500}
\definecolor{GreenDarker}{HTML}{00AA00}
\definecolor{YellowDarker}{HTML}{AAAA00}

\beginArtNoToc

\generatetitle{Fundamental theorem for volume and four-volume integrals.}
%\chapter{Fundamental theorem for volume and four-volume integrals.}
%\label{chap:volumeAndFourVolume}
\section{Three volume integrals.}
\maketheorem{Fundamental theorem for volume integrals.}{thm:volumeAndFourVolume:11}{
Given multivector functions \( F, G \),
and
a three parameter curve \( x(u^1,u^2,u^3) \),
\begin{equation*}
\int_S F d^3\Bx \lrpartial G = \int_{\partial S} F d^2\Bx G,
\end{equation*}
with
volume element \( d^3 \Bx = \lr{ d\Bx_1 \wedge d\Bx_2 \wedge d\Bx_3 }\), and
with bounding surface
\begin{equation*}
d^2 x = d \Bx_1 \wedge d\Bx_2 + d \Bx_3 \wedge d\Bx_1 + d\Bx_2 \wedge d\Bx_3,
\end{equation*}
where the differentials are given by \( d\Bx_k = du^k \Bx_k \) (no sum), and
where
\( S \) is the integration volume,
and \( \partial S \) designates its boundary.
} % theorem
\begin{proof}
Let's write \( \Bx_{ij} = \Bx_i \wedge \Bx_j \), \( \Bx_{ijk} = \Bx_i \wedge \Bx_j \wedge \Bx_k \), and
\( \partial_k = \PDi{u^k}{} \).
We can expand the three-volume integral as
\begin{dmath}\label{eqn:volumeAndFourVolume:60}
\int_S F d^3\Bx \lrpartial G
=
\int_S F d^3\Bx \cdot \lrpartial G
=
\int_S
du^1 du^2 du^3\,
F
\Bx_{123} \cdot \lr{ \Bx^1 \lrpartial_1 + \Bx^2 \lrpartial_2 + \Bx^3 \lrpartial_3 }
G
=
\int_S
du^1 du^2 du^3
\,
F
\lr{
\Bx_{12} \lrpartial_3
-
\Bx_{13} \lrpartial_2
+
\Bx_{23} \lrpartial_1
}
G
=
\int_S
du^1 du^2 du^3
\,
\lr{
	\partial_3 \lr{ F \Bx_{12} G }
	-
	\partial_2 \lr{ F \Bx_{13} G }
	+
	\partial_1 \lr{ F \Bx_{23} G }
}
-
\int_S
du^1 du^2 du^3
\,
\lr{
	F \lr{ \partial_3 \Bx_{12} } G
	-
	F \lr{ \partial_2 \Bx_{13} } G
	+
	F \lr{ \partial_1 \Bx_{23} } G
}
.
\end{dmath}
This is an integration by parts style expansion, where we pull out a set of perfect differentials, and hope that antisymmetry kills off the second integral, as it did for the line integral case.  Let's verify the latter
\begin{dmath}\label{eqn:volumeAndFourVolume:80}
\partial_3 \Bx_{12}
-
\partial_2 \Bx_{13}
+
\partial_1 \Bx_{23}
=
  \lr{ \partial_3 \Bx_{1} } \wedge \Bx_{2} + \Bx_{1} \wedge {\partial_3 \Bx_{2} }
- \lr{ \partial_2 \Bx_{1} } \wedge \Bx_{3} - \Bx_{1} \wedge {\partial_2 \Bx_{3} }
+ \lr{ \partial_1 \Bx_{2} } \wedge \Bx_{3} + \Bx_{2} \wedge {\partial_1 \Bx_{3} }
=
  \lr{ \partial_{13} x } \wedge {\color{BlueDarker}\Bx_{2}}  + {\color{RedDarker}\Bx_{1}} \wedge \lr{\partial_{32} x }
- \lr{ \partial_{12} x } \wedge {\color{GreenDarker}\Bx_{3}} - {\color{RedDarker}\Bx_{1}} \wedge \lr{\partial_{23} x }
+ \lr{ \partial_{12} x } \wedge {\color{GreenDarker}\Bx_{3}} + {\color{BlueDarker}\Bx_{2}} \wedge \lr{\partial_{13} x },
\end{dmath}
where
\begin{dmath}\label{eqn:volumeAndFourVolume:40}
\partial_{jk}x = \PDD{u^k}{u^j}{x}.
\end{dmath}
Assuming equality of mixed partials, everything in \cref{eqn:volumeAndFourVolume:80} cancels perfectly.  We are left with
\begin{dmath}\label{eqn:volumeAndFourVolume:100}
\int_S F d^3\Bx \lrpartial G
=
\int_S
du^1 du^2 du^3
\,
\lr{
	\partial_3 \lr{ F \Bx_{12} G }
	-
	\partial_2 \lr{ F \Bx_{13} G }
	+
	\partial_1 \lr{ F \Bx_{23} G }
}
=
\int
\,
\lr{
  du^1 du^2 \evalbar{\lr{F \Bx_{12} G}}{\Delta u^3 }
- du^1 du^3 \evalbar{\lr{F \Bx_{13} G}}{\Delta u^2 }
+ du^2 du^3 \evalbar{\lr{F \Bx_{23} G}}{\Delta u^1 }
}
=
\int
\,
\lr{
  \evalbar{\lr{F d\Bx_{1} \wedge d\Bx_{2} G}}{\Delta u^3 }
+ \evalbar{\lr{F d\Bx_{3} \wedge d\Bx_{1} G}}{\Delta u^2 }
+ \evalbar{\lr{F d\Bx_{2} \wedge d\Bx_{3} G}}{\Delta u^1 }
}
.
\end{dmath}
This is an explicit representation of what we mean by \( \int_{\partial S} F d^2 \Bx G \).
\end{proof}
\section{Four volume integrals.}
\maketheorem{Fundamental theorem for four-volume integrals.}{thm:volumeAndFourVolume:13}{
Given multivector functions \( F, G \),
and
a four parameter curve \( x(u^1,u^2,u^3,u^4) \),
\begin{equation*}
\int_S F d^4\Bx \lrpartial G = \int_{\partial S} F d^3\Bx G,
\end{equation*}
with
four-volume element \( d^3 \Bx = \lr{ d\Bx_1 \wedge d\Bx_2 \wedge d\Bx_3 \wedge d\Bx_4 }\), and
with bounding surface
\begin{equation*}
d^3 x
=
- d \Bx_2 \wedge d\Bx_3 \wedge d\Bx_4 % 1
+ d \Bx_1 \wedge d\Bx_3 \wedge d\Bx_4 % 2
- d \Bx_1 \wedge d\Bx_2 \wedge d\Bx_4 % 3
+ d \Bx_1 \wedge d\Bx_2 \wedge d\Bx_3 % 4
,
\end{equation*}
where the differentials are given by \( d\Bx_k = du^k \Bx_k \) (no sum), and
where
\( S \) is the integration volume,
and \( \partial S \) designates its boundary.
} % theorem
\begin{proof}
As before we will write \( \Bx_{ijk} = \Bx_i \wedge \Bx_j \wedge \Bx_k \), and
\( \partial_k = \PDi{u^k}{} \).
We can expand the four-volume integral as
\begin{dmath}\label{eqn:volumeAndFourVolume:120}
\int_S F d^3\Bx \lrpartial G
=
\int_S F d^3\Bx \cdot \lrpartial G
=
\int_S
du^1 du^2 du^3 du^4 \,
F
\Bx_{1234} \cdot \lr{
  \Bx^1 \lrpartial_1
+ \Bx^2 \lrpartial_2
+ \Bx^3 \lrpartial_3
+ \Bx^4 \lrpartial_4
}
G
=
\int_S
du^1 du^2 du^3 du^4
\,
F
\lr{
- \Bx_{234} \lrpartial_1
+ \Bx_{134} \lrpartial_2
- \Bx_{124} \lrpartial_3
+ \Bx_{123} \lrpartial_4
}
G
=
\int_S
du^1 du^2 du^3 du^4
\,
\lr{
- \partial_1 \lr{ F \Bx_{234} G }
+ \partial_2 \lr{ F \Bx_{134} G }
- \partial_3 \lr{ F \Bx_{124} G }
+ \partial_4 \lr{ F \Bx_{123} G }
}
-
\int_S
du^1 du^2 du^3 du^4
\,
F \lr{
- \partial_1 \Bx_{234}
+ \partial_2 \Bx_{134}
- \partial_3 \Bx_{124}
+ \partial_4 \Bx_{123}
} G
.
\end{dmath}
As before, this is an integration by parts style expansion, where we pull out a set of perfect differentials, with the expectation that the partials of all the wedges to sum to zero
\begin{dmath}\label{eqn:volumeAndFourVolume:140}
\begin{aligned}
&- \partial_1 \Bx_{234}
+ \partial_2 \Bx_{134}
- \partial_3 \Bx_{124}
+ \partial_4 \Bx_{123} \\
&=
- \partial_{12} x \wedge {\color{BlueDarker}\Bx_{34} }
+ \partial_{21} x \wedge {\color{BlueDarker}\Bx_{34} }
- \partial_{31} x \wedge {\color{GreenDarker}\Bx_{24} }
+ \partial_{41} x \wedge {\color{RedDarker}\Bx_{23} } \\
&\quad + \partial_{13} x \wedge {\color{GreenDarker}\Bx_{24} }
- \partial_{23} x \wedge {\color{PurpleDarker}\Bx_{14} }
+ \partial_{32} x \wedge {\color{PurpleDarker}\Bx_{14} }
- \partial_{42} x \wedge {\color{OrangeDarker}\Bx_{13} } \\
&\quad - \partial_{14} x \wedge {\color{RedDarker}\Bx_{23} }
+ \partial_{24} x \wedge {\color{OrangeDarker}\Bx_{13} }
- \partial_{34} x \wedge {\color{YellowDarker}\Bx_{12} }
+ \partial_{43} x \wedge {\color{YellowDarker}\Bx_{12} }
\end{aligned}
\end{dmath}
Again, due to
equality of mixed partials, everything in \cref{eqn:volumeAndFourVolume:140} cancels out perfectly, leaving
\begin{dmath}\label{eqn:volumeAndFourVolume:180}
\int_S F d^4\Bx \lrpartial G
=
\int
\lr{
- \evalbar{ F \,d\Bx_2 \wedge d\Bx_3 \wedge d\Bx_4 \,G }{\Delta u^1}
+ \evalbar{ F \,d\Bx_1 \wedge d\Bx_3 \wedge d\Bx_4 \,G }{\Delta u^2}
- \evalbar{ F \,d\Bx_1 \wedge d\Bx_2 \wedge d\Bx_4 \,G }{\Delta u^3}
+ \evalbar{ F \,d\Bx_1 \wedge d\Bx_2 \wedge d\Bx_3 \,G }{\Delta u^4}
}
\end{dmath}
This is the explicit representation of \( \int_{\partial S} F d^3 \Bx G \).
\end{proof}

%}
%\EndArticle
\EndNoBibArticle
