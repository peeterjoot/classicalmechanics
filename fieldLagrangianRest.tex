%
% Copyright © 2012 Peeter Joot.  All Rights Reserved.
% Licenced as described in the file LICENSE under the root directory of this GIT repository.
%
%\subsubsection{Electrodynamic Potential Wave Equation.}
%
\subsubsection{Schr\"{o}dinger's equation}
%
Problem \(11.3\) in \citep{goldstein1951cm} is to take the Lagrangian
%
\begin{equation}\label{eqn:fieldLagrangian:540}
\begin{aligned}
\LL
&= \frac{\Hbar^2}{2 m} \spacegrad \psi \cdot \spacegrad \psi^\conj + V \psi \psi^\conj + \frac{\Hbar}{2i}
\lr{  \psi^\conj \partial_t \psi - \psi \partial_t \psi^\conj  }
 \\
&= \frac{\Hbar^2}{2 m} \partial_k \psi \partial_k \psi^\conj + V \psi \psi^\conj +
\frac{\Hbar}{2i}
\lr{  \psi^\conj \partial_t \psi - \psi \partial_t \psi^\conj  }
 .
\end{aligned}
\end{equation}
%
treating \(\psi\), and \(\psi^\conj\) as separate fields and show that Schr\"{o}dinger's equation and its conjugate follows.  (note: I have added a 1/2 fact in the
commutator term that was not in the Goldstein problem.  Believe that to have been a typo in the original (first edition)).
%
We have
\begin{equation}\label{eqn:fieldLagrangian:560}
\PD{\psi^\conj}{\LL} = V\psi + \frac{\Hbar}{2i} \partial_t \psi .
\end{equation}
%
and canonical momenta
\begin{equation}\label{eqn:fieldLagrangian:580}
\begin{aligned}
\PD{
\lr{ \partial_m \psi^\conj }
}{\LL} &= \frac{\Hbar^2}{2 m} \partial_{m} \psi \\
\PD{
\lr{ \partial_t \psi^\conj }
}{\LL} &= -\frac{\Hbar}{2i} {\psi} .
\end{aligned}
\end{equation}
%
\begin{equation}\label{eqn:fieldLagrangian:600}
\begin{aligned}
\PD{\psi^\conj}{\LL} &= \sum_m \partial_m \PD{
\lr{ \partial_m \psi^\conj }
}{\LL} + \partial_t \PD{
\lr{ \partial_t \psi^\conj }
}{\LL} \\
V\psi + \frac{\Hbar}{2i} \partial_t \psi &= \frac{\Hbar^2}{2 m} \sum_m \partial_{mm} \psi -\frac{\Hbar}{2i} \PD{t}{\psi} .
\end{aligned}
\end{equation}
%
which is the desired result
\begin{equation}\label{eqn:fieldLagrangian:620}
-\frac{\Hbar^2}{2 m} \spacegrad^2 \psi + V\psi = {\Hbar i}{} \PD{t}{\psi} .
\end{equation}
%
The conjugate result
%
%\begin{align*}
%\PD{\psi}{\LL} &= V\psi^\conj - \frac{\Hbar}{2i} \partial_t \psi^\conj \\
%\partial_m \PD{(\partial_m \psi)}{\LL} &=
%\frac{\Hbar^2}{2 m}
%\partial_{mm} \psi^\conj \\
%\partial_t \PD{(\partial_t \psi)}{\LL} &= \frac{\Hbar}{2i} \partial_{t}{\psi^\conj} \\
%\implies \\
%V\psi^\conj - \frac{\Hbar}{2i} \partial_t \psi^\conj &=
%\frac{\Hbar^2}{2 m}
%\sum_m \partial_{mm} \psi^\conj +\frac{\Hbar}{2i} \PD{t}{\psi^\conj} \\
%\end{align*}
%
%which is
\begin{equation}\label{eqn:fieldLag:schrod}
-\frac{\Hbar^2}{2 m} \spacegrad^2 \psi^\conj + V\psi^\conj = -{\Hbar i}{} \PD{t}{\psi^\conj} .
\end{equation}
%
follows by inspection since all terms except the time partial are symmetric in \(\psi\) and \(\psi^\conj\).  The time partial has a negation in sign from the commutator of the Lagrangian.

FIXME: Goldstein also wanted the Hamiltonian, but I do not know what that is yet.  Got to go read the earlier parts of the book!
%
\subsubsection{Relativistic Schr\"{o}dinger's equation}
%
The
\href{https://en.wikipedia.org/wiki/Noether%27s_theorem}{wiki article on Noether's theorem} lists the relativistic quantum Lagrangian in the form
%
\begin{equation}\label{eqn:fieldLagrangian:640}
\LL = -\eta^{\mu\nu}\partial_\mu \psi \partial_\nu \psi^\conj + \frac{m^2 c^2}{\Hbar^2}\psi \psi^\conj.
\end{equation}
%
That article uses \(\Hbar = c = 1\), and appears to use a \(-+++\) metric, both
of which are adjusted for here.
%
Calculating the derivatives
%
\begin{equation}\label{eqn:fieldLagrangian:660}
\PD{\psi^\conj}{\LL} = \frac{m^2 c^2}{\Hbar^2} \psi.
\end{equation}
%
\begin{equation}\label{eqn:fieldLagrangian:680}
\begin{aligned}
\partial_\mu \PD{
\lr{ \partial_\mu \psi^\conj }
}{\LL}
&= -\partial_\mu \left(\eta^{\alpha\beta}\partial_\alpha \psi \PD{
\lr{ \partial_\mu \psi^\conj }
}{} \partial_\beta \psi^\conj\right) \\
&= -\partial_\mu
\lr{ \eta^{\alpha\mu}\partial_\alpha \psi } \\
&= -\partial_\mu \partial^\mu \psi .
\end{aligned}
\end{equation}
%
So we have
\begin{equation}\label{eqn:fieldLagrangian:700}
\partial_\mu \partial^\mu \psi = \frac{-m^2 c^2}{\Hbar^2} \psi.
\end{equation}
%
%There is an ambiguity here
%since it is not specified what signature metric tensor \(\eta\) is used.  The Klein-Gordon wikipedia page uses an equivalent action, and the talk
%page for that indicates they use \(\eta^{00} = -1/c^2\).  In this case we have
%
%\begin{align*}
%-\grad^2 \psi = \frac{m^2 c^2}{\Hbar^2} \psi
%\end{align*}
%
%which is also consistent with the Klein-Gordon equation as given in that same page:
%
%
With the metric dependency made explicit this is
\begin{equation}\label{eqn:fieldLagrangian:720}
\left(\spacegrad^2 - \inv{c^2}\PDsq{t}{}\right) \psi = \frac{m^2 c^2}{\Hbar^2} \psi.
\end{equation}
%
Much different looking than the classical time dependent Schr\"{o}dinger's equation in \eqnref{eqn:fieldLag:schrod}.
\citep{srednicki2007qft} has a nice discussion about this equation and its relation to the non-relativistic Schr\"{o}dinger's equation.
