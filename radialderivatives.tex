%
% Copyright � 2023 Peeter Joot.  All Rights Reserved.
% Licenced as described in the file LICENSE under the root directory of this GIT repository.
%
%{
%\input{../latex/blogpost.tex}
%\renewcommand{\basename}{radialderivatives}
%%\renewcommand{\dirname}{notes/phy1520/}
%\renewcommand{\dirname}{notes/ece1228-electromagnetic-theory/}
%%\newcommand{\dateintitle}{}
%%\newcommand{\keywords}{}
%
%\input{../latex/peeter_prologue_print2.tex}
%
%\usepackage{peeters_layout_exercise}
%\usepackage{peeters_braket}
%\usepackage{peeters_macros}
%\usepackage{peeters_figures}
%\usepackage{siunitx}
%\usepackage{verbatim}
%\usepackage{amsthm}
%%\usepackage{mhchem} % \ce{}
%%\usepackage{macros_bm} % \bcM
%%\usepackage{macros_qed} % \qedmarker
%%\usepackage{txfonts} % \ointclockwise
%
%\beginArtNoToc
%
%\generatetitle{Radial vector representation, momentum, and angular momentum.}
%%\chapter{Radial derivatives.}
%%\label{chap:radialderivatives}
%
\section{Radial vector representation, momentum, and angular momentum.}
\subsection{Motivation.}
%In my last couple GA YouTube videos, circular and spherical coordinates were examined.  We found the form of the unit vector derivatives in both cases.
In contrast to specific parameterizations, such as circular and spherical coordinates, let's now look at a more general case, where we write
\begin{equation}\label{eqn:radialderivatives:20}
\Bx = r \rcap,
\end{equation}
leaving the angular dependence of \( \rcap \) unspecified.  We want to find both \( \Bv = \Bx' \) and \( \rcap'\), and apply this to computation of the momentum of a point particle, also finding the kinetic energy of that particle.  This will expose a radial component and an angular momentum dependent term in both the momentum and the energy.
\subsection{Derivatives.}
\makelemma{Radial length derivative.}{lemma:radialderivatives:40}{
The derivative of a spherical length \( r \) can be expressed as
\begin{equation*}
\frac{dr}{dt} = \rcap \cdot \frac{d\Bx}{dt}.
\end{equation*}
} % lemma
\begin{proof}
We write \( r^2 = \Bx \cdot \Bx \), and take derivatives of both sides, to find
\begin{equation}\label{eqn:radialderivatives:60}
2 r \frac{dr}{dt} = 2 \Bx \cdot \frac{d\Bx}{dt},
\end{equation}
or
\begin{equation}\label{eqn:radialderivatives:80}
\frac{dr}{dt} = \frac{\Bx}{r} \cdot \frac{d\Bx}{dt} = \rcap \cdot \frac{d\Bx}{dt}.
\end{equation}
\end{proof}

Application of the chain rule to \cref{eqn:radialderivatives:20} is straightforward
\begin{equation}\label{eqn:radialderivatives:100}
\Bx' = r' \rcap + r \rcap',
\end{equation}
but we don't know the form for \( \rcap' \).  We could proceed with a niave expansion of
\begin{equation}\label{eqn:radialderivatives:120}
\frac{d}{dt} \lr{ \frac{\Bx}{r} },
\end{equation}
but we can be sneaky, and perform a projective and rejective split of \( \Bx' \) with respect to \( \rcap \).  That is
\begin{equation}\label{eqn:radialderivatives:140}
\begin{aligned}
\Bx'
&= \rcap \rcap \Bx' \\
&= \rcap \lr{ \rcap \Bx' } \\
&= \rcap \lr{ \rcap \cdot \Bx' + \rcap \wedge \Bx'} \\
&= \rcap \lr{ r' + \rcap \wedge \Bx'}.
\end{aligned}
\end{equation}
We used \cref{lemma:radialderivatives:40} in the last step above, and after distribution, find
\begin{equation}\label{eqn:radialderivatives:160}
\Bx' = r' \rcap + \rcap \lr{ \rcap \wedge \Bx' }.
\end{equation}
Comparing to \cref{eqn:radialderivatives:100}, we see that
\begin{equation}\label{eqn:radialderivatives:180}
r \rcap' = \rcap \lr{ \rcap \wedge \Bx' }.
\end{equation}
We see that the radial unit vector derivative is proportional to the rejection of \( \rcap \) from \( \Bx' \)
\begin{equation}\label{eqn:radialderivatives:200}
\rcap' = \inv{r} \Rej{\rcap}{\Bx'} = \inv{r^3} \Bx \lr{ \Bx \wedge \Bx' }.
\end{equation}
The vector \( \rcap' \) is perpendicular to \( \rcap \) for any parameterization of it's orientation, or in symbols
\begin{equation}\label{eqn:radialderivatives:220}
\rcap \cdot \rcap' = 0.
\end{equation}
We saw this for the circular and spherical parameterizations, and see now that this also holds more generally.
\subsection{Angular momentum.}
Let's now write out the momentum \( \Bp = m \Bv \) for a point particle with mass \( m \), and determine the kinetic energy \( m \Bv^2/2 = \Bp^2/2m \) for that particle.

The momentum is
\begin{equation}\label{eqn:radialderivatives:320}
\begin{aligned}
\Bp
&= m r' \rcap + m \rcap \lr{ \rcap \wedge \Bv } \\
&= m r' \rcap + \inv{r} \rcap \lr{ \Br \wedge \Bp }.
\end{aligned}
\end{equation}
Observe that \( p_r = m r' \) is the radial component of the momentum.  It is natural to introduce a bivector valued angular momentum operator
\begin{equation}\label{eqn:radialderivatives:340}
L = \Br \wedge \Bp,
\end{equation}
splitting the momentum into a component that is strictly radial and a component that lies purely on the surface of a spherical surface in momentum space.  That is
\begin{equation}\label{eqn:radialderivatives:360}
\Bp = p_r \rcap + \inv{r} \rcap L.
\end{equation}
Making use of the fact that \( \rcap \) and \( \Rej{\rcap}{\Bx'} \) are perpendicular (so there are no cross terms when we square the momentum), the
kinetic energy is
\begin{equation}\label{eqn:radialderivatives:380}
\begin{aligned}
\inv{2m} \Bp^2
&= \inv{2m} \lr{ p_r \rcap + \inv{r} \rcap L }^2 \\
&= \inv{2m} p_r^2 + \inv{2 m r^2 } \rcap L \rcap L \\
&= \inv{2m} p_r^2 - \inv{2 m r^2 } \rcap L^2 \rcap \\
&= \inv{2m} p_r^2 - \inv{2 m r^2 } L^2 \rcap^2,
\end{aligned}
\end{equation}
where we've used the anticommutative nature of \( \rcap \) and \( L \) (i.e.: a sign swap is needed to swap them), and used the fact that \( L^2 \) is a scalar, allowing us to commute \( \rcap \) with \( L^2 \).  This leaves us with
\begin{equation}\label{eqn:radialderivatives:400}
E = \inv{2m} \Bp^2 = \inv{2m} p_r^2 - \inv{2 m r^2 } L^2.
\end{equation}
Observe that both the radial momentum term and the angular momentum term are both strictly postive, since \( L \) is a bivector and \( L^2 \le 0 \).

\subsection{Problems.}
%\makeproblem{Perform a niave computation of \( \rcap' \).}{problem:radialderivatives:240}{
\makeproblem{}{problem:radialderivatives:240}{
Find \cref{eqn:radialderivatives:200} without being sneaky.
} % problem
\makeanswer{problem:radialderivatives:240}{
\begin{equation}\label{eqn:radialderivatives:280}
\begin{aligned}
\rcap'
&= \frac{d}{dt} \lr{ \frac{\Bx}{r} } \\
&= \inv{r} \Bx' - \inv{r^2} \Bx r' \\
&= \inv{r} \Bx' - \inv{r} \rcap r' \\
&= \inv{r} \lr{ \Bx' - \rcap r' } \\
&= \inv{r} \lr{ \rcap \rcap \Bx' - \rcap r' } \\
&= \inv{r} \rcap \lr{ \rcap \Bx' - r' } \\
&= \inv{r} \rcap \lr{ \rcap \Bx' - \rcap \cdot \Bx' } \\
&= \inv{r} \rcap \lr{ \rcap \wedge \Bx' }.
\end{aligned}
\end{equation}
} % answer
\makeproblem{}{problem:radialderivatives:300}{
Show that \cref{eqn:radialderivatives:200} can be expressed as a triple vector cross product
\begin{equation}\label{eqn:radialderivatives:230}
\rcap' = \inv{r^3} \lr{ \Bx \cross \Bx' } \cross \Bx,
\end{equation}
} % problem
\makeanswer{problem:radialderivatives:300}{
While this may be familiar from elementary calculus, such as in \citep{salas1990coa}, we can show follows easily from our GA result
\begin{equation}\label{eqn:radialderivatives:300}
\begin{aligned}
\rcap'
&= \inv{r} \rcap \lr{ \rcap \wedge \Bx' } \\
&= \inv{r} \gpgradeone{ \rcap \lr{ \rcap \wedge \Bx' } } \\
&= \inv{r} \gpgradeone{ \rcap I \lr{ \rcap \cross \Bx' } } \\
&= \inv{r} \gpgradeone{ I \lr{ \rcap \cdot \lr{ \rcap \cross \Bx' } + \rcap \wedge \lr{ \rcap \cross \Bx' } } } \\
&= \inv{r} \gpgradeone{ I^2 \rcap \cross \lr{ \rcap \cross \Bx' } } \\
&= \inv{r} \lr{ \rcap \cross \Bx' } \cross \rcap.
\end{aligned}
\end{equation}
} % answer
%
%}
%\EndArticle
%\EndNoBibArticle
