%
% Copyright � 2012 Peeter Joot.  All Rights Reserved.
% Licenced as described in the file LICENSE under the root directory of this GIT repository.
%
%
%
%
%\input{../peeter_prologue.tex}
%
%\chapter{Spherical polar pendulum for one and multiple masses (Take II)}
\index{spherical pendulum!multiple masses}
\index{spherical pendulum!matrix}
\label{chap:multiPendulumSpherical2}
%
%\blogpage{http://sites.google.com/site/peeterjoot/math2009/multiPendulumSpherical2.pdf?revision=1}
%\date{Nov 4, 2009}
%\revisionInfo{\(RCSfile: multiPendulumSpherical2.tex,v \) Last \(Revision: 1.24 \) \(Date: 2009/12/03 03:24:40 \)}
%
%\beginArtWithToc
%\beginArtNoToc
%
\section{Motivation.}
%
%Attempting the multiple spherical pendulum problem with a bivector parametrized Lagrangian has just been attempted
%\chapcite{sPolarMultiPendulum}, but did not turn out to be an effective approach.  Here a variation is used, 
We will employing regular plain old scalar spherical angle parametrized Kinetic energy, and use geometric algebra to express the Hermitian quadratic form associated with this energy term.

%FIXME: url reference to this since it is implictly referenced.  That doc also implictly referenced my Hamiltionan notes, so pointing back to that would be good for connectedness.
Some simplifying assumptions will be made, including point masses, zero friction at the pivots and rigid non springy massless connecting rods between the masses.
%
\section{The Lagrangian.}
\index{spherical pendulum}
%
A two particle spherical pendulum is depicted in \cref{fig:multiPendulumSpherical2:pendulumDouble}
%
\imageFigure{../figures/classicalmechanics/pendulumDouble}{Double spherical pendulum.}{fig:multiPendulumSpherical2:pendulumDouble}{0.4}
%
The position vector for each particle can be expressed relative to the mass it is connected to (or the origin for the first particle), as in
%
\begin{equation}\label{eqn:multiPendulumSpherical2:boo1}
\begin{aligned}
z_k &= z_{k-1} + \Be_3 l_k e^{j_k \theta_k} \\
j_k &= \Be_3 \wedge \left( \Be_1 e^{i \phi_k} \right) \\
i &= \Be_1 \wedge \Be_2.
\end{aligned}
\end{equation}
%
To express the Kinetic energy for any of the masses \(m_k\), we need the derivative of the incremental difference in position
%
\begin{equation}\label{eqn:multiPendulumSpherical2:85}
\begin{aligned}
\frac{d}{dt} \left( \Be_3 e^{j_k \theta_k} \right)
&=
\Be_3 \left( j_k \dot{\theta}_k e^{j_k \theta_k} + \frac{d j_k }{dt} \sin\theta_k \right)  \\
&=
\Be_3 \left( j_k \dot{\theta}_k e^{j_k \theta_k} + \Be_3 \Be_2 \dot{\phi}_k e^{i \phi_k} \sin\theta_k \right)  \\
&=
\left( \frac{d}{dt}
\begin{bmatrix}
\theta_k & \phi_k
\end{bmatrix} \right)
\begin{bmatrix}
\Be_1 e^{i \phi_k} e^{j_k \theta_k} \\
\Be_2 e^{i \phi_k} \sin\theta_k
\end{bmatrix}.
\end{aligned}
\end{equation}
%
Introducing a Hermitian conjugation \(A^\dagger = \tilde{A}^\T\), reversing and transposing the matrix, and writing
%
\begin{equation}\label{eqn:multiPendulumSpherical2:boo2}
\begin{aligned}
A_k &=
\begin{bmatrix}
\Be_1 e^{i \phi_k} e^{j_k \theta_k} \\
\Be_2 e^{i \phi_k} \sin\theta_k
\end{bmatrix} \\
\BTheta_k &=
\begin{bmatrix}
\theta_k \\
\phi_k
\end{bmatrix}.
\end{aligned}
\end{equation}
%
We can now write the relative velocity differential as
%
\begin{equation}\label{eqn:multiPendulumSpherical2:boo3}
(\dot{z}_k - \dot{z}_{k-1})^2 = l_k^2 {\dot{\BTheta}_k}^\dagger A_k A_k^\dagger \dot{\BTheta}_k.
\end{equation}
%
Observe that the inner product is Hermitian under this definition since \((A_k A_k^\dagger)^\dagger = A_k A_k^\dagger\).  \footnote{Realized later, and being too lazy to adjust everything in these notes, the use of reversion here is not necessary.  Since the generalized coordinates are scalars we could use transposition instead of Hermitian conjugation.  All the matrix elements are vectors so reversal does not change anything.}

The total (squared) velocity of the \(k\)th particle is then
%
\begin{equation}\label{eqn:multiPendulumSpherical2:boo4}
\begin{aligned}
\BTheta &=
\begin{bmatrix}
\BTheta_1 \\
\BTheta_2 \\
\vdots \\
\BTheta_N \\
\end{bmatrix} \\
B_k &=
\begin{bmatrix}
l_1 A_1 \\
l_2 A_2 \\
\vdots \\
l_k A_k \\
0 \\
\end{bmatrix} \\
(\dot{z}_k)^2 &=
\dot{\BTheta}^\dagger B_k B_k^\dagger \dot{\BTheta}.
\end{aligned}
\end{equation}
%
(where the zero matrix in \(B_k\) is a \(N-k\) by one zero).  Summing over all masses and adding in the potential energy we have for the Lagrangian of the system
%
%m_1 A_1 A_1^\dagger
%+ m_2
%A_1 A_1^\dagger
%A_2 A_2^\dagger
%+ m_3
%A_1 A_1^\dagger
%A_2 A_2^\dagger
%A_3 A_3^\dagger
%...
%=
%(m_1 + m_2 + m_3) A_1 A_1^\dagger
%+(m_2 + m_3) A_2 A_2^\dagger
%+(m_3) A_3 A_3^\dagger
%
\begin{equation}\label{eqn:multiPendulumSpherical2:boo5b}
\begin{aligned}
K &=
\inv{2} \sum_{k=1}^N m_k
\dot{\BTheta}^\dagger
B_k B_k^\dagger
\dot{\BTheta} \\
\mu_k &= \sum_{j=k}^N m_j \\
\Phi &=
g \sum_{k=1}^N \mu_k l_k \cos\theta_k \\
\LL &= K - \Phi.
\end{aligned}
\end{equation}
%
There is a few layers of equations involved and we still have an unholy mess of matrix and geometric algebra in the kernel of the kinetic energy quadratic form, but at least this time all the generalized coordinates of the system are scalars.
%
\section{Some tidy up.}
%
Before continuing with evaluation of the Euler-Lagrange equations it is helpful to make a couple of observations about the structure of the matrix products that make up our velocity quadratic forms
%
\begin{equation}\label{eqn:multiPendulumSpherical2:boo8}
\begin{aligned}
\dot{\BTheta}^\dagger
B_k B_k^\dagger
\dot{\BTheta}
&=
\dot{\BTheta}^\dagger
\begin{bmatrix}
\begin{bmatrix}
l_1^2 A_1 A_1^\dagger & l_1 l_2 A_1 A_2^\dagger & \hdots & l_1 l_k A_1 A_k^\dagger \\
l_2 l_1 A_2 A_1^\dagger & l_2^2 A_2 A_2^\dagger & \hdots & l_2 l_k A_2 A_k^\dagger \\
\vdots \\
l_k l_1 A_k A_1^\dagger & l_k l_2 A_k A_2^\dagger & \hdots & l_k^2 A_k A_k^\dagger
\end{bmatrix} & 0 \\
0 & 0
\end{bmatrix}
\dot{\BTheta}.
\end{aligned}
\end{equation}
%
Specifically, consider the \(A_a A_b^\dagger\) products that make up the elements of the matrices \(Q_k = B_k B_k^\dagger\).  Without knowing anything about the grades that make up the elements of \(Q_k\), since it is Hermitian (by this definition of Hermitian) there can be no elements of grade order two or three in the final matrix.  This is because reversion of such grades inverts the sign, and the matrix elements in \(Q_k\) all equal their reverse.  Additionally, the elements of the multivector column matrices \(A_k\) are vectors, so in the product \(A_a A_b^\dagger\) we can only have scalar and bivector (grade two) elements.  The resulting one by one scalar matrix is a sum over all the mixed angular velocities \(\dot{\theta}_a \dot{\theta}_b\), \(\dot{\theta}_a \dot{\phi}_b\), and \(\dot{\phi}_a \dot{\phi}_b\), so once this summation is complete any bivector grades of \(A_a A_b^\dagger\) must cancel out.  This is consistent with the expectation that we have a one by one scalar matrix result out of this in the end (i.e. a number).  The end result is a freedom to exploit the convenience of explicitly using a scalar selection operator that filters out any vector, bivector, and trivector grades in the products \(A_a A_b^\dagger\).  We will get the same result if we write
%
\begin{equation}\label{eqn:multiPendulumSpherical2:boo9}
\begin{aligned}
\dot{\BTheta}^\dagger
B_k B_k^\dagger
\dot{\BTheta}
&=
\dot{\BTheta}^\dagger
\begin{bmatrix}
\begin{bmatrix}
l_1^2 \gpgradezero{A_1 A_1^\dagger} & l_1 l_2 \gpgradezero{A_1 A_2^\dagger} & \hdots & l_1 l_k \gpgradezero{A_1 A_k^\dagger} \\
l_2 l_1 \gpgradezero{A_2 A_1^\dagger} & l_2^2 \gpgradezero{A_2 A_2^\dagger} & \hdots & l_2 l_k \gpgradezero{A_2 A_k^\dagger} \\
\vdots \\
l_k l_1 \gpgradezero{A_k A_1^\dagger} & l_k l_2 \gpgradezero{A_k A_2^\dagger} & \hdots & l_k^2 \gpgradezero{A_k A_k^\dagger}
\end{bmatrix} & 0 \\
0 & 0
\end{bmatrix}
\dot{\BTheta}.
\end{aligned}
\end{equation}
%
Pulling in the summation over \(m_k\) we have
%
\begin{equation}\label{eqn:multiPendulumSpherical2:boo15}
\sum_k m_k
\dot{\BTheta}^\dagger
B_k B_k^\dagger
\dot{\BTheta}
=
\dot{\BTheta}^\dagger
{\begin{bmatrix}
\mu_{\max(r,c)} l_r l_c \gpgradezero{A_r A_c^\dagger}
\end{bmatrix}}_{rc}
\dot{\BTheta}.
\end{equation}
%
It appears justifiable to label the \(\mu_{\max(r,c)} l_r l_c\) factors of the angular velocity matrices as moments of inertia in a generalized sense.  Using this block matrix form, and scalar selection, we can now write the Lagrangian in a slightly tidier form
%
\begin{equation}\label{eqn:multiPendulumSpherical2:boo5}
\begin{aligned}
\mu_k &= \sum_{j=k}^N m_j \\
Q &=
{\begin{bmatrix}
\mu_{\max(r,c)} l_r l_c A_r A_c^\dagger
\end{bmatrix}}_{rc} \\
K &=
\inv{2} \dot{\BTheta}^\dagger Q
\dot{\BTheta}
=\inv{2} \dot{\BTheta}^\T \gpgradezero{Q}
\dot{\BTheta} \\
\Phi &=
g \sum_{k=1}^N \mu_k l_k \cos\theta_k \\
\LL &= K - \Phi.
\end{aligned}
\end{equation}
%
After some expansion, writing \(S_\theta = \sin\theta\), \(C_\phi = \cos\phi\) and so forth, one can find that the scalar parts of the block matrices \(A_r A_c^\dagger\) contained in \(Q\) are
%
\begin{equation}\label{eqn:multiPendulumSpherical2:boo20}
\begin{aligned}
\gpgradezero{A_r A_c^\dagger}
=
\begin{bmatrix}
C_{\phi_c - \phi_r}
C_{\theta_r}
C_{\theta_c}
+S_{\theta_r}
S_{\theta_c} &
-S_{\phi_c - \phi_r} C_{\theta_r} S_{\theta_c} \\
S_{\phi_c - \phi_r} C_{\theta_c} S_{\theta_r} &
C_{\phi_c - \phi_r} S_{\theta_r} S_{\theta_c}
\end{bmatrix}.
\end{aligned}
\end{equation}
%
The diagonal blocks are particularly simple and have no \(\phi\) dependence
%
\begin{equation}\label{eqn:multiPendulumSpherical2:boo20d}
\begin{aligned}
\gpgradezero{A_r A_r^\dagger}
=
\begin{bmatrix}
1 & 0 \\
0 & \sin^2 \theta_r
\end{bmatrix}.
\end{aligned}
\end{equation}
%
Observe also that \(\gpgradezero{A_r A_c^\dagger}^T = \gpgradezero{A_c A_r^\dagger}\), so the scalar matrix
%
\begin{equation}\label{eqn:multiPendulumSpherical2:boo21}
\gpgradezero{Q} =
{\begin{bmatrix}
\mu_{\max(r,c)} l_r l_c \gpgradezero{ A_r A_c^\dagger }
\end{bmatrix}}_{rc}.
\end{equation}
%
is a real symmetric matrix.  We have the option of using this explicit scalar expansion if desired for further computations associated with this problem.  That completely eliminates the Geometric algebra from the problem, and is probably a logical way to formulate things for numerical work since one can then exploit any pre existing matrix algebra system without having to create one that understands non-commuting variables and vector products.
%
\section{Evaluating the Euler-Lagrange equations.}
%
For the acceleration terms of the Euler-Lagrange equations our computation reduces nicely to a function of only \(\gpgradezero{Q}\)
%
\begin{equation}\label{eqn:multiPendulumSpherical2:105}
\begin{aligned}
\frac{d}{dt} \PD{\dot{\theta}_a}{\LL}
&=
\inv{2} \frac{d}{dt}
\left(
\PD{\dot{\theta}_a}{\dot{\BTheta}}^\T
\gpgradezero{Q}
\dot{\BTheta}
+
\dot{\BTheta}^\T
\gpgradezero{Q}
\PD{\dot{\theta}_a}{\dot{\BTheta}}
\right)  \\
&=
\frac{d}{dt} \left(
{\begin{bmatrix}
\delta_{ac}
\begin{bmatrix}
1 & 0
\end{bmatrix}
\end{bmatrix}}_c
\gpgradezero{Q}
\dot{\BTheta}
\right).
\end{aligned}
\end{equation}
%
and
%
\begin{equation}\label{eqn:multiPendulumSpherical2:125}
\begin{aligned}
\frac{d}{dt} \PD{\dot{\phi}_a}{\LL}
&=
\inv{2} \frac{d}{dt}
\left(
\PD{\dot{\phi}_a}{\dot{\BTheta}}^\T
\gpgradezero{Q}
\dot{\BTheta}
+
\dot{\BTheta}^\T
\gpgradezero{Q}
\PD{\dot{\phi}_a}{\dot{\BTheta}}
\right)  \\
&=
\frac{d}{dt} \left(
{\begin{bmatrix}
\delta_{ac}
\begin{bmatrix}
0 & 1
\end{bmatrix}
\end{bmatrix}}_c
\gpgradezero{Q}
\dot{\BTheta}
\right).
\end{aligned}
\end{equation}
%
The last groupings above made use of \(\gpgradezero{Q} = \gpgradezero{Q}^\T\), and in particular \((\gpgradezero{Q} + \gpgradezero{Q}^\T)/2 = \gpgradezero{Q}\).  We can now form a column matrix putting all the angular velocity gradient in a tidy block matrix representation
%
\begin{equation}\label{eqn:multiPendulumSpherical2:boo22}
\begin{aligned}
\grad_{\dot{\BTheta}} \LL =
{\begin{bmatrix}
\begin{bmatrix}
\PD{\dot{\theta}_r}{\LL} \\
\PD{\dot{\phi}_r}{\LL} \\
\end{bmatrix}
\end{bmatrix}}_r = \gpgradezero{Q} \dot{\BTheta}.
\end{aligned}
\end{equation}
%
A small aside on Hamiltonian form.  This velocity gradient is also the conjugate momentum of the Hamiltonian, so if we wish to express the Hamiltonian in terms of conjugate momenta, we require invertability of \(\gpgradezero{Q}\) at the point in time that we evaluate things.  Writing
%
\begin{equation}\label{eqn:multiPendulumSpherical2:boo30}
P_{\BTheta} = \grad_{\dot{\BTheta}} \LL.
\end{equation}
%
and noting that \((\gpgradezero{Q}^{-1})^\T = \gpgradezero{Q}^{-1}\), we get for the kinetic energy portion of the Hamiltonian
%
\begin{equation}\label{eqn:multiPendulumSpherical2:boo31}
K = \inv{2} {P_{\BTheta}}^\T \gpgradezero{Q}^{-1} P_{\BTheta}.
\end{equation}
%
Now, the invertability of \(\gpgradezero{Q}\) cannot be taken for granted.  Even in the single particle case we do not have invertability.  For the single particle case we have
%
\begin{equation}\label{eqn:multiPendulumSpherical2:boo45}
\begin{aligned}
\gpgradezero{Q} =
m l^2
\begin{bmatrix}
1 & 0 \\
0 & \sin^2 \theta
\end{bmatrix}.
\end{aligned}
\end{equation}
%
so at \(\theta = \pm \pi/2\) this quadratic form is singular, and the planar angular momentum becomes a constant of motion.
%
Returning to the evaluation of the Euler-Lagrange equations, the problem is now reduced to calculating the right hand side of the following system
%
\begin{equation}\label{eqn:multiPendulumSpherical2:boo23}
\begin{aligned}
\frac{d}{dt} \left( \gpgradezero{Q} \dot{\BTheta} \right) =
{\begin{bmatrix}
\begin{bmatrix}
\PD{\theta_r}{\LL} \\
\PD{\phi_r}{\LL} \\
\end{bmatrix}
\end{bmatrix}}_r.
\end{aligned}
\end{equation}
%
With back substitution of \eqnref{eqn:multiPendulumSpherical2:boo20}, and \eqnref{eqn:multiPendulumSpherical2:boo21} we have a complete non-multivector expansion of the left hand side.  For the right hand side taking the \(\theta_a\) and \(\phi_a\) derivatives respectively we get
%
\begin{equation}\label{eqn:multiPendulumSpherical2:boo6}
\PD{\theta_a}{\LL}
=
\inv{2} \dot{\BTheta}^\dagger
{\begin{bmatrix}
\mu_{\max(r,c)} l_r l_c \gpgradezero{
\PD{\theta_a}{A_r} A_c^\dagger
+A_r \PD{\theta_a}{A_c}^\dagger
}
\end{bmatrix}}_{rc}
\dot{\BTheta}
-g \mu_a l_a \sin\theta_a.
\end{equation}
%
\begin{equation}\label{eqn:multiPendulumSpherical2:boo6c}
\PD{\phi_a}{\LL}
=
\inv{2} \dot{\BTheta}^\dagger
{\begin{bmatrix}
\mu_{\max(r,c)} l_r l_c \gpgradezero{
\PD{\phi_a}{A_r} A_c^\dagger
+A_r \PD{\phi_a}{A_c}^\dagger
}
\end{bmatrix}}_{rc}
\dot{\BTheta}.
\end{equation}
%
So to proceed we must consider the \(\gpgradezero{A_r A_c^\dagger}\) partials.  A bit of thought shows that the matrices of partials above are mostly zeros.  Illustrating by example, consider \(\PDi{\theta_2}{\gpgradezero{Q}}\), which in block matrix form is
%
\begin{equation}\label{eqn:multiPendulumSpherical2:boo33}
%\begin{aligned}
%&\PD{\theta_2}{\gpgradezero{Q}} \\
%&=
\begin{bsmallmatrix}
0 & \inv{2} \mu_2 l_1 l_2 \gpgradezero{A_1 \PD{\theta_2}{A_2}^\dagger} & 0 & \hdots & 0 \\
\inv{2} \mu_2 l_2 l_1 \gpgradezero{\PD{\theta_2}{A_2} A_1^\dagger} &
\inv{2} \mu_2 l_2 l_2 \gpgradezero{A_2 \PD{\theta_2}{A_2}^\dagger + \PD{\theta_2}{A_2} A_2^\dagger} &
\inv{2} \mu_3 l_2 l_3 \gpgradezero{\PD{\theta_2}{A_2} A_3^\dagger} & \hdots &
\inv{2} \mu_N l_2 l_N \gpgradezero{\PD{\theta_2}{A_2} A_N^\dagger} \\
0 & \inv{2} \mu_3 l_3 l_2 \gpgradezero{A_3 \PD{\theta_2}{A_2}^\dagger} & 0 & \hdots & 0 \\
0 & \vdots & 0 & \hdots & 0 \\
0 & \inv{2} \mu_N l_N l_2 \gpgradezero{A_N \PD{\theta_2}{A_2}^\dagger} & 0 & \hdots & 0 \\
\end{bsmallmatrix}.
%\end{aligned}
\end{equation}
%
Observe that the diagonal term has a scalar plus its reverse, so we can drop the one half factor and one of the summands for a total contribution to \(\PDi{\theta_2}{\LL}\) of just
%
\begin{equation}\label{eqn:multiPendulumSpherical2:145}
\mu_2 {l_2}^2 {\dot{\BTheta}_2}^\T \gpgradezero{\PD{\theta_2}{A_2} A_2^\dagger} \dot{\BTheta}_2.
\end{equation}
%
Now consider one of the pairs of off diagonal terms.  Adding these we contributions to \(\PDi{\theta_2}{\LL}\) of
%
\begin{equation}\label{eqn:multiPendulumSpherical2:165}
\begin{aligned}
\inv{2} &\mu_2 l_1 l_2
{\dot{\BTheta}_1}^\T
\gpgradezero{A_1 \PD{\theta_2}{A_2}^\dagger}
\dot{\BTheta}_2
+\inv{2} \mu_2 l_2 l_1
{\dot{\BTheta}_2}^\T
\gpgradezero{\PD{\theta_2}{A_2} A_1^\dagger}
\dot{\BTheta}_1 \\
&=
\inv{2} \mu_2 l_1 l_2
{\dot{\BTheta}_1}^\T
\gpgradezero{
A_1 \PD{\theta_2}{A_2}^\dagger
+A_1 \PD{\theta_2}{A_2}^\dagger
}
\dot{\BTheta}_2 \\
&=
\mu_2 l_1 l_2
{\dot{\BTheta}_1}^\T
\gpgradezero{
A_1 \PD{\theta_2}{A_2}^\dagger
}
\dot{\BTheta}_2 .
\end{aligned}
\end{equation}
%
This has exactly the same form as the diagonal term, so summing over all terms we get for the position gradient components of the Euler-Lagrange equation just
%
\begin{subequations}
\label{eqn:multiPendulumSpherical2:boo41aa}
\begin{equation}\label{eqn:multiPendulumSpherical2:boo41}
\PD{\theta_a}{\LL}
=
%\mu_a {l_a}^2 {\dot{\BTheta}_a}^\T \gpgradezero{ A_a \PD{\theta_a}{A_a}^\dagger} \dot{\BTheta}_a
%+
%\sum_{k \ne a}
\sum_{k}
\mu_{\max(k,a)} l_k l_a
{\dot{\BTheta}_k}^\T
\gpgradezero{
A_k \PD{\theta_a}{A_a}^\dagger
}
\dot{\BTheta}_a
-g \mu_a l_a \sin\theta_a,
\end{equation}
%
\begin{equation}\label{eqn:multiPendulumSpherical2:boo40}
\PD{\phi_a}{\LL}
=
%\mu_a {l_a}^2 {\dot{\BTheta}_a}^\T \gpgradezero{ A_a \PD{\phi_a}{A_a}^\dagger} \dot{\BTheta}_a
%+
%\sum_{k \ne a}
\sum_{k}
\mu_{\max(k,a)} l_k l_a
{\dot{\BTheta}_k}^\T
\gpgradezero{
A_k \PD{\phi_a}{A_a}^\dagger
}
\dot{\BTheta}_a.
\end{equation}
\end{subequations}
%
The only thing that remains to do is evaluate the \(\gpgradezero{A_k \PDi{\phi_a}{A_a}^\dagger}\) matrices.

It should be possible but it is tedious to calculate the block matrix derivative terms from the \(A_a\) partials using
%
\begin{subequations}
\label{eqn:multiPendulumSpherical2:boo10a}
\begin{equation}\label{eqn:multiPendulumSpherical2:boo10}
\begin{aligned}
\PD{\theta_a}{A_a} &=
\begin{bmatrix}
-\Be_3 e^{j_a \theta_a} \\
\Be_2 e^{i \phi_a} C_{\theta_a}
\end{bmatrix},
\end{aligned}
\end{equation}
\begin{equation}\label{eqn:multiPendulumSpherical2:boo11}
\begin{aligned}
\PD{\phi_a}{A_a}
&=
\begin{bmatrix}
\Be_2 e^{i \phi_a} C_{\theta_a} \\
-\Be_1 e^{i \phi_a} S_{\theta_a}
\end{bmatrix}.
\end{aligned}
\end{equation}
\end{subequations}
%
However multiplying this out and reducing is a bit tedious and would be a better job for a symbolic algebra package.  With \eqnref{eqn:multiPendulumSpherical2:boo20} available to use, one gets easily
%
\begin{subequations}
\label{eqn:multiPendulumSpherical2:boo51a}
\begin{equation}\label{eqn:multiPendulumSpherical2:boo51}
\begin{aligned}
\gpgradezero{ A_k \PD{\theta_a}{A_a}^\dagger }
&=
\begin{bmatrix}
-C_{\phi_a - \phi_k} C_{\theta_k} S_{\theta_a} + S_{\theta_k} C_{\theta_a} &
-S_{\phi_a - \phi_k} C_{\theta_k} C_{\theta_a} \\
-S_{\phi_a - \phi_k} S_{\theta_a} S_{\theta_k} &
C_{\phi_a - \phi_k} S_{\theta_k} C_{\theta_a}
\end{bmatrix},
\end{aligned}
\end{equation}
%
\begin{equation}\label{eqn:multiPendulumSpherical2:boo50}
\begin{aligned}
\gpgradezero{ A_k \PD{\phi_a}{A_a}^\dagger }
&=
\begin{bmatrix}
-S_{\phi_a - \phi_k} C_{\theta_k} C_{\theta_a}  &
-C_{\phi_a - \phi_k} C_{\theta_k} S_{\theta_a} \\
C_{\phi_a - \phi_k} C_{\theta_a} S_{\theta_k} &
-S_{\phi_a - \phi_k} S_{\theta_k} S_{\theta_a}
\end{bmatrix}.
\end{aligned}
\end{equation}
\end{subequations}
%
The right hand side of the Euler-Lagrange equations now becomes
%
\begin{equation}\label{eqn:multiPendulumSpherical2:boo55}
\begin{aligned}
\grad_{\BTheta} \LL =
\sum_k
{\begin{bmatrix}
\begin{bmatrix}
\mu_{\max(k,r)} l_k l_r {\dot{\BTheta}_k}^\T \gpgradezero{ A_k \PD{\theta_r}{A_r}^\dagger } \dot{\BTheta}_r \\
\mu_{\max(k,r)} l_k l_r {\dot{\BTheta}_k}^\T \gpgradezero{ A_k \PD{\phi_r}{A_r}^\dagger } \dot{\BTheta}_r
\end{bmatrix}
\end{bmatrix}}_r
- g
{\begin{bmatrix}
\mu_r l_r \sin\theta_r
\begin{bmatrix}
1 \\
0
\end{bmatrix}
\end{bmatrix}}_r.
\end{aligned}
\end{equation}
%
Can the \(\dot{\BTheta}_a\) matrices be factored out, perhaps allowing for expression as a function of \(\dot{\BTheta}\)?  How to do that if it is possible is not obvious.  The driving reason to do so would be to put things into a tidy form where things are a function of the system angular velocity vector \(\BTheta\), but this is not possible anyways since the gradient is non-linear.
%
\section{Hamiltonian form and linearization.}
%
Having calculated the Hamiltonian equations for the multiple mass planar pendulum in
\chapcite{hamiltonian}
, doing so for the spherical pendulum can now be done by inspection.  With the introduction of a phase space vector for the system using the conjugate momenta (for angles where these conjugate momenta are non-singular)
%
\begin{equation}\label{eqn:multiPendulumSpherical2:boo60}
\Bz =
\begin{bmatrix}
P_{\BTheta}
\BTheta
\end{bmatrix},
\end{equation}
%
we can write the Hamiltonian equations
%
\begin{equation}\label{eqn:multiPendulumSpherical2:boo61}
\frac{d\Bz}{dt} =
\begin{bmatrix}
\grad_{\BTheta} \LL
\gpgradezero{Q}^{-1} P_{\BTheta}
\end{bmatrix}.
\end{equation}
%
The position gradient is given explicitly in \eqnref{eqn:multiPendulumSpherical2:boo55}, and that can be substituted here.  That gradient is expressed in terms of \(\dot{\BTheta}_k\) and not the conjugate momenta, but the mapping required to express the whole system in terms of the conjugate momenta is simple enough
%
\begin{equation}\label{eqn:multiPendulumSpherical2:boo62}
\dot{\BTheta}_k =
{\begin{bmatrix}
\delta_{kc} I_{22}
\end{bmatrix}}_c \gpgradezero{Q}^{-1} P_{\BTheta}.
\end{equation}
%
It is apparent that for any sort of numerical treatment use of a angular momentum and angular position phase space vector is not prudent.  If the aim is nothing more than working with a first order system instead of second order, then we are probably better off with an angular velocity plus angular position phase space system.
%
\begin{equation}\label{eqn:multiPendulumSpherical2:boo60a}
\begin{aligned}
\frac{d}{dt}
\begin{bmatrix}
\gpgradezero{Q} \dot{\BTheta} \\
\BTheta
\end{bmatrix}
=
\begin{bmatrix}
\grad_{\BTheta} \LL \\
\dot{\BTheta}
\end{bmatrix}.
\end{aligned}
\end{equation}
%
This eliminates the requirement for inverting the sometimes singular matrix \(\gpgradezero{Q}\), but one is still left with something that is perhaps tricky to work with since we have the possibility of zeros on the left hand side.  The resulting equation is of the form
%
\begin{equation}\label{eqn:multiPendulumSpherical2:boo64}
M \Bx' = f(\Bx).
\end{equation}
%
where \(M = \left[\begin{smallmatrix}\gpgradezero{Q} & 0 \\ 0 & I\end{smallmatrix}\right]\) is a possibly singular matrix, and \(f\) is a non-linear function of the components of \(\BTheta\), and \(\dot{\BTheta}\).  This is conceivably linearizable in the neighborhood of a particular phase space point \(\Bx_0\).  If that is done, resulting in an equation of the form
%
\begin{equation}\label{eqn:multiPendulumSpherical2:boo65}
M \By' = f(\Bx_0) + B \By.
\end{equation}
%
where \(\Bx = \By + \Bx_0\) and \(B\) is an appropriate matrix of partials (the specifics of which do not really have to be spelled out here).  Because of the possible singularities of \(M\) the exponentiation techniques applied to the linearized planar pendulum may not be possible with such a linearization.  Study of this less well formed system of LDEs probably has interesting aspects, but is also likely best tackled independently of the specifics of the spherical pendulum problem.
%
\subsection{Thoughts about the Hamiltonian singularity.}
%
The fact that the Hamiltonian goes singular on the horizontal in this spherical polar representation is actually what I think is the most interesting bit in the problem (the rest being a lot mechanical details).  On the horizontal \(\phi=0\) or \(\dot{\phi} = 37000\) radians/sec makes no difference to the dynamics.  All you can say is that the horizontal plane angular momentum is a constant of the system.  It seems very much like the increasing uncertainty that you get in the corresponding radial QM equation.  Once you start pinning down the \(\theta\) angle, you loose the ability to say much about \(\phi\).

It is also kind of curious how the energy of the system is never ill defined but a choice of a particular orientation to use as a reference for observations of the momenta introduces the singularity as the system approaches the horizontal in that reference frame.

Perhaps there are some deeper connections relating these classical and QM similarity.  Would learning about symplectic flows and phase space volume invariance shed some light on this?
%
\section{A summary.}
%
A fair amount of notation was introduced along the way in the process of formulating the spherical pendulum equations.  It is worthwhile to do a final concise summary of notation and results before moving on for future reference.

The positions of the masses are given by
%
\begin{equation}\label{eqn:multiPendulumSpherical2:zoo3}
\begin{aligned}
z_k &= z_{k-1} + \Be_3 l_k e^{j_k \theta_k} \\
j_k &= \Be_3 \wedge \left( \Be_1 e^{i \phi_k} \right) \\
i &= \Be_1 \wedge \Be_2.
\end{aligned}
\end{equation}
%
With the introduction of a column vector of vectors (where we multiply matrices using the Geometric vector product),
%
\begin{equation}\label{eqn:multiPendulumSpherical2:zoo1}
\begin{aligned}
\BTheta_k &=
\begin{bmatrix}
\theta_k \\
\phi_k
\end{bmatrix},
\end{aligned}
\end{equation}
\begin{equation}\label{eqn:multiPendulumSpherical2:zoo2}
\begin{aligned}
\BTheta &=
{\begin{bmatrix}
\BTheta_1 &
\BTheta_2 &
\hdots &
\BTheta_N
\end{bmatrix}}^\T.
\end{aligned}
\end{equation}
%
and a matrix of velocity components (with matrix multiplication of the vector elements using the Geometric vector product), we can form the Lagrangian
%
\begin{equation}\label{eqn:multiPendulumSpherical2:zoo4}
\begin{aligned}
A_k &=
\begin{bmatrix}
\Be_1 e^{i \phi_k} e^{j_k \theta_k} \\
\Be_2 e^{i \phi_k} S_{\theta_k}
\end{bmatrix}.
\end{aligned}
\end{equation}
\begin{equation}\label{eqn:multiPendulumSpherical2:zoo5}
\begin{aligned}
\mu_k &= \sum_{j=k}^N m_j \\
\gpgradezero{Q} &=
{\begin{bmatrix}
\mu_{\max(r,c)} l_r l_c \gpgradezero{A_r A_c^\T}
\end{bmatrix}}_{rc} \\
K &= \inv{2} \dot{\BTheta}^\T \gpgradezero{Q}
\dot{\BTheta} \\
\Phi &=
g \sum_{k=1}^N \mu_k l_k C_{\theta_k} \\
\LL &= K - \Phi.
\end{aligned}
\end{equation}
%
An explicit scalar matrix evaluation of the (symmetric) block matrix components of \(\gpgradezero{Q}\) was evaluated and found to be
%
\begin{equation}\label{eqn:multiPendulumSpherical2:zoo6}
\begin{aligned}
\gpgradezero{A_r A_c^\T}
=
\begin{bmatrix}
C_{\phi_c - \phi_r}
C_{\theta_r}
C_{\theta_c}
+S_{\theta_r}
S_{\theta_c} &
-S_{\phi_c - \phi_r} C_{\theta_r} S_{\theta_c} \\
S_{\phi_c - \phi_r} C_{\theta_c} S_{\theta_r} &
C_{\phi_c - \phi_r} S_{\theta_r} S_{\theta_c}
\end{bmatrix}.
\end{aligned}
\end{equation}
%
These can be used if explicit evaluation of the Kinetic energy is desired, avoiding redundant summation over the pairs of skew entries in the quadratic form matrix \(\gpgradezero{Q}\)
%
\begin{equation}\label{eqn:multiPendulumSpherical2:zoo7}
K =
\inv{2} \sum_k \mu_k l_k^2 {\dot{\BTheta}_k}^T \gpgradezero{A_k A_k^\T}
\dot{\BTheta}_k
+ \sum_{r<c} \mu_{\max(r,c)} l_r l_c
{\dot{\BTheta}_r}^\T
\gpgradezero{A_r A_c^\T}
\dot{\BTheta}_c.
\end{equation}
%
We utilize angular position and velocity gradients
%
\begin{equation}\label{eqn:multiPendulumSpherical2:zoo8}
\begin{aligned}
\grad_{\BTheta_k} &=
\begin{bmatrix}
\PD{\theta_k}{} \\
\PD{\phi_k}{}
\end{bmatrix} \\
\grad_{\dot{\BTheta}_k} &=
\begin{bmatrix}
\PD{\dot{\theta}_k}{} \\
\PD{\dot{\phi}_k}{}
\end{bmatrix} \\
\grad_{\BTheta} &=
{\begin{bmatrix}
\grad_{\BTheta_1}^\T &
\grad_{\BTheta_2}^\T &
\hdots  &
\grad_{\BTheta_N}^\T
\end{bmatrix}}^\T \\
\grad_{\dot{\BTheta}} &=
{\begin{bmatrix}
{\grad_{\dot{\BTheta}_1}}^\T &
{\grad_{\dot{\BTheta}_2}}^\T &
\hdots  &
{\grad_{\dot{\BTheta}_N}}^\T
\end{bmatrix}}^\T.
\end{aligned}
\end{equation}
%
and use these to form the Euler-Lagrange equations for the system in column vector form
%
\begin{equation}\label{eqn:multiPendulumSpherical2:zoo9}
\frac{d}{dt} \grad_{\dot{\BTheta}} \LL = \grad_{\BTheta} \LL.
\end{equation}
%
For the canonical momenta we found the simple result
%
\begin{equation}\label{eqn:multiPendulumSpherical2:zoo10}
\grad_{\dot{\BTheta}} \LL = \gpgradezero{Q} \dot{\BTheta}.
\end{equation}
%
For the position gradient portion of the Euler-Lagrange equations \eqnref{eqn:multiPendulumSpherical2:zoo9} we found in block matrix form
%
\begin{equation}\label{eqn:multiPendulumSpherical2:zoo11}
\begin{aligned}
\grad_{\BTheta} \LL =
\sum_k
{\begin{bmatrix}
\begin{bmatrix}
\mu_{\max(k,r)} l_k l_r {\dot{\BTheta}_k}^\T \gpgradezero{ A_k \PD{\theta_r}{A_r}^\dagger } \dot{\BTheta}_r \\
\mu_{\max(k,r)} l_k l_r {\dot{\BTheta}_k}^\T \gpgradezero{ A_k \PD{\phi_r}{A_r}^\dagger } \dot{\BTheta}_r
\end{bmatrix}
\end{bmatrix}}_r
- g
{\begin{bmatrix}
\mu_r l_r S_{\theta_r}
\begin{bmatrix}
1 \\
0
\end{bmatrix}
\end{bmatrix}}_r,
\end{aligned}
\end{equation}
%
\begin{equation}\label{eqn:multiPendulumSpherical2:zoo12}
\begin{aligned}
\gpgradezero{ A_k \PD{\theta_a}{A_a}^\dagger }
&=
\begin{bmatrix}
-C_{\phi_a - \phi_k} C_{\theta_k} S_{\theta_a} + S_{\theta_k} C_{\theta_a} &
-S_{\phi_a - \phi_k} C_{\theta_k} C_{\theta_a} \\
-S_{\phi_a - \phi_k} S_{\theta_a} S_{\theta_k} &
C_{\phi_a - \phi_k}  S_{\theta_k} C_{\theta_a}
\end{bmatrix}.
\end{aligned}
\end{equation}
%
\begin{equation}\label{eqn:multiPendulumSpherical2:zoo13}
\begin{aligned}
\gpgradezero{ A_k \PD{\phi_a}{A_a}^\dagger }
&=
\begin{bmatrix}
-S_{\phi_a - \phi_k} C_{\theta_k} C_{\theta_a} &
-C_{\phi_a - \phi_k} C_{\theta_k} S_{\theta_a} \\
C_{\phi_a - \phi_k} C_{\theta_a} S_{\theta_k} &
-S_{\phi_a - \phi_k} S_{\theta_k} S_{\theta_a}
\end{bmatrix}.
\end{aligned}
\end{equation}
%
A set of Hamiltonian equations for the system could also be formed.  However, this requires that one somehow restrict attention to the subset of phase space where the canonical momenta matrix \(\gpgradezero{Q}\) is non-singular, something not generally possible.
%
%\EndArticle
