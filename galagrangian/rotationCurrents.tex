%
% Copyright � 2012 Peeter Joot.  All Rights Reserved.
% Licenced as described in the file LICENSE under the root directory of this GIT repository.
%
%
%
%
%\input{../peeter_prologue.tex}
%
%\chapter{Translation and rotation Noether field currents}
\label{chap:rotationCurrents}
%
%\blogpage{http://sites.google.com/site/peeterjoot/math2009/rotationCurrents.pdf}
%\date{Sept 4, 2009}
%\revisionInfo{\(RCSfile: rotationCurrents.tex,v \) Last \(Revision: 1.14 \) \(Date: 2009/10/22 02:07:20 \)}
%
\beginArtWithToc
%
\section{Motivation}
The article \citep{montesinos2006sem} details the calculation for a conserved current associated with an incremental Poincare transformation.  Instead of starting with the canonical energy momentum tensor (arising from spacetime translation) which is not symmetric but can be symmetrized with other arguments, the paper of interest obtains the symmetric energy momentum tensor for Maxwell's equations directly.

I believe that I am slowly accumulating the tools required to understand this paper.  One such tool is likely the exponential rotational generator examined in \citep{gabookI:rotationGenerator}, utilizing the angular momentum operator.

Here I review the Noether conservation relations and the associated Noether currents for a single parameter alteration of the Lagrangian, incremental spacetime translation of the Lagrangian, and incremental Lorentz transform of the Lagrangian.

By reviewing these I hope that understanding the referenced article will be easier, or I independently understand (in my own way) how to apply similar techniques to the incremental Poincare transformed Lagrangian.
%
\section{Field Euler-Lagrange equations}
%
The extremization of the action integral
%
\begin{equation}\label{eqn:rotationCurrents:EOF1}
\begin{aligned}
S &= \int \LL d^4 x
\end{aligned}
\end{equation}
%
can be dealt with (following Feynman) as a first order Taylor expansion and integration by parts exercise.  A single field variable example serves to illustrate.  A first order Lagrangian of a single field variable has the form
%
\begin{equation}\label{eqn:rotationCurrents:EOF2}
\begin{aligned}
\LL = \LL(\phi, \partial_\mu \phi)
\end{aligned}
\end{equation}
%
Let us vary the field \(\phi \rightarrow \phi + \overbar{\phi}\) around the stationary field \(\overbar{\phi}\), inducing a corresponding variation in the action
%
\begin{equation}\label{eqn:rotationCurrents:35}
\begin{aligned}
S + \delta S
&= \int \LL(\phi + \overbar{\phi}, \partial_\mu (phi + \overbar{\phi}) d^4 x \\
&= \int d^4 x \left(
\LL(\overbar{\phi}, \partial_\mu \overbar{\phi})
+
\overbar{\phi} \PD{\phi}{\LL}
+\partial_\mu \overbar{\phi} \PD{(\partial_\mu \phi)}{\LL}
+ \cdots \right)
\end{aligned}
\end{equation}
%
Neglecting any second or higher order terms the change in the action from the assumed solution is
%
\begin{equation}\label{eqn:rotationCurrents:EOF3}
\begin{aligned}
\delta S
&=
\int d^4 x \left( \overbar{\phi} \PD{\phi}{\LL} +\partial_\mu \overbar{\phi} \PD{(\partial_\mu \phi)}{\LL} \right)
\end{aligned}
\end{equation}
%
This is now integrable by parts yielding
%
\begin{equation}\label{eqn:rotationCurrents:EOF4}
\begin{aligned}
\delta S
&=
\int d^3 x \left( {\left. \overbar{\phi} \partial_\mu \LL \right\vert}_{\partial x^\mu} \right)
+
\int d^4 x \overbar{\phi} \left( \PD{\phi}{\LL} - \partial_\mu \PD{(\partial_\mu \phi)}{\LL} \right)
\end{aligned}
\end{equation}
%
Here \(d^3 x\) is taken to mean that part of the integration not including \(dx_\mu\).  The field \(\overbar{\phi}\) is always required to vanish on the boundary as in the dynamic Lagrangian arguments, so the first integral is zero.  If the remainder is zero for all fields \(\overbar{\phi}\), then the inner term must be zero, and we the field Euler-Lagrange equations as a result
%
\begin{equation}\label{eqn:rotationCurrents:EOF5}
\begin{aligned}
\PD{\phi}{\LL} - \partial_\mu \PD{(\partial_\mu \phi)}{\LL} = 0
\end{aligned}
\end{equation}
%
When we have multiple field variables, say \(A_\nu\), the chain rule expansion leading to \eqnref{eqn:rotationCurrents:EOF3} will have to be modified to sum over all the field variables, and we end up instead with
%
\begin{equation}\label{eqn:rotationCurrents:EOF6}
\begin{aligned}
\delta S
&=
\int d^4 x \sum_{\nu} \overbar{A_\nu} \left( \PD{A_\nu}{\LL} - \partial_\mu \PD{(\partial_\mu A_\nu)}{\LL} \right)
\end{aligned}
\end{equation}
%
So for \(\delta S = 0\) for all \(\overbar{A}_\nu\) we have a set of equations, one for each \(\nu\)
%
\begin{equation}\label{eqn:rotationCurrents:EOF7}
\begin{aligned}
\PD{A_\nu}{\LL} - \partial_\mu \PD{(\partial_\mu A_\nu)}{\LL} = 0
\end{aligned}
\end{equation}
%
\section{Field Noether currents}
%
The single parameter Noether conservation equation again is mainly application of the chain rule.  Illustrating with the one field variable case, with an altered field variable \(\phi \rightarrow \phi'(\theta)\), and
%
\begin{equation}\label{eqn:rotationCurrents:FieldNoeth1}
\begin{aligned}
\LL' = \LL(\phi', \partial_\mu \phi')
\end{aligned}
\end{equation}
%
Examining the change of \(\LL'\) with \(\theta\) we have
%
\begin{equation}\label{eqn:rotationCurrents:55}
\begin{aligned}
\frac{d \LL'}{d \theta}
&=
\PD{\phi'}{\LL} \PD{\theta}{\phi'}
+\PD{(\partial_\mu \phi')}{\LL}
\PD{\theta}{(\partial_\mu \phi')}
\end{aligned}
\end{equation}
%
For the last term we can switch up the order of differentiation
%
\begin{equation}\label{eqn:rotationCurrents:75}
\begin{aligned}
\PD{\theta}{(\partial_\mu \phi')}
&=
\PD{\theta}{}
\PD{x^\mu}{\phi'} \\
&= \PD{x^\mu}{} \PD{\theta}{\phi'}
\end{aligned}
\end{equation}
%
Additionally, with substitution of the Euler-Lagrange equations in the first term we have
%
\begin{equation}\label{eqn:rotationCurrents:95}
\begin{aligned}
\frac{d \LL'}{d \theta}
&=
\left( \PD{x^\mu}{} \PD{(\partial_\mu \phi')}{\LL} \right) \PD{\theta}{\phi'}
+\PD{(\partial_\mu \phi')}{\LL} \PD{x^\mu}{} \PD{\theta}{\phi'} \\
\end{aligned}
\end{equation}
%
But this can be directly anti-differentiated yielding the Noether conservation equation
%
\begin{equation}\label{eqn:rotationCurrents:FieldNoeth2}
\begin{aligned}
\frac{d \LL'}{d \theta}
=
\PD{x^\mu}{} \left( \PD{(\partial_\mu \phi')}{\LL} \PD{\theta}{\phi'} \right)
\end{aligned}
\end{equation}
%
With multiple field variables we will have a term in the chain rule expansion for each field variable.  The end result is pretty much the same, but we have to sum over all the fields
%
\begin{equation}\label{eqn:rotationCurrents:FieldNoeth3}
\begin{aligned}
\frac{d \LL'}{d \theta}
=
\sum_\nu \PD{x^\mu}{} \left( \PD{(\partial_\mu {A'}_\nu)}{\LL} \PD{\theta}{{A'}_\nu} \right)
\end{aligned}
\end{equation}
%
Unlike the field Euler-Lagrange equations we have just one here, not one for each field variable.  In this multivariable case, expression in vector form can eliminate the sum over field variables.  With \(A' = {A'}_\nu \gamma^\nu\), we have
%
\begin{equation}\label{eqn:rotationCurrents:FieldNoeth4}
\begin{aligned}
\frac{d \LL'}{d \theta}
=
\PD{x^\mu}{} \left( \gamma_\nu \PD{(\partial_\mu {A'}_\nu)}{\LL} \cdot \PD{\theta}{A'} \right)
\end{aligned}
\end{equation}
%
With an evaluation at \(\theta = 0\), we have finally
%
\begin{equation}\label{eqn:rotationCurrents:FieldNoeth5}
\begin{aligned}
{\left. \frac{d \LL'}{d \theta} \right\vert}_{\theta=0}
=
\PD{x^\mu}{} \left( \gamma_\nu \PD{(\partial_\mu {A}_\nu)}{\LL} \cdot {\left. \PD{\theta}{A'} \right\vert}_{\theta=0}\right)
\end{aligned}
\end{equation}
%
When the Lagrangian alteration is independent of \(\theta\) (i.e. is invariant), it is said that there is a symmetry.  By \eqnref{eqn:rotationCurrents:FieldNoeth5} we have a conserved quantity associated with this symmetry, some quantity, say \(J\) that has a zero divergence.   That is
%
\begin{equation}\label{eqn:rotationCurrents:FieldNoeth6}
\begin{aligned}
J^\mu &= \gamma_\nu \PD{(\partial_\mu {A}_\nu)}{\LL} \cdot {\left. \PD{\theta}{A'} \right\vert}_{\theta=0} \\
0 &= \partial_\mu J^\mu
\end{aligned}
\end{equation}
%
\section{Spacetime translation symmetries and Noether currents}
\index{Noether currents!field translation}
%
Considering the effect of spacetime translation on the Lagrangian we examine the application of the first order linear Taylor series expansion shifting the vector parameters by an increment \(a\).  The Lagrangian alteration is
%
\begin{equation}\label{eqn:rotationCurrents:transCurrent1}
\begin{aligned}
\LL \rightarrow e^{a \cdot \grad }\LL \approx \LL + a \cdot \grad \LL
\end{aligned}
\end{equation}
%
Similar to the addition of derivatives to the Lagrangians of dynamics, we can add in some types of total derivatives \(\partial_\mu F^\mu\) to the Lagrangian without changing the resulting field equations (i.e. there is an associated ``symmetry'' for this Lagrangian alteration).  The directional derivative \(a \cdot \grad \LL = a^\mu \partial_\mu \LL\) appears to be an example of a total derivative alteration that leaves the Lagrangian unchanged.
%
\subsection{On the symmetry}
%
The fact that this translation necessarily results in the same field equations is not necessarily obvious.  Using one of the simplest field Lagrangians, that of the Coulomb electrostatic law, we can illustrate that this is true in at least one case, and also see what is required in the general case
%
\begin{equation}\label{eqn:rotationCurrents:transCurrent2}
\begin{aligned}
\LL = \inv{2} (\spacegrad \phi)^2 - \inv{\epsilon_0}\rho \phi = \inv{2} \sum_m(\partial_m \phi)^2 - \inv{\epsilon_0}\rho \phi
\end{aligned}
\end{equation}
%
With partials written \(\partial_m f = f_m\) we summarize the field Euler-Lagrange equations using the variational derivative
%
\begin{equation}\label{eqn:rotationCurrents:transCurrent3}
\begin{aligned}
\frac{\delta }{\delta \phi} &=
\frac{\partial }{\partial \phi} - \sum_m \partial_m \frac{\partial }{\partial \phi_m}
\end{aligned}
\end{equation}
%
Where the extremum condition \({\delta \LL}/{\delta \phi} = 0\) produces the field equations.

For the Coulomb Lagrangian without (spatial) translation, we have
%
\begin{equation}\label{eqn:rotationCurrents:transCurrent4}
\begin{aligned}
\frac{\delta \LL}{\delta \phi} &=
- \inv{\epsilon_0}\rho - \partial_{mm} \phi
\end{aligned}
\end{equation}
%
So the extremum condition \({\delta \LL}/{\delta \phi} = 0\) gives
%
\begin{equation}\label{eqn:rotationCurrents:transCurrent5}
\begin{aligned}
\spacegrad^2 \phi = - \inv{\epsilon_0}\rho
\end{aligned}
\end{equation}
%
Equivalently, and probably more familiar, we write \(\BE = -\spacegrad \phi\), and get the differential form of Coulomb's law in terms of the electric field
%
\begin{equation}\label{eqn:rotationCurrents:transCurrent6}
\begin{aligned}
\spacegrad \cdot \BE = \inv{\epsilon_0}\rho
\end{aligned}
\end{equation}
%
To consider the translation case we have to first evaluate the first order translation produced by the directional derivative.  This is
%
\begin{equation}\label{eqn:rotationCurrents:115}
\begin{aligned}
\Ba \cdot \spacegrad \LL
&= \sum_m a_m \partial_m \LL \\
&= -\frac{\Ba}{\epsilon_0} \cdot (\rho \spacegrad \phi + \phi \spacegrad \rho)
\end{aligned}
\end{equation}
%
For the translation to be a symmetry the evaluation of the variational derivative must be zero.  In this case we have
%
\begin{equation}\label{eqn:rotationCurrents:135}
\begin{aligned}
\frac{\delta }{\delta \phi} \Ba \cdot \spacegrad \LL
&= -\frac{\Ba}{\epsilon_0} \cdot \frac{\delta }{\delta \phi} (\rho \spacegrad \phi + \phi \spacegrad \rho) \\
&= -\sum_m \frac{a_m}{\epsilon_0} \frac{\delta }{\delta \phi} (\rho \partial_m \phi + \phi \partial_m \rho) \\
&= -\sum_m \frac{a_m}{\epsilon_0} \left( \frac{\partial }{\partial \phi} - \sum_k \partial_k \frac{\partial }{\partial \phi_k}\right) (\rho \phi_m + \phi \rho_m) \\
\end{aligned}
\end{equation}
%
We see that the \(\phi\) partials select only \(\rho\) derivatives whereas the \(\phi_k\) partials select only the \(\rho\) term.  All told we have zero
%
\begin{equation}\label{eqn:rotationCurrents:155}
\begin{aligned}
\left( \frac{\partial }{\partial \phi} - \sum_k \partial_k \frac{\partial }{\partial \phi_k}\right) (\rho \phi_m + \phi \rho_m)
&=
\rho_m - \sum_k \partial_k \rho \delta_{km} \\
&=
\rho_m - \partial_m \rho  \\
&= 0
\end{aligned}
\end{equation}
%
This example illustrates that we have a symmetry provided we can ``commute'' the variational derivative with the gradient
%
\begin{equation}\label{eqn:rotationCurrents:transCurrent7}
\begin{aligned}
\frac{\delta }{\delta \phi} \Ba \cdot \spacegrad \LL
&=
\Ba \cdot \spacegrad \frac{\delta \LL}{\delta \phi}
\end{aligned}
\end{equation}
%
Since \({\delta \LL}/{\delta \phi} = 0\) by construction, the resulting field equations are unaltered by such a modification.

Are there conditions where this commutation is not possible?  Some additional exploration on symmetries associated with addition of derivatives to field Lagrangians was made previously in \chapcite{stressEnergyNoethers}.  After all was said and done, the conclusion motivated by this simple example was also reached.  Namely, we require the commutation condition \eqnref{eqn:rotationCurrents:transCurrent7} between the variational derivative and the gradient of the Lagrangian.
%
\subsection{Existence of a symmetry for translational variation}
%
Considering an example Lagrangian we found that there was a symmetry provided we could commute the variational derivative with the gradient, as in \eqnref{eqn:rotationCurrents:transCurrent7}
%
% blog
%\begin{align}
%\frac{\delta }{\delta \phi} \Ba \cdot \spacegrad \LL
%&=
%\Ba \cdot \spacegrad \frac{\delta \LL}{\delta \phi}
%\end{align}
%
What this really means is not clear in general and a better answer to the existence question for incremental translation can be had by considering the transformation of the action directly around the stationary fields.

Without really any loss of generality we can consider an action with a four dimensional spacetime volume element, and apply the incremental translation operator to this
%
\begin{equation}\label{eqn:rotationCurrents:175}
\begin{aligned}
\int &d^4 x a \cdot \grad \LL( A^\beta + \overbar{A}^\beta, \partial_\alpha A^\beta + \partial_\alpha \overbar{A}^\beta) \\
&=
\int d^4 x a \cdot \grad
\LL( \overbar{A}^\beta, \partial_\alpha \overbar{A}^\beta)
+
\int d^4 x a \cdot \grad
\left(
\PD{A^\beta}{\LL} \overbar{A^\beta}
+\PD{(\partial_\alpha A^\beta)}{\LL} \partial_\alpha \overbar{A^\beta}
\right)
+ \cdots
\end{aligned}
\end{equation}
%
For the first term we have \(a \cdot \grad \int d^4 x \LL( \overbar{A}^\beta, \partial_\alpha \overbar{A}^\beta)\), but this integral is our stationary action.
The remainder, to first order in the field variables, can then be expanded and integrated by parts
%
\begin{equation}\label{eqn:rotationCurrents:195}
\begin{aligned}
\int &d^4 x a^\mu \partial_\mu
\left(
   \PD{A^\beta}{\LL} \overbar{A^\beta}
   +\PD{(\partial_\alpha A^\beta)}{\LL} \partial_\alpha \overbar{A^\beta}
\right) \\
&=
\int d^4 x a^\mu
\lr{
   \left( \partial_\mu \PD{A^\beta}{\LL} \right) \overbar{A^\beta}
   +\PD{A^\beta}{\LL} \left( \partial_\mu \overbar{A^\beta} \right)
} \\
&+\quad \int d^4 x a^\mu
\lr{
   \left( \partial_\mu \PD{(\partial_\alpha A^\beta)}{\LL} \right) \partial_\alpha \overbar{A^\beta}
   +\PD{(\partial_\alpha A^\beta)}{\LL} \left( \partial_\mu \partial_\alpha \overbar{A^\beta} \right)
} \\
&=
\int d^4 x
\lr{
   \left(
      a^\mu
      \partial_\mu \PD{A^\beta}{\LL}
   \right)
   \overbar{A^\beta}
   -
   \left(
      \partial_\mu a^\mu
      \PD{A^\beta}{\LL}
   \right)
   \overbar{A^\beta}
} \\
&\quad +
\int d^4 x
\lr{
   \left( \partial_\mu \PD{(\partial_\alpha A^\beta)}{\LL} \right) \partial_\alpha \overbar{A^\beta}
   -
   \left( \partial_\mu
   a^\mu
   \PD{(\partial_\alpha A^\beta)}{\LL} \right) \partial_\alpha \overbar{A^\beta}
}
.
\end{aligned}
\end{equation}
Since \(a^\mu\) are constants, this is zero, so there can be no contribution to the field equations by the addition of the translation increment to the Lagrangian.
%
\subsection{Noether current derivation}
%
With the assumption that the Lagrangian translation induces a symmetry, we can proceed with the calculation of the Noether current.  This procedure for deriving the Noether current for an incremental spacetime translation follows along similar lines as the scalar alteration considered previously.

We start with the calculation of the first order alteration, expanding the derivatives.  Let us work with a multiple field Lagrangian \(\LL = \LL(A^\beta, \partial_\alpha A^\beta)\) right from the start
%
\begin{equation}\label{eqn:rotationCurrents:215}
\begin{aligned}
a \cdot \grad \LL
&=
a^\mu \partial_\mu \LL \\
&=
a^\mu \left(
\PD{A^\sigma}{\LL} \PD{x^\mu}{A^\sigma}
+\PD{(\partial_\alpha A^\beta)}{\LL} \PD{x^\mu}{(\partial_\alpha A^\beta)}
\right) \\
\end{aligned}
\end{equation}
%
Using the Euler-Lagrange field equations in the first term, and switching integration order in the second this can be written as a single derivative
%
\begin{equation}\label{eqn:rotationCurrents:235}
\begin{aligned}
a \cdot \grad \LL
&=
a^\mu \left(
\partial_\alpha \PD{(\partial_\alpha A^\beta)}{\LL} \PD{x^\mu}{A^\beta}
+\PD{(\partial_\alpha A^\beta)}{\LL} \partial_\alpha \PD{x^\mu}{A^\beta}
\right) \\
&=
a^\mu \partial_\alpha \left(
\PD{(\partial_\alpha A^\beta)}{\LL} \PD{x^\mu}{A^\beta}
\right) \\
%&=
%\partial_\alpha \left(
%\PD{(\partial_\alpha A^\beta)}{\LL} (a \cdot \grad) A^\beta
%\right) \\
\end{aligned}
\end{equation}
%
In the scalar Noether current we were able to form an similar expression, but one that was a first order derivative that could be set to zero, to fix the conservation relationship.  Here there is no such freedom, but we can sneakily subtract \(a \cdot \grad \LL\) from itself to calculate such a zero
%
\begin{equation}\label{eqn:rotationCurrents:transCurrent8}
\begin{aligned}
0 =
\partial_\alpha \left(
\PD{(\partial_\alpha A^\beta)}{\LL} a^\mu \PD{x^\mu}{A^\beta} - a^\alpha \LL
\right)
\end{aligned}
\end{equation}
%
Since this must hold for any vector \(a\), we have the freedom to choose the simplest such vector, a unit vector \(a = \gamma_\nu\), for which \(a^\mu = {\delta^\mu}_\nu\).  Our current and its zero divergence relationship then becomes
%
\begin{equation}\label{eqn:rotationCurrents:transCurrent9}
\begin{aligned}
{T^\alpha}_\nu &= \PD{(\partial_\alpha A^\beta)}{\LL} \partial_\nu A^\beta - {\delta^\alpha}_\nu \LL \\
0 &= \partial_\alpha {T^\alpha}_\nu
\end{aligned}
\end{equation}
%
This is not the symmetric energy momentum tensor that we want in the electrodynamics context although it can be obtained from it by adding just the right zero.
%
\subsection{Relating the canonical energy momentum tensor to the Lagrangian gradient}
%
In \citep{doran2003gap} many tensor quantities are not written in index form, but instead using a vector notation.  In particular, the symmetric energy momentum tensor is expressed as
%
\begin{equation}\label{eqn:rotationCurrents:hoo1}
\begin{aligned}
T(a) = -\frac{\epsilon_0}{2} F a F
\end{aligned}
\end{equation}
%
where the usual tensor form following by taking dot products with \(\gamma^\mu\) and substituting \(a = \gamma^\nu\).  The conservation equation for the canonical energy momentum tensor of \eqnref{eqn:rotationCurrents:transCurrent9} can be put into a similar vector form
%
\begin{equation}\label{eqn:rotationCurrents:hoo2}
\begin{aligned}
T(a) &= \gamma_\alpha \PD{(\partial_\alpha A^\beta)}{\LL} (a \cdot \grad) A^\beta - a \LL \\
0 &= \grad \cdot T(a)
\end{aligned}
\end{equation}
%
The adjoint \(\overbar{T}\) of the tensor can be calculated from the definition
%
\begin{equation}\label{eqn:rotationCurrents:hoo3}
\begin{aligned}
\grad \cdot T(a) = a \cdot \overbar{T}(\grad)
\end{aligned}
\end{equation}
%
Somewhat unintuitively, this is a function of the gradient.  Playing around with factoring out the displacement vector \(a\) from \eqnref{eqn:rotationCurrents:hoo2} that the energy momentum adjoint essentially provides an expansion of the gradient of the Lagrangian.  To prepare, let us introduce some helper notation
%
\begin{equation}\label{eqn:rotationCurrents:hoo4}
\begin{aligned}
\Pi_\beta \equiv \gamma_\alpha \PD{(\partial_\alpha A^\beta)}{\LL}
\end{aligned}
\end{equation}
%
With this our Noether current equation becomes
%
\begin{equation}\label{eqn:rotationCurrents:255}
\begin{aligned}
\grad \cdot T(a)
&= \gpgradezero{ \grad T(a) } \\
&= \gpgradezero{ \grad (\Pi_\beta (a \cdot \grad) A^\beta - a \grad \LL ) } \\
&= \gpgradezero{ \grad \left(\inv{2} \Pi_\beta (a (\grad A^\beta) + (\grad A^\beta) a) - a \LL \right) } \\
\end{aligned}
\end{equation}
%
Cyclic permutation of the vector products \(\gpgradezero{a b c} = \gpgradezero{ c a b}\) can be used in the scalar selection.  This is a little more tractable with some helper notation for the \(A^\beta\) gradients, say \(v^\beta = \grad A^\beta\).  Because of the operator nature of the gradient once the vector order is permuted we have to allow for the gradient to act left or right or both, so arrows are used to disambiguate this where appropriate.
%
\begin{equation}\label{eqn:rotationCurrents:275}
\begin{aligned}
\grad \cdot T(a)
&= \gpgradezero{ \grad \left(\inv{2} \Pi_\beta a v^\beta +\Pi_\beta v^\beta a \right) - \grad \LL a } \\
&= \gpgradezero{
\left( \inv{2} v^\beta \lrgrad \Pi_\beta
\inv{2} \grad (\Pi_\beta v^\beta)
- \grad \LL \right) a } \\
&=
a \cdot \left(
\inv{2} \gpgradeone{ v^\beta \lrgrad \Pi_\beta + \grad (\Pi_\beta v^\beta) } - \grad \LL
\right)
\end{aligned}
\end{equation}
%
This dotted with quantity is the adjoint of the canonical energy momentum tensor
%
\begin{equation}\label{eqn:rotationCurrents:hoo5}
\begin{aligned}
\overbar{T}(\grad) &=
\inv{2} \gpgradeone{ v^\beta \lrgrad \Pi_\beta + \grad (\Pi_\beta v^\beta) } - \grad \LL
\end{aligned}
\end{equation}
%
This can however, be expanded further.  First tackling the
bidirectional gradient vector term we can utilize the property that the reverse of a vector leaves the vector unchanged.  This gives us
%
\begin{equation}\label{eqn:rotationCurrents:295}
\begin{aligned}
\gpgradeone{ v^\beta \lrgrad \Pi_\beta }
&=
\gpgradeone{ v^\beta (\rgrad \Pi_\beta) }
+\gpgradeone{ (v^\beta \lgrad) \Pi_\beta } \\
&=
\gpgradeone{ v^\beta (\rgrad \Pi_\beta) }
+\gpgradeone{ \Pi_\beta (\rgrad v^\beta) } \\
\end{aligned}
\end{equation}
%
In the remaining term, using the Hestenes overdot notation clarify the scope of the operator, we have
%
\begin{equation}\label{eqn:rotationCurrents:315}
\begin{aligned}
\overbar{T}(\grad)
&=
\inv{2} \left(
\gpgradeone{ v^\beta (\grad \Pi_\beta) }
+\gpgradeone{ \Pi_\beta (\grad v^\beta) }
+\gpgradeone{ (\grad \Pi_\beta) v^\beta } + \gpgradeone{ \grad' \Pi_\beta {v^\beta}'}
\right)
- \grad \LL \\
\end{aligned}
\end{equation}
%
The grouping of the first and third terms above simplifies nicely
%
\begin{equation}\label{eqn:rotationCurrents:335}
\begin{aligned}
\inv{2} &
\gpgradeone{ v^\beta (\grad \Pi_\beta) } +\inv{2} \gpgradeone{ (\grad \Pi_\beta) v^\beta } \\
&=
v^\beta (\grad \cdot \Pi_\beta)
+\inv{2} \gpgradeone{ v^\beta (\grad \wedge \Pi_\beta) } +\gpgradeone{ (\grad \wedge \Pi_\beta) v^\beta }.
\end{aligned}
\end{equation}
%
Since \(a (b \wedge c) + (b \wedge c) a = 2 a \wedge b \wedge c\), which is purely a trivector, the vector grade selection above is zero.   This leaves the adjoint reduced to
%
\begin{equation}\label{eqn:rotationCurrents:355}
\begin{aligned}
\overbar{T}(\grad)
&=
v^\beta (\grad \cdot \Pi_\beta)
+\inv{2} \left(
\gpgradeone{ \Pi_\beta (\grad v^\beta) }
+ \gpgradeone{ \grad' \Pi_\beta {v^\beta}'}
\right)
- \grad \LL \\
\end{aligned}
\end{equation}
%
For the remainder vector grade selection operators we have something that is of the following form
%
\begin{equation}\label{eqn:rotationCurrents:375}
\begin{aligned}
\inv{2} \gpgradeone{ a b c + b a c } = (a \cdot b ) c
\end{aligned}
\end{equation}
%
And we are finally able to put the adjoint into a form that has no remaining grade selection operators
%
\begin{equation}\label{eqn:rotationCurrents:395}
\begin{aligned}
\overbar{T}(\grad)
&= (\grad A^\beta) (\grad \cdot \Pi_\beta) +(\Pi_\beta \cdot \grad) (\grad A^\beta) -\grad \LL \\
&= (\grad A^\beta) (\rgrad \cdot \Pi_\beta) +
(\grad A^\beta) (\lgrad \cdot \Pi_\beta)
-\grad \LL \\
&= (\grad A^\beta) (\lrgrad \cdot \Pi_\beta) -\grad \LL
\end{aligned}
\end{equation}
%
Recapping, we have for the tensor and its adjoint
%
\begin{equation}\label{eqn:rotationCurrents:hoo6}
\begin{aligned}
0 &= \grad \cdot T(a) = a \cdot \overbar{T}(\grad)   \\
\Pi_\beta &\equiv \gamma_\alpha \PD{(\partial_\alpha A^\beta)}{\LL} \\
T(a) &= \Pi_\beta (a \cdot \grad) A^\beta - a \grad \LL  \\
\overbar{T}(\grad) &= (\grad A^\beta) (\lrgrad \cdot \Pi_\beta) - \grad \LL
\end{aligned}
\end{equation}
%
For the adjoint, since \(a \cdot \overbar{T}(\grad) = 0\) for all \(a\), we must also have \(\overbar{T}(\grad) = 0\), which means the adjoint of the canonical energy momentum tensor really provides not much more than a recipe for computing the Lagrangian gradient
%
\begin{equation}\label{eqn:rotationCurrents:hoo10}
\begin{aligned}
\grad \LL &=
(\grad A^\beta) (\lrgrad \cdot \Pi_\beta)
\end{aligned}
\end{equation}
%
Having seen the adjoint notation, it was natural to see what this was for a multiple scalar field variable Lagrangian, even if it is not intrinsically useful.  Observe that the identity \eqnref{eqn:rotationCurrents:hoo10}, obtained so laboriously, is not more than syntactic sugar for the chain rule expansion of the Lagrangian partials (plus application of the Euler-Lagrange field equations).  We could obtain this directly if desired much more easily than by factoring out \(a\) from \(\grad \cdot T(a) = 0\).
%
\begin{equation}\label{eqn:rotationCurrents:415}
\begin{aligned}
\partial_\mu \LL
&=
\PD{A^\beta}{\LL} \partial_\mu A^\beta
+\PD{(\partial_\alpha A^\beta)}{\LL} \partial_\mu \partial_\alpha A^\beta \\
&=
\left( \partial_\alpha \PD{(\partial_\alpha A^\beta)}{\LL} \right) \partial_\mu A^\beta
+\PD{(\partial_\alpha A^\beta)}{\LL} \partial_\alpha \partial_\mu A^\beta \\
&=
\partial_\alpha
\left(
\left( \PD{(\partial_\alpha A^\beta)}{\LL} \right) \partial_\mu A^\beta
\right) \\
\end{aligned}
\end{equation}
%
Summing over \(\mu\) for the gradient, this reproduces \eqnref{eqn:rotationCurrents:hoo10}, with much less work
%
\begin{equation}\label{eqn:rotationCurrents:435}
\begin{aligned}
\grad \LL
&= \gamma^\mu \partial_\mu \LL \\
&=
\partial_\alpha
\left(
\left( \PD{(\partial_\alpha A^\beta)}{\LL} \right) (\grad A^\beta)
\right) \\
&=
(\Pi_\beta \cdot \lrgrad) (\grad A^\beta)
\end{aligned}
\end{equation}
%
Observe that the Euler-Lagrange field equations are implied in this relationship, so perhaps it has some utility.  Also note that while it is simpler to directly compute this, without having started with the canonical energy momentum tensor, we would not know how the two of these were related.
%
\section{Noether current, infinitesimal Lorentz transformation.}
%
Let us assume that we can use the exponential generator of rotations
%
\begin{equation}\label{eqn:rotationCurrents:LorTx1}
\begin{aligned}
e^{(i \cdot x) \cdot \grad} = 1 + (i \cdot x) \cdot \grad + \cdots
\end{aligned}
\end{equation}
%
to alter a Lagrangian density.
%
In particular, that we can use the first order approximation of this Taylor series, applying the incremental rotation operator \((i \cdot x) \cdot \grad = i \cdot (x \wedge \grad)\) to transform the Lagrangian.
%
\begin{equation}\label{eqn:rotationCurrents:LorTx2}
\begin{aligned}
\LL \rightarrow \LL + (i \cdot x) \cdot \grad \LL
\end{aligned}
\end{equation}
%
Suppose that we parametrize the rotation bivector \(i\) using two perpendicular unit vectors \(u\), and \(v\).  Here perpendicular is in the sense \(u v = -v u\) so that \(i = u \wedge v = u v\).  For the bivector expressed this way our incremental rotation operator takes the form
%
\begin{equation}\label{eqn:rotationCurrents:455}
\begin{aligned}
(i \cdot x) \cdot \grad
&=
((u \wedge v) \cdot x) \cdot \grad \\
&=
(u (v \cdot x) - v (u \cdot x)) \cdot \grad \\
&=
(v \cdot x) u \cdot \grad - (u \cdot x)) v \cdot \grad \\
\end{aligned}
\end{equation}
%
The operator is reduced to a pair of torque-like scaled directional derivatives, and we have already examined the Noether currents for the translations induced by the directional derivatives.  It is not unreasonable to take exactly the same approach to consider rotation symmetries as we did for translation.  We found for incremental translations
%
\begin{equation}\label{eqn:rotationCurrents:LorTx3}
\begin{aligned}
a \cdot \grad \LL
&=
\partial_\alpha \left(
\PD{(\partial_\alpha A^\beta)}{\LL} (a \cdot \grad) {A^\beta}
\right)
\end{aligned}
\end{equation}
%
So for incremental rotations the change to the Lagrangian is
%
\begin{equation}\label{eqn:rotationCurrents:LorTx4}
\begin{aligned}
(i \cdot x) \cdot \grad \LL
&=
(v \cdot x)
\partial_\alpha \left(
\PD{(\partial_\alpha A^\beta)}{\LL} (u \cdot \grad) {A^\beta}
\right)
-(u \cdot x)
\partial_\alpha \left(
\PD{(\partial_\alpha A^\beta)}{\LL} (v \cdot \grad) {A^\beta}
\right)
\end{aligned}
\end{equation}
%
Since the choice to make \(u\) and \(v\) both unit vectors and perpendicular has been made, there is really no loss in generality to align these with a pair of the basis vectors, say \(u = \gamma_\mu\) and \(v = \gamma_\nu\).
%
The incremental rotation operator is reduced to
\begin{equation}\label{eqn:rotationCurrents:475}
\begin{aligned}
(i \cdot x) \cdot \grad
&=
(\gamma_\nu \cdot x) \gamma_\mu \cdot \grad - (\gamma_\mu \cdot x)) \gamma_\nu \cdot \grad \\
&=
x_\nu \partial_\mu - x_\mu \partial_\nu \\
\end{aligned}
\end{equation}
%
Similarly the change to the Lagrangian is
%
\begin{equation}\label{eqn:rotationCurrents:LorTx5}
\begin{aligned}
(i \cdot x) \cdot \grad \LL
&=
x_\nu
\partial_\alpha \left(
\PD{(\partial_\alpha A^\beta)}{\LL} \partial_\mu {A^\beta}
\right)
-
x_\mu
\partial_\alpha \left(
\PD{(\partial_\alpha A^\beta)}{\LL} \partial_\nu {A^\beta}
\right)
\end{aligned}
\end{equation}
%
Subtracting the two, essentially forming \((i \cdot x) \cdot \grad \LL - (i \cdot x) \cdot \grad \LL = 0\), we have
%
\begin{equation}\label{eqn:rotationCurrents:LorTx6}
\begin{aligned}
0 =
x_\nu
\partial_\alpha \left(
\PD{(\partial_\alpha A^\beta)}{\LL} \partial_\mu {A^\beta}
- {\delta^\alpha}_\mu \LL
\right)
-
x_\mu
\partial_\alpha \left(
\PD{(\partial_\alpha A^\beta)}{\LL} \partial_\nu {A^\beta}
- {\delta^\alpha}_\nu \LL
\right)
\end{aligned}
\end{equation}
%
We previously wrote
%
\begin{equation}\label{eqn:rotationCurrents:495}
\begin{aligned}
{T^\alpha}_\nu &= \PD{(\partial_\alpha A^\beta)}{\LL} \partial_\nu A^\beta - {\delta^\alpha}_\nu \LL \\
\end{aligned}
\end{equation}
%
for the Noether current of spacetime translation, and with that our conservation equation becomes
%
\begin{equation}\label{eqn:rotationCurrents:LorTx7}
\begin{aligned}
0 = x_\nu \partial_\alpha {T^\alpha}_\mu - x_\mu \partial_\alpha {T^\alpha}_\nu
\end{aligned}
\end{equation}
%
As is, this does not really appear to say much, since we previously also found \(\partial_\alpha {T^\alpha}_\nu = 0\).  We appear to need a way to pull the x coordinates into the derivatives to come up with a more interesting statement.  A test expansion of \(\grad \cdot (i \cdot x) \LL\) to see what is left over compared to \((i \cdot x) \cdot \grad \LL\) shows that there is in fact no difference, and we actually have the identity
%
\begin{equation}\label{eqn:rotationCurrents:LorTx8}
\begin{aligned}
i \cdot (x \wedge \grad) \LL = (i \cdot x) \cdot \grad \LL = \grad \cdot (i \cdot x) \LL
\end{aligned}
\end{equation}
%
This suggests that we can pull the \(x\) coordinates into the derivatives of \eqnref{eqn:rotationCurrents:LorTx7} as in
%
\begin{equation}\label{eqn:rotationCurrents:LorTx9}
\begin{aligned}
0 = \partial_\alpha \left( {T^\alpha}_\mu x_\nu - {T^\alpha}_\nu x_\mu \right)
\end{aligned}
\end{equation}
%
However, expanding this derivative shows that this is fact not the case.  Instead we have
%
\begin{equation}\label{eqn:rotationCurrents:515}
\begin{aligned}
\partial_\alpha \left( {T^\alpha}_\mu x_\nu - {T^\alpha}_\nu x_\mu \right)
&=
{T^\alpha}_\mu \partial_\alpha x_\nu
- {T^\alpha}_\nu \partial_\alpha x_\mu  \\
&=
{T^\alpha}_\mu \eta_{\alpha\nu}
- {T^\alpha}_\nu \eta_{\alpha\mu}  \\
&=
T_{\nu\mu} - T_{\mu\nu}
\end{aligned}
\end{equation}
%
So instead of a Noether current, following the procedure used to calculate the spacetime translation current, we have only a mediocre compromise
%
\begin{equation}\label{eqn:rotationCurrents:LorTx10}
\begin{aligned}
{M^{\alpha}}_{\mu\nu} &\equiv {T^\alpha}_\mu x_\nu - {T^\alpha}_\nu x_\mu \\
\partial_\alpha {M^{\alpha}}_{\mu\nu} &= T_{\nu\mu} - T_{\mu\nu}
\end{aligned}
\end{equation}
%
Jackson \citep{jackson1975cew} ends up with a similar index upper expression
%
\begin{equation}\label{eqn:rotationCurrents:LorTx11}
\begin{aligned}
M^{\alpha\beta\gamma} &\equiv T^{\alpha\beta} x^\gamma - T^{\alpha\gamma} x^\beta \\
\end{aligned}
\end{equation}
%
and then uses a requirement for vanishing 4-divergence of this quantity
%
\begin{equation}\label{eqn:rotationCurrents:LorTx12}
\begin{aligned}
0 &= \partial_\alpha M^{\alpha\beta\gamma}
\end{aligned}
\end{equation}
%
to symmetries this tensor by subtracting off all the antisymmetric portions.  The differences compared to Jackson with upper verses lower indices are minor for we can follow the same arguments and arrive at the same sort of \(0 - 0 = 0\) result as we had in \eqnref{eqn:rotationCurrents:LorTx7}
%
\begin{equation}\label{eqn:rotationCurrents:LorTx13}
\begin{aligned}
0 = x^\nu \partial_\alpha T^{\alpha\mu} - x^\mu \partial_\alpha T^{\alpha\nu}
\end{aligned}
\end{equation}
%
The only difference is that our not-really-a-conservation equation becomes
%
\begin{equation}\label{eqn:rotationCurrents:LorTx14}
\begin{aligned}
\partial_\alpha M^{\alpha\mu\nu} =  T^{\nu\mu} - T^{\mu\nu}
\end{aligned}
\end{equation}
%
\subsection{An example of the symmetry}
%
While not a proof that application of the incremental rotation operator is a symmetry, an example at least provides some comfort that this is a reasonable thing to attempt.  Again, let us consider the Coulomb Lagrangian
%
\begin{equation}\label{eqn:rotationCurrents:535}
\begin{aligned}
\LL = \inv{2} (\spacegrad \phi)^2 - \inv{\epsilon_0}\rho \phi
\end{aligned}
\end{equation}
%
For this we have
%
\begin{equation}\label{eqn:rotationCurrents:555}
\begin{aligned}
\LL'
&= \LL + (i \cdot \Bx) \cdot \spacegrad \LL \\
&= \LL - (i \cdot \Bx) \cdot \inv{\epsilon_0} \left( \rho \spacegrad \phi + \phi \spacegrad \rho \right)
\end{aligned}
\end{equation}
%
If the variational derivative of the incremental rotation contribution is zero, then we have a symmetry.
%
\begin{equation}\label{eqn:rotationCurrents:575}
\begin{aligned}
\frac{\delta }{\delta \phi} (i \cdot \Bx) \cdot \spacegrad \LL \\
&=
(i \cdot \Bx) \cdot \inv{\epsilon_0} \spacegrad \rho
- \sum_m \partial_m \left( (i \cdot \Bx) \cdot \inv{\epsilon_0} \rho \Be_m \right) \\
&=
(i \cdot \Bx) \cdot \inv{\epsilon_0} \spacegrad \rho
- \spacegrad \cdot \left( (i \cdot \Bx) \inv{\epsilon_0} \rho \right) \\
\end{aligned}
\end{equation}
%
As found in \eqnref{eqn:rotationCurrents:LorTx8}, we have \((i \cdot \Bx) \cdot \spacegrad = \spacegrad \cdot (i \cdot \Bx)\), so we have
%
\begin{equation}\label{eqn:rotationCurrents:LorTx15}
\begin{aligned}
\frac{\delta }{\delta \phi} (i \cdot \Bx) \cdot \spacegrad \LL = 0
\end{aligned}
\end{equation}
%
for this specific Lagrangian as expected.

Note that the test expansion I used to state \eqnref{eqn:rotationCurrents:LorTx8} was done using only the bivector \(i = \gamma_\mu \wedge \gamma_\nu\).  An expansion with \(i = u^\alpha u^\beta \gamma_\alpha \wedge \gamma_\beta\) shows that this is also the case in shows that this is true more generally.  Specifically, this expansion gives
%
\begin{equation}\label{eqn:rotationCurrents:595}
\begin{aligned}
\grad \cdot (i \cdot x) \LL
&= (i \cdot x) \cdot \grad \LL + (\eta_{\alpha\beta} - \eta_{\beta\alpha}) u^\alpha v^\beta \LL \\
&= (i \cdot x) \cdot \grad \LL
\end{aligned}
\end{equation}
%
(since the metric tensor is symmetric).

Loosely speaking, the geometric reason for this is that \(\grad \cdot f(x)\) takes its maximum (or minimum) when \(f(x)\) is colinear with \(x\) and is zero when \(f(x)\) is perpendicular to \(x\).  The vector \(i \cdot x\) is a combined projection and 90 degree rotation in the plane of the bivector, and the divergence is left with no colinear components to operate on.

While this commutation of the \(i \cdot \Bx\) with the divergence operator did not help with finding the Noether current, it does at least show that we have a symmetry.  Demonstrating the invariance for the general Lagrangian (at least the single field variable case) likely follows the same procedure as in this specific example above.

\subsection{General existence of the rotational symmetry}
\index{Noether currents!field rotation}
The example above hints at a general method to demonstrate that the incremental Lorentz transform produces a symmetry.  It will be sufficient to consider the variation around the stationary field variables for the change due to the action from the incremental rotation operator.  That is
\begin{equation}\label{eqn:rotationCurrents:LorSymExistence1}
\begin{aligned}
\delta S = \int d^4 x (i \cdot x) \cdot \grad \LL( A^\beta + \overbar{A}^\beta, \partial_\alpha A^\beta + \partial_\alpha \overbar{A}^\beta)
\end{aligned}
\end{equation}
%
Performing a first order Taylor expansion of the Lagrangian around the stationary field variables we have
%
\begin{equation}\label{eqn:rotationCurrents:615}
\begin{aligned}
\delta S
&= \int d^4 x (i \cdot x) \cdot \gamma^\mu \partial_\mu \LL( A^\beta + \overbar{A}^\beta, \partial_\alpha A^\beta + \partial_\alpha \overbar{A}^\beta) \\
&= \int d^4 x (i \cdot x) \cdot \gamma^\mu \partial_\mu \left(
\PD{A^\beta}{\LL} \overbar{A}^\beta
+\PD{(\partial_\alpha A^\beta)}{\LL} (\partial_\alpha \overbar{A}^\beta)
\right) \\
&=
\int d^4 x (i \cdot x) \cdot \gamma^\mu \\
&\qquad
\left(
   \left(\partial_\mu \PD{A^\beta}{\LL}\right) \overbar{A}^\beta
   +\PD{A^\beta}{\LL} \partial_\mu \overbar{A}^\beta
\right) \\
&+\,\int d^4 x (i \cdot x) \cdot \gamma^\mu \\
&\,\qquad
\left(
   \left(\partial_\mu \PD{(\partial_\alpha A^\beta)}{\LL}\right) (\partial_\alpha \overbar{A}^\beta)
   +\PD{(\partial_\alpha A^\beta)}{\LL} \partial_\mu (\partial_\alpha \overbar{A}^\beta)
\right).
\end{aligned}
\end{equation}
%
Doing the integration by parts we have
%
\begin{equation}\label{eqn:rotationCurrents:635}
\begin{aligned}
\delta S
&=
\int d^4 x \overbar{A}^\beta \gamma^\mu \cdot
\left(
(i \cdot x)
\left(\partial_\mu \PD{A^\beta}{\LL}\right)
-\partial_\mu \left(\PD{A^\beta}{\LL} (i \cdot x)\right)
\right)
\\
&+
\int d^4 x
(\partial_\alpha \overbar{A}^\beta)
\gamma^\mu \cdot
\left(
(i \cdot x)
\left(\partial_\mu \PD{(\partial_\alpha A^\beta)}{\LL}\right)
-\partial_\mu \left( \PD{(\partial_\alpha A^\beta)}{\LL} (i \cdot x) \right)
\right)
\\
&=
\int d^4 x \overbar{A}^\beta
\left(
(i \cdot x) \cdot \grad
\PD{A^\beta}{\LL}
- \grad \cdot (i \cdot x) \PD{A^\beta}{\LL}
\right)
\\
&\qquad
+
(\partial_\alpha \overbar{A}^\beta)
\left(
(i \cdot x) \cdot \grad
\PD{(\partial_\alpha A^\beta)}{\LL}
- \grad \cdot (i \cdot x) \PD{(\partial_\alpha A^\beta)}{\LL}
\right)
\end{aligned}
\end{equation}
%
Since \((i \cdot x) \cdot \grad f = \grad \cdot (i \cdot x) f\) for any \(f\), there is no change to the resulting field equations due to this incremental rotation, so we have a symmetry for any Lagrangian that is first order in its derivatives.
%
%\EndArticle
