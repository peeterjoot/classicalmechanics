%
% Copyright � 2012 Peeter Joot.  All Rights Reserved.
% Licenced as described in the file LICENSE under the root directory of this GIT repository.
%
%
%
%
%\input{../peeter_prologue.tex}
%
%\chapter{Lorentz force from Lagrangian (non-covariant)}
\index{Lorentz force!non-covariant}
\label{chap:jackson12Dash9}
%
%\blogpage{http://sites.google.com/site/peeterjoot/math2009/jackson12Dash9.pdf}
%\date{Sept 22, 2009}
%\revisionInfo{\(RCSfile: jackson12Dash9.tex,v \) Last \(Revision: 1.4 \) \(Date: 2009/10/22 02:07:20 \)}
%
%\beginArtWithToc
\beginArtNoToc
%
\section{Motivation}
%
Jackson \citep{jackson1975cew} gives the Lorentz force non-covariant Lagrangian
%
\begin{equation}\label{eqn:jacksonLorLag:foo1}
\begin{aligned}
L = - m c^2 \sqrt{1 -\Bu^2/c^2} + \frac{e}{c} \Bu \cdot \BA - e \phi
\end{aligned}
\end{equation}
%
and leaves it as an exercise for the reader to verify that this produces the Lorentz force law.  Felt like trying this anew since I recall having trouble the first time I tried it (the covariant derivation was easier).
%
\section{Guts}
%
Jackson gives a tip to use the convective derivative (yet another name for the chain rule), and using this in the Euler Lagrange equations we have
%
\begin{equation}\label{eqn:jacksonLorLag:foo2}
\begin{aligned}
\spacegrad \LL = \frac{d}{dt} \spacegrad_\Bu \LL = \left( \frac{\partial}{\partial t} + \Bu \cdot \spacegrad \right) \sigma_a \frac{\partial \LL}{\partial \xdot^a}
\end{aligned}
\end{equation}
%
where \(\{\sigma_a\}\) is the spatial basis.  The first order of business is calculating the gradient and conjugate momenta.  For the latter we have
%
\begin{equation}\label{eqn:jackson12Dash9:23}
\begin{aligned}
\sigma_a \frac{\partial \LL}{\partial \xdot^a}
&=
\sigma_a \left(- m c^2 \gamma \inv{2} (-2) \xdot^a/c^2 + \frac{e}{c} A^a \right) \\
&=
m \gamma \Bu + \frac{e}{c} \BA \\
&\equiv \Bp + \frac{e}{c}\BA
\end{aligned}
\end{equation}
%
Applying the convective derivative we have
%
\begin{equation}\label{eqn:jackson12Dash9:43}
\begin{aligned}
\frac{d}{dt}
\sigma_a \frac{\partial \LL}{\partial \xdot^a}
&=
\frac{d\Bp}{dt}
+ \frac{e}{c} \frac{\partial \BA}{\partial t}
+ \frac{e}{c} \Bu \cdot \spacegrad \BA
\end{aligned}
\end{equation}
%
For the gradient we have
%
\begin{equation}\label{eqn:jackson12Dash9:63}
\begin{aligned}
\sigma_a \frac{\partial \LL}{\partial x^a} = e\left( \inv{c}\xdot^b \spacegrad A^b - \spacegrad \phi \right)
\end{aligned}
\end{equation}
%
Rearranging \eqnref{eqn:jacksonLorLag:foo2} for this Lagrangian we have
%
\begin{equation}\label{eqn:jackson12Dash9:83}
\begin{aligned}
\frac{d\Bp}{dt}
=
e \left(
- \spacegrad \phi
- \frac{1}{c} \frac{\partial \BA}{\partial t}
- \frac{1}{c} \Bu \cdot \spacegrad \BA
 +\inv{c} \xdot^b \spacegrad A^b
\right)
\end{aligned}
\end{equation}
%
The first two terms are the electric field
%
\begin{equation}\label{eqn:jackson12Dash9:103}
\begin{aligned}
\BE \equiv
- \spacegrad \phi
- \frac{1}{c} \frac{\partial \BA}{\partial t}
\end{aligned}
\end{equation}
%
So it remains to be shown that the remaining two equal \((\Bu/c) \cross \BB = (\Bu/c) \cross (\spacegrad \cross \BA)\).  Using the Hestenes notation using primes to denote what the gradient is operating on, we have
%
\begin{equation}\label{eqn:jackson12Dash9:123}
\begin{aligned}
\xdot^b \spacegrad A^b - \Bu \cdot \spacegrad \BA
&=
\spacegrad' \Bu \cdot \BA' - \Bu \cdot \spacegrad \BA \\
&=
-\Bu \cdot (\spacegrad \wedge \BA) \\
&=
\inv{2} \left(
(\spacegrad \wedge \BA) \Bu  -
\Bu (\spacegrad \wedge \BA)
\right) \\
&=
\frac{I}{2} \left(
(\spacegrad \cross \BA) \Bu -
\Bu (\spacegrad \cross \BA)
\right) \\
&=
-I (\Bu \wedge \BB) \\
&=
-I I (\Bu \cross \BB) \\
&=
\Bu \cross \BB \\
\end{aligned}
\end{equation}
%
I have used the Geometric Algebra identities I am familiar with to regroup things, but this last bit can likely be done with index manipulation too.  The exercise is complete, and we have from the Lagrangian
%
\begin{equation}\label{eqn:jacksonLorLag:foo3}
\begin{aligned}
\frac{d\Bp}{dt} = e \left( \BE + \inv{c} \Bu \cross \BB \right)
\end{aligned}
\end{equation}
%
%\EndArticle
%%\EndNoBibArticle
