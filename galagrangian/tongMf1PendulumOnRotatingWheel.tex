%
% Copyright © 2012 Peeter Joot.  All Rights Reserved.
% Licenced as described in the file LICENSE under the root directory of this GIT repository.
%
\makeoproblem{Pendulum on a rotating wheel.}{tongmf1:pr5}{\citep{TongMf1} p5.}{
Lagrangian and equations of motion for pendulum with pivot moving on a circle.
}
%
\makeanswer{tongmf1:pr5}{
%
Express the position of the pivot point on the wheel with:
%
\begin{equation}\label{eqn:tongMf1:1480}
q_1 = R e^{-i\omega t}.
\end{equation}
%
The position of the mass is then:
\begin{equation}\label{eqn:tongMf1:1500}
q_2 = R e^{-i\omega t} - il e^{i\theta}.
\end{equation}
%
The velocity of the mass is then:
\begin{equation}\label{eqn:tongMf1:1520}
\qdot_2 = -i(\dotomega t + \omega)R e^{-i\omega t} + l \dottheta e^{i\theta}.
\end{equation}
%
Let \(\omega t = \alpha\), we have a Kinetic energy of:
%
\begin{equation}\label{eqn:tongMf1:480}
\begin{aligned}
\inv{2} m \abs{\qdot_2}^2
&= \inv{2} m \Abs{-i\dotalpha R e^{-i\omega t} + l \dottheta e^{i\theta}}^2 \\
&= \inv{2} m \left( R^2 \dotalpha^2 + l^2 \dottheta^2 + 2 R l \dotalpha \dottheta \Real
\lr{  -i e^{-i\alpha -i\theta} }
 \right) \\
&= \inv{2} m \left( R^2 \dotalpha^2 + l^2 \dottheta^2 + 2 R l \dotalpha \dottheta \cos( -\alpha -\theta -\pi/2 ) \right) \\
&= \inv{2} m \left( R^2 \dotalpha^2 + l^2 \dottheta^2 - 2 R l \dotalpha \dottheta \sin( \alpha +\theta ) \right) .
\end{aligned}
\end{equation}
%
The potential energy in the Lagrangian does not depend on the position of the pivot, only the angle from vertical, so it is thus:
%
\begin{equation}\label{eqn:tongMf1:500}
\begin{aligned}
V &= mg l ( 1 - \cos\theta ) \\
V' &= 0 - m g l \cos\theta.
\end{aligned}
\end{equation}
%
Depending on whether one measures the potential up from the lowest potential point, or measures decreasing potential from zero at the horizontal.
Either way, combining the kinetic and potential terms, and dividing through by \(ml^2\) we have the Lagrangian of:
%
\begin{equation}\label{eqn:tongMf1:1540}
\Lq = \inv{2} \left( (R/l)^2 \dotalpha^2 + \dottheta^2 - 2 (R/l) \dotalpha \dottheta \sin( \alpha +\theta ) \right) + (g/l) \cos\theta.
\end{equation}
%
\paragraph{Digression.  Reduction of the Lagrangian.}
\label{chap:PJTongMf1:addDerivative}
%
Now, in Tong's solutions for this problem (which he emailed me since I questioned problem 2), he had \(\dotalpha = \omega = \text{constant}\), which allows the
Lagrangian above to be expressed as:
%
\begin{dmath}\label{eqn:tongMf1:1560}
\Lq =
\inv{2} \left( (R/l)^2 \omega^2 + \dottheta^2 \right)
+ \frac{d}{dt} ((R/l) \cos( \omega t + \theta )) + \omega (R/l) \sin(\omega t + \theta)
+ (g/l) \cos\theta.
\end{dmath}
%
and he made the surprising step of removing that cosine term completely, with a statement that it would not effect the dynamics because it was a time derivative.
That turns out to be a generalized result, but I had to prove it to myself.  I also asked around on PF about this, and it was not any named property of Lagrangians,
but was a theorem in some texts.
%
First consider the simple example of a Lagrangian with such a cosine derivative term added to it:
%
\begin{equation}\label{eqn:tongMf1:1580}
\Lq' = \Lq + \frac{d}{dt} A \cos(\omega t + \theta).
\end{equation}
%
and compute the equations of motion from this:
\begin{equation}\label{eqn:tongMf1:520}
\begin{aligned}
0
&= \PD{\theta}{\Lq'} - \frac{d}{dt}\left(\PD{\dottheta}{\Lq'}\right) \\
&= \PD{\theta}{\Lq} - \frac{d}{dt}\left(\PD{\dottheta}{\Lq}\right)
+ \PD{\theta}{} {\frac{d}{dt} A \cos(\omega t + \theta)}
- \frac{d}{dt} \PD{\dottheta}{} \frac{d}{dt} A \cos(\omega t + \theta) \\
&= \PD{\theta}{\Lq} - \frac{d}{dt}\left(\PD{\dottheta}{\Lq}\right)
- \PD{\theta}{} A \dottheta \sin(\omega t + \theta) +
\frac{d}{dt} \PD{\dottheta}{} A \dottheta \sin(\omega t + \theta) \\
&= \PD{\theta}{\Lq} - \frac{d}{dt}\left(\PD{\dottheta}{\Lq}\right)
- \PD{\theta}{} A \dottheta \sin(\omega t + \theta) + \frac{d}{dt} A \sin(\omega t + \theta) \\
&= \PD{\theta}{\Lq} - \frac{d}{dt}\left(\PD{\dottheta}{\Lq}\right)
- A \dottheta \cos(\omega t + \theta) + A \dottheta \cos(\omega t + \theta) \\
&= \PD{\theta}{\Lq} - \frac{d}{dt}\left(\PD{\dottheta}{\Lq}\right)  .
\end{aligned}
\end{equation}
%
Now consider the general case, altering a Lagrangian by adding the time derivative of a positional dependent function:
%
\begin{equation}\label{eqn:tongMf1:1600}
\Lq' = \Lq + \frac{df}{dt}.
\end{equation}
%
and compute the equations of motion from this more generally altered function:
\begin{equation}\label{eqn:tongMf1:540}
\begin{aligned}
0
&= \PD{q^i}{\Lq'} - \frac{d}{dt}\left(\PD{q^i}{\Lq'}\right) \\
&= \PD{q^i}{\Lq} - \frac{d}{dt}\left(\PD{q^i}{\Lq}\right)
+ \PD{q^i}{}\frac{df}{dt} - \frac{d}{dt} \PD{q^i}{} \frac{df}{dt} .
\end{aligned}
\end{equation}
%
Now, if \(f(q^j,\qdot^j,t) = f(q^j,t)\) we have:
\begin{equation}\label{eqn:tongMf1:1620}
\frac{df}{dt} = \sum \PD{q^j}{f} \qdot^j + \PD{t}{f}.
\end{equation}
%
We want to see if the following sums to zero:
\begin{equation}\label{eqn:tongMf1:560}
\begin{aligned}
\PD{q^i}{}\frac{df}{dt} - \frac{d}{dt} \PD{q^i}{} \frac{df}{dt}
&= \sum \PD{q^i}{} \PD{q^j}{f} \left(\qdot^j + \PD{t}{f}\right) - \frac{d}{dt} \PD{\qdot^i}{} \left(\sum \PD{q^j}{f} \qdot^j + \PD{t}{f}\right) \\
&= \sum \PDD{q^i}{q^j}{f} \qdot^j + \PDD{q^i}{t}{f} - \frac{d}{dt} \left(\sum \delta_{ij} \PD{q^j}{f} + \PDD{\qdot^i}{t}{f}\right) \\
&= \sum \PDD{q^i}{q^j}{f} \qdot^j + \PDD{q^i}{t}{f} - \frac{d}{dt} \PD{q^i}{f} \\
&= \sum \PDD{q^i}{q^j}{f} \qdot^j + \PDD{q^i}{t}{f} - \sum \qdot^j \PDD{q^j}{q^i}{f} - \PDD{t}{q^i}{f} .
\end{aligned}
\end{equation}
%
Therefore provided the function is sufficiently continuous that all mixed pairs of mixed partials are equal, this is zero, and the \(df/dt\) addition does not
change the equations of motion that the Lagrangian generates.
%
\paragraph{Back to the problem.}
%
Now, return to the Lagrangian for this problem, and compute the equations of motion.  Writing \(\mu = R/l\), we have:
%
\begin{equation}\label{eqn:tongMf1:1640}
\Lq = \inv{2} \left( \mu^2 \dotalpha^2 + \dottheta^2 - 2 \mu \dotalpha \dottheta \sin( \alpha +\theta ) \right) + (g/l) \cos\theta.
\end{equation}
%
\begin{equation}\label{eqn:tongMf1:580}
\begin{aligned}
0 &= \frac{d}{dt}\PD{\dottheta} -\PD{\theta} \Lq \\
&= \frac{d}{dt} \left(\dottheta - \mu \dotalpha \sin(\alpha + \theta)\right) + \mu \dotalpha \dottheta \cos(\alpha + \theta) + (g/l) \sin\theta \\
&= \ddottheta - \mu \ddotalpha \sin(\alpha + \theta) - \mu \dotalpha \cos(\alpha + \theta) (\dotalpha + \dottheta) + \mu \dotalpha \dottheta \cos(\alpha + \theta) + (g/l) \sin\theta .
\end{aligned}
\end{equation}
%
Sure enough we have a cancellation of terms for constant \(\omega\).  In general we are left with:
%
\begin{equation}\label{eqn:tongMf1:1660}
\ddottheta = \mu \ddotalpha \sin(\alpha + \theta) + \mu \dotalpha^2 \cos(\alpha + \theta) - (g/l) \sin\theta.
\end{equation}
%
This expands to
\begin{equation}\label{eqn:tongMf1:1680}
\ddottheta = \mu (\ddotomega t + 2\dotomega) \sin(\omega t + \theta) + \mu (\dotomega t + \omega)^2 \cos(\omega t + \theta) - (g/l) \sin\theta.
\end{equation}
%
For constant \(\omega\), this is just:
\begin{equation}\label{eqn:tongMf1:1700}
\ddottheta = \mu \omega^2 \cos(\omega t + \theta) - (g/l) \sin\theta.
\end{equation}
}
%
