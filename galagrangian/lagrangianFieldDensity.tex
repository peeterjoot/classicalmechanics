%
% Copyright � 2012 Peeter Joot.  All Rights Reserved.
% Licenced as described in the file LICENSE under the root directory of this GIT repository.
%
%\chapter{Direct variation of Maxwell equations}
\index{Maxwell action!variation}
\label{chap:PJMaxwellLagrangian}
\label{chap:lagrangianFieldDensity}
%\date{Sept 8, 2008.  lagrangianFieldDensity.tex}
%
\section{Motivation, definitions and setup.}
%
This document will attempt to calculate Maxwell's equation, which in multivector form is
%
\begin{equation}\label{eqn:maxFieldLag:maxwell}
\grad F = J/\epsilon_0 c.
\end{equation}
%
using a Lagrangian energy density variational approach.
%
\subsection{Tensor form of the field.}
%
Explicit expansion of the field bivector in terms of coordinates one has
%
\begin{equation}\label{eqn:lagrangianFieldDensity:20}
\begin{aligned}
F
&= \BE + I c \BB \\
&= E^k \gamma_{k0} + \gamma_{0123k0} c B^k \\
&= E^k \gamma_{k0} + {(\gamma_{0})}^2 {(\gamma_{k})}^2 {\epsilon^{ij}}_k c \gamma_{ij} B^k .
\end{aligned}
\end{equation}
%
The complete coordinate expansion of the field is
\begin{equation}
F = E^k \gamma_{k0} - c {\epsilon^{ij}}_k B^k \gamma_{ij}.
\end{equation}
%
When this bivector is expressed in terms of basis bivectors \(\gamma_{\mu\nu}\) we have
%
\begin{equation}\label{eqn:lagrangianFieldDensity:40}
F
= \sum_{\mu<\nu} (F \cdot \gamma^{\nu\mu}) \gamma_{\mu\nu}
= \inv{2} (F \cdot \gamma^{\nu\mu}) \gamma_{\mu\nu}.
\end{equation}
%
As shorthand for the coordinates the field can be expressed with respect to various bivector basis sets in tensor form
%
\begin{equation}\label{eqn:lagrangianFieldDensity:1254}
\begin{array}{l l l}
F^{\mu\nu} &= F \cdot \gamma^{\nu\mu} & \quad F = (1/2) F^{\mu\nu} \gamma_{\mu\nu} \\
F_{\mu\nu} &= F \cdot \gamma_{\nu\mu} & \quad F = (1/2) F_{\mu\nu} \gamma^{\mu\nu} \\
{F_{\mu}}^\nu &= F \cdot {\gamma_{\nu}}^{\mu} & \quad F = (1/2) {F_{\mu}}^{\nu} {\gamma^{\mu}}_{\nu} \\
{F^{\mu}}_\nu &= F \cdot {\gamma^{\nu}}_{\mu} & \quad F = (1/2) {F^{\mu}}_{\nu} {\gamma_{\mu}}^{\nu}.
\end{array}
\end{equation}
%
In particular, we can extract the electric field components by dotting with a spacetime mix of indices
%
\begin{equation}\label{eqn:lagrangianFieldDensity:1274}
F^{i0} = E^k \gamma_{k0} \cdot \gamma^{0i} = E^i = -F_{i0}.
\end{equation}
%
and the magnetic field components by dotting with the bivectors having a pure spatial mix of indices
%
\begin{equation}\label{eqn:lagrangianFieldDensity:1294}
F^{ij} = - c {\epsilon^{a b}}_k B^k \gamma_{a b} \cdot \gamma^{ji} = - c {\epsilon^{i j}}_k B^k = F_{ij}.
\end{equation}
%
It is customary to summarize these tensors in matrix form
\begin{subequations}
\label{eqn:lagrangianFieldDensity:1010}
\begin{equation}\label{eqn:maxFieldLag:matrixtensor}
F^{\mu\nu} =
\begin{bmatrix}
0   & -E^1 & -E^2 & -E^3 \\
E^1 &   0  & -c B^3 &  c B^2 \\
E^2 &  c B^3 &   0  & -c B^1 \\
E^3 & -c B^2 &  c B^1 &   0  \\
\end{bmatrix},
\end{equation}
%
\begin{equation}
F_{\mu\nu} =
\begin{bmatrix}
0   & E^1 & E^2 & E^3 \\
-E^1 &   0  & -c B^3 &  c B^2 \\
-E^2 &  c B^3 &   0  & -c B^1 \\
-E^3 & -c B^2 &  c B^1 &   0  \\
\end{bmatrix}.
\end{equation}
\end{subequations}
%
Neither of these matrices will be needed explicitly, but are included for comparison since there is some variation in the sign conventions and units used for the field tensor.  Observe that these matrix representations are both sparse and filled with redudancy, and are not a particularily great representation of the field.
%
\subsubsection{Potential form.}
%
With the assumption that the field can be expressed in terms of the curl of a potential vector
%
\begin{equation}\label{eqn:maxFieldLag:potentialdef}
F = \grad \wedge A,
\end{equation}
the tensor expression of the field becomes
\begin{equation}\label{eqn:maxFieldLag:tensorpot}
\begin{aligned}
F^{\mu\nu} &= F \cdot (\gamma^{\nu} \wedge \gamma^{\mu}) = \partial^{\mu} A^{\nu} - \partial^{\nu} A^{\mu} \\
F_{\mu\nu} &= F \cdot (\gamma_{\nu} \wedge \gamma_{\mu}) = \partial_{\mu} A_{\nu} - \partial_{\nu} A_{\mu} \\
{F^{\mu}}_{\nu} &= F \cdot (\gamma^{\nu} \wedge \gamma_{\mu}) = \partial^{\mu} A_{\nu} - \partial_{\nu} A^{\mu} \\
{F_{\mu}}^{\nu} &= F \cdot (\gamma_{\nu} \wedge \gamma^{\mu}) = \partial_{\mu} A^{\nu} - \partial^{\nu} A_{\mu}.
\end{aligned}
\end{equation}
%
These field bivector coordinates will be used in the Lagrangian calculations.
%
