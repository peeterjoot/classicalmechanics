%
% Copyright © 2012 Peeter Joot.  All Rights Reserved.
% Licenced as described in the file LICENSE under the root directory of this GIT repository.
%
\makeoproblem{Two circular constrained paths.}{tongmf1:pr7}{\citep{TongMf1} p7}{
Masses connected by a spring.
}
%
\makeanswer{tongmf1:pr7}{
%
With \(i = \Be_1 \wedge \Be_2\), the paths, (squared) speeds and separation of the masses can be written:
%
\begin{equation}\label{eqn:tongMf1:820}
\begin{aligned}
q_1 &= \Be_1 R_1 e^{i\theta} \\
q_2 &= c\Be_3 + \Be_1
\lr{  ai + R_2 e^{i\alpha} }
\\
\Abs{\qdot_1}^2 &=
\lr{ R_1 \dottheta }^2 \\
\Abs{\qdot_2}^2 &=
\lr{ R_2 \dotalpha }^2 \\
d^2 &=
\lr{ q_1 - q_2 }^2 \\
&= c^2 + \Abs{ai + R_2 e^{i\alpha} - R_1 e^{i\theta}}^2 \\
&= c^2 + a^2 + {R_2}^2 + {R_1}^2 + a i \left( R_2 e^{-i\alpha} - R_1 e^{-i\theta} -R_2 e^{i\alpha} + R_1 e^{i\theta} \right) \\
& \quad - R_1 R_2
\lr{  e^{i\alpha} e^{-i\theta} + e^{-i\alpha} e^{i\theta}  } \\
&= c^2 + a^2 + {R_2}^2 + {R_1}^2
% a i \left( R_2 e^{-i\alpha} - R_1 e^{-i\theta} -R_2 e^{i\alpha} + R_1 e^{i\theta} \right)
% a/i \left( -R_2 e^{-i\alpha} + R_1 e^{-i\theta} +R_2 e^{i\alpha} - R_1 e^{i\theta} \right)
+ 2 a
(R_2 \sin\alpha - R_1 \sin\theta)
- 2 R_1 R_2 \cos(\alpha - \theta).
\end{aligned}
\end{equation}
%
With the given potential:
%
\begin{equation}\label{eqn:tongMf1:2160}
V = \inv{2} \omega^2 d^2.
\end{equation}
%
We have the following Lagrangian (where the constant terms in the separation have been dropped) :
%
\begin{dmath}\label{eqn:tongMf1:2180}
\LL = \inv{2} m_1 (R_1 \dottheta)^2 +\inv{2} m_2 (R_2 \dotalpha)^2
+ \omega^2 \left( a (R_2 \sin\alpha - R_1 \sin\theta) - R_1 R_2 \cos(\alpha - \theta) \right).
\end{dmath}
%
Last part of the problem was to show that there is an additional conserved quantity when \(a=0\).  The Lagrangian in that case is:
%
\begin{equation}\label{eqn:tongMf1:2200}
\LL = \inv{2} m_1
\lr{ R_1 \dottheta }^2 +\inv{2} m_2
\lr{ R_2 \dotalpha }^2
- R_1 R_2 \omega^2 \cos(\alpha - \theta).
\end{equation}
%
Evaluating the Lagrange equations, for this condition one has:
%
\begin{equation}\label{eqn:tongMf1:840}
\begin{aligned}
-R_1 R_2 \omega^2 \sin(\alpha - \theta) &=
\lr{ m_1 {R_1}^2 \dottheta }' \\
R_1 R_2 \omega^2 \sin(\alpha - \theta) &=
\lr{ m_2 {R_2}^2 \dotalpha }'.
\end{aligned}
\end{equation}
%
Summing these one has:
\begin{equation}\label{eqn:tongMf1:2220}
\lr{ m_1 {R_1}^2 \dottheta }'
 + \lr{ m_2 {R_2}^2 \dotalpha }' = 0.
\end{equation}
%
Therefore the additional conserved quantity is:
\begin{equation}\label{eqn:tongMf1:2240}
m_1 {R_1}^2 \dottheta + m_2 {R_2}^2 \dotalpha = K.
\end{equation}
%
FIXME: Is there a way to identify such a conserved quantity without evaluating the derivatives?  Noether's?
%
\paragraph{Spring Potential?}
%
Small digression.  Let us take the gradient of this spring potential and see if this matches our expectations for a \(-kx\) spring force.
%
\begin{equation}\label{eqn:tongMf1:2260}
-{\grad}_d V = -\omega^2 d \dcap = -\omega^2 \Bd.
\end{equation}
%
Okay, this works, \(\omega^2 = k\), which just expresses the positiveness of this constant.
}
%
