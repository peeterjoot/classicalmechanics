%
% Copyright © 2012 Peeter Joot.  All Rights Reserved.
% Licenced as described in the file LICENSE under the root directory of this GIT repository.
%
\makeoproblem{Double pendulum.}{tongmf1:pr4}{\citep{TongMf1} p4.}{
Derive the Lagrangian for a double pendulum.
}
%
\makeanswer{tongmf1:pr4}{
%
First consider a single pendulum (fixed length \(l\)).
%
\begin{equation}\label{eqn:tongMf1:200}
\begin{aligned}
x &= l \exp(i\theta) \\
\xdot &= l i \dottheta \exp(i\theta) \\
\abs{\xdot}^2 &= l^2 \dottheta^2 .
\end{aligned}
\end{equation}
%
Now, if \(\theta = 0\) represents the downwards position at rest, the height above that rest point is \(h = l - l \cos\theta\).  Therefore the Lagrangian is:
%
\begin{equation}\label{eqn:tongMf1:220}
\begin{aligned}
\Lq
&= \inv{2} m v^2 - mgh \\
&= \inv{2} m l^2 \dottheta^2 - mg l (1 - \cos\theta) .
\end{aligned}
\end{equation}
%
The constant term can be dropped resulting in the equivalent Lagrangian:
%
\begin{equation}\label{eqn:tongMf1:1100}
\Lq' = \inv{2} m l^2 \dottheta^2 + mg l \cos\theta.
\end{equation}
%
This amounts to a difference in the reference point for the potential energy, so
instead of measuring the potential energy \(V = m g h\) from a reference position below the mass, one could consider that the potential has a maximum of zero at the highest position, and decreases from there as:
%
\begin{equation}\label{eqn:tongMf1:1120}
V' = 0 - m g l \cos\theta.
\end{equation}
%
Moving back to the EOMs that result from either form of Lagrangian, we have after taking our derivatives:
%
\begin{equation}\label{eqn:tongMf1:1140}
-m g l \sin\theta = \frac{d}{dt} m l^2 \dottheta = m l^2 \ddottheta.
\end{equation}
%
Dividing out the \( m l^2 \) we are left with
%
\begin{equation}\label{eqn:tongMf1:1160}
\ddottheta = - g/l \sin\theta.
\end{equation}
%
This is consistent with our expectations, and recovers the familiar small angle SHM equation:
%
\begin{equation}\label{eqn:tongMf1:1180}
\ddottheta \approx - g/l \theta.
\end{equation}
%
Now, move on to the double pendulum, and compute the Kinetic energies of the two particles:
%
\begin{equation}\label{eqn:tongMf1:240}
\begin{aligned}
x_1 &= l_1 \exp(i\theta_1) \\
\xdot_1 &= l_1 i \dottheta_1 \exp(i\theta_1) \\
\abs{\xdot_1}^2 &= {l_1}^2 {\dottheta_1}^2,
\end{aligned}
\end{equation}
%
\begin{equation}\label{eqn:tongMf1:260}
\begin{aligned}
x_2 &= x_1 + l_2 \exp(i\theta_2) \\
\xdot_2 &= \xdot_1 + l_2 i \dottheta_2 \exp(i\theta_2) \\
        &= l_1 i \dottheta_1 \exp(i\theta_1) + l_2 i \dottheta_2 \exp(i\theta_2) \\
\abs{\xdot_2}^2
&=
(l_1 \dottheta_1)^2 + (l_2 \dottheta_2)^2
+ l_1 i \dottheta_1 \exp(i\theta_1) l_2 (-i) \dottheta_2 \exp(-i\theta_2) \\
&\quad + l_1 (-i) \dottheta_1 \exp(-i\theta_1) l_2 i \dottheta_2 \exp(i\theta_2) \\
&=
(l_1 \dottheta_1)^2 + (l_2 \dottheta_2)^2
+ l_1 l_2 \dottheta_1 \dottheta_2 \left( \exp(i(\theta_1 -\theta_2)) + \exp(-i(\theta_1 - \theta_2)) \right) \\
&=
(l_1 \dottheta_1)^2 + (l_2 \dottheta_2)^2
+ 2 l_1 l_2 \dottheta_1 \dottheta_2 \cos(\theta_1 -\theta_2).
\end{aligned}
\end{equation}
%
Now calculate the potential energies for the two masses.  The first has potential of
%
\begin{equation}\label{eqn:tongMf1:1200}
V_1 = m_1 g l_1 (1 - \cos\theta_1).
\end{equation}
%
and the potential energy of the second mass, relative to the position of the first mass is:
\begin{equation}\label{eqn:tongMf1:1220}
V_2' = m_2 g l_2 (1 - \cos\theta_2).
\end{equation}
%
But that is the potential only if the first mass is at rest.  The total difference in height from the dual rest position is:
\begin{equation}\label{eqn:tongMf1:1240}
l_1 (1 - \cos\theta_1) + l_2 (1 - \cos\theta_2).
\end{equation}
%
So, the potential energy for the second mass is:
\begin{equation}\label{eqn:tongMf1:1260}
V_2 = m_2 g \left( l_1 (1 - \cos\theta_1) + l_2 (1 - \cos\theta_2) \right).
\end{equation}
%
Dropping constant terms the total Lagrangian for the system is:
%
\begin{equation}\label{eqn:tongMf1:280}
\begin{aligned}
\Lq
&= \inv{2}m_1 {v_1}^2 +\inv{2}m_2 {v_2}^2 + m_1 g l_1 \cos\theta_1 + m_2 g \left( l_1 \cos\theta_1 + l_2 \cos\theta_2 \right) \\
&=
\inv{2} m_1 {l_1}^2 {\dottheta_1}^2
+ \inv{2} m_2 \left( (l_1 \dottheta_1)^2 + (l_2 \dottheta_2)^2 + 2 l_1 l_2 \dottheta_1 \dottheta_2 \cos(\theta_1 -\theta_2) \right) \\
&+ m_1 g l_1 \cos\theta_1 + m_2 g \left( l_1 \cos\theta_1 + l_2 \cos\theta_2 \right) .
\end{aligned}
\end{equation}
%
Again looking at the resulting Lagrangian, we see that it would have been more natural to measure the potential energy from a reference point of zero potential at the horizontal position, and measure downwards from there:
%
\begin{equation}\label{eqn:tongMf1:300}
\begin{aligned}
V_1' &= 0 - m_1 g l_1 \cos\theta_1 \\
V_2' &= 0 - m_2 g \left( l_1 \cos\theta_1 + l_2 \cos\theta_2 \right).
\end{aligned}
\end{equation}
%
\paragraph{N coupled pendulums.}
%
Now, with just two masses it is not too messy to expand out those kinetic energy terms, but for more the trig gets too messy.  With the \(K_2\) term of the Lagrangian in complex form we have:
%
\begin{equation}\label{eqn:tongMf1:320}
\begin{aligned}
\Lq
&=
\inv{2} m_1 {l_1}^2 {\dottheta_1}^2
+ \inv{2} m_2 \Abs{ l_1 \dottheta_1 + l_2 \dottheta_2 \exp(i(\theta_2 -\theta_1)) }^2 \\
&+ m_1 g l_1 \cos\theta_1 + m_2 g \left( l_1 \cos\theta_1 + l_2 \cos\theta_2 \right) \\
&=
\inv{2} m_1 \Abs{l_1 \dottheta_1 \exp(i\theta_1)}^2
+ \inv{2} m_2 \Abs{ l_1 \dottheta_1 \exp(i\theta_1) + l_2 \dottheta_2 \exp(i\theta_2) }^2 \\
&+ m_1 g l_1 \cos\theta_1 + m_2 g \left( l_1 \cos\theta_1 + l_2 \cos\theta_2 \right) .
\end{aligned}
\end{equation}
%
By inspection we can also write the Lagrangian for the N particle variant:
%
\begin{equation}\label{eqn:tongMf1:1280}
\Lq
=
\inv{2} \sum_{j=1}^N m_j \Abs{ \sum_{k=1}^j l_k \dottheta_k \exp(i\theta_k) }^2
+ g \sum_{j=1}^N l_j \cos\theta_j \sum_{k=j}^N m_k.
\end{equation}
%
Can this be used to derive the wave equation?
%
If each of the \(N\) masses is a fraction \(m_j = \Delta m = M/N\) of the total mass, and the lengths are all uniformly divided into segments of length
\(l_j = \Delta l = L/N\), then the Lagrangian becomes:
%
\begin{equation}\label{eqn:tongMf1:340}
\begin{aligned}
\Lq
&= \frac{\Delta l}{2g} \sum_{j=1}^N \Abs{ \sum_{k=1}^j \dottheta_k \exp(i\theta_k) }^2 + \sum_{j=1}^N \cos\theta_j \sum_{k=j}^N 1 \\
&= \frac{\Delta l}{2g} \sum_{j=1}^N \Abs{ \sum_{k=1}^j \dottheta_k \exp(i\theta_k) }^2 + ( N - j + 1 ) \sum_{j=1}^N \cos\theta_j .
\end{aligned}
\end{equation}
%
FIXME: return to this later?
\paragraph{Double pendulum.}
%
First consider a single pendulum (fixed length \(l\)).
%
\begin{equation}\label{eqn:tongMf1:360}
\begin{aligned}
x &= l \exp(i\theta) \\
\xdot &= l i \dottheta \exp(i\theta) \\
\abs{\xdot}^2 &= l^2 \dottheta^2 .
\end{aligned}
\end{equation}
%
Now, if \(\theta = 0\) represents the downwards position at rest, the height above that rest point is \(h = l - l \cos\theta\).  Therefore the Lagrangian is:
%
\begin{equation}\label{eqn:tongMf1:380}
\begin{aligned}
\Lq
&= \inv{2} m v^2 - mgh \\
&= \inv{2} m l^2 \dottheta^2 - mg l (1 - \cos\theta) .
\end{aligned}
\end{equation}
%
The constant term can be dropped resulting in the equivalent Lagrangian:
%
\begin{equation}\label{eqn:tongMf1:1300}
\Lq' = \inv{2} m l^2 \dottheta^2 + mg l \cos\theta.
\end{equation}
%
This amounts to a difference in the reference point for the potential energy, so
instead of measuring the potential energy \(V = m g h\) from a reference position below the mass, one could consider that the potential has a maximum of zero at the highest position, and decreases from there as:
%
\begin{equation}\label{eqn:tongMf1:1320}
V' = 0 - m g l \cos\theta.
\end{equation}
%
Moving back to the EOMs that result from either form of Lagrangian, we have after taking our derivatives:
%
\begin{equation}\label{eqn:tongMf1:1340}
-m g l \sin\theta = \frac{d}{dt} m l^2 \dottheta = m l^2 \ddottheta.
\end{equation}
%
Dividing out the \( m l^2 \) we are left with
%
\begin{equation}\label{eqn:tongMf1:1360}
\ddottheta = - g/l \sin\theta.
\end{equation}
%
This is consistent with our expectations, and recovers the familiar small angle SHM equation:
%
\begin{equation}\label{eqn:tongMf1:1380}
\ddottheta \approx - g/l \theta.
\end{equation}
%
Now, move on to the double pendulum, and compute the Kinetic energies of the two particles:
%
\begin{equation}\label{eqn:tongMf1:400}
\begin{aligned}
x_1 &= l_1 \exp(i\theta_1) \\
\xdot_1 &= l_1 i \dottheta_1 \exp(i\theta_1) \\
\abs{\xdot_1}^2 &= {l_1}^2 {\dottheta_1}^2 .
\end{aligned}
\end{equation}
%
\begin{equation}\label{eqn:tongMf1:420}
\begin{aligned}
x_2 &= x_1 + l_2 \exp(i\theta_2) \\
\xdot_2 &= \xdot_1 + l_2 i \dottheta_2 \exp(i\theta_2) \\
        &= l_1 i \dottheta_1 \exp(i\theta_1) + l_2 i \dottheta_2 \exp(i\theta_2) \\
\abs{\xdot_2}^2
&=
(l_1 \dottheta_1)^2 + (l_2 \dottheta_2)^2
+ l_1 i \dottheta_1 \exp(i\theta_1) l_2 (-i) \dottheta_2 \exp(-i\theta_2) \\
&\quad
+ l_1 (-i) \dottheta_1 \exp(-i\theta_1) l_2 i \dottheta_2 \exp(i\theta_2) \\
&=
(l_1 \dottheta_1)^2 + (l_2 \dottheta_2)^2
+ l_1 l_2 \dottheta_1 \dottheta_2 \left( \exp(i(\theta_1 -\theta_2)) + \exp(-i(\theta_1 - \theta_2)) \right) \\
&=
(l_1 \dottheta_1)^2 + (l_2 \dottheta_2)^2
+ 2 l_1 l_2 \dottheta_1 \dottheta_2 \cos(\theta_1 -\theta_2) .
\end{aligned}
\end{equation}
%
Now calculate the potential energies for the two masses.  The first has potential of
%
\begin{equation}\label{eqn:tongMf1:1400}
V_1 = m_1 g l_1 (1 - \cos\theta_1).
\end{equation}
%
and the potential energy of the second mass, relative to the position of the first mass is:
\begin{equation}\label{eqn:tongMf1:1420}
V_2' = m_2 g l_2 (1 - \cos\theta_2).
\end{equation}
%
But that is the potential only if the first mass is at rest.  The total difference in height from the dual rest position is:
\begin{equation}\label{eqn:tongMf1:1440}
l_1 (1 - \cos\theta_1) + l_2 (1 - \cos\theta_2).
\end{equation}
%
So, the potential energy for the second mass is:
\begin{equation}\label{eqn:tongMf1:1460}
V_2 = m_2 g \left( l_1 (1 - \cos\theta_1) + l_2 (1 - \cos\theta_2) \right).
\end{equation}
%
Dropping constant terms the total Lagrangian for the system is:
%
\begin{equation}\label{eqn:tongMf1:440}
\begin{aligned}
\Lq
&= \inv{2}m_1 {v_1}^2 +\inv{2}m_2 {v_2}^2 + m_1 g l_1 \cos\theta_1 + m_2 g \left( l_1 \cos\theta_1 + l_2 \cos\theta_2 \right) \\
&=
\inv{2} m_1 {l_1}^2 {\dottheta_1}^2
+ \inv{2} m_2 \left( (l_1 \dottheta_1)^2 + (l_2 \dottheta_2)^2 + 2 l_1 l_2 \dottheta_1 \dottheta_2 \cos(\theta_1 -\theta_2) \right) \\
&+ m_1 g l_1 \cos\theta_1 + m_2 g \left( l_1 \cos\theta_1 + l_2 \cos\theta_2 \right) .
\end{aligned}
\end{equation}
%
Again looking at the resulting Lagrangian, we see that it would have been more natural to measure the potential energy from a reference point of zero potential at the horizontal position, and measure downwards from there:
%
\begin{equation}\label{eqn:tongMf1:460}
\begin{aligned}
V_1' &= 0 - m_1 g l_1 \cos\theta_1 \\
V_2' &= 0 - m_2 g \left( l_1 \cos\theta_1 + l_2 \cos\theta_2 \right).
\end{aligned}
\end{equation}
}
%
