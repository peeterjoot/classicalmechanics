%
% Copyright � 2012 Peeter Joot.  All Rights Reserved.
% Licenced as described in the file LICENSE under the root directory of this GIT repository.
%
%
%
%
%\chapter{Covariant Lagrangian, and electrodynamic potential}
\index{electrodynamic potential!covariant}
\label{chap:PJSrLagrangian}
\label{chap:srLagrangian}
%\date{August 21, 2008.  srLagrangian.tex}
%
\section{Motivation}
%
In \citep{gabookII:PJSrGAFPLorentzForce}, it was observed that insertion of \(F = \grad \wedge A\) into
the covariant form of the Lorentz force:
%
\begin{equation}\label{eqn:srLag:lorentz}
\pdot = q (F \cdot v/c)
\end{equation}
%
allowed this law to be expressed as a gradient equation:
%
\begin{equation}
\pdot = q \grad (A \cdot v/c).
\end{equation}
%
Now, this suggests the possibility of a covariant potential that could be
used in a Lagrangian to produce \eqnref{eqn:srLag:lorentz} directly.  An
initial incorrect guess at what this Lagrangian would be was done, and
here some better guesses are made as well as a bit of raw algebra to verify
that it works out.
%
\section{Guess at the Lagrange equations for relativistic correctness}
%
Now, the author does not at the moment know any variational calculus worth
speaking of, but can guess at what the Lagrangian equations that would
solve the relativistic minimization problem.  Specifically, use proper
time in place of any local time derivatives:
%
\begin{equation}\label{eqn:srLag:properLagrange}
\frac{\partial \calL}{\partial x^{\mu}} =
\frac{d}{d\tau} \frac{\partial \calL}{\partial \xdot^{\mu}}.
\end{equation}
%
Note that in this equation \(\xdot^{\mu} = \frac{d x^{\mu}}{d\tau}\).
%
\subsection{Try it with a non-velocity dependent potential}
%
Lets see if this works as expected, by applying it to the simplest general
kinetic and potential Lagrangian.
%
\begin{equation}
\calL = \inv{2} m v^2 + \phi
\end{equation}
%
Calculate the Lagrangian equations:
%
\begin{equation}\label{eqn:srLagrangian:20}
\begin{aligned}
\frac{\partial \calL}{\partial x^{\mu}} &= \frac{d}{d\tau} \frac{\partial \calL}{\partial \xdot^{\mu}} \\
+ \frac{\partial \phi}{\partial x^{\mu}}
&= \inv{2} m \frac{d}{d\tau} \frac{\partial}{\partial \xdot^{\mu}} \sum \gamma_{\alpha} \cdot \gamma_{\beta} \xdot^{\alpha} \xdot^{\beta} \\
&= \inv{2} m \sum \gamma_{\alpha} \cdot \gamma_{\beta} \frac{d}{d\tau} \left({\delta^{\alpha}}_{\mu} \xdot^{\beta} + \xdot^{\alpha} {\delta^{\beta}}_{\mu}\right) \\
&= \inv{2} m \sum \frac{d}{d\tau}
\left(\gamma_{\mu} \cdot \gamma_{\beta} \xdot^{\beta} + \gamma_{\alpha} \cdot \gamma_{\mu} \xdot^{\alpha} \right) \\
&= m \sum \frac{d}{d\tau} \gamma_{\mu} \cdot \gamma_{\alpha} \xdot^{\alpha} \\
&= m \sum \gamma_{\mu} \cdot \gamma_{\alpha} \ddot{x}^{\alpha} \\
\end{aligned}
\end{equation}
%
Now, as in the Newtonian case, where we could show the correct form of the gradient for non-orthonormal frames could be derived from the Lagrangian equations using appropriate reciprocal vector pairs, we do the same thing here, summing the product of this last result with the reciprocal frame vectors:
%
\begin{equation}\label{eqn:srLagrangian:40}
\begin{aligned}
\sum \gamma^{\mu} \left(\frac{\partial \phi}{\partial x^{\mu}}\right) &= \sum \gamma^{\mu} \left(m \gamma_{\mu} \cdot \gamma_{\alpha} \ddot{x}^{\alpha}\right) \\
\left(\sum \gamma^{\mu} \frac{\partial}{\partial x^{\mu}}\right) \phi
&= m \sum \gamma_{\alpha} \ddot{x}^{\alpha} \\
&= m \ddot{x}
\end{aligned}
\end{equation}
%
Now, this left hand operator quantity is exactly our spacetime gradient:
%
\begin{equation}
\grad = \sum \gamma^{\mu} \frac{\partial}{\partial x^{\mu}},
\end{equation}
%
and the right hand side is our proper momentum.  Therefore the result of following through with the assumed Lagrangian equations yield the expected result:
%
\begin{equation}
\pdot = \grad \phi.
\end{equation}
%
Additionally, this demonstrates that the spacetime gradient used in GAFP is appropriate for any spacetime basis, regardless of whether the chosen basis vectors are orthonormal.

There are two features that are of interest here, one is that this result is independent of dimension, and the other is that there is also no requirement for any particular metric signature, Minkowski, Euclidean, or other.  That has to come another source.
%
\subsection{Velocity dependent potential}
%
The simplest scalar potential that is dependent on velocity is a potential that is composed of the dot product of a vector with that velocity.  Lets calculate the Lagrangian
equation for such an abstract potential, \(\phi(x,v) = A \cdot v\) (where any required unit adjustment to make this physically meaningful can be thought of as temporarily incorporated into \(A\)).
%
\begin{equation}
\calL = \inv{2}m v^2 + A \cdot v
\end{equation}
%
\begin{equation}\label{eqn:srLagrangian:60}
\begin{aligned}
\frac{\partial \calL}{\partial x^{\mu}} &= \frac{d}{d\tau} \frac{\partial \calL}{\partial \xdot^{\mu}} \\
\frac{\partial A}{\partial x^{\mu}} \cdot v &= m \gamma_{\mu} \cdot \gamma_{\alpha} \ddot{x}^{\alpha} +\frac{d}{d\tau} \frac{\partial A \cdot v}{\partial \xdot^{\mu}} \\
\end{aligned}
\end{equation}
%
To simplify matters this last term can be treated separately.  First observe that the coordinate representation of the proper velocity \(v\) follows from the worldline
position vector as follows
%
\begin{equation}\label{eqn:srLagrangian:80}
\begin{aligned}
x &= x^\mu \gamma_\mu \\
v &= \frac{dx}{d\tau} = \xdot^\mu \gamma_\mu.
\end{aligned}
\end{equation}
%
This gives
%
\begin{equation}\label{eqn:srLagrangian:100}
\begin{aligned}
\PD{\xdot^\mu}{A \cdot v}
&= \PD{\xdot^\mu}{} A \cdot \xdot^\nu \gamma_\nu \\
&= \left( A \PD{\xdot^\mu}{\xdot^\nu} \right) \cdot \gamma_\nu \\
&= {\delta^\mu}_\nu A \cdot \gamma_\nu \\
&= A \cdot \gamma_\mu.
\end{aligned}
\end{equation}
%
The taking the derivative of this conjugate momentum term we have
%
\begin{equation}\label{eqn:srLagrangian:120}
\begin{aligned}
\frac{d}{d\tau} \PD{\xdot^\mu}{A \cdot v}
&= \frac{d}{d\tau} A \cdot \gamma_\mu \\
&= \frac{d}{d\tau} (A^\nu \gamma_\nu) \cdot \gamma_\mu \\
&= \frac{dA^\nu}{d\tau} \gamma_\nu \cdot \gamma_\mu \\
\end{aligned}
\end{equation}
%
Reassembling things this is
\begin{equation}\label{eqn:srLagrangian:140}
\begin{aligned}
m \gamma_{\mu} \cdot \gamma_{\alpha} \ddot{x}^{\alpha} &=
\frac{\partial A}{\partial x^{\mu}} \cdot v -\frac{d}{d\tau} \frac{\partial A \cdot v}{\partial \xdot^{\mu}} \\
&= \frac{\partial A}{\partial x^{\mu}} \cdot v - \gamma_{\mu} \cdot \gamma_{\alpha} \xdot^{\beta} \frac{\partial A^{\alpha}}{\partial x^{\beta}} \\
&= - v^{\beta} \gamma_{\mu} \cdot \gamma_{\alpha} \frac{\partial A^{\alpha}}{\partial x^{\beta}} + \frac{\partial A^{\alpha}}{\partial x^{\mu}} {v}^{\beta} \gamma_{\alpha} \cdot \gamma_{\beta} \\
&= v^{\beta} \gamma_{\alpha} \cdot \left( -\gamma_{\mu} \frac{\partial}{\partial x^{\beta}} + \frac{\partial}{\partial x^{\mu}} \gamma_{\beta} \right) A^{\alpha} \\
\implies \\
\pdot &= v^{\beta} \gamma^{\mu} \gamma_{\alpha} \cdot \left( -\gamma_{\mu} \frac{\partial}{\partial x^{\beta}} + \gamma_{\beta} \frac{\partial}{\partial x^{\mu}} \right) A^{\alpha} \\
\end{aligned}
\end{equation}
%
Now, this last result has an alternation that suggests the wedge product is somehow involved, but is something slightly different.  Working (guessing) backwards, lets
see if this matches the following:
%
\begin{equation}\label{eqn:srLagrangian:160}
\begin{aligned}
(\grad \wedge A) \cdot v
&= \sum ( \gamma^{\mu} \wedge \gamma_{\alpha} ) \cdot \gamma_{\beta} \partial_{\mu} A^{\alpha} v^{\beta} \\
&= \sum (\gamma^{\mu} \gamma_{\alpha} \cdot \gamma_{\beta} - \gamma_{\alpha} \gamma^{\mu} \cdot \gamma_{\beta}) \partial_{\mu} A^{\alpha} v^{\beta} \\
&= \sum (\gamma^{\mu} \gamma_{\alpha} \cdot \gamma_{\beta} - \gamma_{\alpha} {\delta^{\mu}}_{\beta}) \partial_{\mu} A^{\alpha} v^{\beta} \\
&= \sum
 \gamma^{\mu} \gamma_{\alpha} \cdot \gamma_{\beta} v^{\beta} \partial_{\mu} A^{\alpha}
-\gamma_{\alpha} v^{\mu} \partial_{\mu} A^{\alpha} \\
&= \sum
 \gamma^{\mu} \gamma_{\alpha} \cdot \gamma_{\beta} v^{\beta} \partial_{\mu} A^{\alpha}
-
%\gamma_{\alpha} = \gamma^{\beta} \gamma_{\beta} \cdot \gamma_{\alpha}
\gamma^{\beta} \gamma_{\beta} \cdot \gamma_{\alpha}
 v^{\mu} \partial_{\mu} A^{\alpha} \\
&= \sum
 \gamma^{\mu} \gamma_{\alpha} \cdot \gamma_{\beta} v^{\beta} \partial_{\mu} A^{\alpha}
-
\gamma^{\mu} \gamma_{\mu} \cdot \gamma_{\alpha} v^{\beta} \partial_{\beta} A^{\alpha} \\
&= \sum v^{\beta} \gamma^{\mu} \gamma_{\alpha} \cdot ( \gamma_{\beta} \partial_{\mu} - \gamma_{\mu} \partial_{\beta} ) A^{\alpha} \\
\end{aligned}
\end{equation}
%
From this we can conclude that the covariant Lagrangian for the Lorentz force law has the form:
%
\begin{equation}\label{eqn:srLag:lorentzLagrange}
\calL = \inv{2}m v^2 + q (A \cdot v/c)
\end{equation}
%
where application of the proper time variant of Lagrange's equation \eqnref{eqn:srLag:properLagrange} results in the equation:
%
\begin{equation}
\pdot = q (\grad \wedge A) \cdot v/c = q F \cdot v/c
\end{equation}
%
Adding in Maxwell's equation:
%
\begin{equation}\label{eqn:srLag:maxwell}
\grad F = \grad \wedge A = J,
\end{equation}
%
we have a complete statement of pre-quantum electrodynamics and relativistic dynamics all buried in three small equations \eqnref{eqn:srLag:maxwell}, \eqnref{eqn:srLag:properLagrange}, and \eqnref{eqn:srLag:lorentzLagrange}.

Wow!  Very cool.  Now, I have also seen that Maxwell's equations can be expressed in Lagrangian form (have seen a tensor something like \(F^{\mu\nu} F_{\mu\nu}\) used to express this).  Next step is probably to work out the details of how that would fit.

Also worth noting here is the fact that no gauge invariance condition was required.  Adding that in yields the ability to solve for \(A\) directly from the wave equation \(\grad^2 A = J\).
