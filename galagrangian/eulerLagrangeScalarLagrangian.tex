%
% Copyright © 2012 Peeter Joot.  All Rights Reserved.
% Licenced as described in the file LICENSE under the root directory of this GIT repository.
%
\section{Scalar form of Euler-Lagrange equations.}
\index{Euler-Lagrange equations}
%
\citep{lasenby1993mda} presents a multivector Lagrangian treatment.  To
preparation for understanding that I have gone
back and derived the scalar
case myself.  As in my recent field Lagrangian derivations Feynman's
\citep{feynman1963flp} simple action procedure will be used.
%
Write
%
\begin{equation}\label{eqn:eulerLagrange:22}
\begin{aligned}
\Lq &= \Lq(q^i, \qdot^i, \lambda) \\
q^i &= \barq^i + n^i \\
S &= \int_{\partial \lambda} \Lq d\lambda.
\end{aligned}
\end{equation}
%
Here \(\barq^i\) are the desired optimal solutions, and the functions \(n^i\)
are all zero at the end points of the integration range \(\partial \lambda\).

A multivariable function \(f(a^i) = f(a^1, a^2, \cdots, a^n)\) may be expanded, to first order, in Taylor series
%
\begin{equation}\label{eqn:eulerLagrange:42}
f(a^i + x^i) \approx f(a^i) + \sum_i (a^i + x^i) \left. \PD{x^i}{f} \right\vert_{x^i = a^i}.
\end{equation}
%
In this case the \(x^i\) take the values \(q^i\), and \(\qdot^i\), so the first
order Lagrangian approximation requires summation over differential contributions for both sets of terms
%
\begin{equation}\label{eqn:euler_lagrange:linearizedLagrangian}
\Lq(q^i, \qdot^i, \lambda)
\approx \Lq(\barq^i, \barqdot^i, \lambda)
+ \sum_i (\barq^i + n^i) \left. \PD{q^i}{\Lq} \right\vert_{q^i = \barq^i}
+ \sum_i (\barqdot^i + \ndot^i) \left. \PD{\qdot^i}{\Lq} \right\vert_{q^i = \barq^i}.
\end{equation}
%
%Here subscript \(0\) represents evaluation of the integrals at \(q_j = \barq_j\),
%or \(\qdot_j = \barqdot_j\).
%
Now form the action, and group the terms in fixed and variable sets
%
\begin{equation}\label{eqn:eulerLagrange:62}
\begin{aligned}
S &= \int \Lq d\lambda \\
&\approx
\int d\lambda
\left(
\Lq(\barq^i, \barqdot^i, \lambda)
+ \sum_i \barq^i \left. \PD{q^i}{\Lq} \right\vert_{q^i = \barq^i}
+ \sum_i \barqdot^i \left. \PD{\qdot^i}{\Lq} \right\vert_{q^i = \barq^i}
\right) \\
&+
\mathLabelBox{
\sum_i \int d\lambda
\left(
n^i \left. \PD{q^i}{\Lq} \right\vert_{q^i = \barq^i}
+\ndot^i \left. \PD{\qdot^i}{\Lq} \right\vert_{q^i = \barq^i}
\right)
}{\(\delta S\)}.
\end{aligned}
\end{equation}
%
For the optimal solution we want \(\delta S = 0\) for all possible paths \(n^i\).  Now do the integration by parts writing
\(u' = \ndot^i\), and \(v = \partial \Lq/{\partial \qdot^i}\)
%
\begin{equation}\label{eqn:eulerLagrange:82}
\int u' v = u v - \int u v'.
\end{equation}
%
The action variation is then
%
\begin{equation}\label{eqn:eulerLagrange:102}
\delta S =
+ \sum_i \left. \left( n^i \PD{\qdot^i}{\Lq} \right) \right\vert_{\partial \lambda}
+ \sum_i \int d\lambda n^i
\left(
\left. \PD{q^i}{\Lq} \right\vert_{q^i = \barq^i}
-\frac{d}{d\lambda} \left. \PD{\qdot^i}{\Lq} \right\vert_{q^i = \barq^i}
\right).
\end{equation}
%
The non-integral term is zero since by definition \(n^i = 0\) on the boundary of the desired integration region, so for the
total variation to equal zero for all possible paths \(n^i\) one must have
%
\begin{equation}\label{eqn:euler_lagrange:eulerlag}
\PD{q^i}{\Lq} -\frac{d}{d\lambda} \PD{\qdot^i}{\Lq} = 0.
\end{equation}
%
Evaluation of these derivatives at the optimal desired paths has been suppressed since these equations now define that path.
%
\subsection{Some comparison to the Goldstein approach.}
%
\citep{goldstein1951cm} calls the quantity \eqnref{eqn:euler_lagrange:eulerlag} the functional derivative
%
\begin{equation}\label{eqn:eulerLagrange:122}
\frac{\delta S}{\delta q^i} = \PD{q^i}{\Lq} -\frac{d}{d\lambda} \PD{\qdot^i}{\Lq}.
\end{equation}
%
(with higher order derivatives if the Lagrangian has dependencies on more than generalized position and velocity terms).  Goldstein's
approach is also harder to follow than Feynman's (Goldstein introduces a parameter \(\epsilon\), writing
%
\begin{equation}\label{eqn:euler_lagrange:epsilonvariation}
q^i = \barq^i + \epsilon n^i.
\end{equation}
%
He then takes derivatives under the integral sign for the end result.

While his approach is a bit harder to follow initially, that additional \(\epsilon\) parametrization of the variation path also fits nicely with this
linearization procedure.
After the integration by parts and subsequent differentiation under integral sign nicely does the job of
discarding all the ``fixed'' \(\barq^i\) contributions to the action leaving:
%
\begin{equation}\label{eqn:eulerLagrange:142}
\frac{dS}{d\epsilon} = \int d\lambda \sum_i n^i \left. \frac{\delta S}{\delta q^i} \right\vert_{q^i = \barq^i}.
\end{equation}
%
Introducing this idea does firm things up, eliminating some hand waving.  To obtain the extremal solution it does
make sense to set the derivative of the action equal to zero, and introducing an additional scalar variational control
in the paths from the optimal solution provides that something to take derivatives with respect to.

Goldstein also writes that this action derivative is then evaluated at \(\epsilon = 0\).  This really says the same
thing as Feynman... toss all the higher order terms, since factors of epsilon will be left associated with of these.
With my initial read of Goldstein this was not the least bit clear... it was really yet another example of the classic
physics approach of solving something with a first order linear approximation.
%
