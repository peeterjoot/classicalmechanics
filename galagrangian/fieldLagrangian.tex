%
% Copyright � 2012 Peeter Joot.  All Rights Reserved.
% Licenced as described in the file LICENSE under the root directory of this GIT repository.
%

%
%
%\chapter{Derivation of Euler-Lagrange field equations}
\index{Euler-Lagrange equations!field}
\label{chap:PJFieldLagrangian}
\label{chap:fieldLagrangian}
%\date{October 10, 2008.  fieldLagrangian.tex}

\section{Motivation}

In \chapcite{PJMaxwellLagrangian} Maxwell's equations were derived from
a Lagrangian action in tensor and STA forms.  This was done with
Feynman's \citep{feynman1963flp} simple, but somewhat non-rigorous, direct variational technique.

An alternate approach is to use a field form of the Euler-Lagrange
equations as done in the wikipedia article \citep{wiki:emtensor}.  I had
trouble understanding that derivation, probably because
I did not understand the notation, nor what the source of that equation.

Here Feynman's approach will be used to derive the field versions of the Euler-Lagrange
equations, which clarifies the notation.  As a verification of the correctness these
will be applied to derive Maxwell's equation.

\section{Deriving the field Lagrangian equations}

That essence of Feynman's method from his
``Principle of Least Action'' entertainment chapter of the Lectures is
to do a first order linear expansion of the function, ignore all the higher order terms,
then do the integration by parts for the remainder.

Looking at the Maxwell field Lagrangian and action for motivation,

\begin{equation}\label{eqn:fieldLagrangian:20}
\begin{aligned}
\LL &=
\lr{ \partial^\mu A^\nu - \partial^\nu A^\mu }
\lr{ \partial_\mu A_\nu - \partial_\nu A_\mu }
+ \kappa J^\sigma A_\sigma \\
S &= \int d^4 x \LL
\end{aligned}
\end{equation}

where the potential functions \(A^\mu\) (or their index lowered variants)
are to be determined by extreme values of the action variation.  Note the use of the shorthand
\(\partial_\mu \equiv \PD{x^\mu}{}\).

We want to consider general Lagrangians of this form.  Write

\begin{equation}\label{eqn:fieldLagrangian:40}
\begin{aligned}
\LL = \LL( A^\mu, \partial_\nu A^\sigma) = \LL(A^0, A^1, \cdots, \partial_0 A^0, \partial_1 A^0, \cdots )
\end{aligned}
\end{equation}

\subsection{First order Taylor expansion of a multi variable function}

Given an abstractly specified function like this, with indices and partials flying around, how to do a first order Taylor series expansion may not be obvious, especially since the variables are all undetermined functions!

Consideration of a simple case guides the way.  Assume that a two variable function can be expressed as a polynomial of some order

\begin{equation}\label{eqn:fieldLagrangian:60}
\begin{aligned}
f(x,y) = a_{i j} x^i y^j
\end{aligned}
\end{equation}

Evaluation of this function or its partials at \((x,y) = (0,0)\) supply the constants \(a_{i j}\).  Simplest is the lowest order constant

\begin{equation}\label{eqn:fieldLagrangian:80}
\begin{aligned}
f(0,0) &= a_{0 0} \\
\end{aligned}
\end{equation}

\begin{equation}\label{eqn:fieldLagrangian:100}
\begin{aligned}
\partial_x f &= i a_{i j} x^{i-1} y^j \\
\partial_y f &= j a_{i j} x^{i} y^{j-1} \\
\partial_{xx} f &= i(i-1) a_{i j} x^{i-2} y^j \\
\partial_{yy} f &= j(j-1) a_{i j} x^{i} y^{j-2} \\
\partial_{xy} f &= i j a_{i j} x^{i-1} y^{j-1} \\
\hdots \\
\implies \\
a_{1 0} &= \lr{ \partial_x f } \vert_0 \\
a_{0 1} &= \lr{ \partial_y f } \vert_0 \\
a_{2 0} &= \inv{2!} \lr{ \partial_{xx} f } \vert_0 \\
a_{0 2} &= \inv{2!} \lr{ \partial_{yy} f } \vert_0 \\
a_{1 1} &= \lr{ \partial_{xy} f } \vert_0 =
\lr{ \partial_{yx} f } \vert_0 \\
\hdots
\end{aligned}
\end{equation}

Or

\begin{equation}\label{eqn:fieldLagrangian:120}
\begin{aligned}
f(x,y) &= f \vert_0 + x
\lr{ \partial_x f }
\vert_0 + y
\lr{ \partial_y f } \vert_0 \\
&+ \inv{2} \left( x^2
\lr{ \partial_{x x} f }
\vert_0 + x y
\lr{ \partial_{x y} f }
\vert_0  + y x
\lr{ \partial_{y x} f }
\vert_0  + y^2
\lr{ \partial_{y y} f }
 \vert_0 \right) \\
&+ \sum_{(i+j) > 2} a_{i j} x^i y^j
\end{aligned}
\end{equation}

\subsection{First order expansion of the Lagrangian function}

It is not hard to see that the same thing can be done for higher degree functions too, although enumerating the
higher order terms will get messier, however for the purposes of this variational exercise the assumption is that only
the first order differential terms are significant.

How to do the first order Taylor expansion of a multivariable function has been established.  Next write \(A^\mu = \barA^\mu + n^\mu\), where the \(\barA^\mu\) functions are the desired solutions and each of \(n^\mu\) vanishes on the boundaries of the integration region.  Expansion of \(\LL\) around the desired solutions one has

\begin{equation}\label{eqn:fieldLagrangian:140}
\begin{aligned}
\LL(\barA^\mu + n^\mu, \partial_\nu( \barA^\sigma + n^\sigma) )
&=
\LL(\barA^\mu, \partial_\nu \barA^\sigma ) \\
&+ \lr{ \barA^\mu + n^\mu }
\left( \left. \PD{A^\mu}{\LL} \right) \right\vert_{A^\mu = \barA^\mu} \\
&+ (\partial_\nu \barA^\sigma + \partial_\nu n^\sigma) \left( \left. \PD{
\lr{ \partial_\nu A^\sigma }
}{\LL} \right) \right\vert_{\partial_\nu A^\sigma = \partial_\nu \barA^\sigma} \\
&+ \sum_{i+j>2}
\lr{ \barA^\mu + n^\mu }^i
\lr{ \partial_\nu \barA^\sigma + \partial_\nu n^\sigma }^j
\mathLabelBox{\left(\cdots\right)}{higher order derivatives}
\end{aligned}
\end{equation}

\subsection{Example for clarification}
Here we see the first use of the peculiar looking partials from the wikipedia article

\begin{equation}\label{eqn:fieldLagrangian:160}
\begin{aligned}
\PD{
\lr{ \partial_\nu A^\sigma }}{\LL}.
\end{aligned}
\end{equation}

Initially looking at that I could not fathom what it meant, but it is just what it says,
differentiation with respect to a variable \(\partial_\nu A^\sigma\).  As an example, for

\begin{equation}\label{eqn:fieldLagrangian:180}
\begin{aligned}
\LL
&= u A^0 + v A^1 + a \partial_1 A^0 + b \partial_0 A^1 \\
&= u A^0 + v A^1 + a \PD{x^1}{A^0} + b \PD{x^0}{A^1}
\end{aligned}
\end{equation}

where \(u\),\(v\),\(a\), and \(b\) are constants.  Then an corresponding example of such a partial term is

\begin{equation}\label{eqn:fieldLagrangian:200}
\begin{aligned}
\PD{\lr{ \partial_1 A^0 }}{\LL} = a.
\end{aligned}
\end{equation}

\subsection{Calculation of the action for the general field Lagrangian}

\begin{equation}\label{eqn:fieldLagrangian:220}
\begin{aligned}
S &= \int d^4 x \LL \\
&= \int d^4 x \LL(\barA^\mu, \partial_\nu \barA^\sigma ) \\
&+ \int d^4 x
\lr{ \barA^\mu + n^\mu }
\left( \left. \PD{A^\mu}{\LL} \right) \right\vert_{A^\mu = \barA^\mu} \\
&+ \int d^4 x
\lr{ \partial_\nu \barA^\sigma + \partial_\nu n^\sigma }
 \left( \left. \PD{
\lr{ \partial_\nu A^\sigma }
}{\LL} \right) \right\vert_{\partial_\nu A^\sigma = \partial_\nu \barA^\sigma} \\
&+ \int d^4 x (\cdots \text{neglected higher order terms} \cdots )
\end{aligned}
\end{equation}

Grouping this into parts associated with the assumed variational solution, and the varied parts we have

\begin{equation}\label{eqn:fieldLagrangian:240}
\begin{aligned}
S &= \int d^4 x
\left(
\LL(\barA^\mu, \partial_\nu \barA^\sigma ) + \barA^\mu \left. \left( \PD{A^\mu}{\LL} \right) \right\vert_{A^\mu = \barA^\mu}
+ \partial_\nu \barA^\sigma \left. \left( \PD{
\lr{ \partial_\nu A^\sigma }
}{\LL} \right) \right\vert_{\partial_\nu A^\sigma = \partial_\nu \barA^\sigma}
\right) \\
&+ \int d^4 x
\left(
n^\mu \left. \left( \PD{A^\mu}{\LL} \right) \right\vert_{A^\mu = \barA^\mu}
+\partial_\nu n^\sigma \left. \left( \PD{
\lr{ \partial_\nu A^\sigma }
}{\LL} \right) \right\vert_{\partial_\nu A^\sigma = \partial_\nu \barA^\sigma}
\right) \\
&+ \cdots
\end{aligned}
\end{equation}

None of the terms in the first integral are of interest since they are fixed.  The second term of the remaining integral is the one to integrate by
parts.  For short, let

\begin{equation}\label{eqn:fieldLagrangian:260}
\begin{aligned}
u &= \left. \left( \PD{
\lr{ \partial_\nu A^\sigma }
}{\LL} \right) \right\vert_{\partial_\nu A^\mu = \partial_\nu \barA^\mu}
\end{aligned}
\end{equation}

then this integral is
%(u v)' = u' v + u v'
%\int u' v = \int (u v)' - \int u v'
%\int u' v = u v - \int u v'
\begin{equation}\label{eqn:fieldLagrangian:280}
\begin{aligned}
\int d^3 x d x^\nu \PD{x^\nu}{n^\sigma} u
&= \int d^3 x \left. \left( {n^\sigma} u \right) \right\vert_{\partial x^\nu} - \int d^4 x n^\sigma \PD{x^\nu}{} u
\end{aligned}
\end{equation}

Here \(\partial x^\nu\) denotes the boundary of the integration.  Because \(n^\sigma\) was by definition zero on all boundaries of the
integral region this first integral is zero.  Denoting the non-variational parts of the action integral by \(\delta S\), we have

\begin{equation}\label{eqn:fieldLagrangian:300}
\begin{aligned}
\delta S
&= \int d^4 x \left(
n^\mu \left. \left( \PD{A^\mu}{\LL} \right) \right\vert_{A^\mu = \barA^\mu}
- n^\sigma \PD{x^\nu}{} \left. \left( \PD{
\lr{ \partial_\nu A^\sigma }
}{\LL} \right) \right\vert_{\partial_\nu A^\sigma = \partial_\nu \barA^\sigma}
\right) \\
&= \int d^4 x
n^\sigma
\left(
\left. \left( \PD{A^\sigma}{\LL} \right) \right\vert_{A^\sigma = \barA^\sigma}
- \PD{x^\nu}{} \left. \left( \PD{
\lr{ \partial_\nu A^\sigma }
}{\LL} \right) \right\vert_{\partial_\nu A^\sigma = \partial_\nu \barA^\sigma}
\right)
\end{aligned}
\end{equation}

Now, for \(\delta S = 0\) for all possible variations \(n^\sigma\) from the optimal solution \(\barA^\sigma\), then the inner expression must also be zero
for all \(\sigma\).  Specifically

\begin{equation}\label{eqn:fieldLag:eulerLagrangeField}
\begin{aligned}
\PD{A^\sigma}{\LL} = \PD{x^\nu}{} \PD{
\lr{ \partial_\nu A^\sigma }
}{\LL}.
\end{aligned}
\end{equation}

Feynman's direct approach does not require too much to understand, and one can intuit through it fairly easily.  Contrast to
\citep{goldstein1951cm}
where the same result appears to be derived in Chapter 13.  That approach requires the use and familiarity with a functional derivative

\begin{equation}\label{eqn:fieldLagrangian:320}
\begin{aligned}
\frac{\delta \LL}{\delta A^\sigma} &= \PD{A^\sigma}{\LL} - \frac{d}{dx^\nu} \PD{
\lr{ \partial_\nu A^\sigma }
}{\LL}
\end{aligned}
\end{equation}

which must be defined and explained somewhere earlier in the book in one of the chapters that I skimmed over to get to the interesting ``Continuous Systems and Fields'' content
at the end of the book.

FIXME: I am also not clear why Goldstein would have a complete derivative \(d/dx^\nu\) here instead of \(\partial_\nu = \partial/{\partial x^\nu}\).  A more thoroughly worked simple example
of the integration by parts in two variables can be found in the plane solution of an electrostatics Lagrangian in \chapcite{PJMaxwellLagrangian}.  Based on the arguments there I think that it has to be a partial derivative.   The partial also happens to be consistent with both the wikipedia article \citep{wiki:emtensor}, and the Maxwell's derivation below.

\section{Verifying the equations}
\subsection{Maxwell's equation derivation from action}

For Maxwell's equation, our Lagrangian density takes the following complex valued form

\begin{equation}\label{eqn:fieldLag:maxlag}
\begin{aligned}
\LL &= -\frac{\epsilon_0 c}{2}
\lr{ \grad \wedge A }^2
+ J \cdot A
\end{aligned}
\end{equation}

In coordinates, writing \(\gamma^{\alpha\beta} = \gamma^\alpha \wedge \gamma^\beta\), and
for convenience \(\kappa = -\epsilon_0 c /2\) this is

\begin{equation}\label{eqn:fieldLag:maxlagcomp}
\begin{aligned}
\LL &= \kappa
\lr{ \gamma^{\mu\nu} }
 \lr{ \gamma^{\alpha\beta} }
 \partial_\mu A_\nu \partial_\alpha A_\beta + J^\sigma A_\sigma
\end{aligned}
\end{equation}

Some intermediate calculations to start
\begin{equation}\label{eqn:fieldLagrangian:340}
\begin{aligned}
\PD{A_\sigma}{\LL} = J^\sigma
\end{aligned}
\end{equation}

For the differentiation with respect to partials it is helpful to introduce a complete switch of indices in \eqnref{eqn:fieldLag:maxlag}
to avoid confusing things with the \(\nu\), and \(\sigma\) indices in our partial.

\begin{equation}\label{eqn:fieldLagrangian:360}
\begin{aligned}
\PD{
\lr{ \partial_\nu A_\sigma }
}{\LL}
&= \kappa \PD{
\lr{ \partial_\nu A_\sigma }
}{\LL} \left(
\lr{ \gamma^{M N} }
 \lr{ \gamma^{B C} }
 \partial_M A_N \partial_B A_C \right) \\
\end{aligned}
\end{equation}

This makes it clearer that the differentiation really just requires evaluation of the product chain rule \((fg)' = f'g + f g'\)

\begin{equation}\label{eqn:fieldLagrangian:380}
\begin{aligned}
\PD{
\lr{ \partial_\nu A_\sigma }
}{\LL}
&=
\kappa
\left(
\lr{ \gamma^{\nu\sigma} }
 \lr{ \gamma^{\alpha\beta} }
 \partial_\alpha A_\beta
+
\lr{ \gamma^{\alpha\beta} }
 \lr{ \gamma^{\nu\sigma} }
 \partial_\alpha A_\beta
\right) \\
&= \kappa \partial_\alpha A_\beta \left(
\lr{ \gamma^{\nu\sigma} }
 \lr{ \gamma^{\alpha\beta} }
 +
\lr{ \gamma^{\alpha\beta} }
 \lr{ \gamma^{\nu\sigma} }
 \right).
\end{aligned}
\end{equation}


Reassembling results the complete field equations are described by the set of relations

\begin{equation}\label{eqn:fieldLagrangian:400}
\begin{aligned}
J^\sigma
&= -\inv{2} \epsilon_0 c \partial_\nu \partial_\alpha A_\beta
%\gpgrade{ (\gamma^{\nu\sigma}) (\gamma^{\alpha\beta}) }{0+4}
\left(
\lr{ \gamma^{\nu\sigma} }
 \lr{ \gamma^{\alpha\beta} } +
\lr{ \gamma^{\alpha\beta} }
 \lr{ \gamma^{\nu\sigma} }
 \right).
\end{aligned}
\end{equation}

Multiplying this by \(\gamma_\sigma\) on both sides and summing produces the current density \(J = \gamma_\sigma J^\sigma\) on the LHS

\begin{equation}\label{eqn:fieldLagrangian:420}
\begin{aligned}
\frac{J}{\epsilon_0 c}
&= -\inv{2}
\partial_\nu \partial_\alpha A_\beta
\gamma_\sigma
\left(
\lr{ \gamma^{\nu\sigma} }
 \lr{ \gamma^{\alpha\beta} } +
\lr{ \gamma^{\alpha\beta} }
 \lr{ \gamma^{\nu\sigma} }
 \right) \\
&= \inv{2}
\partial_\nu \partial_\alpha A_\beta
\gamma_\sigma
\left(
\lr{ \gamma^{\sigma\nu} }
 \lr{ \gamma^{\alpha\beta} } +
\lr{ \gamma^{\alpha\beta} }
\lr{ \gamma^{\sigma\nu} }
\right) \\
\implies \\
\end{aligned}
\end{equation}
\begin{equation}\label{eqn:fieldLag:gettingthere}
\begin{aligned}
\frac{J}{\epsilon_0 c}
&= \inv{2}
\partial_\nu \partial_\alpha A_\beta
\left(
\gamma^{\nu}
\lr{ \gamma^{\alpha\beta} }
 +
\gamma_\sigma
\lr{ \gamma^{\alpha\beta} }
\lr{ \gamma^{\sigma\nu} }
\right)
\end{aligned}
\end{equation}

It appears that there is a cancellation of \(\sigma\) terms possible in that last term above too.  Algebraically
for vectors \(a\), \(b\), and bivector \(B\) where \(a \cdot b = 0\), a reduction of the algebraic product is required

\begin{equation}\label{eqn:fieldLagrangian:440}
\begin{aligned}
\inv{a} B b a
\end{aligned}
\end{equation}

Attempting this reduction to cleanly cancel the \(a\) terms goes nowhere fast.  The trouble is that there
is a dependence between B and the vectors, and exploiting that dependence is required to cleanly obtain the desires
result.

Step back and observe that the original Lagrangian of \eqnref{eqn:fieldLag:maxlag},
had only have scalar and pseudoscalar grades from the bivector square, plus a pure scalar grade part

\begin{equation}\label{eqn:fieldLagrangian:460}
\begin{aligned}
\LL = \gpgrade{\LL}{0+4}
\end{aligned}
\end{equation}

This implies that there is an dependency in the indices of the
bivector pairs of the Lagrangian in coordinate form \eqnref{eqn:fieldLag:maxlagcomp}.
Since scalar differentiation will not change the grades, the pairs
of indices in the symmetric product above in \eqnref{eqn:fieldLag:gettingthere} are also not all free.
In particular, either \(\{\nu, \sigma\} \in \{\alpha, \beta\}\), or these indices are all distinct, since two but only two
of these indices equal would mean there is a bivector grade in the sum.

The end of a long story is that the bivector product

\begin{equation}\label{eqn:fieldLagrangian:480}
\begin{aligned}
\lr{ \gamma^{\alpha\beta} }
\lr{ \gamma^{\sigma\nu} }
&=
\lr{ \gamma^{\sigma\nu} }
\lr{ \gamma^{\alpha\beta} }
\end{aligned}
\end{equation}

can be commuted, which leaves

\begin{equation}\label{eqn:fieldLagrangian:500}
\begin{aligned}
\frac{J}{\epsilon_0 c}
&= \partial_\nu \partial_\alpha A_\beta \gamma^{\nu}
\lr{ \gamma^{\alpha\beta} }
\\
&= \gamma^\nu \partial_\nu \partial_\alpha A_\beta
\lr{ \gamma^{\alpha\beta} }
\\
&= \grad \partial_\alpha A_\beta
\lr{ \gamma^{\alpha\beta} }
\\
&= \inv{2} \grad \left(\partial_\alpha A_\beta - \partial_\beta A_\alpha \right)
\lr{ \gamma^{\alpha\beta} }
 \\
&= \inv{2} \grad F_{\alpha\beta}
\lr{ \gamma^{\alpha\beta} }
\\
\end{aligned}
\end{equation}

This is Maxwell's equation in its full glory

\begin{equation}\label{eqn:fieldLagrangian:520}
\begin{aligned}
\grad F = \frac{J}{\epsilon_0 c}.
\end{aligned}
\end{equation}

\citep{doran2003gap} contains additional treatment, albeit a dense one, of
this form of Maxwell's equation.

\subsection{Electrodynamic Potential Wave Equation}

\subsection{Schr\"{o}dinger's equation}

Problem \(11.3\) in \citep{goldstein1951cm} is to take the Lagrangian

\begin{equation}\label{eqn:fieldLagrangian:540}
\begin{aligned}
\LL
&= \frac{\Hbar^2}{2 m} \spacegrad \psi \cdot \spacegrad \psi^\conj + V \psi \psi^\conj + \frac{\Hbar}{2i}
\lr{  \psi^\conj \partial_t \psi - \psi \partial_t \psi^\conj  }
 \\
&= \frac{\Hbar^2}{2 m} \partial_k \psi \partial_k \psi^\conj + V \psi \psi^\conj +
\frac{\Hbar}{2i}
\lr{  \psi^\conj \partial_t \psi - \psi \partial_t \psi^\conj  }
 \\
\end{aligned}
\end{equation}

treating \(\psi\), and \(\psi^\conj\) as separate fields and show that Schr\"{o}dinger's equation and its conjugate follows.  (note: I have added a 1/2 fact in the
commutator term that was not in the Goldstein problem.  Believe that to have been a typo in the original (first edition)).

We have
\begin{equation}\label{eqn:fieldLagrangian:560}
\begin{aligned}
\PD{\psi^\conj}{\LL} &= V\psi + \frac{\Hbar}{2i} \partial_t \psi \\
\end{aligned}
\end{equation}

and canonical momenta
\begin{equation}\label{eqn:fieldLagrangian:580}
\begin{aligned}
\PD{
\lr{ \partial_m \psi^\conj }
}{\LL} &= \frac{\Hbar^2}{2 m} \partial_{m} \psi \\
\PD{
\lr{ \partial_t \psi^\conj }
}{\LL} &= -\frac{\Hbar}{2i} {\psi} \\
\end{aligned}
\end{equation}

\begin{equation}\label{eqn:fieldLagrangian:600}
\begin{aligned}
\PD{\psi^\conj}{\LL} &= \sum_m \partial_m \PD{
\lr{ \partial_m \psi^\conj }
}{\LL} + \partial_t \PD{
\lr{ \partial_t \psi^\conj }
}{\LL} \\
V\psi + \frac{\Hbar}{2i} \partial_t \psi &= \frac{\Hbar^2}{2 m} \sum_m \partial_{mm} \psi -\frac{\Hbar}{2i} \PD{t}{\psi} \\
\end{aligned}
\end{equation}

which is the desired result
\begin{equation}\label{eqn:fieldLagrangian:620}
\begin{aligned}
-\frac{\Hbar^2}{2 m} \spacegrad^2 \psi + V\psi &= {\Hbar i}{} \PD{t}{\psi} \\
\end{aligned}
\end{equation}

The conjugate result

%\begin{align*}
%\PD{\psi}{\LL} &= V\psi^\conj - \frac{\Hbar}{2i} \partial_t \psi^\conj \\
%\partial_m \PD{(\partial_m \psi)}{\LL} &=
%\frac{\Hbar^2}{2 m}
%\partial_{mm} \psi^\conj \\
%\partial_t \PD{(\partial_t \psi)}{\LL} &= \frac{\Hbar}{2i} \partial_{t}{\psi^\conj} \\
%\implies \\
%V\psi^\conj - \frac{\Hbar}{2i} \partial_t \psi^\conj &=
%\frac{\Hbar^2}{2 m}
%\sum_m \partial_{mm} \psi^\conj +\frac{\Hbar}{2i} \PD{t}{\psi^\conj} \\
%\end{align*}
%
%which is
\begin{equation}\label{eqn:fieldLag:schrod}
\begin{aligned}
-\frac{\Hbar^2}{2 m} \spacegrad^2 \psi^\conj + V\psi^\conj &= -{\Hbar i}{} \PD{t}{\psi^\conj} \\
\end{aligned}
\end{equation}

follows by inspection since all terms except the time partial are symmetric in \(\psi\) and \(\psi^\conj\).  The time partial has a negation in sign from the commutator of the Lagrangian.

FIXME: Goldstein also wanted the Hamiltonian, but I do not know what that is yet.  Got to go read the earlier parts of the book!

\subsection{Relativistic Schr\"{o}dinger's equation}

The
\href{https://en.wikipedia.org/wiki/Noether%27s_theorem}{wiki article on Noether's theorem} lists the relativistic quantum Lagrangian in the form

\begin{equation}\label{eqn:fieldLagrangian:640}
\begin{aligned}
\LL = -\eta^{\mu\nu}\partial_\mu \psi \partial_\nu \psi^\conj + \frac{m^2 c^2}{\Hbar^2}\psi \psi^\conj
\end{aligned}
\end{equation}

That article uses \(\Hbar = c = 1\), and appears to use a \(-+++\) metric, both
of which are adjusted for here.

Calculating the derivatives

\begin{equation}\label{eqn:fieldLagrangian:660}
\begin{aligned}
\PD{\psi^\conj}{\LL} = \frac{m^2 c^2}{\Hbar^2} \psi
\end{aligned}
\end{equation}

\begin{equation}\label{eqn:fieldLagrangian:680}
\begin{aligned}
\partial_\mu \PD{
\lr{ \partial_\mu \psi^\conj }
}{\LL}
&= -\partial_\mu \left(\eta^{\alpha\beta}\partial_\alpha \psi \PD{
\lr{ \partial_\mu \psi^\conj }
}{} \partial_\beta \psi^\conj\right) \\
&= -\partial_\mu
\lr{ \eta^{\alpha\mu}\partial_\alpha \psi } \\
&= -\partial_\mu \partial^\mu \psi \\
\end{aligned}
\end{equation}

So we have
\begin{equation}\label{eqn:fieldLagrangian:700}
\begin{aligned}
\partial_\mu \partial^\mu \psi = \frac{-m^2 c^2}{\Hbar^2} \psi
\end{aligned}
\end{equation}

%There is an ambiguity here
%since it is not specified what signature metric tensor \(\eta\) is used.  The Klein-Gordon wikipedia page uses an equivalent action, and the talk
%page for that indicates they use \(\eta^{00} = -1/c^2\).  In this case we have
%
%\begin{align*}
%-\grad^2 \psi = \frac{m^2 c^2}{\Hbar^2} \psi
%\end{align*}

%which is also consistent with the Klein-Gordon equation as given in that same page:
%

With the metric dependency made explicit this is
\begin{equation}\label{eqn:fieldLagrangian:720}
\begin{aligned}
\left(\spacegrad^2 - \inv{c^2}\PDsq{t}{}\right) \psi = \frac{m^2 c^2}{\Hbar^2} \psi
\end{aligned}
\end{equation}

Much different looking than the classical time dependent Schr\"{o}dinger's equation in \eqnref{eqn:fieldLag:schrod}.
\citep{srednicki2007qft} has a nice discussion about this equation and its relation to the non-relativistic Schr\"{o}dinger's equation.
