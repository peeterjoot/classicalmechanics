%
% Copyright � 2012 Peeter Joot.  All Rights Reserved.
% Licenced as described in the file LICENSE under the root directory of this GIT repository.
%
%
%
%
%\chapter{Canonical energy momentum tensor and Lagrangian translation}
\index{energy momentum tensor}
\index{translation}
\label{chap:stressEnergyNoethers}
%\date{June 5, 2009.  stressEnergyNoethers.tex}
%
\section{Motivation and direction.}
%
In \citep{gabookII:PJstressEnergyLorentz} we saw that it was possible to express the Lorentz force equation for the charge per unit volume in terms of the energy momentum tensor.

Repeating %from \eqnref{eqn:seLorentz:lorentzForceT}
% and
%\eqnref{eqn:seLorentz:lorentzForceT} ???
%
\begin{equation}\label{eqn:stressEnergyNoethers:20}
\begin{aligned}
\grad \cdot T(\gamma_\mu) &= \inv{c} \gpgradezero{ F \gamma_\mu J } \\
T(a) &= \frac{\epsilon_0}{2} F a \tilde{F}.
\end{aligned}
\end{equation}
%
While these may not appear too much like the Lorentz force equation as we are used to seeing it, with some manipulation we found %\eqnref{eqn:seLorentz:lorentzForcePair}
%
\begin{equation}\label{eqn:stressEnergyNoethers:40}
\begin{aligned}
\inv{c} \gpgradezero{ F \gamma_0 J } &= -\Bj \cdot \BE \\
\inv{c} \gpgradezero{ F \gamma_k J } &= (\rho \BE + \Bj \cross \BB) \cdot \sigma_k,
\end{aligned}
\end{equation}
%
where we now have an energy momentum pair of equations, the second
of which if integrated over a volume is the Lorentz force for the charge
in that volume.
%
We have also seen
%in
%\eqnref{eqn:seLorentz:lorentzForceGA}
that we can express the Lorentz force equation in GA form
\begin{equation}\label{eqn:stressEnergyNoethers:60}
\begin{aligned}
m \ddot{x} &= q F \cdot \dot{x}/c.
\end{aligned}
\end{equation}
%
%In
%\eqnref{eqn:seLorentz:lorentzForceTensor}
This was expressed in tensor form, toggling indices that was
\begin{equation}\label{eqn:stressEnergyNoethers:80}
\begin{aligned}
m \ddot{x}_\mu &= {q} F_{\mu\alpha} \dot{x}^\alpha.
\end{aligned}
\end{equation}
%
We then saw in \citep{gabookII:PJenMtensor}
%in equations \eqnref{eqn:stressEnTen:miracle} and
%\eqnref{eqn:stressEnTen:covariantTensor}
that the covariant form of the energy momentum tensor relation was
\begin{equation}\label{eqn:stressEnergyNoethers:100}
\begin{aligned}
T^{\mu\nu} &= {\epsilon_0} \left( F^{\alpha\mu} {F^{\nu}}_{\alpha} + \inv{4} F^{\alpha\beta} F_{\alpha\beta} \eta^{\mu\nu} \right) \\
\partial_\nu T^{\mu\nu} &= F^{\alpha\mu} J_\alpha/c.
\end{aligned}
\end{equation}
This has identical structure (FIXME: sign error here?) to the covariant Lorentz force equation.

Now the energy momentum conservation equations above did not require
the Lorentz force equations at all for their derivation, nor have we
used the Lorentz force interaction Lagrangian to arrive at them.
With Maxwell's equation and the Lorentz force equation together (
or the equivalent field and interaction Lagrangians) we have
the complete specification of classical electrodynamics.
Curiously it appears that we have most of the structure of the Lorentz force equation
(except for the association with mass) all in embedded in Maxwell's equation
or the Maxwell field Lagrangian.

Now, a proper treatment of the field and charged mass interaction
likely requires the Dirac Lagrangian, and hiding in there if one
could extract it, is probably everything that could be said on the
topic.  It will be a long journey to get to that point, but how
much can we do considering just the field Lagrangian?

For these reasons it seems desirable to
understand the background behind the energy momentum tensor much better.
In particular, it is natural to then expect that these conservation
relations may also be found as a
consequence of a symmetry and an associated Noether current (see \chapcite{PJNoethersField}).  What
is that symmetry?  That symmetry should leave the field equations as
calculated by the field Euler-Lagrange equations
Given that symmetry how would one go about
actually showing that this is the case?   These are the questions
to tackle here.
%
\section{On translation and divergence symmetries.}
%
\subsection{Symmetry due to total derivative addition to the Lagrangian.}
%
In \citep{doran2003gap} the energy momentum tensor is treated
by considering spacetime translation, but I have unfortunately
not understood much more than vague direction in that treatment.

In \citep{srednicki2007qft} it is also stated that the energy momentum tensor
is the result of a Lagrangian spacetime translation, but I did not find
details there.

There are examples
of the canonical energy momentum tensor (in the simpler non-GA tensor form)
and the symmetric energy momentum tensor in
\citep{jackson1975cew}.  However, that treatment relies on analogy with
mechanical form of Noether's theorem, and I had rather see it developed
explicitly.

Finally, in an unexpected place (since I am not studying QFT but was merely curious), the
clue required to understand the details of
how this spacetime translation results in the energy momentum tensor was found in
\citep{TongQFT}.
%  Also helpful was problem on exponential operator form of taylor series
% in byram & ...

In Tong's treatment it is pointed out there is a symmetry for the Lagrangian
if it is altered by a divergence
%
\begin{equation}\label{eqn:stressEnergyNoethers:120}
\LL \rightarrow \LL + \partial_\mu F^\mu.
\end{equation}
%
It took me a while to figure out how this was a symmetry, but after a
nice refreshing motorcycle ride, the answer suddenly surfaced.  One
can add a derivative to a mechanical Lagrangian and not change
the resulting equations of motion.  While tackling problem 5 of
Tong's mechanics in \chapcite{PJTongMf1}, such an invariance was considered in
detail in one of the problems for Tong's classical mechanics notes
\chapcite{PJTongMf1:addDerivative}
.
%
%FOLLOWUP: No Noether current was
%derived for this sort of mechanical symmetry.  From
%what I have read, such a Noether's current calculation is
%likely to provide a demonstration of linear momentum conservation.  Worth
%verifying.
%
If one has altered the Lagrangian by adding an arbitrary function \(f\)
to it.
%
\begin{equation}\label{eqn:stressEnergyNoethers:140}
\LL' = \LL + f.
\end{equation}
%
Assuming to start a Lagrangian that is a function of a single field variable
\(\LL = \LL(\phi, \partial_\mu \phi)\), then the
variation of the Lagrangian for the field equations yields
%
\begin{equation}\label{eqn:stressEnergyNoethers:160}
\begin{aligned}
\frac{\delta \LL'}{\delta \phi}
&=
\frac{\partial \LL'}{\partial \phi}
-
\partial_\sigma \frac{\partial \LL'}{\partial (\partial_\sigma \phi)} \\
&=
\mathLabelBox
[
   labelstyle={below of=m\themathLableNode, below of=m\themathLableNode}
]
{
\frac{\partial \LL}{\partial \phi}
-
\partial_\sigma \frac{\partial \LL}{\partial (\partial_\sigma \phi)}
}{\(=0\)}
+\frac{\partial f}{\partial \phi}
-
\partial_\sigma \frac{\partial f}{\partial (\partial_\sigma \phi)} .
\end{aligned}
\end{equation}
%
So, if this transformed Lagrangian is a symmetry, it is sufficient to
find the conditions for the variation of additional part to be zero
%
\begin{equation}\label{eqn:stressEnergyNoethers:requirement}
\frac{\delta f}{\delta \phi} = 0.
\end{equation}
%
\subsection{Some examples adding a divergence.}
%
To validate the fact that we can add a divergence to the Lagrangian without
changing the field equations lets work out a few concrete examples
of \eqnref{eqn:stressEnergyNoethers:requirement} of for Lagrangian alterations by a divergence
\(f = \partial_\mu F^\mu\).

Each of these examples will be for a single field variable Lagrangian
with generalized coordinates \(x^1 = x\), and \(x^1 = y\).
%
\subsubsection{Simplest case.  No partials.}
%
Let
%
\begin{equation}\label{eqn:stressEnergyNoethers:180}
\begin{aligned}
F^1 &= \phi \\
F^2 &= 0 .
\end{aligned}
\end{equation}
%
With this the divergence is
%
\begin{equation}\label{eqn:stressEnergyNoethers:200}
\begin{aligned}
f
&= \partial_x F^x + \partial_y F^y  \\
&=
\PD{x}{\phi}.
\end{aligned}
\end{equation}
%
Now the variation is
\begin{equation}\label{eqn:stressEnergyNoethers:220}
\begin{aligned}
\frac{\delta f}{\delta \phi}
&=
\left( \PD{\phi}{}
- \PD{x}{}\PD{(\PDi{x}{\phi})}{}
- \PD{y}{}\PD{(\PDi{y}{\phi})}{}
\right) \PD{x}{\phi} \\
&=
\PD{x}{}\PD{\phi}{\phi} - \PD{x}{1} \\
&= 0.
\end{aligned}
\end{equation}
%
Okay, so far so good.
%
\subsubsection{One partial.}
%
Now, let
%
\begin{equation}\label{eqn:stressEnergyNoethers:240}
\begin{aligned}
F^1 &= \PD{x}{\phi} \\
F^2 &= 0 .
\end{aligned}
\end{equation}
%
With this the divergence is
%
\begin{equation}\label{eqn:stressEnergyNoethers:260}
\begin{aligned}
f
&= \partial_x F^x + \partial_y F^y  \\
&=
\PD{x}{}\PD{x}{\phi}.
\end{aligned}
\end{equation}
%
And the variation is
\begin{equation}\label{eqn:stressEnergyNoethers:280}
\begin{aligned}
\frac{\delta f}{\delta \phi}
&=
\left( \PD{\phi}{}
- \PD{x}{}\PD{(\PDi{x}{\phi})}{}
- \PD{y}{}\PD{(\PDi{y}{\phi})}{}
\right) \PD{x}{}\PD{x}{\phi} \\
&=
\PD{\phi}{} \PD{x}{}\PD{x}{\phi} \\
&=
\PD{x}{} \PD{x}{}\PD{\phi}{\phi} \\
&=
\PD{x}{} \PD{x}{1} \\
&= 0.
\end{aligned}
\end{equation}
%
Again, assuming I am okay to switch the differentiation order, we have zero.
%
\subsubsection{Another partial.}
%
For the last concrete example before going on to the general case, try
%
\begin{equation}\label{eqn:stressEnergyNoethers:300}
\begin{aligned}
F^1 &= \PD{y}{\phi} \\
F^2 &= 0 .
\end{aligned}
\end{equation}
%
The divergence is
%
\begin{equation}\label{eqn:stressEnergyNoethers:320}
\begin{aligned}
f
&= \partial_x F^x + \partial_y F^y  \\
&=
\PD{x}{}\PD{y}{\phi}.
\end{aligned}
\end{equation}
%
And the variation is
\begin{equation}\label{eqn:stressEnergyNoethers:340}
\begin{aligned}
\frac{\delta f}{\delta \phi}
&=
\left( \PD{\phi}{}
- \PD{x}{}\PD{(\PDi{x}{\phi})}{}
- \PD{y}{}\PD{(\PDi{y}{\phi})}{}
\right) \PD{x}{}\PD{y}{\phi} \\
&=
- \PD{y}{}\PD{x}{1} \\
&= 0.
\end{aligned}
\end{equation}
%
\subsubsection{The general case.}
%
Because of linearity we have now seen that we can construct functions with
any linear combinations of first and second derivatives
%
\begin{equation}\label{eqn:stressEnergyNoethers:360}
F^\mu = a^\mu \phi + \sum_\sigma {b_\sigma}^\mu \PD{x^\sigma}{\phi}.
\end{equation}
%
and for such a function we will have
%
\begin{equation}\label{eqn:stressEnergyNoethers:380}
\frac{\delta (\partial_\mu F^\mu)}{\delta \phi}  = 0.
\end{equation}
%
How general can the function \(F^\mu = F^\mu(\phi, \partial_\sigma \phi)\) be
made and still yield a zero variational derivative?

To answer this, let us compute the derivative for a general divergence
added to a single field variable Lagrangian.  This is
%
\begin{equation}\label{eqn:stressEnergyNoethers:400}
\begin{aligned}
\frac{\delta (\partial_\mu F^\mu)}{\delta \phi}
&=
\sum_\mu \left( \PD{\phi}{}
- \sum_\sigma \PD{x^\sigma}{}\PD{(\PDi{x^\sigma}{\phi})}{}
\right) \PD{x^\mu}{F^\mu} \\
&=
\sum_\mu \PD{x^\mu}{} \PD{\phi}{F^\mu} \\
&\quad
- \sum_{\mu,\sigma} \PD{x^\sigma}{}\PD{(\PDi{x^\sigma}{\phi})}{}
\left(
   \PD{\phi}{F^\mu} \PD{x^\mu}{\phi}
   + \sum_\alpha \PD{(\PDi{x^\alpha}{\phi})}{F^\mu} \PD{x^\mu}{(\PDi{x^\alpha}{\phi})}
\right) \\
&=
\partial_\mu \PD{\phi}{F^\mu}
- \partial_\sigma \PD{(\partial_\sigma \phi)}{}
\left(
\PD{\phi}{F^\mu} \partial_\mu {\phi}
+ \PD{(\partial_\alpha \phi)}{F^\mu} \partial_{\mu\alpha} \phi \right) .
\end{aligned}
\end{equation}
%
For tractability in this last line the shorthand for the partials has been injected.
Sums over \(\alpha\), \(\mu\), and \(\sigma\) are also now implied (this was made explicit prior to
this in all cases where upper and lower indices were matched).

Treating these two last derivatives separately, we have for the first
%
\begin{equation}\label{eqn:stressEnergyNoethers:420}
\begin{aligned}
\partial_\sigma
\PD{(\partial_\sigma \phi)}{}
\PD{\phi}{F^\mu} \partial_\mu {\phi}
&=
\partial_\sigma
\left(\PD{(\partial_\sigma \phi)}{} \PD{\phi}{F^\mu} \right) \partial_\mu {\phi}
+
\partial_\sigma
\PD{\phi}{F^\mu} \PD{(\partial_\sigma \phi)}{} \partial_\mu {\phi} \\
&=
\partial_\sigma
\left(\PD{(\partial_\sigma \phi)}{} \PD{\phi}{F^\mu} \right) \partial_\mu {\phi}
+
\partial_\mu \PD{\phi}{F^\mu}.
\end{aligned}
\end{equation}
%
So our \(\PDi{\phi}{F^\mu}\)'s cancel out, and we are left with
%
\begin{equation}\label{eqn:stressEnergyNoethers:440}
\begin{aligned}
\frac{\delta (\partial_\mu F^\mu)}{\delta \phi}
&=
-\partial_\sigma
\left(
\left(\PD{(\partial_\sigma \phi)}{} \PD{\phi}{F^\mu} \right) \partial_\mu {\phi}
+
\PD{(\partial_\sigma \phi)}{}
\left(
\PD{(\partial_\alpha \phi)}{F^\mu} \partial_{\mu\alpha} \phi
\right)
\right)
\\
&=
-\partial_\sigma
\left(
\partial_\mu {\phi}
\left(\PD{(\partial_\sigma \phi)}{} \PD{\phi}{F^\mu} \right)
+
\partial_{\mu\alpha} \phi
\PD{(\partial_\sigma \phi)}{}
\left(
\PD{(\partial_\alpha \phi)}{F^\mu}
\right)
\right)
\\
&=
-\partial_\sigma
\left(
(\partial_\mu {\phi})
\PD{\phi}{}
\PD{(\partial_\sigma \phi)}{}
F^\mu
+
\left(\partial_{\mu}
\PD{x^\alpha}{\phi}\right)
\PD{(\partial_\alpha \phi)}{}
\PD{(\partial_\sigma \phi)}{}
{F^\mu}
\right)
.
\end{aligned}
\end{equation}
%
Now there is a lot of indices and derivatives floating around.  Writing \(g^\mu = \PDi{(\partial_\sigma \phi)}{F^\mu}\), we have something a bit easier to look at
%
\begin{equation}\label{eqn:stressEnergyNoethers:460}
\begin{aligned}
\frac{\delta (\partial_\mu F^\mu)}{\delta \phi}
&=
-\partial_\sigma
\left(
(\partial_\mu {\phi})
\PD{\phi}{ g^\mu }
+
\left(\partial_{\mu}
\PD{x^\alpha}{\phi}\right)
\PD{(\partial_\alpha \phi)}{ g^\mu }
\right)
.
\end{aligned}
\end{equation}
%
But this is a chain rule expansion of the
derivative \(\partial_\mu g^\mu\)
%
\begin{equation}\label{eqn:stressEnergyNoethers:480}
\begin{aligned}
\PD{x^\mu}{g^\mu} &=
\PD{x^\mu}{\phi}\PD{\phi}{g^\mu}
+ \PD{x^\mu}{\partial_\beta \phi}\PD{\partial_\beta \phi}{g^\mu}.
\end{aligned}
\end{equation}
%
So, we finally have
%
\begin{equation}\label{eqn:stressEnergyNoethers:500}
\frac{\delta (\partial_\mu F^\mu)}{\delta \phi}
=
-\partial_{\sigma \mu} g^\mu.
\end{equation}
%
This is
\begin{equation}\label{eqn:stressEnergyNoethers:requiredZeroForSymmetry}
\frac{\delta (\partial_\mu F^\mu)}{\delta \phi}
=
-\partial_{\sigma \mu} \PD{(\partial_\sigma \phi)}{F^\mu}.
\end{equation}
%
I do not think we have any right asserting that this is zero for arbitrary \(F^\mu\).  However
if the Taylor expansion of \(F^\mu\) with respect to variables \(\phi\), and \(\partial_\sigma \phi\) has
no higher than first order terms in the field variables \(\partial_\sigma \phi\), we will
certainly have a zero variational derivative and a corresponding symmetry.
\subsubsection{More examples to confirm the symmetry requirements.}
As a confirmation that a zero in \eqnref{eqn:stressEnergyNoethers:requiredZeroForSymmetry} requires linear field derivatives, lets try two more example calculations.

First with non-linear powers of \(\phi\) to show that we have more freedom to construct the function first powers.
Let
\begin{equation}\label{eqn:stressEnergyNoethers:520}
\begin{aligned}
F^1 &= \phi^2 \\
F^2 &= 0.
\end{aligned}
\end{equation}
%
We have
%
\begin{equation}\label{eqn:stressEnergyNoethers:540}
\begin{aligned}
\frac{\delta (\partial_\mu F^\mu)}{\delta \phi}
&=
\left(\PD{\phi}{} - \partial_\sigma \PD{(\partial_\sigma \phi)}{}\right) 2 \phi \phi_x \\
&=
2 \phi_x - \partial_x (2 \phi) \\
&= 0.
\end{aligned}
\end{equation}
%
Zero as expected.  Generalizing the function to include arbitrary polynomial powers is no harder.
%
Let
\begin{equation}\label{eqn:stressEnergyNoethers:560}
\begin{aligned}
F^1 &= \phi^k \\
F^2 &= 0 \\
\partial_\mu F^\mu &= k \phi^{k-1} \phi_x .
\end{aligned}
\end{equation}
%
So we have
\begin{equation}\label{eqn:stressEnergyNoethers:580}
\begin{aligned}
\frac{\delta (\partial_\mu F^\mu)}{\delta \phi}
&=
k (k-1) \phi^{k-2} \phi_x - \partial_x (k \phi^{k-1})  \\
&= 0.
\end{aligned}
\end{equation}
%
Okay, now moving on to the derivatives.  Picking a divergence that should not will not generate a
symmetry, something with a non-linear derivative should do the trick.  Let us Try
%
\begin{equation}\label{eqn:stressEnergyNoethers:600}
\begin{aligned}
F^1 &= (\phi_x)^2 \\
F^2 &= 0.
\end{aligned}
\end{equation}
%
\begin{equation}\label{eqn:stressEnergyNoethers:620}
\begin{aligned}
\frac{\delta (\partial_\mu F^\mu)}{\delta \phi}
&=
\left(\PD{\phi}{} - \partial_\sigma \PD{(\partial_\sigma \phi)}{}\right) 2 \phi_x \phi_{xx} \\
&=
- 2 \partial_x \phi_{xx} \\
&=
- 2 \phi_{xxx} .
\end{aligned}
\end{equation}
%
So, sure enough, unless additional conditions can be imposed on \(\phi\), such a transformation
will not be a symmetry.
%
\subsection{Symmetry for Wave equation under spacetime translation.}
%
The Lagrangian for a one dimensional wave equation is
%
\begin{equation}\label{eqn:stressEnergyNoethers:oneDimWave}
\LL =
\inv{2 v^2} \left(\PD{t}{\phi}\right)^2 - \inv{2} \left(\PD{x}{\phi}\right)^2.
\end{equation}
%
Under a transformation of variables
%
\begin{equation}\label{eqn:stressEnergyNoethers:640}
\begin{aligned}
x &\rightarrow x' = x + a \\
t &\rightarrow t' = t + \tau.
\end{aligned}
\end{equation}
%
Employing a multivariable Taylor expansion
(see
\citep{gabookI:PJmultiTaylors}
)
for our Lagrangian having no explicit dependence on \(t\) and \(x\), we have
%
\begin{equation}\label{eqn:stressEnergyNoethers:660}
\begin{aligned}
\LL' &= \LL +
\mathLabelBox{(a \partial_x + \tau \partial_t)\LL}{\(\conj\)}
+ \cdots
\end{aligned}
\end{equation}
%
That first order term of the Taylor expansion \(\conj\),
can be written as a divergence \(\partial_\mu F^\mu\), with \(F^1 = a \LL\), and \(F^2 = \tau\LL\), however
both of these are quadratic in \(\phi_x\), and \(\phi_t\), which is not linear.
That linearity in the derivatives was required for \eqnref{eqn:stressEnergyNoethers:requiredZeroForSymmetry} to be
definitively zero for the transformation to be a symmetry.  So after all that goofing around
with derivatives and algebra it is defeated by the
simplest field Lagrangian.

Now, if we continue we find that we do in fact still have a symmetry by introducing a linearized spacetime translation.
This follows from direct expansion
%
\begin{equation}\label{eqn:stressEnergyNoethers:680}
\begin{aligned}
(*)
&= (a \partial_x + \tau \partial_t) \LL \\
&=
a
\left(
\inv{v^2} \phi_t \partial_x \phi_t
- \phi_x \partial_x \phi_x
\right)
+ \tau
\left(
\inv{v^2} \phi_t \partial_t \phi_t
- \phi_x \partial_t \phi_x
\right) .
\end{aligned}
\end{equation}
%
Next, calculation of the variational derivative we have
%
\begin{equation}\label{eqn:stressEnergyNoethers:700}
\begin{aligned}
\frac{\delta(*)}{\delta \phi}
&=
\left( \PD{\phi}{} - \partial_x \PD{\phi_x}{} - \partial_t \PD{\phi_t}{} \right) (*) \\
&=
-\partial_x \left( -a \partial_{xx} \phi - \tau \partial_{tx}\phi \right)
-\inv{v^2} \partial_t \left( a \partial_{xt} \phi + \tau \partial_{tt}\phi \right) \\
&=
a \left(
\partial_x \left(
\phi_{xx} - \inv{v^2} \phi_{tt}
\right)
\right)
+ \tau \left(
\partial_t \left(
\phi_{xx} - \inv{v^2} \phi_{tt}
\right)
\right) .
\end{aligned}
\end{equation}
%
Since we have \(\phi_{xx} = \inv{v^2} \phi_{tt}\) by variation of \eqnref{eqn:stressEnergyNoethers:oneDimWave}.  So we do in fact have a symmetry from the
linearized spacetime translation for any shift \((t,x) \rightarrow (t+\tau, x+a)\).
%
\subsection{Symmetry condition for arbitrary linearized spacetime translation.}
%
If we want to be able to alter the Lagrangian with a linearized vector translation of the generalized coordinates by some arbitrary
shift, since we do not have the linear derivatives for many Lagrangians of interest (wave equations, Maxwell equation, ...)
then can we find a general condition that is responsible for the translation symmetry that we have observed must exist for
the simple wave equation.

For a general Lagrangian \(\LL = \LL(\phi(x), \partial_\mu \phi(x))\) under shift by some vector \(a\)
%
\begin{equation}\label{eqn:stressEnergyNoethers:shiftXbyA}
x \rightarrow x' = x + a,
\end{equation}
%
we have
%
\begin{equation}\label{eqn:stressEnergyNoethers:720}
\LL' = \left( e^{a \cdot \grad}\right) \LL = \LL + (a \cdot \grad)\LL + \inv{2!}(a \cdot \grad)^2 \LL + \cdots
\end{equation}
%
Now, if we have
%
\begin{equation}\label{eqn:stressEnergyNoethers:740}
\begin{aligned}
\frac{\delta ((a \cdot \grad) \LL)}{\delta \phi}
&
\questionEquals
(a \cdot \grad)
\mathLabelBox{\frac{\delta \LL}{\delta \phi}}{\(=0\)}
 .
\end{aligned}
\end{equation}
%
then this would explain the fact that we have a symmetry under linearized translation for the wave equation Lagrangian.  Can this interchange
of differentiation order be justified?

Writing out this variational derivative in full we have
\begin{equation}\label{eqn:stressEnergyNoethers:760}
\begin{aligned}
\frac{\delta ((a \cdot \grad) \LL)}{\delta \phi}
&=
\left( \PD{\phi}{} - \partial_\sigma \PD{\phi_\sigma}{} \right) a^\mu \partial_\mu \LL \\
&=
a^\mu \left(
\PD{\phi}{} \PD{x^\mu}{}
 - \PD{x^\sigma}{} \PD{\phi_\sigma}{} \PD{x^\mu}{}
\right)
\LL .
\end{aligned}
\end{equation}
%
Now, one can impose continuity conditions on the
field variables and Lagrangian sufficient to allow the commutation
of the coordinate partials.  Namely
%
\begin{equation}\label{eqn:stressEnergyNoethers:780}
\begin{aligned}
\PD{x^\mu}{}\PD{x^\nu}{} f(\phi, \partial_\sigma \phi)
&=
\PD{x^\nu}{}\PD{x^\mu}{} f(\phi, \partial_\sigma \phi).
\end{aligned}
\end{equation}
%
However, we have a dependence between the field variables and the coordinates
%
\begin{equation}\label{eqn:stressEnergyNoethers:800}
\PD{x^\mu}{} =
\PD{x^\mu}{\phi} \PD{\phi}{}
+\sum_\sigma \PD{x^\mu}{\phi_\sigma} \PD{\phi_\sigma}{}.
\end{equation}
%
Given this, can we commute the field partials and the coordinate partials like so
%
\begin{equation}\label{eqn:stressEnergyNoethers:820}
\begin{aligned}
\PD{\phi}{} \PD{x^\mu}{} &\questionEquals \PD{x^\mu}{} \PD{\phi}{} \\
\PD{\phi_\sigma}{} \PD{x^\mu}{}  &\questionEquals \PD{x^\mu}{} \PD{\phi_\sigma}{}.
\end{aligned}
\end{equation}
%
This is not obvious to me due to the dependence between the two.

If that is a reasonable thing to do, then the variational derivative of this directional derivative is zero
%
\begin{equation}\label{eqn:stressEnergyNoethers:840}
\begin{aligned}
\frac{\delta ((a \cdot \grad) \LL)}{\delta \phi}
&=
a^\mu \PD{x^\mu}{}
\left(
\PD{\phi}{}
 - \PD{x^\sigma}{} \PD{\phi_\sigma}{}
\right)
\LL \\
&=
(a \cdot \grad) \frac{\delta \LL}{\delta \phi} \\
&= 0.
\end{aligned}
\end{equation}
%
To make any progress below I had to assume that this is justifiable.  With this assumption or requirement
we therefore have a symmetry for any Lagrangian
altered by the addition of a directional derivative, as is required for the first order Taylor series
approximation associated with a spacetime (or spatial or timelike) translation.
%
\subsubsection{An error above to revisit.}
%
In an email discussing what I initially thought was a typo in
\citep{TongQFT}, he says
%There is this subtlety with active vs. passive transformations which is explained after (1.26). If the x coordinate gets shifted one way, then the argument of the field goes the other way. After (1.26), I describe this for Lorentz transformations, but it is also true for translations which is the situation in (1.39).
%
%Yes - if you did it the other way, you would also get the right Noether current. And you are right -- you should just be expanding out the first term in the Taylor series. The point is that when the spacetime coordinate
%changes as
%
%x -> x-a
%
%you do not just put this into the argument \phi(x). Instead the field transforms as
%
%\phi(x) -> \phi(x+a)
%
%which you then expand. The reason for this strange change of sign is to do with the difference between an active and a passive transformation. It is a little subtle, and it is what I tried to explain in the words below (1.26)
%
%
that while it is correct to transform the Lagrangian
using a Taylor expansion in \(\phi(x+a)\) as I have done, this actually results
from \(x \rightarrow x - a\), as opposed to the positive shift given in
\eqnref{eqn:stressEnergyNoethers:shiftXbyA}.  There was discussion of this in the context of Lorentz transformations around (1.26) of his QFT course notes, also applicable to translations.
The subtlety is apparently due to differences between passive and active
transformations.
I am sure he is right, and I think this is actually consistent with the
treatment of
\citep{doran2003gap} where they include an inverse operation in the
transformed Lagrangian (that minus is surely associated with the inverse
of the translation transformation).
It will take further study for me to completely understand this point, but
provided the starting point is really considered the Taylor series expansion
based on \(\phi(x) \rightarrow \phi(x+a)\) and not based on \eqnref{eqn:stressEnergyNoethers:shiftXbyA}
then nothing else I have done here is wrong.  Also note that in the end our
Noether current can be adjusted by an arbitrary multiplicative constant so
the direction of the translation will also not change the final result.
%
\section{Noether current.}
%
\subsection{Vector parametrized Noether current.}
%
In \chapcite{PJFieldLagrangian} the derivation of Noether's theorem given a single variable parametrized
alteration of the Lagrangian was seen to essentially be an exercise in the
application of the chain rule.

How to extend that argument to the multiple variable case is not immediately
obvious.  In GA we can divide by vectors but attempting to formulate
a derivative this way gives us left and right sided derivatives.  How do we
overcome this to examine change of the Lagrangian with respect to a
vector parametrization?
One possibility is a scalar parametrization of the magnitude
of the translation vector.  If the translation is along \(a = \alpha u\),
where \(u\) is a unit vector we can write
%
\begin{equation}\label{eqn:stressEnergyNoethers:860}
\begin{aligned}
\LL' &= \LL + \delta \LL  \\
&= \LL + (a \cdot \grad) \LL \\
&= \LL + \alpha (u \cdot \grad) \LL .
\end{aligned}
\end{equation}
%
So we have
%
\begin{equation}\label{eqn:stressEnergyNoethers:880}
\begin{aligned}
\frac{d\LL'}{d\alpha}
&=
(u \cdot \grad) \LL.
\end{aligned}
\end{equation}
%
Now our previous Noether's current was derived by considering just
the sort of derivative on the LHS above, but on the RHS we are back
to working with a directional derivative.  The key is finding a
logical starting point for the chain rule like expansion that we expect
to produce the conservation current.
%
\begin{equation}\label{eqn:stressEnergyNoethers:900}
\begin{aligned}
\delta \LL
&=  (a \cdot \grad ) \LL \\
&=  a^\mu \partial_\mu \LL \\
&=  a^\mu \left(
\PD{x^\mu}{\phi} \PD{\phi}{\LL}
+\sum_\sigma
\PD{x^\mu}{\phi_\sigma} \PD{\phi_\sigma}{\LL}
\right) \\
&=
\PD{\phi}{\LL} (a \cdot \grad) \phi
+
\sum_\sigma
\PD{\phi_\sigma}{\LL}
(a \cdot \grad) \phi_\sigma
\\
&=
\left(
\sum_\sigma
\partial_\sigma
\PD{\phi_\sigma}{\LL}
\right)
(a \cdot \grad) \phi
+
\sum_\sigma
\PD{\phi_\sigma}{\LL}
(a \cdot \grad) \phi_\sigma
\\
&=
\left(
\sum_\sigma
\partial_\sigma
\PD{\phi_\sigma}{\LL}
\right)
(a \cdot \grad) \phi
+
\sum_\sigma
\PD{\phi_\sigma}{\LL}
\partial_\sigma
((a \cdot \grad) \phi)
\\
&=
\sum_\sigma
\partial_\sigma
\left(
\PD{\phi_\sigma}{\LL}
(a \cdot \grad) \phi
\right)
.
\end{aligned}
\end{equation}
%
So far so good, but where to go from here?  The trick (again from Tong) is that
the difference with itself is zero.  With a switch of dummy indices \(\sigma \rightarrow \mu\), we have
%
\begin{equation}\label{eqn:stressEnergyNoethers:920}
\begin{aligned}
0 &=
\delta \LL - \delta \LL  \\
&=
\sum_\mu
\partial_\mu
\left(
\PD{\phi_\mu}{\LL}
(a \cdot \grad) \phi
\right) -
 a^\mu \partial_\mu \LL
\\
&=
\sum_\mu
\partial_\mu
\left(
\PD{\phi_\mu}{\LL}
(a \cdot \grad) \phi
- a^\mu \LL
\right)
.
\end{aligned}
\end{equation}
%
Now we have a quantity that is zero for any vector \(a\), and can say we have
a conserved current \(T(a)\) with coordinates
%
\begin{equation}\label{eqn:stressEnergyNoethers:JmuOfAoneVar}
T^\mu(a)
=
\PD{\phi_\mu}{\LL}
(a \cdot \grad) \phi
- a^\mu \LL.
\end{equation}
%
Finally, putting this back into vector form
%
\begin{equation}\label{eqn:stressEnergyNoethers:940}
\begin{aligned}
T(a) &= \gamma_\mu T^\mu(a) \\
&=
\left( \gamma_\mu \PD{\phi_\mu}{\LL} \right)
(a \cdot \grad) \phi
- \gamma_\mu a^\mu \LL  .
\end{aligned}
\end{equation}
%
So we have
%
\begin{equation}\label{eqn:stressEnergyNoethers:vCurrent}
\begin{aligned}
T(a) &=
\left( \left(\gamma_\mu \PD{\phi_\mu}{} \right) \LL \right)
(a \cdot \grad) \phi
- a \LL  \\
\grad \cdot T(a) &= 0.
\end{aligned}
\end{equation}
%
So after a long journey, I have in
\eqnref{eqn:stressEnergyNoethers:vCurrent}
a derivation of a conservation current associated
with
a linearized vector displacement of the generalized coordinates.  I
recalled that the treatment in
\citep{doran2003gap} somehow eliminated the \(a\).  That argument is still tricky involving
their linear operator theory, but I have at least obtained their equation (13.15).
They treat a multivector displacement whereas I only looked at
vector displacement.  They also do it in three lines, whereas building up to this
(or even understanding it) based on what I know required 13 pages.
%
\subsection{Comment on the operator above.}
%
We have something above that is gradient like in
\eqnref{eqn:stressEnergyNoethers:vCurrent}.  Our spacetime gradient operator is
%
\begin{equation}\label{eqn:stressEnergyNoethers:960}
\grad = \gamma^\mu \PD{x^\mu}{}.
\end{equation}
%
Whereas this unknown field variable derivative operator
%
\begin{equation}\label{eqn:stressEnergyNoethers:980}
\something = \gamma_\mu \PD{\phi_\mu}{}.
\end{equation}
%
is somewhat like a velocity gradient with respect to the field variable.
It would be reasonable to expect that this will have a role in the field canonical momentum.
%
\subsection{In tensor form.}
%
The conserved current
of \eqnref{eqn:stressEnergyNoethers:vCurrent}
can be put into tensor form by considering the action on
each of the basis vectors.
%
\begin{equation}\label{eqn:stressEnergyNoethers:1000}
\begin{aligned}
T(\gamma_\nu) \cdot \gamma^\mu
&=
\left( \left(\PD{\phi_\mu}{} \right) \LL \right)
(\gamma_\nu \cdot (\gamma^\sigma \partial_\sigma)) \phi
- \gamma_\nu \cdot \gamma^\mu \LL  .
\end{aligned}
\end{equation}
%
Thus writing \({T^\mu}_\nu = T(\gamma_\nu) \cdot \gamma^\mu\) we have
%
\begin{equation}\label{eqn:stressEnergyNoethers:currentTensor}
{T^\mu}_\nu = \PD{\phi_\mu}{\LL} \partial_\nu \phi - {\delta_\nu}^\mu \LL.
\end{equation}
%
\subsection{Multiple field variables.}
%
In order to deal with the Maxwell Lagrangian a generalization to multiple
field variables is required.  Suppose now that we have a Lagrangian
density \(\LL = \LL(\phi^\alpha, \partial_\beta \phi^\alpha)\).  Proceeding
with the chain rule application again we have after some latex
search and replace
adding in indices in all the right places (proof by regular expressions)
%
\begin{equation}\label{eqn:stressEnergyNoethers:1020}
\begin{aligned}
\delta \LL
&=  (a \cdot \grad ) \LL \\
&=  a^\mu \partial_\mu \LL \\
&=  a^\mu \left(
\PD{x^\mu}{\phi^\alpha} \PD{\phi^\alpha}{\LL}
+
\PD{x^\mu}{\partial_\sigma \phi^\alpha} \PD{\partial_\sigma \phi^\alpha}{\LL}
\right) \\
&=
\PD{\phi^\alpha}{\LL} (a \cdot \grad) \phi^\alpha
+
\PD{\partial_\sigma \phi^\alpha}{\LL}
(a \cdot \grad) \partial_\sigma \phi^\alpha
\\
&=
\left(
\partial_\sigma
\PD{\partial_\sigma \phi^\alpha}{\LL}
\right)
(a \cdot \grad) \phi^\alpha
+
\PD{\partial_\sigma \phi^\alpha}{\LL}
(a \cdot \grad) \partial_\sigma \phi^\alpha
\\
&=
\left(
\partial_\sigma
\PD{\partial_\sigma \phi^\alpha}{\LL}
\right)
(a \cdot \grad) \phi^\alpha
+
\PD{\partial_\sigma \phi^\alpha}{\LL}
\partial_\sigma
((a \cdot \grad) \phi^\alpha)
\\
&=
\partial_\sigma
\left(
\PD{\partial_\sigma \phi^\alpha}{\LL}
(a \cdot \grad) \phi^\alpha
\right)
.
\end{aligned}
\end{equation}
%
In the above manipulations (and those below), any repeated index, regardless of whether upper and lower indices are matched implies summation.

Using this we have a multiple field generalization of
\eqnref{eqn:stressEnergyNoethers:JmuOfAoneVar}.   The
Noether current and its conservation law in coordinate form is
%
\begin{equation}\label{eqn:stressEnergyNoethers:JmuOfAmanyVar}
\begin{aligned}
T^\mu(a)
&=
\PD{\partial_\mu \phi^\alpha}{\LL}
(a \cdot \grad) \phi^\alpha
- a^\mu \LL \\
\partial_\mu T^\mu(a) &= 0.
\end{aligned}
\end{equation}
%
Or in vector form, corresponding to \eqnref{eqn:stressEnergyNoethers:vCurrent}
%
\begin{equation}\label{eqn:stressEnergyNoethers:vCurrentmanyField}
\begin{aligned}
T(a) &=
\left( \left(\gamma_\mu \PD{\partial_\mu \phi^\alpha}{} \right) \LL \right)
(a \cdot \grad) \phi^\alpha
- a \LL  \\
\grad \cdot T(a) &= 0.
\end{aligned}
\end{equation}
%
And finally
in tensor form, as in \eqnref{eqn:stressEnergyNoethers:currentTensor}
%
\begin{equation}\label{eqn:stressEnergyNoethers:currentTensormany}
\begin{aligned}
{T^\mu}_\nu &= \PD{\partial_\mu \phi^\alpha}{\LL} \partial_\nu \phi^\alpha - {\delta_\nu}^\mu \LL \\
\partial_\mu {T^\mu}_\nu &= 0.
\end{aligned}
\end{equation}
%
\subsection{Spatial Noether current.}
%
The conservation arguments above have been expressed with the assumption that the Lagrangian density is a function
of both spatial and time coordinates, and this was made explicit with the use of the Dirac basis to express the
Noether current.

It should be pointed out that for a purely spatial Lagrangian density, such as that of electrostatics
%
\begin{equation}\label{eqn:stressEnergyNoethers:1040}
\begin{aligned}
\LL &= -\frac{\epsilon_0}{2} (\spacegrad \phi)^2 + \rho \phi.
\end{aligned}
\end{equation}
%
the same results apply.  In this case it would be reasonable to summarize the conservation under translation
using the Pauli basis and write
%
\begin{equation}\label{eqn:stressEnergyNoethers:1060}
\begin{aligned}
T(\Ba) &= \sigma_k \PD{\partial_k \phi}{\LL} \Ba \cdot \spacegrad \phi - \Ba \LL \\
\spacegrad \cdot T(\Ba) &= 0.
\end{aligned}
\end{equation}
%
Without the time translation, calling the vector Noether current the energy momentum tensor is not likely appropriate.  Perhaps just
the canonical energy momentum tensor?  Working with such a spatial Lagrangian density later should help clarify how to label things.
%
\section{Field Hamiltonian.}
%
A special case of \eqnref{eqn:stressEnergyNoethers:currentTensor} is for time translation of the
Lagrangian.

For that, our Noether current, writing \(\calH^\mu = {T^\mu}_0\) is
%
%\newcommand{\HH}[0]{\boldsymbol{\calH}}
\begin{equation}\label{eqn:stressEnergyNoethers:1080}
\begin{aligned}
\calH^0 &= \PD{\dotphi}{\LL} \dotphi - \LL \\
\calH^k &= \PD{\phi_k}{\LL} \dotphi.
\end{aligned}
\end{equation}
%
These are expected to have a role associated with field energy and
momentum respectively.

For the Maxwell Lagrangian we will need the multiple field current
%
\begin{equation}\label{eqn:stressEnergyNoethers:1100}
\begin{aligned}
\calH^0 &= \PD{\partial_0 \phi^\alpha}{\LL} \partial_0 \phi^\alpha - \LL \\
\calH^k &= \PD{\partial_k \phi^\alpha}{\LL} \partial_0 \phi^\alpha.
\end{aligned}
\end{equation}
%
%
\section{Wave equation.}
%
Having computed the general energy momentum tensor for field Lagrangians, this can now be applied
to some specific field equations.  The Lagrangian for the relativistic wave equation is an
obvious first candidate due to simplicity.
%
\subsection{Tensor components and energy term.}
%
%  One of the simplest
%Lagrangian with a single field variable, and the next ones are for multiple field Lagrangians where
%we will need to think through a generalization of the Noether conservation current equation first.
%
\begin{dmath}\label{eqn:stressEnergyNoethers:1120}
\LL
= \inv{2} \partial_\mu \phi \partial^\mu \phi
= \inv{2} \phi_\mu \phi^\mu
= \inv{2} \lr{\grad \phi}^2
= \inv{2}
\lr{\dotphi^2 - (\spacegrad \phi)^2}.
\end{dmath}
%
In the explicit spacetime split above we have a split into terms that appear
to correspond to kinetic and potential terms
%
\begin{equation}\label{eqn:stressEnergyNoethers:1140}
\LL = K - V.
\end{equation}
%
To compute the tensor, we first need
\(\PDi{\phi_\mu}{\LL} = \phi^\mu\), which gives us
%
\begin{equation}\label{eqn:stressEnergyNoethers:tensorInIndexForm}
\begin{aligned}
{T^\mu}_\nu
&= \phi^\mu \phi_\nu - {\delta_\nu}^\mu \LL.
\end{aligned}
\end{equation}
%
Writing this out in matrix form (with rows \(\mu\), and columns \(\nu\)), we have
%
\begin{equation}\label{eqn:stressEnergyNoethers:bigTensorMatrix}
\begin{bsmallmatrix}
\inv{2}(\dotphi^2 + \phi_x^2 + \phi_y^2 + \phi_z^2) & \dotphi \phi_x & \dotphi \phi_y & \dotphi \phi_z \\
-\phi_x \dotphi & \inv{2}(-\dotphi^2 - \phi_x^2 + \phi_y^2 + \phi_z^2) & -\phi_x \phi_y & -\phi_x \phi_z \\
-\phi_y \dotphi & -\phi_y \phi_x & \inv{2}(-\dotphi^2 + \phi_x^2 - \phi_y^2 + \phi_z^2) & -\phi_y \phi_z \\
-\phi_z \dotphi & -\phi_z \phi_x & -\phi_z \phi_y & \inv{2}(-\dotphi^2 + \phi_x^2 + \phi_y^2 - \phi_z^2) \\
\end{bsmallmatrix}.
\end{equation}
%
As mentioned by Jackson, the canonical energy momentum tensor is not necessarily symmetric, and we see that here.
We have what is expected for the wave energy in the \(0,0\) element
%
\begin{equation}\label{eqn:stressEnergyNoethers:1160}
\begin{aligned}
{T^0}_0 &= K + V  \\
&= \inv{2} (\dotphi^2 + (\spacegrad \phi)^2).
\end{aligned}
\end{equation}
%
\subsection{Conservation equations.}
%
How about the conservation equations when written in full.  The first is
%
\begin{equation}\label{eqn:stressEnergyNoethers:1180}
\begin{aligned}
0
&= \partial_\mu {T^\mu}_0 \\
&=
\inv{2} \partial_t (\dotphi^2 + \phi_x^2 + \phi_y^2 + \phi_z^2)
-\partial_x(\phi_x \dotphi )
-\partial_y(\phi_y \dotphi )
-\partial_z(\phi_z \dotphi ) \\
&=
\dotphi \ddotphi
+ \phi_x \phi_{xt}
+ \phi_y \phi_{yt}
+ \phi_z \phi_{zt}
-\phi_{xx} \dotphi
-\phi_{yy} \dotphi
-\phi_{zz} \dotphi
-\phi_x \phi_{tx}
-\phi_y \phi_{ty}
-\phi_z \phi_{tz}
\\
&=
\dotphi (\ddotphi -\phi_{xx} -\phi_{yy} -\phi_{zz} ).
\end{aligned}
\end{equation}
%
So our first conservation equation is
%
\begin{equation}\label{eqn:stressEnergyNoethers:1200}
0 = \dotphi (\grad^2 \phi).
\end{equation}
%
But \(\grad^2 \phi = 0\) is just our wave equation, the result of the variation of the Lagrangian itself.
So curiously the divergence of energy-momentum four vector \({T^\mu}_0\)
ends up as another method of supplying the wave equation!

How about one of the other conservation equations?  The pattern will all be the same, so calculating one is sufficient.
%
\begin{equation}\label{eqn:stressEnergyNoethers:1220}
\begin{aligned}
0
&= \partial_\mu {T^\mu}_1 \\
&=
\partial_t(\dotphi \phi_x )
+\inv{2} \partial_x (-\dotphi^2 - \phi_x^2 + \phi_y^2 + \phi_z^2)
-\partial_y(\phi_y \phi_x )
-\partial_z(\phi_z \phi_x ) \\
&=
\ddotphi \phi_x
+ \dotphi \phi_{xt}
-\dotphi \phi_{tx} - \phi_x \phi_{xx} + \phi_y \phi_{yx}+ \phi_z \phi_{zx}
- \phi_{yy} \phi_x - \phi_y \phi_{xy}
- \phi_{zz} \phi_x - \phi_z \phi_{xz}
\\
&=
\phi_x (\ddotphi -\phi_{xx} -\phi_{yy} -\phi_{zz} ).
\end{aligned}
\end{equation}
%
It should probably not be surprising that we have such a symmetric relation between space and time for the wave equations
and we can summarize the spacetime translation conservation equations by
%
\begin{equation}\label{eqn:stressEnergyNoethers:1240}
\begin{aligned}
0
&= \partial_\mu {T^\mu}_\nu \\
&= \phi_\nu (\grad^2 \phi).
\end{aligned}
\end{equation}
%
\subsection{Invariant length.}
%
It has been assumed that \(T(\gamma_\mu)\) are four vectors.  If that is the
cast we ought to have an invariant length.

Let us calculate the vector square of \(T(\gamma_0)\).
Picking off first column of our tensor in \eqnref{eqn:stressEnergyNoethers:bigTensorMatrix}, we have
%
\begin{equation}\label{eqn:stressEnergyNoethers:1260}
\begin{aligned}
(T(\gamma_0))^2
&= (\gamma_\mu {T^\mu}_0) \cdot ( \gamma_\nu {T^\nu}_0 ) \\
&= ({T^0}_0)^2 -({T^1}_0)^2 -({T^2}_0)^2 -({T^3}_0)^2 \\
&=
 \inv{4}\left( \dotphi^2 + \phi_x^2 + \phi_y^2 + \phi_z^2 \right)^2
- \phi_x^2 \dotphi^2
- \phi_y^2 \dotphi^2
- \phi_z^2 \dotphi^2 \\
&=
 \inv{4}\left( \dotphi^4 + \phi_x^4 + \phi_y^4 + \phi_z^4 \right)
-\inv{2}
\left(
   \dotphi^2 \phi_x^2
   +\dotphi^2 \phi_y^2
   +\dotphi^2 \phi_z^2
\right) \\
&\quad
+\inv{2}\left(
+\phi_x^2 \phi_y^2
+\phi_y^2 \phi_z^2
+\phi_z^2 \phi_x^2 \right)
%- \phi_x^2 \dotphi^2
%- \phi_y^2 \dotphi^2
%- \phi_z^2 \dotphi^2
\\
&=
 \inv{4}\left( \dotphi^2 - \phi_x^2 - \phi_y^2 - \phi_z^2 \right)^2 .
\end{aligned}
\end{equation}
%
But this is just our (squared) Lagrangian density, and we therefore have
%
\begin{equation}\label{eqn:stressEnergyNoethers:1280}
\begin{aligned}
(T(\gamma_0))^2 &= \LL^2.
\end{aligned}
\end{equation}
%
%\begin{align*}
%T(\gamma_\nu)^2
%&= (\gamma_\mu {T^\mu}_\nu)^2 \\
%&=
%\sum_\mu (\gamma_\mu)^2 (\phi^\mu \phi_\nu - {\delta_\nu}^\mu \LL)^2 \\
%&=
%\sum_\mu (\gamma_\mu)^2 (-\phi_\mu \phi_\nu - {\delta_\nu}^\mu \LL)^2 \\
%&=
%\sum_\mu (\gamma_\mu)^2 (
%\phi_\mu^2 \phi_\nu^2 + {\delta_\nu}^\mu \LL^2
%+2 \phi_\mu \phi_\nu {\delta_\nu}^\mu \LL
%) \\
%&=
%\sum_\mu (\gamma_\mu)^2
%\phi_\mu^2 \phi_\nu^2  + (\gamma_\nu)^2 (\LL^2 +2 \phi_\nu^2 \LL)
%\\
%\end{align*}
%
Doing the same calculation for the second column, which is representative of the other two by symmetry, we have
\begin{equation}\label{eqn:stressEnergyNoethers:1300}
\begin{aligned}
(T(\gamma_k))^2 &= -\LL^2.
\end{aligned}
\end{equation}
%
Summarizing all four squares we have
\begin{equation}\label{eqn:stressEnergyNoethers:invariantLength}
\begin{aligned}
(T(\gamma_\mu))^2 &= (\gamma_\mu)^2 \LL^2.
\end{aligned}
\end{equation}
%
All of these conservation current four vectors have the same length up to a sign, where \(T(\gamma_0)\) is timelike (positive square),
whereas \(T(\gamma_k)\) is spacelike (negative square).

Now, is \(\LL^2\) a Lorentz invariant?  If so we can justify calling \(T(\gamma_\mu)\) four vectors.  Reflection shows that this is in fact the case, since \(\LL\) is a Lorentz invariant.  The transformation properties of \(\LL\) go with the gradient.  Writing \(\grad' = R \grad \tilde{R}\), we have
%
\begin{equation}\label{eqn:stressEnergyNoethers:1320}
\begin{aligned}
\LL'
&= \inv{2} \grad' \phi \cdot \grad' \phi \\
&= \inv{2} \gpgradezero{ R \grad \tilde{R} \phi R \grad \tilde{R} \phi} \\
&= \inv{2} \gpgradezero{ R \grad \phi \grad \tilde{R} \phi} \\
&= \inv{2} \gpgradezero{
\mathLabelBox
[
   labelstyle={xshift=2cm},
   linestyle={out=270,in=90, latex-}
]
{\tilde{R} R}{\(=1\)}
\grad \phi \grad \phi} \\
&= \inv{2} \grad \phi \cdot \grad \phi \\
&= \LL.
\end{aligned}
\end{equation}
%
\subsection{Diagonal terms of the tensor.}
%
There is a conjugate structure evident in the diagonal terms of the matrix for
the tensor.  In particular, the \({T^0}_0\) can be expressed using the
Hermitian conjugate from QM.  For a multivector \(F\), this was defined
as
%
\begin{equation}\label{eqn:stressEnergyNoethers:HermitianConj}
F^\dagger = \gamma_0 \tilde{F} \gamma_0.
\end{equation}
%
We have for \({T^0}_0\)
%
\begin{equation}\label{eqn:stressEnergyNoethers:1340}
\begin{aligned}
{T^0}_0
&= \inv{2} (\grad \phi)^\dagger \cdot (\grad \phi) \\
&= \inv{2} \gpgradezero{ \gamma_0 \grad \gamma_0 \phi \grad \phi } \\
&= \inv{2} \gpgradezero{ (\gamma_0 \grad \phi)^2 } \\
&= \inv{2} \gpgradezero{ (\gamma_0 (\gamma^0 \partial_0 + \gamma^k \partial_k ) \phi )^2 } \\
&= \inv{2} \gpgradezero{ ((\partial_0 - \gamma^k \gamma_0 \partial_k ) \phi)^2 } \\
&= \inv{2} \gpgradezero{ ((\partial_0 + \spacegrad) \phi)^2 } \\
&= \inv{2} \left( \dotphi^2 + (\spacegrad \phi)^2 \right) .
\end{aligned}
\end{equation}
%
Now conjugation with respect to the time basis vector should not be special in any way, and should be equally justified defining a
conjugation operation along any of the spatial directions too.  Is there a symbol for this?  Let us write for now
%
\begin{equation}\label{eqn:stressEnergyNoethers:HermitianConjMu}
F^{\dagger_\mu} \equiv \gamma_\mu \tilde{F} \gamma^\mu.
\end{equation}
%
There is a possibility that the sign picked here is not appropriate for all purposes.
It is hard to tell for now since we have a vector \(F\) that equals its reverse, and in fact
after a computation with both \(\mu\) indices down I have raised an index altering
an
initial choice of \(F^{\dagger_\mu} = \gamma_\mu \tilde{F} \gamma_\mu\).

Applying this, for \(\mu \ne 0\) we have
%
\begin{equation}\label{eqn:stressEnergyNoethers:1360}
\begin{aligned}
(\grad \phi)^{\dagger_\mu} \cdot (\grad \phi)
&=
-\gpgradezero{ \gamma_\mu \grad \gamma_\mu \phi \grad \phi } \\
&=
-\gpgradezero{ ((\partial_\mu + \gamma_\mu \sum_{\nu \ne \mu} \gamma^\nu \partial_\nu) \phi )^2 } \\
&=
-((\partial_\mu \phi)^2 + \sum_{\nu \ne \mu} (\gamma_\mu \gamma^\nu)^2 (\partial_\nu \phi )^2) \\
&=
-((\partial_\mu \phi)^2 - \sum_{\nu \ne \mu} (\gamma_\mu)^2 (\gamma^\nu)^2 (\partial_\nu \phi )^2) \\
&=
-((\partial_\mu \phi)^2 + \sum_{\nu \ne \mu} (\gamma^\nu)^2 (\partial_\nu \phi )^2) \\
&=
-(\partial_\mu \phi)^2 - (\partial_0 \phi )^2 + \sum_{k \ne \mu, k \ne 0} (\partial_k \phi )^2 .
\end{aligned}
\end{equation}
%
This recovers the diagonal terms, and allows us to write (no sum)
%
\begin{equation}\label{eqn:stressEnergyNoethers:1380}
{T^\mu}_\mu = \inv{2} (\grad \phi)^{\dagger_\mu} \cdot (\grad \phi).
\end{equation}
%
\subsubsection{As a projection?}
%
As a vector (a projection of \(T(\gamma_\mu)\) onto the \(\gamma_\mu\) direction) this is (again no sum)
%
\begin{equation}\label{eqn:stressEnergyNoethers:1400}
\begin{aligned}
\gamma_\mu {T^\mu}_\mu
&= \inv{2} \gamma_\mu (\grad \phi)^{\dagger_\mu} \cdot (\grad \phi) \\
&= \inv{2} \gamma_\mu \gpgradezero{ \gamma^\mu \grad \phi \gamma_\mu \grad \phi } \\
&= \inv{4} \gamma_\mu ( \gamma^\mu \grad \phi \gamma_\mu \grad \phi + \grad \phi \gamma_\mu \grad \phi \gamma^\mu ) \\
&= \inv{4} ( (\grad \phi \gamma_\mu \grad \phi) + \gamma_\mu (\grad \phi \gamma_\mu \grad \phi) \gamma^\mu ).
\end{aligned}
\end{equation}
%
Intuition says this may have a use when assembling a complete vector representation of \(T(\gamma_\mu)\) in
terms of the gradient, but what that is now is not clear.
%Is the grade selection here required?  Does this reduce to \((\grad \phi \gamma_\mu \grad \phi)/2 \) ?
%
\subsection{Momentum.}
%
Now, let us look at the four vector \(T(\gamma_0) = \gamma_\mu{T^\mu}_0\) more carefully.
We have seen the energy term of this, but have not looked at the spatial part (momentum).

We can calculate the spatial component by wedging with the observer unit velocity \(\gamma_0\), and get
%
\begin{equation}\label{eqn:stressEnergyNoethers:1420}
\begin{aligned}
T(\gamma_0) \wedge \gamma_0
&= \gamma_k \gamma_0 {T^k}_0 \\
&= -\sigma_k \dotphi \phi_k \\
&= - \dotphi \spacegrad \phi .
\end{aligned}
\end{equation}
%
Right away we have something interesting!  The wave momentum is related to the gradient operator, exactly as
we have in quantum physics, despite the fact that we are only looking at the classical wave equation (for
light or some other massless field effect).
%
\section{Wave equation.  GA form for the energy momentum tensor.}
%
Some of the playing around above was attempting to find more structure for the terms of the energy momentum tensor.  For
the diagonal terms this was done successfully.  However, doing so for the remainder is harder when working backwards
from the tensor in coordinate form.
%
\subsection{Calculate GA form.}
%
Let us step back to the defining relation \eqnref{eqn:stressEnergyNoethers:vCurrentmanyField}, from which we
see that we wish to calculate
%
\begin{equation}\label{eqn:stressEnergyNoethers:1440}
\begin{aligned}
\gamma_\mu \PD{\partial_\mu \phi^\alpha}{\LL}
&=
\gamma_\mu \partial^\mu \phi \\
&=
\gamma^\mu \partial_\mu \phi \\
&=
\grad \phi.
\end{aligned}
\end{equation}
%
This completely removes the indices from the tensor, leaving us with
%
\begin{equation}\label{eqn:stressEnergyNoethers:1460}
\begin{aligned}
T(a)
&= (\grad \phi) a \cdot \grad \phi - \frac{a}{2} (\grad \phi)^2 \\
&= (\grad \phi) \left( \inv{2}(a \grad \phi + \grad \phi a) - \grad \phi \frac{a}{2} \right) .
\end{aligned}
\end{equation}
%
Thus we have
%
\begin{equation}\label{eqn:stressEnergyNoethers:GAwaveStressEnergy}
\begin{aligned}
T(a) &= \inv{2} (\grad \phi) a (\grad \phi).
\end{aligned}
\end{equation}
%
This meets the intuitive expectation that the energy momentum tensor for the wave equation could be expressed
completely in terms of the gradient.
%
\subsection{Verify against tensor expression.}
%
There is in fact a surprising simplicity to the result of \eqnref{eqn:stressEnergyNoethers:GAwaveStressEnergy}.
It is somewhat hard to believe that it summarizes the messy matrix we have calculated above.
To verify this let us derive the tensor relation of \eqnref{eqn:stressEnergyNoethers:tensorInIndexForm}.
%
\begin{equation}\label{eqn:stressEnergyNoethers:1480}
\begin{aligned}
{T^\mu}_\nu
&= T(\gamma_\nu) \cdot \gamma^\mu \\
&= \inv{2} \gpgradezero{ (\grad \phi) \gamma_\nu (\grad \phi) \gamma^\mu } \\
&= \inv{2} \gpgradezero{ \gamma^\alpha \partial_\alpha \phi \gamma_\nu \gamma_\beta \partial^\beta \phi \gamma^\mu } \\
&= \inv{2}
\partial_\alpha \phi \partial^\beta \phi
\gpgradezero{ \gamma^\alpha \gamma_\nu \gamma_\beta \gamma^\mu } \\
&= \inv{2}
\partial_\alpha \phi \partial^\beta \phi
\left(
{\delta^\alpha}_\nu {\delta_\beta}^\mu
+
(\gamma^\alpha \wedge \gamma_\nu) \cdot (\gamma_\beta \wedge \gamma^\mu )
\right) \\
&= \inv{2} \left(
\partial_\nu \phi \partial^\mu \phi
+
\partial^\alpha \phi \partial_\beta \phi (\gamma_\alpha \wedge \gamma_\nu) \cdot (\gamma^\beta \wedge \gamma^\mu )
\right) \\
&= \inv{2} \left(
\partial_\nu \phi \partial^\mu \phi
+
(\partial^\alpha \phi \partial_\beta \phi) \gamma_\alpha \cdot (
\mathLabelBox{\gamma_\nu \cdot (\gamma^\beta \wedge \gamma^\mu )}{\(={\delta_\nu}^\beta \gamma^\mu -{\delta_\nu}^\mu \gamma^\beta \)}
)
\right) \\
&= \inv{2} \left(
\partial_\nu \phi \partial^\mu \phi
+
(\partial^\alpha \phi \partial_\beta \phi) ( {\delta_\nu}^\beta {\delta_\alpha}^\mu - {\delta_\nu}^\mu {\delta_\alpha}^\beta )
\right) \\
&= \inv{2} \left( \partial_\nu \phi \partial^\mu \phi + \partial^\mu \phi \partial_\nu \phi - {\delta_\nu}^\mu (\partial^\alpha \phi \partial_\alpha \phi) \right) \\
&= \partial_\nu \phi \partial^\mu \phi - {\delta_\nu}^\mu \LL
\qedmarker.
\end{aligned}
\end{equation}
%
\subsection{Invariant length.}
%
Putting the energy momentum tensor in GA form makes the demonstration of the invariant length almost trivial.  We have for any \(a\)
%
\begin{equation}\label{eqn:stressEnergyNoethers:1500}
\begin{aligned}
(T(a))^2
&= \inv{4} \grad \phi a \grad \phi \grad \phi a \grad \phi \\
&= \inv{4} (\grad \phi)^2 \grad \phi a^2 \grad \phi \\
&= \inv{4} (\grad \phi)^4 a^2 \\
&= \LL^2 a^2 .
\end{aligned}
\end{equation}
%
This recovers \eqnref{eqn:stressEnergyNoethers:invariantLength}, which came at considerably higher cost in terms of guesswork.
%
\subsection{Energy and Momentum split (again).}
%
By wedging with \(\gamma_0\) we can extract the momentum terms of \(T(\gamma_0)\).  That is
%
\begin{equation}\label{eqn:stressEnergyNoethers:1520}
\begin{aligned}
T(\gamma_0) \wedge \gamma_0
&=
\left( (\gamma_0 \cdot \grad \phi) \grad \phi - \inv{2} (\grad \phi)^2 \gamma_0 \right) \wedge \gamma_0 \\
&=
(\gamma_0 \cdot \grad \phi) (\grad \phi \wedge \gamma_0) - \inv{2} (\grad \phi)^2
\mathLabelBox{(\gamma_0 \wedge \gamma_0)}{\(=0\)}
\\
&=
\dotphi (\gamma^k \gamma_0 \partial_k \phi ) \\
&=
-\dotphi \spacegrad \phi.
\end{aligned}
\end{equation}
%
For the energy term, dotting with \(\gamma_0\) we have
%
\begin{equation}\label{eqn:stressEnergyNoethers:1540}
\begin{aligned}
T(\gamma_0) \cdot \gamma_0
&=
\left( (\gamma_0 \cdot \grad \phi) \grad \phi - \inv{2} (\grad \phi)^2 \gamma_0 \right) \cdot \gamma_0 \\
&=
(\gamma_0 \cdot \grad \phi)^2 - \inv{2} (\grad \phi)^2  \\
&=
\dotphi^2 - \inv{2}( \dotphi^2 - (\spacegrad \phi)^2 ) \\
&=
\inv{2} ( \dotphi^2 + (\spacegrad \phi)^2 ) .
\end{aligned}
\end{equation}
%
Wedging with \(\gamma_0\) itself does not provide us with a relative spatial vector.  For example, consider the proper time velocity four vector
(still working with \(c=1\))
%
\begin{equation}\label{eqn:stressEnergyNoethers:1560}
\begin{aligned}
v
&= \frac{dt}{d\tau}\frac{d}{dt}\left( t \gamma_0 + \gamma_k x^k \right) \\
&= \frac{dt}{d\tau}\left( \gamma_0 + \gamma_k \frac{dx^k}{dt} \right) .
\end{aligned}
\end{equation}
%
We have
\begin{equation}\label{eqn:stressEnergyNoethers:1580}
\begin{aligned}
v \cdot \gamma_0 &= \frac{dt}{d\tau} = \gamma.
\end{aligned}
\end{equation}
%
and
\begin{equation}\label{eqn:stressEnergyNoethers:1600}
\begin{aligned}
v \wedge \gamma_0 &= \frac{dt}{d\tau} \sigma_k \frac{dx^k}{dt}.
\end{aligned}
\end{equation}
%
Or
\begin{equation}\label{eqn:stressEnergyNoethers:1620}
\begin{aligned}
\Bv
&\equiv \sigma_k \frac{dx^k}{dt} \\
&= \frac{v \wedge \gamma_0}{v \cdot \gamma_0}.
\end{aligned}
\end{equation}
%
This suggests that the form for the relative momentum (spatial) vector for the field should therefore be
%
\begin{equation}\label{eqn:stressEnergyNoethers:1640}
\begin{aligned}
\Bp
&\equiv \frac{T(\gamma_0) \wedge \gamma_0}{T(\gamma_0) \cdot \gamma_0} \\
&=
-\frac{\dotphi}{\inv{2}(\dotphi^2 + (\spacegrad \phi)^2)} \spacegrad \phi \\
&=
-\frac{2}{1 + \left(\frac{\spacegrad \phi}{\dotphi}\right)^2} \frac{\spacegrad \phi}{\dotphi} \\
&=
-\frac{2}{\frac{\dotphi}{\spacegrad \phi} + \frac{\spacegrad \phi}{\dotphi}} .
\end{aligned}
\end{equation}
%
This has been written in a few different ways, looking for something familiar, and not really finding it.
It would be useful to
revisit this after considering in detail wave momentum in a mechanical sense, perhaps with a limiting argument as given
in \citep{goldstein1951cm} (ie: one dimensional Lagrangian density considering infinite sequence of springs in a line).
%
\section{Scalar Klein Gordon.}
%
A number of details have been extracted considering the scalar wave equation.  Now lets move to a two field variable Lagrangian.
%
\begin{equation}\label{eqn:stressEnergyNoethers:1660}
\LL = \inv{2} \partial_\mu \psi \partial^\mu \psi - \frac{m^2 c^2}{2 \Hbar^2} \psi^2.
\end{equation}
%
This forced wave equation will have almost the same energy momentum tensor.  The exception will be the diagonal terms
for which we have an additional factor of \(m^2 c^2 \psi^2/ 2\Hbar^2\).

This also means that the conservation equations will be altered slightly
%
\begin{equation}\label{eqn:stressEnergyNoethers:1680}
\begin{aligned}
0 &= \partial_\mu {T^\mu}_\nu \\
&= \phi_\nu \left( \grad^2 \phi + \frac{m^2 c^2}{\Hbar^2} \phi \right).
\end{aligned}
\end{equation}
%
Again the divergence of the individual canonical energy momentum tensor four vectors reproduces the field equations
that we also obtain from the variation.
%
\section{Complex Klein Gordon.}
%
\subsection{Tensor in GA form.}
%
\begin{equation}\label{eqn:stressEnergyNoethers:1700}
\LL = \partial_\mu \psi \partial^\mu \psi^\conj - \frac{m^2 c^2}{\Hbar^2} \psi \psi^\conj.
\end{equation}
%
We first want to calculate what perhaps could be called the field velocity gradient
%
\begin{equation}\label{eqn:stressEnergyNoethers:1720}
\begin{aligned}
\gamma_\mu \PD{(\partial_\mu \psi)}{\LL}
&=
\gamma_\mu \partial^\mu \psi \\
&=
\grad \psi .
\end{aligned}
\end{equation}
%
Similarly
\begin{equation}\label{eqn:stressEnergyNoethers:1740}
\begin{aligned}
\gamma_\mu \PD{(\partial_\mu \psi^\conj)}{\LL}
&=
\gamma_\mu \partial^\mu \psi^\conj \\
&=
\grad \psi^\conj .
\end{aligned}
\end{equation}
%
Assembling results into an application of \eqnref{eqn:stressEnergyNoethers:vCurrentmanyField}, we have
%
\begin{equation}\label{eqn:stressEnergyNoethers:1760}
\begin{aligned}
T(a)
&= \grad \psi (a \cdot \grad) \psi^\conj +\grad \psi^\conj (a \cdot \grad) \psi - a \LL  \\
&= \grad \psi (a \cdot \grad) \psi^\conj +\grad \psi^\conj (a \cdot \grad) \psi - a \inv{2}(\grad \psi \grad \psi^\conj + \grad \psi^\conj \grad \psi )
+ a \frac{m^2 c^2}{\Hbar^2} \psi \psi^\conj \\
&=
\grad \psi ( a \cdot \grad \psi^\conj - \inv{2} \grad \psi^\conj a)
+\grad \psi^\conj ( a \cdot \grad \psi - \inv{2} \grad \psi a)
+ a \frac{m^2 c^2}{\Hbar^2} \psi \psi^\conj \\
&=
\inv{2} \left(
(\grad \psi ) a (\grad \psi^\conj)
+(\grad \psi^\conj ) a (\grad \psi) \right)
+ a \frac{m^2 c^2}{\Hbar^2} \psi \psi^\conj .
\end{aligned}
\end{equation}
%
Since vectors equal their own reverse this is just
%
\begin{equation}\label{eqn:stressEnergyNoethers:1780}
\begin{aligned}
T(a) &= (\grad \psi ) a (\grad \psi^\conj) + a \frac{m^2 c^2}{\Hbar^2} \psi \psi^\conj.
\end{aligned}
\end{equation}
%
\subsection{Tensor in index form.}
%
Expanding the energy momentum tensor in index notation we have
%
\begin{equation}\label{eqn:stressEnergyNoethers:1800}
\begin{aligned}
{T^\mu}_\nu &= T(\gamma_\nu) \cdot \gamma^\mu \\
&=
\partial_\alpha \psi \partial_\beta \psi^\conj
\gpgradezero{ \gamma^\alpha \gamma_\nu \gamma^\beta \gamma^\mu } + {\delta_\nu}^\mu \frac{m^2 c^2}{\Hbar^2} \psi \psi^\conj \\
&=
\partial_\nu \psi \partial^\mu \psi^\conj
+\partial^\mu \psi \partial_\nu \psi^\conj
-\partial^\alpha \psi \partial_\alpha \psi^\conj {\delta_\nu}^\mu
+ {\delta_\nu}^\mu \frac{m^2 c^2}{\Hbar^2} \psi \psi^\conj .
\end{aligned}
\end{equation}
%
So we have
\begin{equation}\label{eqn:stressEnergyNoethers:1820}
\begin{aligned}
{T^\mu}_\nu &= \partial^\mu \psi \partial_\nu \psi^\conj + \partial^\mu \psi^\conj \partial_\nu \psi - {\delta_\nu}^\mu \LL.
\end{aligned}
\end{equation}
%
This index representation also has a nice compact elegance.
%
\subsection{Invariant Length?}
%
Writing for short \(b = \grad \psi\), and working in natural units \(m^2 c^2 = \Hbar^2\), we have
%
\begin{equation}\label{eqn:stressEnergyNoethers:1840}
\begin{aligned}
(T(a))^2
&=
(b a b^\conj + a \psi \psi^\conj)^2 \\
&=
\gpgradezero{ a b^\conj b a b^\conj b } + a^2 \psi^2 (\psi^\conj)^2 + 2 a \cdot (b a b^\conj) .
\end{aligned}
\end{equation}
%
Unlike the light wave equation this does not (obviously) appear to have a natural split into something times \(a^2\).  Is there a way to do it?
%
\subsection{Divergence relation.}
%
Borrowing notation from above to calculate the divergence we want
%
\begin{equation}\label{eqn:stressEnergyNoethers:1860}
\begin{aligned}
\grad \cdot (b a b^\conj)
&=
\gpgradezero{ \grad (b a b^\conj) } \\
&=
\gpgradezero{ (b^\conj \lrgrad b) a } \\
&=
a \cdot \gpgradeone{ b^\conj \lrgrad b } .
\end{aligned}
\end{equation}
%
Here cyclic reordering of factors within the scalar product was used.  In order for that to be a meaningful operation the gradient must
be allowed to operate bidirectionally, so this is really just shorthand for
%
\begin{equation}\label{eqn:stressEnergyNoethers:1880}
b^\conj \lrgrad b
\equiv
\dot{b}^\conj \dot{\grad} b
+{b}^\conj \dot{\grad} \dot{b},
\end{equation}
%
where the more conventional overdot notation is used to indicate the scope of the operation.
In particular, for \(b = \grad \psi\), we have
%
\begin{equation}\label{eqn:stressEnergyNoethers:1900}
\begin{aligned}
\gpgradeone{ b^\conj \lrgrad b }
&=
(\grad^2 \psi^\conj) (\grad \psi) + (\grad \psi^\conj) (\grad^2 \psi).
\end{aligned}
\end{equation}
%
Our tensor also has a vector scalar product that we need the divergence of.  That is
%
\begin{equation}\label{eqn:stressEnergyNoethers:1920}
\begin{aligned}
\grad \cdot (a \psi \psi^\conj)
&=
\gpgradezero{ \grad (a \psi \psi^\conj) } \\
&=
a \cdot \grad (\psi \psi^\conj) .
\end{aligned}
\end{equation}
%
Putting things back together we have
\begin{equation}\label{eqn:stressEnergyNoethers:1940}
\begin{aligned}
\grad \cdot T(a)
&= a \cdot \left(
\gpgradeone{ (\grad \psi^\conj) \lrgrad (\grad \psi) } + \frac{m^2 c^2}{\Hbar^2} \grad (\psi \psi^\conj)
\right).
\end{aligned}
\end{equation}
%
This is
\begin{equation}\label{eqn:stressEnergyNoethers:1960}
0 = \grad \cdot T(a) = a \cdot \left(
(\grad^2 \psi^\conj) (\grad \psi) + (\grad \psi^\conj) (\grad^2 \psi) + \frac{m^2 c^2}{\Hbar^2} \grad (\psi \psi^\conj)
\right).
\end{equation}
%
Again, we see that the divergence of the canonical energy momentum tensor produces the field equations that we get by direct variation!  Put explicitly
we have zero for all displacements \(a\), so must also have
%
\begin{equation}\label{eqn:stressEnergyNoethers:adjointEqualsZero}
0 =
(\grad \psi) \left(\grad^2 \psi^\conj + \frac{m^2 c^2}{\Hbar^2} \psi^\conj \right)
+ (\grad \psi^\conj) \left(\grad^2 \psi + \frac{m^2 c^2}{\Hbar^2} \psi \right).
\end{equation}
%
Also noteworthy above is the adjoint relationship.  The adjoint \(\overbar{F}\) of a an operator \(F\) was defined via the dot product
%
\begin{equation}\label{eqn:stressEnergyNoethers:1980}
a \cdot F(b) \equiv b \cdot \overbar{F}(a).
\end{equation}
%
So we have a concrete example of the adjoint applied to the gradient, and for this energy momentum tensor we have
%
\begin{equation}\label{eqn:stressEnergyNoethers:2000}
\begin{aligned}
\overbar{T}(\grad) &= \gpgradeone{ (\grad \psi^\conj) \grad (\grad \psi) } + \frac{m^2 c^2}{\Hbar^2} \grad (\psi \psi^\conj).
\end{aligned}
\end{equation}
%
Here the arrows notation has been dropped, where it is implied that this derivative acts on all neighboring vectors either unidirectionally or bidirectionally
as appropriate.

Now, this adjoint tensor is a curious beastie.  Intuition says this one will have a Lorentz invariant length.  A moment of reflection
shows that this is in fact the case since the adjoint was fully expanded in \eqnref{eqn:stressEnergyNoethers:adjointEqualsZero}.  That vector is zero, and the
length is therefore also necessarily invariant.
\subsection{TODO.}
How about the energy and momentum split in this adjoint form?  Could also write out adjoint in index
notation for comparison to non-adjoint tensor in index form.
\section{Electrostatics Poisson Equation.}
\subsection{Lagrangian and spatial Noether current.}
\begin{equation}\label{eqn:stressEnergyNoethers:2020}
\LL = -\frac{\epsilon_0}{2} (\spacegrad \phi)^2 + \rho \phi.
\end{equation}
Evaluating this yields the desired \(\spacegrad^2 \phi = -\rho/\epsilon_0\), or \(\spacegrad \cdot \BE = \rho/\epsilon_0\).
\subsection{Energy momentum tensor.}
In this particular case we then have
\begin{equation}\label{eqn:stressEnergyNoethers:2040}
\begin{aligned}
T(\Ba)
&= \sigma_k (-\epsilon_0 \partial_k \phi) \Ba \cdot \spacegrad \phi - \Ba \LL \\
&= -\epsilon_0 (\spacegrad \phi) \Ba \cdot \spacegrad \phi - \Ba (-\frac{\epsilon_0}{2} (\spacegrad \phi)^2 + \rho \phi) \\
&= -\epsilon_0 (\spacegrad \phi) \Ba \cdot \spacegrad \phi + (\spacegrad \phi)^2 \Ba \frac{\epsilon_0}{2} - \Ba \rho \phi \\
&= \frac{\epsilon_0}{2} (\spacegrad \phi) \left( -2 \Ba \cdot \spacegrad \phi + \spacegrad \phi \Ba \right) - \Ba \rho \phi \\
&= \frac{\epsilon_0}{2} (\spacegrad \phi) \left( -\Ba \spacegrad \phi - \spacegrad \phi \Ba + \spacegrad \phi \Ba \right) - \Ba \rho \phi \\
&= -\frac{\epsilon_0}{2} (\spacegrad \phi) \Ba \spacegrad \phi - \Ba \rho \phi .
\end{aligned}
\end{equation}
%
It in terms of \(\BE = -\spacegrad \phi\) this is
%
\begin{equation}\label{eqn:stressEnergyNoethers:electrostaticTensor}
\begin{aligned}
T(\Ba) &= -\frac{\epsilon_0}{2} \BE \Ba \BE - \Ba \rho \phi.
\end{aligned}
\end{equation}
%
This is not immediately recognizable (at least to me), and also does not appear to be easily separable into something times \(\Ba\).
%
\subsection{Divergence and adjoint tensor.}
%
What will we get with the divergence calculation?
%
\begin{equation}\label{eqn:stressEnergyNoethers:2060}
\begin{aligned}
\spacegrad \cdot (\BE \Ba \BE)
&=
\gpgradezero{ \spacegrad (\BE \Ba \BE) } \\
&=
\Ba \cdot \gpgradeone{ \BE \lrspacegrad \BE }.
\end{aligned}
\end{equation}
%
Also want
\begin{equation}\label{eqn:stressEnergyNoethers:2080}
\begin{aligned}
\spacegrad \cdot (\Ba \rho \phi)
&=
\gpgradezero{ \spacegrad (\Ba \rho \phi)} \\
&=
\Ba \cdot \spacegrad (\rho\phi).
\end{aligned}
\end{equation}
%
Assembling these we have
\begin{equation}\label{eqn:stressEnergyNoethers:2100}
\begin{aligned}
\spacegrad \cdot T(\Ba) &=
-\Ba \cdot \left( \gpgradeone{ \frac{\epsilon_0}{2} \BE \lrspacegrad \BE } + \spacegrad (\rho\phi) \right)
.
\end{aligned}
\end{equation}
%
From this we can pick off the adjoint
\begin{equation}\label{eqn:stressEnergyNoethers:2120}
\begin{aligned}
\overbar{T}(\spacegrad)
&=
-\frac{\epsilon_0}{2} \gpgradeone{ \BE \lrspacegrad \BE } - \spacegrad (\rho\phi)  \\
&=
-\frac{\epsilon_0}{2}\left(
(\dot{\BE} \cdot \dot{\spacegrad}) \BE
\BE (\spacegrad \cdot \BE)
\right)
- \spacegrad (\rho\phi)  \\
&=
-\epsilon_0 (\spacegrad^2 \phi) \spacegrad \phi - \spacegrad (\rho\phi)  \\
&=
-\epsilon_0 \spacegrad (\spacegrad \phi)^2 - \spacegrad (\rho\phi)  \\
&=
\spacegrad \left( -\epsilon_0 (\spacegrad \phi)^2 - \rho\phi \right)  .
\end{aligned}
\end{equation}
%
If we write \(\LL = K - V\), then we have in this case
%
\begin{equation}\label{eqn:stressEnergyNoethers:2140}
\overbar{T}(\spacegrad) = \spacegrad (K + V) = 0.
\end{equation}
%
Since the gradient of this quantity is zero everywhere it must be constant
%
\begin{equation}\label{eqn:stressEnergyNoethers:electrostaticConst}
K + V = \text{constant}.
\end{equation}
%
We did not have any time dependence in the Lagrangian, and blindly
following the math to calculate the
associated symmetry with the field translation, we end up with a
conservation statement that appears to be about energy.

TODO: am used to (as in \citep{feynman1963flp})
seeing electrostatic energy written
%
\begin{equation}\label{eqn:stressEnergyNoethers:2160}
U = \inv{2} \epsilon_0 \int \BE^2 dV = \inv{2} \int \rho \phi dV.
\end{equation}
%
Reconcile this with \eqnref{eqn:stressEnergyNoethers:electrostaticConst}.
%
%\section{Magnetostatics Equation}
%
%FIXME: What is the Lagrangian for this case?  work it through.
%
\section{Schr\"{o.}dinger equation}
%
While not a Lorentz invariant Lagrangian, we do not have a dependence on that,
and can still calculate a Noether current on spatial translation.
%
\begin{equation}\label{eqn:stressEnergyNoethers:2180}
\LL = \frac{\Hbar^2}{2m}
(\spacegrad \psi) \cdot (\spacegrad \psi^\conj) + V \psi \psi^\conj + {i \Hbar} \left( \psi \partial_t \psi^\conj - \psi^\conj \partial_t \psi \right).
\end{equation}
%
For this Lagrangian density it is worth noting that the action is in fact
%
\begin{equation}\label{eqn:stressEnergyNoethers:2200}
S = \int d^3 x \LL.
\end{equation}
%
... ie: \(\partial_t \psi\) is not a field variable in the variation (this is why there is no factor of \(1/2\) in the probability current term).
%
Calculating the Noether current for a vector translation \(\Ba\) we have
%
\begin{equation}\label{eqn:stressEnergyNoethers:2220}
\begin{aligned}
T(\Ba)
&=
\frac{\Hbar^2}{2m} \spacegrad \psi \Ba \cdot \spacegrad \psi^\conj
+\frac{\Hbar^2}{2m} \spacegrad \psi^\conj \Ba \cdot \spacegrad \psi
- \Ba \LL.
\end{aligned}
\end{equation}
%
Expanding the divergence is messy but straightforward
%
\begin{equation}\label{eqn:stressEnergyNoethers:2240}
\begin{aligned}
&\spacegrad \cdot T(\Ba) \\
&=
\frac{\Hbar^2}{2m}
\gpgradezero{
\spacegrad \left( \spacegrad \psi \Ba \cdot \spacegrad \psi^\conj + \spacegrad \psi^\conj \Ba \cdot \spacegrad \psi \right)
- \spacegrad(\spacegrad \psi \cdot \spacegrad \psi^\conj) \Ba
} \\
   &\quad- \Ba \cdot \spacegrad \left( V \psi \psi^\conj + i\Hbar (\psi \dotpsi^\conj - \psi^\conj \dotpsi) \right) \\
&=
\frac{\Hbar^2}{4m}
\gpgradezero{
   \spacegrad \left( \spacegrad \psi (\Ba \spacegrad \psi^\conj + \spacegrad \psi^\conj \Ba) + \spacegrad \psi^\conj
   (\Ba \spacegrad \psi
   +\spacegrad \psi \Ba)
   \right)
   - 2 \spacegrad(\spacegrad \psi \cdot \spacegrad \psi^\conj) \Ba
} \\
   &\quad- \Ba \cdot \spacegrad \left( V \psi \psi^\conj + i\Hbar (\psi \dotpsi^\conj - \psi^\conj \dotpsi) \right) \\
&=
\frac{\Hbar^2}{4m}
\Ba \cdot
\gpgradeone{
     \spacegrad \psi^\conj \lrspacegrad \spacegrad \psi
   + \spacegrad \psi \lrspacegrad \spacegrad \psi^\conj
} \\
&\quad+
\frac{\Hbar^2}{4m}
\Ba \cdot
\gpgradeone{
     \spacegrad ( \spacegrad \psi \spacegrad \psi^\conj )
   + \spacegrad ( \spacegrad \psi^\conj \spacegrad \psi )
   -2 \spacegrad(\spacegrad \psi \cdot \spacegrad \psi^\conj)
} \\
&\quad- \Ba \cdot \spacegrad \left( V \psi \psi^\conj + i\Hbar (\psi \dotpsi^\conj - \psi^\conj \dotpsi) \right) \\
&=
\frac{\Hbar^2}{4m}
\Ba \cdot
\gpgradeone{
  \spacegrad \psi^\conj \lrspacegrad \spacegrad \psi
+ \spacegrad \psi \lrspacegrad \spacegrad \psi^\conj
} \\
&\quad- \Ba \cdot \spacegrad \left( V \psi \psi^\conj + i\Hbar (\psi \dotpsi^\conj - \psi^\conj \dotpsi) \right) \\
&=
\frac{\Hbar^2}{4m}
\Ba \cdot 2 \left( \spacegrad \psi^\conj \spacegrad^2 \psi + \spacegrad \psi \spacegrad^2 \psi^\conj \right)
- \Ba \cdot \spacegrad \left( V \psi \psi^\conj + i\Hbar (\psi \dotpsi^\conj - \psi^\conj \dotpsi) \right) \\
&=
\frac{\Hbar^2}{2m}
\Ba \cdot \spacegrad \left( \spacegrad \psi^\conj \cdot \spacegrad \psi \right)
- \Ba \cdot \spacegrad \left( V \psi \psi^\conj + i\Hbar (\psi \dotpsi^\conj - \psi^\conj \dotpsi) \right) .
\end{aligned}
\end{equation}
%
Which is, finally,
%
\begin{equation}\label{eqn:stressEnergyNoethers:2260}
\spacegrad \cdot T(\Ba)
=
\Ba \cdot \spacegrad
\left(
\frac{\Hbar^2}{2m}
\spacegrad \psi^\conj \cdot \spacegrad \psi
- V \psi \psi^\conj - i\Hbar (\psi \dotpsi^\conj - \psi^\conj \dotpsi)
\right).
\end{equation}
%
Picking off the adjoint we have
%
\begin{equation}\label{eqn:stressEnergyNoethers:2280}
\overbar{T}(\spacegrad)
=
\frac{\Hbar^2}{2m}
\spacegrad \psi^\conj \cdot \spacegrad \psi
- V \psi \psi^\conj - i\Hbar (\psi \dotpsi^\conj - \psi^\conj \dotpsi).
\end{equation}
%
Just like the electrostatics equation, it appears that we can make an association with Kinetic (\(K\)) and Potential (\(\phi\)) energies with the adjoint
stress tensor.
%
\begin{equation}\label{eqn:stressEnergyNoethers:2300}
\begin{aligned}
K &= \frac{\Hbar^2}{2m} \spacegrad \psi^\conj \cdot \spacegrad \psi \\
\phi &= V \psi \psi^\conj + i\Hbar (\psi \dotpsi^\conj - \psi^\conj \dotpsi) \\
\LL &= K - \phi \\
\overbar{T}(\spacegrad) &= K + \phi.
\end{aligned}
\end{equation}
%
FIXME: Unlike the electrostatics case however, there is no
conserved scalar quantity that is obvious.
The association in this case with energy is by analogy, not
connected to anything reasonably physical seeming.  How to connect this
with actual physical concepts?
Can this be written as the
gradient of something?  Because of the time derivatives perhaps the space
time gradient would be required, however, because of the non-Lorentz
invariant nature I had expect that terms may have to be added or subtracted
to make that possible.
%
\section{Maxwell equation.}
%
Wanting to see some of the connections between the Maxwell equation and the Lorentz force was the
original reason for examining this canonical energy momentum tensor concept in detail.
%
\subsection{Lagrangian.}
%
Recall that the Lagrangian for the vector grades of Maxwell's equation
%
\begin{equation}\label{eqn:stressEnergyNoethers:maxwell}
\grad F = J/\epsilon_0 c.
\end{equation}
%
is of the form
%
\begin{equation}\label{eqn:stressEnergyNoethers:2320}
\begin{aligned}
\LL
&= \kappa (\grad \wedge A) \cdot (\grad \wedge A) + J \cdot A \\
&= \kappa (\gamma^\mu \wedge \gamma^\nu) \cdot (\gamma_\alpha \wedge \gamma_\beta) \partial_\mu A_\nu \partial^\alpha A^\beta + J^\sigma A_\sigma .
\end{aligned}
\end{equation}
%
We can fix the constant \(\kappa\) by taking variational derivatives and comparing with \eqnref{eqn:stressEnergyNoethers:maxwell}
%
\begin{equation}\label{eqn:stressEnergyNoethers:2340}
\begin{aligned}
0
&= \PD{A_\sigma}{\LL} - \partial_\mu \PD{(\partial_\mu A_\sigma)}{\LL} \\
&= J^\sigma - 2 \kappa (\gamma^\mu \wedge \gamma^\sigma) \cdot (\gamma_\alpha \wedge \gamma_\beta) \partial_\mu \partial^\alpha A^\beta  .
\end{aligned}
\end{equation}
%
Taking \(\gamma^\sigma\) dot products with \eqnref{eqn:stressEnergyNoethers:maxwell} we have
%
\begin{equation}\label{eqn:stressEnergyNoethers:2360}
\begin{aligned}
0
&= \gamma^\sigma \cdot (J - \epsilon_0 c \grad \cdot F ) \\
&= J^\sigma - \epsilon_0 c \gamma^\sigma \cdot (\gamma^\mu \cdot (\gamma_\alpha \wedge \gamma_\beta)) \partial_\mu \partial^\alpha A^\beta .
\end{aligned}
\end{equation}
%
So we have \(2\kappa = -\epsilon_0 c\), and can write our Lagrangian density as
\begin{equation}\label{eqn:stressEnergyNoethers:2380}
\begin{aligned}
\LL
&= -\frac{\epsilon_0}{2} (\grad \wedge A) \cdot (\grad \wedge A) + \frac{J}{c} \cdot A \\
&= -\frac{\epsilon_0}{2} (\gamma^\mu \wedge \gamma^\nu) \cdot (\gamma_\alpha \wedge \gamma_\beta) \partial_\mu A_\nu \partial^\alpha A^\beta + \frac{J^\sigma}{c} A_\sigma.
\end{aligned}
\end{equation}
%
\subsection{Energy momentum tensor.}
%
For the Lagrangian density we have
%
\begin{equation}\label{eqn:stressEnergyNoethers:2400}
\begin{aligned}
\gamma_\mu \PD{(\partial_\mu A_\nu)}{\LL}
&= -{\epsilon_0} \gamma_\mu (\gamma^\mu \wedge \gamma^\nu) \cdot (\gamma_\alpha \wedge \gamma_\beta) \partial^\alpha A^\beta \\
&= -{\epsilon_0} \gamma_\mu ( {\delta^\mu}_\beta {\delta^\nu}_\alpha -{\delta^\mu}_\alpha {\delta^\nu}_\beta ) \partial^\alpha A^\beta \\
&=
-{\epsilon_0} \gamma_\mu (\partial^\nu A^\mu -\partial^\mu A^\nu) \\
&= {\epsilon_0} \gamma_\mu F^{\mu\nu} .
\end{aligned}
\end{equation}
%
One can guess that the vector contraction of \(F^{\mu\nu}\) above is an expression of a dot product with our bivector field.  This is in fact the case
%
\begin{equation}\label{eqn:stressEnergyNoethers:2420}
\begin{aligned}
F \cdot \gamma^\nu
&=
(\gamma_\alpha \wedge \gamma_\beta) \cdot \gamma^\nu \partial^\alpha A^\beta \\
&=
(\gamma_\alpha {\delta_\beta}^\nu -\gamma_\beta {\delta_\alpha}^\nu ) \partial^\alpha A^\beta
\\
&=
\gamma_\mu (\partial^\mu A^\nu - \partial^\nu A^\mu )
\\
&=
\gamma_\mu F^{\mu\nu}
.
\end{aligned}
\end{equation}
%
We therefore have
%
\begin{equation}\label{eqn:stressEnergyNoethers:2440}
\begin{aligned}
T(a)
&= {\epsilon_0} (F \cdot \gamma^\nu) a \cdot \grad A_\nu - a \LL \\
&= {\epsilon_0} \left( (F \cdot \gamma^\nu) a \cdot \grad A_\nu + \frac{a}{2} F \cdot F \right) - a \left( A \cdot J/c \right).
\end{aligned}
\end{equation}
%
\subsection{Index form of tensor.}
%
Before trying to factor out \(a\), let us expand the tensor in abstract index form.  This is
\begin{equation}\label{eqn:stressEnergyNoethers:2460}
\begin{aligned}
{T_\nu}^\mu
&=
T(\gamma_\nu) \cdot \gamma^\mu \\
&= {\epsilon_0} \left( F^{\mu\beta} \partial_\nu A_\beta + \frac{{\delta_\nu}^\mu}{2} F \cdot F \right) - {\delta_\nu}^\mu A^\sigma J_\sigma/c \\
&= {\epsilon_0} \left( F^{\mu\beta} \partial_\nu A_\beta - \frac{{\delta_\nu}^\mu}{4} F^{\alpha\beta}F_{\alpha\beta} \right) - {\delta_\nu}^\mu A^\sigma J_\sigma/c .
\end{aligned}
\end{equation}
%
In particular, note that this is not the familiar symmetric tensor from the Poynting relations.
%
\subsection{Expansion in terms of \texorpdfstring{\(\BE\) and \(\BB\).}{E and B}}
%
TODO.
%
\subsection{Adjoint.}
%
Now, we want to move on to a computation of the adjoint so that \(a\) can essentially be factored out.
Doing so is resisting initial attempts.  As an aid, introduce a few vector valued helper variables
%
\begin{equation}\label{eqn:stressEnergyNoethers:2480}
\begin{aligned}
F^\mu &= F \cdot \gamma^\mu \\
G_\mu &= \grad A_\nu .
\end{aligned}
\end{equation}
%
Then we have
%
\begin{equation}\label{eqn:stressEnergyNoethers:2500}
\begin{aligned}
\grad \cdot T(a)
&= \frac{\epsilon_0}{2} \left( \gpgradezero{ \grad(F^\nu (a G_\nu + G_\nu a)} + {a} \cdot \gpgradeone{\grad(F^2)} \right) - a \cdot \grad \left( A \cdot J/c \right) \\
&= \frac{\epsilon_0}{2} a \cdot
\gpgradeone{
G_\nu \lrgrad F^\nu
+\grad(F^\nu G_\nu)
+ \grad(F^2)} - a \cdot \grad \left( A \cdot J/c \right) .
\end{aligned}
\end{equation}
%
This provides the adjoint energy momentum tensor, albeit in a form that looks like it can be reduced further
%
\begin{equation}\label{eqn:stressEnergyNoethers:2520}
\begin{aligned}
0 &= \overbar{T}(\grad) = \frac{\epsilon_0}{2} \gpgradeone{ G_\nu \lrgrad F^\nu  +\grad(F^\nu G_\nu) + \grad(F^2)} - \grad \left( A \cdot J/c \right).
\end{aligned}
\end{equation}
%
We want to write this as a gradient of something, to determine the conserved quantity.  Getting part way is not too hard.
%
\begin{dmath}\label{eqn:stressEnergyNoethers:2540}
\overbar{T}(\grad)
= \frac{\epsilon_0}{2}
\mathLabelBox{
\left(
\gpgradeone{ G_\nu \lrgrad F^\nu } + \grad \cdot (F^\nu \wedge G_\nu)
\right)
}{\(\conj\)}
+ \grad \left( \frac{\epsilon_0}{2} (F^\nu \cdot G_\nu + F \cdot F) - A \cdot J/c \right).
\end{dmath}
%
It would be nice if these first two terms \(\conj\) cancel.
Can we be so lucky?
%
\begin{equation}\label{eqn:stressEnergyNoethers:2560}
\begin{aligned}
(*) &=
\gpgradeone{ G_\nu \lrgrad F^\nu } + \grad \cdot (F^\nu \wedge G_\nu) \\
&=
\gpgradeone{
(G_\nu \lgrad) F^\nu
+G_\nu (\rgrad F^\nu)
}
+ (\grad \cdot F^\nu) G_\nu
- F^\nu (\grad \cdot G_\nu)  \\
&=
(\grad \cdot G_\nu) F^\nu
+F^\nu \cdot (\grad \wedge G_\nu )
+G_\nu (\grad \cdot F^\nu) \\
&\quad+G_\nu \cdot (\grad \wedge F^\nu)
+ (\grad \cdot F^\nu) G_\nu
- F^\nu (\grad \cdot G_\nu)  \\
&=
2 G_\nu (\grad \cdot F^\nu) +G_\nu \cdot (\grad \wedge F^\nu) \\
%&=
% G_\nu (\grad \cdot F^\nu) + \gpgradeone{G_\nu (\grad F^\nu)} .
\end{aligned}
\end{equation}
%
This is not obviously zero.  How about \(F^\nu \cdot G_\nu\)?
%
\begin{equation}\label{eqn:stressEnergyNoethers:2580}
\begin{aligned}
F^\nu \cdot G_\nu
&=
\gpgradezero{((\gamma_\alpha \wedge \gamma_\beta) \cdot \gamma^\nu) \gamma^\sigma } \partial^\alpha A^\beta \partial_\sigma A_\nu \\
&=
({\delta_\alpha}^\sigma {\delta_\beta}^\nu - {\delta_\beta}^\sigma {\delta_\alpha}^\nu) \partial^\alpha A^\beta \partial_\sigma A_\nu \\
&=
\partial^\alpha A^\beta (\partial_\alpha A_\beta - \partial_\beta A_\alpha) \\
&=
\partial^\alpha A^\beta F_{\alpha\beta} \\
&=
\inv{2} F_{\alpha\beta} F^{\alpha\beta}.
\end{aligned}
\end{equation}
%
Ah.  Up to a sign, this was \(F \cdot F\).  What is the sign?
%
\begin{equation}\label{eqn:stressEnergyNoethers:2600}
\begin{aligned}
F \cdot F
&=
(\gamma_\alpha \wedge \gamma_\beta) \cdot (\gamma^\mu \wedge \gamma^\nu) \partial^\alpha A^\beta \partial_\mu A_\nu \\
&=
({\delta_\alpha}^\nu {\delta_\beta}^\mu - {\delta_\beta}^\nu {\delta_\alpha}^\mu) \partial^\alpha A^\beta \partial_\mu A_\nu \\
&=
\partial^\alpha A^\beta (\partial_\beta A_\alpha - \partial_\alpha A_\beta) \\
&=
\partial^\alpha A^\beta F_{\beta\alpha} \\
&=
\inv{2}
F_{\beta\alpha} ( \partial^\alpha A^\beta -\partial^\beta A^\alpha ) \\
&=
\inv{2} F_{\beta\alpha} F^{\alpha\beta} \\
&= - F^\nu \cdot G_\nu.
\end{aligned}
\end{equation}
%
Bad first guess.  It is the second two terms that cancel, not the first, leaving us with
%
\begin{equation}\label{eqn:stressEnergyNoethers:2620}
\begin{aligned}
\overbar{T}(\grad)
&= \frac{\epsilon_0}{2} \left(
\gpgradeone{ G_\nu \lrgrad F^\nu }
+ \grad \cdot (F^\nu \wedge G_\nu)
\right)
- \grad \left( A \cdot J/c \right).
\end{aligned}
\end{equation}
%
Now, intuition tells me that it ought to be possible to simplify this further,
in particular, eliminating the \(\nu\) indices.

Think I will take a break from this for a while, and come back to it later.
%
\section{Nomenclature.  Linearized spacetime translation.}
%
Applying the translation \(x^\mu \rightarrow x^\mu + e^\mu\), is what I thought
would be called ``spacetime translation''.  But to do so we need higher
order powers of the exponential vector translation operator (ie:
multivariable Taylor series operator)
%
\begin{equation}\label{eqn:stressEnergyNoethers:2640}
\sum_k (1/k!) (e^\mu \partial_\mu)^k.
\end{equation}
%
The transformation that appears to result in the canonical
energy momentum tensor has only the linear term of this operator, so I called
it ``linearized spacetime translation operator'', which seemed like a
better name (to me).  That is all.  My guess is that what is typically
referred to as the spacetime translation that generates the canonical
energy momentum tensor is really just the first order term of the
translation operation, and not truly a complete translation.  If
that is the case, then dropping the linearized adjective would probably
be reasonable.

It is somewhat odd that the derived conditions for a divergence added
to the Lagrangian are immediately busted by the wave equation.
I think the saving grace is the fact that
an arbitrary \(\partial_\mu F^\mu\) is not necessarily a symmetry is the
fact the translation of the coordinates is not an arbitrary divergence.
 This directional derivative operator is applied to the Lagrangian
itself and not to an arbitrary function.  This builds in the required
symmetry (you could also add in or subtract out additional divergence
terms that meet the derived conditions and not change anything).

Now, if the first order term of the Taylor expansion is a symmetry
because we can commute the field partials and the coordinate partials
then the higher order terms should also be symmetries.  This would mean
that a true translation \(\LL \rightarrow \exp(e^\mu \partial_\mu) \LL\)
would also be a symmetry.  What conservation current would we get from
that?  Would it be the symmetric energy momentum tensor?
%
%Yes, I had expect that the first order term of the Taylor expansion of
%the translation would be related to an infinitesimal operation, but
%have not thought that through in detail.  Additionally both Tong and
%D&L include inverse operators when they treat more general
%transformations than translations, and I have not figured out where
%those come from.
