%
% Copyright © 2012 Peeter Joot.  All Rights Reserved.
% Licenced as described in the file LICENSE under the root directory of this GIT repository.
%
\makeoproblem{Alternate Lagrangian.}{tongmf1:pr2}{\citep{TongMf1} p2.}{
%
%\begin{equation*}
%\LL = \inv{12}m^2 \xdot^4 - m \xdot^2 V + V^2.
%\end{equation*}
%
%\begin{align*}
%\frac{\partial L}{\partial x} &= \frac{d}{dt} \frac{\partial L}{\partial \xdot} \\
%-m \xdot^2 V_x + 2V V_x &= \frac{d}{dt} \left( \inv{3}m^2 \xdot^3 - 2 m \xdot V \right) \\
%                        &= m^2 \xdot^2 \xddot - 2m \xddot V \\
%-(m \xdot^2 -2V) V_x    &= m \xddot (m \xdot^2 - 2 V) \\
%\end{align*}
%
%Cancelling left and right common factors:
%
%\begin{equation*}
%m \xddot = -\frac{\partial V}{\partial x}.
%\end{equation*}
%
%THIS IS WRONG, since \(dV/dt \ne 0\).
%
%Although \(V(x)\) has no explicit time
%dependence on \(t\), \(x = x(t)\), so \(dV/dt = dV/dx dx/dt\).
%
\begin{equation}
\LL = \inv{12}m^2 \xdot^4 + m \xdot^2 V - V^2.
\end{equation}
}
%
\makeanswer{tongmf1:pr2}{
%
Digging in
%
\begin{equation}\label{eqn:tongMf1:100}
\begin{aligned}
\frac{\partial L}{\partial x} &= \frac{d}{dt} \frac{\partial L}{\partial \xdot} \\
m \xdot^2 V_x - 2V V_x &= \frac{d}{dt} \left( \inv{3}m^2 \xdot^3 + 2 m \xdot V \right) .
\end{aligned}
\end{equation}
%
When taking the time derivative of \(V\), \(dV/dt \ne 0\), despite no explicit time dependence.
Take an example, such as \(V = mgx\), where the positional parameter is dependent on time, so the chain rule is required:
%
\begin{equation}\label{eqn:tongMf1:1040}
\frac{d V}{dt} = \frac{d V}{dx} \frac{dx}{dt} = \xdot V_x.
\end{equation}
%
Perhaps that is obvious, but I made that mistake first doing this problem (which would have been harder to make if I had used an example potential) the first time.  I subsequently constructed an alternate Lagrangian
\(\lr{ \LL = \inv{12}m^2 \xdot^4 - m \xdot^2 V + V^2 }\)
that worked when this mistake was made, and emailed the author suggesting that I believed he had a sign typo in his problem set.
%
Anyways, continuing with the calculation:
%
\begin{equation}\label{eqn:tongMf1:120}
\begin{aligned}
m \xdot^2 V_x - 2V V_x &= m^2 \xdot^2 \xddot + 2m \xddot V + 2 m \xdot^2 V_x \\
m \xdot^2 V_x - 2V V_x - 2 m \xdot^2 V_x &= m \xddot
\lr{  m \xdot^2 + 2 V  }
 \\
-
\lr{  2V + m \xdot^2  }
 V_x &= 
\end{aligned}
\end{equation}
%
Canceling left and right common factors, which perhaps not coincidentally equal \(2E = V + \inv{2}mv^2\) we have:
%
\begin{equation}\label{eqn:tongMf1:1060}
m \xddot = -V_x.
\end{equation}
%
This is what we would get for our standard kinetic and position dependent Lagrangian too:
%
\begin{equation}\label{eqn:tongMf1:1080}
\LL = \inv{2}m \xdot^2 - V.
\end{equation}
%
\begin{equation}\label{eqn:tongMf1:140}
\begin{aligned}
\frac{\partial L}{\partial x} &= \frac{d}{dt} \frac{\partial L}{\partial \xdot} \\
-V_x &= \frac{d (m \xdot) }{dt} \\
-V_x &= m \xddot.
\end{aligned}
\end{equation}
}
%
