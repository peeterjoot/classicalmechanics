%
% Copyright � 2012 Peeter Joot.  All Rights Reserved.
% Licenced as described in the file LICENSE under the root directory of this GIT repository.
%
%
%
%
%\chapter{Vector canonical momentum}
\index{canonical momentum}
\label{chap:PJCanMomentum}
\label{chap:canonicalMomentum}
%\date{Sept. 1, 2008.  canonicalMomentum.tex}
%
\citep{goldstein1951cm} defines the canonical momentum as:
%
\begin{equation*}
\PD{\xdot^{\mu}}{\LL}
\end{equation*}
%
and gives an example (Lorentz force) about how this can generalize the
concept of momentum to include contributions from velocity dependent
potentials.

Lets look at his example, but put into the more natural covariant form
with the Lorentz Lagrangian (using summation convention here)
%
\begin{equation*}
\LL = \inv{2} m v^2 + q A \cdot v /c =
\inv{2} m \gamma_{\alpha} \cdot \gamma_{\beta} \xdot^{\alpha} \xdot^{\beta}
+ \frac{q}{c} \gamma_{\alpha} \cdot \gamma_{\beta} A^{\alpha} \xdot^{\beta}
\end{equation*}
%
Calculation of the canonical momentum gives:
%
\begin{equation}\label{eqn:canonicalMomentum:20}
\begin{aligned}
\PD{\xdot^{\mu}}{\LL}
&=
m \gamma_{\alpha} \cdot \gamma_{\beta} {\delta^{\alpha}}_{\mu} \xdot^{\beta}
+ \frac{q}{c} \gamma_{\alpha} \cdot \gamma_{\beta} A^{\alpha} {\delta^{\beta}}_{\mu} \\
&=
m \gamma_{\mu} \cdot \gamma_{\alpha} \xdot^{\alpha}
+ \frac{q}{c} \gamma_{\alpha} \cdot \gamma_{\mu} A^{\alpha} \\
&=
\gamma_{\mu} \cdot \left(
m \gamma_{\alpha} \xdot^{\alpha}
+ \frac{q}{c} \gamma_{\alpha} A^{\alpha}
\right) \\
&= \gamma_{\mu} \cdot \left( m v + \frac{q}{c} A \right) .
\end{aligned}
\end{equation}
%
So, if we are to call this modified quantity \(p = m v + q A / c\) the total general momentum for the system, then the canonical momentum conjugate to \(x^{\mu}\) is:
%
\begin{equation*}
\PD{\xdot^{\mu}}{\LL} = \gamma_{\mu} \cdot p.
\end{equation*}
%
In terms of our reciprocal frame vectors, the components of \(p\) are:
%
\begin{equation*}
p = \gamma_{\mu} \gamma^{\mu} \cdot p = \gamma_{\mu} p^{\mu}
\end{equation*}
\begin{equation*}
p = \gamma^{\mu} \gamma_{\mu} \cdot p = \gamma^{\mu} p_{\mu}
\end{equation*}
%
From this we see that the conjugate momentum gives us our vector momentum
component with respect to the reciprocal frame.  We can therefore recover
our total momentum by summing over the reciprocal frame vectors.
%
\begin{equation}\label{eqn:canonicalMomentum:40}
\begin{aligned}
\PD{x^{\mu}}{\LL}
&= \frac{d}{d\tau}\PD{\xdot^{\mu}}{\LL} \\
&= \frac{d}{d\tau} p_{\mu} \\
\implies \\
\sum \gamma^{\mu} \PD{x^{\mu}}{\LL} &= \sum \frac{d}{d\tau} \gamma^{\mu} p_{\mu} .
\end{aligned}
\end{equation}
%
Observe that we have nothing more than our spacetime gradient on the left hand side, and a velocity
specific spacetime gradient on the right hand side.  Summarizing, this allows for writing the Euler-Lagrange equations in vector form as follows:
%
\begin{equation}\label{eqn:canonicalMomentum:60}
\begin{aligned}
\frac{d p}{d\tau} &= \grad \LL \\
p &= \grad_v \LL \\
\grad &= \gamma^{\mu} \PD{x^{\mu}}{} \\
\grad_v &= \gamma^{\mu} \PD{\xdot^{\mu}}{}
\end{aligned}
\end{equation}
%
Now, perhaps this is a step backwards, since the Lagrangian formulation allows for not having to use vector representations explicitly, nor to be constrained to specific parameterizations such as this constant frame vector representation.  However, it is nice to see things in a form that is closer to what one is used to, and this is not too different seeming than the familiar spatial Newtonian formulation:
%
\begin{equation*}
\frac{d\Bp}{dt} = -\spacegrad \phi
\end{equation*}
%
%(as seen previously, the negation is due to the Minkowski metric, and selection of the spatial components of the gradient).
