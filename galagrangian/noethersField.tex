%
% Copyright � 2012 Peeter Joot.  All Rights Reserved.
% Licenced as described in the file LICENSE under the root directory of this GIT repository.
%
%
%
%
%\chapter{Field form of Noether's Law}
\index{Noether's law!field}
\label{chap:PJNoethersField}
\label{chap:noethersField}
%\date{October 29, 2008.  noethersField.tex}
%
\section{Derivation.}
%
It was seen in
%euler_lagrange.ltx
\chapcite{PJEulerLagrange}
that Noether's law for a line integral action was shown
to essentially be an application of the chain rule, coupled with
an application of the Euler-Lagrange equations.

For a field Lagrangian a similar conservation statement can be made, where
it takes the form of a divergence relationship instead of derivative
with respect to the integration parameter associated with the line integral.

The following derivation follows \citep{doran2003gap}, but is dumbed down to the scalar field variable case, and
additional details are added.

The Lagrangian to be considered is
%
\begin{equation}\label{eqn:noethersField:20}
\begin{aligned}
\LL &= \LL(\psi, \partial_\mu \psi),
\end{aligned}
\end{equation}
%
and the single field case is sufficient to see how this works.  Consider the following transformation:
%
\begin{equation}\label{eqn:noethersField:40}
\begin{aligned}
\psi &\rightarrow f(\psi, \alpha) = \psi' \\
\LL' &= \LL(f, \partial_\mu f).
\end{aligned}
\end{equation}
%
Taking derivatives of the transformed Lagrangian with respect to the free transformation variable \(\alpha\), we have
%
\begin{equation}\label{eqn:noetherField:noetherschain}
\begin{aligned}
\frac{d\LL'}{d\alpha}
&= \PD{f}{\LL}\PD{\alpha}{f} + \sum_\mu \PD{(\partial_\mu f)}{\LL}\PD{\alpha}{(\partial_\mu f)}.
\end{aligned}
\end{equation}
%
The Euler-Lagrange field equations for the transformed Lagrangian are
%
\begin{equation}\label{eqn:noetherField:eulerLag}
\PD{f}{\LL} = \sum_\mu \partial_\mu \PD{(\partial_\mu f)}{\LL}.
\end{equation}
%
For some
for background discussion, examples, and derivation of the field form of Noether's equation see
%field_lagrangian.ltx
\chapcite{PJFieldLagrangian}.
%
Now substitute back into \eqnref{eqn:noetherField:noetherschain} for
%
\begin{equation}\label{eqn:noethersField:60}
\begin{aligned}
\frac{d\LL'}{d\alpha}
&=
\sum_\mu \left( \partial_\mu \PD{(\partial_\mu f)}{\LL}\right)
\PD{\alpha}{f} + \sum_\mu \PD{(\partial_\mu f)}{\LL}\PD{\alpha}{(\partial_\mu f)} \\
&=
\sum_\mu \left( \partial_\mu \PD{(\partial_\mu f)}{\LL} \right)
\PD{\alpha}{f} + \sum_\mu \PD{(\partial_\mu f)}{\LL} \partial_\mu \PD{\alpha}{f} .
\end{aligned}
\end{equation}
%
Using the product rule we have
\begin{equation}\label{eqn:noethersField:80}
\begin{aligned}
\frac{d\LL'}{d\alpha}
&= \sum_\mu \partial_\mu \left( \PD{(\partial_\mu f)}{\LL} \PD{\alpha}{f} \right) \\
&= \sum_\mu \gamma^\mu \partial_\mu \cdot \left( \gamma_\mu \PD{(\partial_\mu f)}{\LL} \PD{\alpha}{f} \right) \\
&= \grad \cdot \left( \gamma_\mu \PD{(\partial_\mu \psi')}{\LL} \PD{\alpha}{\psi'} \right) .
\end{aligned}
\end{equation}
%
Here the field does not have to be a relativistic field which could be implied by the use of the standard
symbols for relativistic four vector
basis \(\{\gamma_\mu\}\) of STA.
This is really a statement that one can form a gradient in the field variable configuration space
using any appropriate reciprocal basis pair.

Noether's law for a field Lagrangian is a statement that if the transformed Lagrangian is unchanged (invariant) by
some type of parametrized field variable transformation, then with \(J' = {J'}^\mu \gamma_\mu\) one has
%
\begin{subequations}
\begin{equation}
\frac{d\LL'}{d\alpha} = \grad \cdot J' = 0
\end{equation}
\begin{equation}
{J'}^\mu = \PD{(\partial_\mu \psi')}{\LL} \PD{\alpha}{\psi'}
\end{equation}
\end{subequations}
%
FIXME: GAFP evaluates things at \(\alpha = 0\) where that is the identity case.  I think this is what allows them to drop the primes later.  Must think
this through.
%
\section{Examples.}
%
\subsection{Klein-Gordan Lagrangian invariance under phase change.}
%
The Klein-Gordan Lagrangian, a relativistic relative of the Schr\"{o}dinger equation is
%
\begin{equation}\label{eqn:noethersField:100}
\LL = \eta^{\mu\nu}\partial_\mu \psi \partial_\nu \psi^\conj - m^2 \psi \psi^\conj.
\end{equation}
%
FIXME: fixed sign above.  Adjust the remainder below.
This provides a simple example application of the field form of Noether's equation, for a
transformation that involves a phase change
%
\begin{equation}\label{eqn:noethersField:120}
\begin{aligned}
\psi &\rightarrow \psi' = e^{i\theta}\psi \\
\psi^\conj &\rightarrow {\psi^\conj}' = e^{-i\theta}\psi^\conj.
\end{aligned}
\end{equation}
%
This transformation leaves the Lagrangian unchanged, so there is an associated conserved
quantity.
%
\begin{equation}\label{eqn:noethersField:140}
\begin{aligned}
%\PD{\psi'}{\LL} &= m^2 {\psi'}^\conj \\
\PD{\theta}{\psi'} &= i \psi' \\
\PD{(\partial_\mu \psi')}{\LL} &= \eta^{\mu\nu}\partial_\nu {\psi'}^\conj = \partial^\mu {\psi'}^\conj.
\end{aligned}
\end{equation}
%
Summing all the field partials, treating \(\psi\), and \(\psi^\conj\) as separate
field variables the divergence conservation statement is
\begin{equation}\label{eqn:noethersField:160}
\partial_\mu
\mathLabelBox
[
   labelstyle={xshift=2cm, yshift=0.5cm},
   linestyle={out=270,in=90, latex-}
]
{
\left(
\partial^\mu {\psi'}^\conj i\psi'
-\partial^\mu {\psi'} i{\psi'}^\conj
\right)
}{\({J'}^\mu\)}
 = 0
\end{equation}
%
%
%
Dropping primes and writing \(J = \gamma_\mu J^\mu\), this is
%
\begin{equation}\label{eqn:noethersField:180}
\begin{aligned}
J &= i (\psi \grad {\psi}^\conj - \psi^\conj \grad {\psi} ) \\
\grad \cdot J &= 0.
\end{aligned}
\end{equation}
%
Apparently with charge added this quantity actually represents electric current density.  It will be interesting to
learn some quantum mechanics and see how this works.
%
\subsection{Lorentz boost and rotation invariance of Maxwell Lagrangian.}
%
\begin{subequations}
\label{eqn:noetherField:lagrangian}
\begin{dmath}
\LL
= -\gpgradezero{(\grad \wedge A)^2} + \kappa A \cdot J
= \partial_\mu A_\nu (\partial^\mu A^\nu - \partial^\nu A^\mu) + \kappa A_\sigma J^\sigma
\end{dmath}
\begin{dmath}
\kappa = \frac{2}{\epsilon_0 c}
\end{dmath}
\end{subequations}
%
The rotation and boost invariance of the Maxwell Lagrangian was demonstrated in
%boost_maxwell_lagrangian.ltx
\chapcite{PJBoostMaxwell}.
%
Following
%Doing the exercise of determining the conserved quantity for the Lorentz force Lagrangian
%in
%noethers_lorentz_force.ltx
\chapcite{PJLorentzTxInteraction}
write the Lorentz boost or rotation in exponential form.
%motivated learning the field form of Noether's law, since the next obvious thing to try was to see what the
%conserved quantity for the field itself was.
%
\begin{equation}\label{eqn:noethersField:200}
L(x) = \exp(-\alpha i/2) x \exp(\alpha i/2), \quad \Lambda = \exp(-\alpha i/2)
\end{equation}
%
where \(i\) is a unit spatial bivector for a rotation of \(-\alpha\) radians, and a boost with rapidity \(\alpha\) when \(i\) is a spacetime unit bivector.

Introducing the transformation
%
\begin{equation}\label{eqn:noethersField:220}
A \rightarrow A' = \Lambda A \Lambda^\dagger
\end{equation}
%
The change in \(A'\) with respect to \(\alpha\) is
\begin{equation}\label{eqn:noethersField:240}
\PD{\alpha}{A'} = -i A' + A' i = 2 A' \cdot i = 2 {A'}_\sigma \gamma^\sigma \cdot i
\end{equation}
%
Next we want to compute
%
\begin{equation}\label{eqn:noethersField:260}
\begin{aligned}
\PD{(\partial_\mu {A'}_\nu)}{\LL}
&= \PD{(\partial_\mu {A'}_\nu)}{} \left( \partial_\alpha {A'}_\beta (\partial^\alpha {A'}^\beta - \partial^\beta {A'}^\alpha) + \kappa {A'}_\sigma J^\sigma \right) \\
&= \left( \PD{(\partial_\mu {A'}_\nu)}{} \partial_\alpha {A'}_\beta \right) \left(\partial^\alpha {A'}^\beta - \partial^\beta {A'}^\alpha \right) \\
&+ \partial^\alpha {A'}^\beta \PD{(\partial_\mu {A'}_\nu)}{} \left(\partial_\alpha {A'}_\beta - \partial_\beta {A'}_\alpha) \right) \\
&= \left( \PD{(\partial_\mu {A'}_\nu)}{} \partial_\mu {A'}_\nu \right) \left(\partial^\mu {A'}^\nu - \partial^\nu {A'}^\mu \right) \\
&+ \partial^\mu {A'}^\nu \PD{(\partial_\mu {A'}_\nu)}{} \partial_\mu {A'}_\nu \\
&- \partial^\nu {A'}^\mu \PD{(\partial_\mu {A'}_\nu)}{} \partial_\mu {A'}_\nu \\
&= 2 \left( \partial^\mu {A'}^\nu - \partial^\nu {A'}^\mu \right) \\
&= 2 F^{\mu\nu}.
\end{aligned}
\end{equation}
%
Employing the vector field form of Noether's equation as in \eqnref{eqn:noetherField:NoethersVector} the conserved current \(C\) components are
%
\begin{equation}\label{eqn:noethersField:280}
\begin{aligned}
C^\mu &= 2 (\gamma_\nu F^{\mu\nu}) \cdot (2 A \cdot i) \\
&\propto (\gamma_\nu F^{\mu\nu}) \cdot (A \cdot i) \\
&\propto (\gamma^\mu \cdot F) \cdot (A \cdot i) .
\end{aligned}
\end{equation}
%
Or
\begin{equation}\label{eqn:noetherField:maxwellConserved}
C = \gamma_\mu ((\gamma^\mu \cdot F) \cdot (A \cdot i))
\end{equation}
%
Here \(C\) was used instead of \(J\) for the conserved current vector since \(J\) is already taken for the current charge density itself.
%
\subsection{Questions.}
%
FIXME: What is this quantity?  It has the look of
angular momentum, or torque, or an inertial tensor.  Does it have a physical significance?  Can the \(i\) be factored out of the expression, leaving a conserved quantity that is some linear function only of \(F\), and \(A\) (this was possible in the Lorentz force Lagrangian for the same invariance considerations).
%
\subsection{Expansion for x-axis boost.}
%
As an example to get a feel for \eqnref{eqn:noetherField:maxwellConserved}, lets expand
this for a specific spacetime boost plane.  Using the x-axis that is \(i=\gamma_1 \wedge \gamma_0\)

First expanding the potential projection one has
%
\begin{equation}\label{eqn:noethersField:300}
\begin{aligned}
A \cdot i &= (A_\mu \gamma^\mu) \cdot (\gamma_1 \wedge \gamma_0) \\
&= A_1 \gamma_0 - A_0 \gamma_1.
\end{aligned}
\end{equation}
%
Next the \(\mu\) component of the field is
\begin{equation}\label{eqn:noethersField:320}
\begin{aligned}
\gamma^\mu \cdot F
&= \inv{2} F^{\alpha\beta} \gamma^\mu \cdot (\gamma_\alpha \wedge \gamma_\beta) \\
&= \inv{2} F^{\mu\beta} \gamma_\beta -\inv{2} F^{\alpha\mu} \gamma_\alpha \\
&= F^{\mu\alpha} \gamma_\alpha .
\end{aligned}
\end{equation}
%
So the \(\mu\) component of the conserved vector is
\begin{equation}\label{eqn:noethersField:340}
\begin{aligned}
C^\mu
&= (\gamma^\mu \cdot F) \cdot (A \cdot i) \\
&= (F^{\mu\alpha} \gamma_\alpha) \cdot (A_1 \gamma_0 - A_0 \gamma_1) \\
&= (F^{\mu\alpha} \gamma_\alpha) \cdot (A^0 \gamma^1 - A^1 \gamma^0) \\
%&= ({F^{\mu}}_{\alpha} \gamma^\alpha) \cdot (A_1 \gamma_0 - A_0 \gamma_1) \\
%&= {F^{\mu}}_{0} A_1 - {F^{\mu}}_{1} A_0 \\
%&= \eta^{\mu\nu} (F_{\nu 0} A_1 - F_{\nu 1} A_0) .
\end{aligned}
\end{equation}
%
Therefore the conservation statement is
%\begin{align*}
%C_\mu &= F_{\mu 0} A_1 - F_{\mu 1} A_0 \\
%\partial^\mu C_\mu &= 0
%\end{align*}
%
%Or
%
\begin{equation}\label{eqn:noetherField:conservedCurrentBoost}
\begin{aligned}
C^\mu &= F^{\mu 1} A^0 - F^{\mu 0} A^1 \\
\partial_\mu C^\mu &= 0.
\end{aligned}
\end{equation}
%
Let us write out the components of \eqnref{eqn:noetherField:conservedCurrentBoost} explicitly, to perhaps get a better feel for them.
%
\begin{equation}\label{eqn:noethersField:360}
\begin{aligned}
C^0 &= F^{0 1} A^0 = -E_x \phi \\
C^1 &= - F^{1 0} A^1 = -E_x A_x \\
C^2 &= F^{2 1} A^0 - F^{2 0} A^1 = B_z \phi - E_y A_x \\
C^3 &= F^{3 1} A^0 - F^{3 0} A^1 = -B_y \phi - E_z A_x .
\end{aligned}
\end{equation}
%
Well, that is not particularly enlightening looking after all.
%
\subsection{Expansion for rotation or boost.}
%
Suppose that one takes \(i = \gamma^{\mu} \wedge \gamma^{\nu}\), so that we have a symmetry for a boost if one of \(\mu\) or \(\nu\) is zero,
and rotational symmetry otherwise.

This gives
%
\begin{equation}\label{eqn:noethersField:380}
\begin{aligned}
A \cdot i
&= (A^\alpha \gamma_\alpha) \cdot (\gamma^{\mu} \wedge \gamma^{\nu}) \\
&= A^\mu \gamma^\nu - A^\nu \gamma^\mu.
\end{aligned}
\end{equation}
%
\begin{equation}\label{eqn:noethersField:400}
\begin{aligned}
C^\alpha
&= (\gamma^\alpha \cdot F) \cdot (A \cdot i) \\
&= ( F^{\alpha\beta} \gamma_\beta ) \cdot ( A^\mu \gamma^\nu - A^\nu \gamma^\mu) .
\end{aligned}
\end{equation}
%\gamma^\mu \cdot F &= F^{\mu\alpha} \gamma_\alpha \\
%\gamma^\alpha \cdot F &= F^{\alpha\beta} \gamma_\beta \\
%
\begin{equation}\label{eqn:noetherField:conservedCurrentGeneralIndexes}
\begin{aligned}
C^\alpha
&= F^{\alpha\nu} A^\mu - F^{\alpha\mu} A^\nu.
\end{aligned}
\end{equation}
%
For a rotation in the \(a,b\), plane with \(\mu = a\), and \(\nu = b\) (say), lets write out the \(C^\alpha\) components explicitly in terms of \(\BE\) and \(\BB\) components, also
writing \(0 < d\), \(a \ne d \ne b\).  That is
%
\begin{equation}\label{eqn:noethersField:420}
\begin{aligned}
C^0 &= F^{0 b} A^a - F^{0 a} A^b = E^a A^b - E^b A^a \\
C^1 &= F^{1 b} A^a - F^{1 a} A^b \\
C^2 &= F^{2 b} A^a - F^{2 a} A^b \\
C^3 &= F^{3 b} A^a - F^{3 a} A^b .
\end{aligned}
\end{equation}
%
Only the first term of this reduces nicely.  Suppose we additionally write \(a = 1\), \(b = 2\) to make things more concrete.  Then we have
%
\begin{equation}\label{eqn:noethersField:440}
\begin{aligned}
C^0 &= F^{0 2} A^1 - F^{0 1} A^2 = E_x A_y - E_y A_x = (\BE \cross \BA)_z \\
C^1 &= F^{1 2} A^1 - F^{1 1} A^2 = -B_z A_x \\
C^2 &= F^{2 2} A^1 - F^{2 1} A^2 = B_z A_x \\
C^3 &= F^{3 2} A^1 - F^{3 1} A^2 = B_x A_x + B_y A_y .
\end{aligned}
\end{equation}
%
The time-like component of whatever this vector is the z component of a cross product (spatial component of the \(\BE \cross \BA\) product in
the direction of the normal to the rotational plane), but what is the rest?
%
\subsubsection{Conservation statement.}
%
Returning to \eqnref{eqn:noetherField:conservedCurrentGeneralIndexes}, the conservation statement can be calculated as
%
\begin{equation}\label{eqn:noethersField:460}
\begin{aligned}
0 &= \partial_\alpha C^\alpha \\
&= \partial_\alpha F^{\alpha\nu} A^\mu - \partial_\alpha F^{\alpha\mu} A^\nu + F^{\alpha\nu} \partial_\alpha A^\mu - F^{\alpha\mu} \partial_\alpha A^\nu  .
\end{aligned}
\end{equation}
%
But the grade one terms of the Maxwell equation in tensor form is
%
\begin{equation}\label{eqn:noethersField:480}
\partial_\mu F^{\mu \alpha} = J^\alpha/\epsilon_0 c
\end{equation}
%
So we have
%
\begin{equation}\label{eqn:noethersField:500}
\begin{aligned}
0
&= \inv{\epsilon_0 c} \left(J^\nu A^\mu - J^\mu A^\nu \right) + {F_{\alpha}}^{\nu} \partial^\alpha A^\mu - {F_{\alpha}}^{\mu} \partial^\alpha A^\nu  \\
&= \inv{\epsilon_0 c} \left(J^\nu A^\mu - J^\mu A^\nu \right) + {F_{\alpha}}^{\nu} F^{\alpha \mu} - {F_{\alpha}}^{\mu} F^{\alpha\nu} .
\end{aligned}
\end{equation}
%
This first part is some sort of current-potential torque like beastie.
That second part, the squared field term is what?  I do not see an obvious way to reduce it to something more structured.
\section{Multivariable derivation.}
For completion sake, cut and pasted with most discussion omitted,
the multiple field variable case follows in the same fashion
as the single field variable Lagrangian.
%
\begin{equation}\label{eqn:noethersField:520}
\LL = \LL(\psi_\sigma, \partial_\mu \psi_\sigma).
\end{equation}
The transformation is now:
\begin{equation}\label{eqn:noethersField:540}
\begin{aligned}
\psi_\sigma &\rightarrow f_\sigma(\psi_\sigma, \alpha) = \psi_\sigma' \\
\LL' &= \LL(f_\sigma, \partial_\mu f_\sigma).
\end{aligned}
\end{equation}
%
Taking derivatives:
\begin{equation}\label{eqn:noetherField:noetherschainMult}
\begin{aligned}
\frac{d\LL'}{d\alpha}
&= \sum_\sigma \PD{f_\sigma}{\LL}\PD{\alpha}{f_\sigma} + \sum_{\mu,\sigma} \PD{(\partial_\mu f_\sigma)}{\LL}\PD{\alpha}{(\partial_\mu f_\sigma)}.
\end{aligned}
\end{equation}
%
Again, making the Euler-Lagrange substitution of \eqnref{eqn:noetherField:eulerLag} (with \(f\rightarrow f_\sigma)\) back into \eqnref{eqn:noetherField:noetherschainMult} gives
\begin{equation}\label{eqn:noethersField:560}
\begin{aligned}
\frac{d\LL'}{d\alpha}
&=
\sum_\sigma \left( \sum_\mu \partial_\mu \PD{(\partial_\mu f_\sigma)}{\LL}\right)
\PD{\alpha}{f_\sigma} + \sum_{\mu,\sigma} \PD{(\partial_\mu f_\sigma)}{\LL}\PD{\alpha}{(\partial_\mu f_\sigma)} \\
&=
\sum_{\mu,\sigma} \left( \left( \partial_\mu \PD{(\partial_\mu f_\sigma)}{\LL} \right)
\PD{\alpha}{f_\sigma} + \PD{(\partial_\mu f_\sigma)}{\LL} \partial_\mu \PD{\alpha}{f_\sigma} \right) \\
&= \sum_{\mu,\sigma} \partial_\mu \left( \PD{(\partial_\mu f_\sigma)}{\LL} \PD{\alpha}{f_\sigma} \right) \\
&= \sum_\mu \gamma^\mu \partial_\mu \cdot \left( \sum_{\sigma,\nu} \gamma_\nu \PD{(\partial_\nu f_\sigma)}{\LL} \PD{\alpha}{f_\sigma} \right) \\
&= \grad \cdot \left( \sum_{\sigma,\nu} \gamma_\nu \PD{(\partial_\nu \psi_\sigma')}{\LL} \PD{\alpha}{\psi_\sigma'} \right) .
\end{aligned}
\end{equation}
%
Or
\begin{subequations}
\label{eqn:noetherField:conCurrent}
\begin{dmath}
\frac{d \LL'}{d\alpha} = \grad \cdot J' = 0
\end{dmath}
\begin{dmath}
J' = {J'}^\mu \gamma_\mu
\end{dmath}
\begin{dmath}
{J'}^\mu = \sum_\sigma \PD{(\partial_\mu \psi_\sigma')}{\LL} \PD{\alpha}{\psi_\sigma'}
\end{dmath}
\end{subequations}
%
A notational convenience for vector valued fields, in particular as we have in the electrodynamic Lagrangian for the vector potential, the chain rule summation in
\eqnref{eqn:noetherField:conCurrent} above can be replaced with a dot product.
%
\begin{equation}\label{eqn:noethersField:580}
\begin{aligned}
{J'}^\mu &= \gamma_\sigma \PD{(\partial_\mu \psi_\sigma')}{\LL} \cdot \PD{\alpha}{\gamma^\sigma \psi_\sigma'}.
\end{aligned}
\end{equation}
%
Dropping primes for convenience, and writing \(\Psi = \gamma^\sigma \psi_\sigma\) for the vector field variable,
the field form of Noether's law takes the form
%
\begin{subequations}
\label{eqn:noetherField:NoethersVector}
\begin{dmath}
J = \gamma_\mu \left( \gamma_\sigma \PD{(\partial_\mu \psi_\sigma)}{\LL} \cdot \PD{\alpha}{\Psi} \right)
\end{dmath}
\begin{dmath}
\grad \cdot J = 0.
\end{dmath}
\end{subequations}
%
That is, a current vector with respect to this configuration space divergence is conserved when the Lagrangian field transformation is invariant.
