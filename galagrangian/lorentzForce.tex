%
% Copyright � 2012 Peeter Joot.  All Rights Reserved.
% Licenced as described in the file LICENSE under the root directory of this GIT repository.
%
%
%
%
%\chapter{Revisit Lorentz force from Lagrangian}
\index{Lorentz force}
\label{chap:PJSrLorentzForce}
\label{chap:lorentzForce}
%\date{October 8, 2008.  lorentzForce.tex}
%
\section{Motivation.}
%
In \chapcite{PJSrLagrangian} a derivation of the Lorentz force in covariant
form was performed.  Intuition says that result, because of the squared
proper velocity, was dependent on the
positive time Minkowski signature.
With many
GR references using the opposite signature, it seems worthwhile to understand
what results are signature dependent and put them in a signature invariant form.

Here the result will be rederived without assuming this signature.

Assume a Lagrangian of the following form
%
\begin{equation}\label{eqn:lorForce:potlag}
\LL = \inv{2}m v^2 + \kappa A \cdot v
\end{equation}
%
where \(v\) is the proper velocity.  Here \(A(x^\mu,\xdot^\nu) = A(x^\mu)\) is a position but not velocity dependent four vector potential.  The constant \(\kappa\) includes the charge of the test mass, and will be determined exactly in due course.
%
\section{Equations of motion.}
%
As observed in \chapcite{PJCanMomentum} the Euler-Lagrange equations can be summarized in four-vector form as
%
\begin{equation}\label{eqn:lorentzForce:20}
\grad \LL = \frac{d}{d\tau}(\grad_v \LL).
\end{equation}
%
To compute this, some intermediate calculations are helpful
%
\begin{equation}\label{eqn:lorentzForce:40}
\begin{aligned}
\grad v^2 &= 0 \\
\grad (A \cdot v)
&= \grad A_\mu \xdot^\mu \\
&= \gamma^\nu \xdot^\mu \partial_\nu A_\mu \\
\inv{2} \grad_v v^2
&= \inv{2} \grad_v (\gamma_\mu)^2 (\xdot^\mu)^2 \\
&= \gamma^\nu (\gamma_\mu)^2 \partial_{\xdot^\nu} \xdot^\mu \\
&= \gamma^\mu (\gamma_\mu)^2 \xdot^\mu \\
&= \gamma_\mu \xdot^\mu \\
&= v \\
\grad_v (A \cdot v)
&= \gamma^\nu \partial_{\xdot^\nu} A_\mu \xdot^\mu \\
&= \gamma^\nu A_\mu {\delta^\mu}_\nu \\
&= \gamma^\mu A_\mu \\
&= A \\
\frac{d}{d\tau} &= \xdot^\mu \partial_\mu
\end{aligned}
\end{equation}
%
Putting all this back together
\begin{equation}\label{eqn:lorentzForce:60}
\begin{aligned}
\grad \LL &= \frac{d}{d\tau}(\grad_v \LL) \\
\kappa \gamma^\nu \xdot^\mu \partial_\nu A_\mu &= \frac{d}{d\tau}\left( m v + \kappa A \right) \\
\implies \\
\pdot
&= \kappa \left( \gamma^\nu \xdot^\mu \partial_\nu A_\mu - \xdot^\nu \partial_\nu \gamma^\mu A_\mu \right) \\
&= \kappa \partial_\nu A_\mu \left( \gamma^\nu \xdot^\mu - \xdot^\nu \gamma^\mu \right) \\
\end{aligned}
\end{equation}
%
We know this will be related to \(F \cdot v\), where \(F = \grad \wedge A\).  Expanding that for comparison
%
\begin{equation}\label{eqn:lorentzForce:80}
\begin{aligned}
F \cdot v
&= (\grad \wedge A) \cdot v \\
&= (\gamma^\mu \wedge \gamma^\nu) \cdot \gamma_\alpha \xdot^\alpha \partial_\mu A_\nu \\
&= \left( \gamma^\mu {\delta^\nu}_\alpha -\gamma^\nu {\delta^\mu}_\alpha \right) \xdot^\alpha \partial_\mu A_\nu \\
&=
\gamma^\mu \xdot^\nu \partial_\mu A_\nu
-\gamma^\nu \xdot^\mu \partial_\mu A_\nu \\
&= \partial_\nu A_\mu \left( \gamma^\nu \xdot^\mu -\gamma^\mu \xdot^\nu \right) \\
\end{aligned}
\end{equation}
%
With the insertion of the \(\kappa\) factor this is an exact match, but working backwards to demonstrate that would have been harder.  The equation of motion associated with the Lagrangian of \eqnref{eqn:lorForce:potlag} is thus
%
\begin{equation}\label{eqn:lorForce:lorentzUndeterminedConst}
\pdot = \kappa F \cdot v.
\end{equation}
%
\section{Correspondence with classical form.}
%
A reasonable approach to fix the constant \(\kappa\) is to put this into correspondence with the classical
vector form of the Lorentz force equation.

Introduce a rest observer, with worldline \(x = ct e_0\).  Computation of the spatial parts of the four vector force \eqnref{eqn:lorForce:lorentzUndeterminedConst} for this rest observer requires taking the wedge product
with the observer velocity \(v = c \gamma e_0\).
For clarity, for the observer frame we use a different set of basis vectors \(\{e_\mu\}\), to point
out that \(\gamma_0\) of the derivation above does not have to equal \(e_0\).  Since the end result of the Lagrangian calculation
ended up being coordinate and signature free, this is perhaps superfluous.

First calculate the field velocity product in terms of electric and magnetic components.  In this new frame of reference write the proper velocity of the charged particle as
\(v = e_\mu \fdot^\mu\)
%
\begin{equation}\label{eqn:lorentzForce:100}
\begin{aligned}
F \cdot v
&= (\BE + I c \BB) \cdot v \\
&= (E^i e_{i0} - \epsilon_{ijk}c B^k e_{ij}) \cdot e_\mu \fdot^\mu \\
&=
  E^i \fdot^0 e_{i0} \cdot e_0
+ E^i \fdot^j e_{i0} \cdot e_j
- \epsilon_{ijk} c B^k \fdot^m e_{ij} \cdot e_m \\
\end{aligned}
\end{equation}
%
Omitting the scale factor \(\gamma = dt/d\tau\) for now, application of a wedge with \(e_0\) operation to both sides
of
\eqnref{eqn:lorForce:lorentzUndeterminedConst}
will suffice to determine this observer dependent expression of the force.
%
\begin{equation}\label{eqn:lorentzForce:120}
\begin{aligned}
(F \cdot v) \wedge e_0
&=
\left(
  E^i \fdot^0 (e_{i0} \cdot e_0)
+ E^i \fdot^j (e_{i0} \cdot e_j)
- \epsilon_{ijk} c B^k \fdot^m e_{ij} \cdot e_m \right) \wedge e_0 \\
&= E^i \fdot^0 e_{i0} (e_0)^2 - \epsilon_{ijk} c B^k \fdot^m (e_i)^2 ( e_{i} \delta_{jm} -e_{j} \delta_{im} ) \wedge e_0 \\
&= (e_0)^2 \left( E^i \fdot^0 e_{i0} + \epsilon_{ijk} c B^k \left( \fdot^j e_{i0} - \fdot^i e_{j0} \right) \right) \\
\end{aligned}
\end{equation}
%
This wedge application has discarded the timelike components of the force equation with respect to this observer rest frame.
Introduce the basis
\(\{\sigma_i = e_i \wedge e_0\}\) for this observers' Euclidean space.  These spacetime bivectors square to unity, and thus behave in every respect like
Euclidean space vector basis vectors.  Writing \(\BE = E^i \sigma_i\), \(\BB = B^i \sigma_i\), and \(\Bv = \sigma_i dx^i/dt\) we have
%
\begin{equation}\label{eqn:lorentzForce:140}
\begin{aligned}
(F \cdot v) \wedge e_0
&= (e_0)^2 c \frac{dt}{d\tau} \left( \BE + \epsilon_{ijk} B^k \left( \frac{df^j}{dt} \sigma_{i} - \frac{df^i}{dt} \sigma_{j} \right) \right) \\
\end{aligned}
\end{equation}
%
This inner antisymmetric sum is just the cross product.  This can be observed by expanding the determinant
%
\begin{equation}\label{eqn:lorentzForce:160}
\begin{aligned}
\Ba \cross \Bb &=
\begin{vmatrix}
\sigma_1 & \sigma_2 & \sigma_3 \\
a_1 & a_2 & a_3 \\
b_1 & b_2 & b_3 \\
\end{vmatrix} \\
&=
  \sigma_1 (a_2 b_3 - a_3 b_2)
+ \sigma_2 (a_3 b_1 - a_1 b_3)
+ \sigma_3 (a_1 b_2 - a_2 b_1) \\
&=
  \sigma_i a_j b_k
\end{aligned}
\end{equation}
%
This leaves
\begin{equation}\label{eqn:lorForce:FdotVwedge}
\begin{aligned}
\kappa (F \cdot v) \wedge e_0
&= \kappa (e_0)^2 c \frac{dt}{d\tau} \left( \BE + \Bv \cross \BB \right)
\end{aligned}
\end{equation}
%
Next expand the left hand side acceleration term in coordinates, and wedge with \(e_0\)
%
\begin{equation}\label{eqn:lorentzForce:180}
\begin{aligned}
\pdot \wedge e_0
&= \left(e_\mu \frac{d m \fdot^\mu}{dt}\frac{dt}{d\tau} \right) \wedge e_0 \\
&= e_{i0} \frac{d m \fdot^i}{dt} \frac{dt}{d\tau}.
\end{aligned}
\end{equation}
%
Equating with \eqnref{eqn:lorForce:FdotVwedge}, with cancellation of the \(\gamma = dt/d\tau\) factors, leaves the traditional Lorentz force law in observer dependent form
%
\begin{equation}\label{eqn:lorentzForce:200}
\begin{aligned}
\frac{d}{dt}\left( m \gamma \Bv \right) &= \kappa (e_0)^2 c \left( \BE + \Bv \cross \BB \right)
\end{aligned}
\end{equation}
%
This supplies the undetermined constant factor from the Lagrangian \(\kappa (e_0)^2 c = q\).  A summary statement of the results is as follows
%
\begin{equation}\label{eqn:lorForce:summarize}
\begin{aligned}
\LL &= \inv{2} m v^2 + q (e_0)^2 A \cdot (v/c) \\
\pdot &= (e_0)^2 q F \cdot (v/c) \\
\Bp &= q (\BE + \Bv \cross \BB)
\end{aligned}
\end{equation}
%
For \((e_0)^2 = 1\), we have the proper Lorentz force equation as found in
\citep{doran2003gap}, which also uses the positive time signature.  In that text the equation was obtained using some subtle relativistic symmetry arguments not especially easy to follow.
%
\section{General potential.}
%
Having written this, it would be more natural to couple the signature dependency into the velocity term of the Lagrangian since that squared velocity was the
signature dependent term to start with
%
\begin{equation}\label{eqn:lorentzForce:220}
\begin{aligned}
\LL &= \inv{2} m v^2 (e_0)^2 + q A \cdot (v/c) \\
\end{aligned}
\end{equation}
%
Although this does not change the equations of motion we can keep that signature factor with the velocity term.  Consider a general potential as an example
%
\begin{equation}\label{eqn:lorentzForce:240}
\begin{aligned}
\LL &= \inv{2}m v^2 (\gamma_0)^2 + \phi \\
\grad \LL &= \frac{d}{d\tau}(\grad_v \LL) \\
\frac{d}{d\tau}(m v (\gamma_0)^2) &= \grad \phi - \frac{d}{d\tau}(\grad_v \phi) \\
\end{aligned}
\end{equation}
%
%WRONG: have ds^2 = (\gamma_0)^2 c^2 d\tau^2
%
%When the time signature is negative this gives a natural coupling of terms in terms of worldline arc length \(ds = -c d\tau\)
%
%\begin{align}
%\frac{d}{ds}(m v) &= \inv{c} \left(\grad \phi - \frac{d}{d\tau}(\grad_v \phi) \right) \\
%\end{align}
%
%Since for a positive time metric we have \(ds = cd\tau\), this happens to also be a signature invariant form for the equations of motion associated with a general position and velocity dependent special relativistic Lagrangian.
