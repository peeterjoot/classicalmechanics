%
% Copyright � 2012 Peeter Joot.  All Rights Reserved.
% Licenced as described in the file LICENSE under the root directory of this GIT repository.
%

%
%
%\chapter{Tensor Derivation of Covariant Lorentz Force from Lagrangian}
\index{Lorentz Force!covariant}
\label{chap:maxwellTensorLagrangian}
%\date{October 12, 2008.  maxwellTensorLagrangian.tex}

\section{Motivation}

In \chapcite{PJSrLorentzForce}, and before that in \chapcite{PJSrLagrangian} Clifford
algebra derivations of the STA form of the covariant Lorentz force equation
were derived.  As an exercise in tensor manipulation try the equivalent
calculation using only tensor manipulation.

\section{Calculation}

The starting point will be an assumed Lagrangian of the following form

\begin{equation}\label{eqn:maxwell_tensor_lagrangian:lagrangian}
\begin{aligned}
\LL &= \inv{2} v^2 + (q/m) A \cdot v/c \\
&= \inv{2} \xdot_\alpha \xdot^\alpha + (q/mc) A_\beta \xdot^\beta
\end{aligned}
\end{equation}

Here \(v\) is the proper (four)velocity, and \(A\) is the four potential.
And following \citep{doran2003gap}, we use a positive time signature for the metric tensor (\(+---\)).

\begin{equation}\label{eqn:maxwellTensorLagrangian:20}
\begin{aligned}
\PD{x^\mu}{\LL} = (q/mc) \PD{x^\mu}{A_\beta} \xdot^\beta
\end{aligned}
\end{equation}

\begin{equation}\label{eqn:maxwellTensorLagrangian:40}
\begin{aligned}
\PD{\xdot^\mu}{\LL}
&= \PD{\xdot^\mu}{} \left( \inv{2} g_{\alpha\beta} \xdot^\beta \xdot^\alpha \right) + (q/mc) \PD{\xdot^\mu}{(A_\alpha \xdot^\alpha)} \\
&= \inv{2} \left( g_{\alpha\mu} \xdot^\alpha +g_{\mu\beta} \xdot^\beta \right) + (q/mc) A_\mu \\
&= \xdot_\mu + (q/mc) A_\mu \\
\end{aligned}
\end{equation}

\begin{equation}\label{eqn:maxwellTensorLagrangian:60}
\begin{aligned}
\PD{x^\mu}{\LL} &= \frac{d}{d\tau} \PD{\xdot^\mu}{\LL} \\
(q/mc) \PD{x^\mu}{A_\beta} \xdot^\beta &= \xddot_\mu + (q/mc) \xdot^\beta \PD{x^\beta}{A_\mu} \\
\implies \\
\xddot_\mu
&= (q/mc) \xdot^\beta \left( \PD{x^\mu}{A_\beta} - \PD{x^\beta}{A_\mu} \right) \\
&= (q/mc) \xdot^\beta \left( \partial_\mu {A_\beta} - \partial_{\beta}{A_\mu} \right) \\
\end{aligned}
\end{equation}

This is

\begin{equation}\label{eqn:maxwell_tensor_lagrangian:fromLagrangian}
\begin{aligned}
m \xddot_\mu &= (q/c) F_{\mu\beta} \xdot^\beta
\end{aligned}
\end{equation}

The wikipedia article \citep{wiki:LorentzForce} writes this in the equivalent indices toggled form
\begin{equation}\label{eqn:maxwellTensorLagrangian:80}
\begin{aligned}
m \xddot^\mu &= (q/c) \xdot_\beta F^{\mu\beta}
\end{aligned}
\end{equation}

\citep{schiller:mm} (22nd edition, equation 467) writes this with the
Maxwell tensor in mixed form

\begin{equation}\label{eqn:maxwellTensorLagrangian:100}
\begin{aligned}
b^\mu = \frac{q}{m} {F_\nu}^\mu u^\nu
\end{aligned}
\end{equation}

where \(b^\mu\) is a proper acceleration.  If one has to put the Lorentz
equation it in tensor
form, using a mixed index tensor seems like the nicest way since all
vector quantities then have consistently placed indices.  Observe that he has used units with \(c=1\), and by comparison must
also be using a time negative metric tensor.

%\subsection{}
%\begin{align}\label{eqn:maxwell_tensor_lagrangian:lagrangian}
%This Lagrangian can also be found in \citep{AFkracklauerDeBroglie}, but
%differs by a factor of two?
%
%\calL = \frac{1}{2} m v^2 + e A \cdot v/c \\
%&= \inv{2} \xdot_\alpha \xdot^\alpha + \kappa A_\beta \xdot^\beta

\section{Compare for reference to GA form}

To verify that this form is identical to familiar STA Lorentz Force equation,

\begin{equation}\label{eqn:maxwellTensorLagrangian:120}
\begin{aligned}
\pdot &= q (F \cdot v/c)
\end{aligned}
\end{equation}

reduce this equation to coordinates.  Starting with the RHS (leaving out the q/c)

\begin{equation}\label{eqn:maxwellTensorLagrangian:140}
\begin{aligned}
(F \cdot v) \cdot \gamma_\mu
&= \inv{2} F_{\alpha\beta} \xdot^\nu ((\gamma^{\alpha} \wedge \gamma^{\beta}) \cdot \gamma_\nu) \cdot \gamma_\mu \\
&= \inv{2} F_{\alpha\beta} \xdot^\nu \left( \gamma^{\alpha} (\gamma^{\beta} \cdot \gamma_\nu) -\gamma^{\beta} (\gamma^{\alpha} \cdot \gamma_\nu) \right) \cdot \gamma_\mu \\
&= \inv{2} \left( F_{\alpha\nu} \xdot^\nu \gamma^{\alpha} - F_{\nu\beta} \xdot^\nu \gamma^{\beta} \right) \cdot \gamma_\mu \\
&= \inv{2} \left( F_{\mu\nu} \xdot^\nu - F_{\nu\mu} \xdot^\nu \right) \\
&= F_{\mu\nu} \xdot^\nu \\
\end{aligned}
\end{equation}

And for the LHS
\begin{equation}\label{eqn:maxwellTensorLagrangian:160}
\begin{aligned}
\pdot \cdot \gamma_\mu &= m \xddot_\alpha \gamma^\alpha \cdot \gamma_\mu \\
&= m \xddot_\mu
\end{aligned}
\end{equation}

Which gives us

\begin{equation}\label{eqn:maxwellTensorLagrangian:180}
\begin{aligned}
m \xddot_\mu &= (q/c) F_{\mu\nu} \xdot^\nu
\end{aligned}
\end{equation}

in agreement with \eqnref{eqn:maxwell_tensor_lagrangian:fromLagrangian}.
