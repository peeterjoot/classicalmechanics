%
% Copyright � 2012 Peeter Joot.  All Rights Reserved.
% Licenced as described in the file LICENSE under the root directory of this GIT repository.
%
%
%
%
%\documentclass{article}
%
%\input{../peeters_macros.tex}
%\input{../peeters_macros2.tex}
%\input{../peeters_macros3.tex}
%
%\usepackage{listings}
%\usepackage{txfonts} % for ointctr... (also appears to make "prettier" \int and \sum's)
% makes \grad look funny though (almost like spacegrad, but narrower)
%\usepackage[bookmarks=true]{hyperref}
%
%\usepackage{color,cite,graphicx}
   % use colour in the document, put your citations as [1-4]
   % rather than [1,2,3,4] (it looks nicer, and the extended LaTeX2e
   % graphics package.
%\usepackage{latexsym,amssymb,epsf} % do not remember if these are
   % needed, but their inclusion can not do any damage
%
%\chapter{Comparison of two covariant Lorentz force Lagrangians}
\index{Lorentz force}
\label{chap:lForceLag2}
%\author{Peeter Joot \quad peeterjoot@protonmail.com }
%\date{ June 17, 2009.  \(RCSfile: lForceLag2.tex,v \) Last \(Revision: 1.6 \) \(Date: 2009/10/22 02:07:20 \) }
%
%\begin{document}
%
%\maketitle{}
%\tableofcontents
\section{Motivation.}
%
In \citep{poisson1999ild},
the covariant Lorentz force Lagrangian is given by
%
\begin{equation}\label{eqn:lforceLag2:poissonLag}
\LL = \int A_\alpha j^\alpha d^4 x - m \int d\tau
\end{equation}
%
which is not quadratic in proper time as seen previously in
\chapcite{PJSrLorentzForce}
%\eqnref{eqn:lforceLag2:lorForce:summarize}
, and
\chapcite{lorentzForcePQA}
%\eqnref{eqn:lforceLag2:lorForcePqA:interactionLagPsq}
%
\begin{equation}\label{eqn:lForceLag2:20}
\begin{aligned}
\LL
%&= \inv{2} m v^2 + q (\gamma_0)^2 A \cdot (v/c) \\
&= \inv{2} m v^2 + q A \cdot (v/c) \\
&= \inv{2 m}\left( m v + \frac{q}{c} A \right)^2 - \frac{q^2}{ 2 m c^2} A^2.
\end{aligned}
\end{equation}
%
These two forms are identical, but the second is expressed explicitly
in terms of the conjugate momentum, and calls out the explicit kinetic
vs potential terms in the Lagrangian nicely.  Note that both forms assume \(\gamma_0^2 = 1\), unlike
\eqnref{eqn:lforceLag2:poissonLag}, which must assume a time negative line element.
%
\section{Lagrangian with Quadratic Velocity.}
%
For review purposes lets once again compute the equations of motion
with an evaluation of the Euler-Lagrange equations.  With hindsight
this can also be done more compactly than in previous notes.

We carry out the evaluation of the Euler-Lagrange equations in vector form
%
\begin{equation}\label{eqn:lForceLag2:40}
\begin{aligned}
0
&= \grad \LL - \frac{d}{d\tau} \grad_v \LL \\
&= \left( \grad - \frac{d}{d\tau} \grad_v \right)
\left( \inv{2} m v^2 + q A \cdot (v/c) \right)
\\
&= q \grad (A \cdot (v/c))
- \frac{d}{d\tau} \grad_v
\left( \inv{2} m v^2 + q A \cdot (v/c) \right)
.
\end{aligned}
\end{equation}
%
The middle term here is the easiest and we essentially want the gradient of a vector square.
%
\begin{equation}\label{eqn:lForceLag2:60}
\begin{aligned}
\grad x^2
&=
\gamma^\mu \partial_\mu x^\alpha x_\alpha \\
&=
2 \gamma^\mu x_\mu .
\end{aligned}
\end{equation}
%
This is
\begin{equation}\label{eqn:lForceLag2:80}
\begin{aligned}
\grad x^2 &= 2 x.
\end{aligned}
\end{equation}
%
The same argument would work for \(\grad_v v^2 = \gamma^\mu \PDi{\xdot^\mu}{(\xdot^\alpha \xdot_\alpha)}\), but is messier to write and read.

Next we need the gradient of the \(A \cdot v\) dot product, where \(v = \gamma_\mu \xdot^\mu\) is essentially a constant.
We have
%
\begin{equation}\label{eqn:lForceLag2:100}
\begin{aligned}
\grad (A \cdot v)
&= \gpgradeone{ \grad (A \cdot v) } \\
&= \inv{2} \gpgradeone{ \dot{\grad} (\dot{A} v + v \dot{A}) } \\
&= \inv{2}
\left(
   (\grad \cdot A) v + (\grad \wedge A) \cdot v + (v \cdot \grad) A -
   \mathLabelBox{(v \wedge \grad) \cdot A}{(**)}
\right) .
\end{aligned}
\end{equation}
((**) = \(v (\grad \cdot A) - \grad (A \cdot v)\)).
%
Canceling \(v(\grad \cdot A)\) terms, and rearranging we have
%
\begin{equation}\label{eqn:lForceLag2:120}
\begin{aligned}
\grad (A \cdot v)
&= (\grad \wedge A) \cdot v + (v \cdot \grad) A.
\end{aligned}
\end{equation}
%
Finally we want
%
\begin{equation}\label{eqn:lForceLag2:140}
\begin{aligned}
\grad_v (A \cdot v)
&=
\gamma^\mu \PD{\xdot^\mu}{ A_\alpha v^\alpha} \\
&=
\gamma^\mu A_\mu .
\end{aligned}
\end{equation}
%
Which is just
%
\begin{equation}\label{eqn:lForceLag2:160}
\begin{aligned}
\grad_v (A \cdot v) &= A.
\end{aligned}
\end{equation}
%
Putting these all together we have
%
\begin{equation}\label{eqn:lForceLag2:180}
\begin{aligned}
0 &=
q \left( (\grad \wedge A) \cdot v/c + (v/c \cdot \grad) A \right)
- \frac{d}{d\tau} ( m v + q A/c).
\end{aligned}
\end{equation}
%
The only thing left is the proper time derivative of \(A\), which by chain rule is
%
\begin{equation}\label{eqn:lForceLag2:200}
\begin{aligned}
\frac{d A}{d\tau}
&=
\PD{x^\mu}{A} \PD{\tau}{x^\mu} \\
&=
v^\mu \partial_\mu A \\
&= (v \cdot \grad) A.
\end{aligned}
\end{equation}
%
So our \((v \cdot \grad)A\) terms cancel and with \(F = \grad \wedge A\) we have our covariant Lorentz force law
%
\begin{equation}\label{eqn:lForceLag2:220}
\frac{d(mv)}{d\tau} = q F \cdot v/c
\end{equation}
%
\section{Lagrangian with Absolute Velocity.}
%
Now, with
%
\begin{equation}\label{eqn:lForceLag2:240}
d\tau = \sqrt{\frac{dx}{d\lambda}} d\lambda
\end{equation}
%
it appears from
\eqnref{eqn:lforceLag2:poissonLag}
that we can form a different Lagrangian
%
\begin{equation}\label{eqn:lForceLag2:260}
\LL = \alpha m \Abs{v} + q A \cdot v/c
\end{equation}
%
where \(\alpha\) is a constant to be determined.  Most of the work of evaluating the variational derivative has been done, but we need \(\grad_v \Abs{v}\), omitting dots this is
%
\begin{equation}\label{eqn:lForceLag2:280}
\begin{aligned}
\grad \Abs{x}
&= \gamma^\mu \partial_\mu \sqrt{x^\alpha x_\alpha} \\
&= \gamma^\mu \inv{2 \sqrt{x^2}} \partial_\mu (x^\alpha x_\alpha) \\
&= \gamma^\mu \inv{\sqrt{x^2}} x_\mu \\
&= \frac{x}{\Abs{x}}.
\end{aligned}
\end{equation}
%
We therefore have
%
\begin{equation}\label{eqn:lForceLag2:300}
\begin{aligned}
\grad_v \Abs{v}
&= \frac{v}{\Abs{v}} \\
&= \frac{v}{c} .
\end{aligned}
\end{equation}
%
which gives us
\begin{equation}\label{eqn:lForceLag2:320}
\alpha \frac{d(mv/c)}{d\tau} = q F \cdot v/c
\end{equation}
%
This fixes the constant \(\alpha = c\), and we now have a new form for the Lagrangian
%
\begin{equation}\label{eqn:lForceLag2:340}
\LL = m \Abs{v} c + q A \cdot v/c
\end{equation}
%
Observe that only after varying the Lagrangian can one make use of the \(\Abs{v} = c\) equality.
%
%TODO: What is the canonical momentum for this Lagrangian?
%
%\bibliographystyle{plainnat}
%\bibliography{myrefs}
%
%\end{document}
