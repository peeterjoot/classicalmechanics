%
% Copyright � 2020 Peeter Joot.  All Rights Reserved.
% Licenced as described in the file LICENSE under the root directory of this GIT repository.
%
%{
\input{../latex/blogpost.tex}
\renewcommand{\basename}{lorentzForceCovariant}
%\renewcommand{\dirname}{notes/phy1520/}
\renewcommand{\dirname}{notes/ece1228-electromagnetic-theory/}
%\newcommand{\dateintitle}{}
%\newcommand{\keywords}{}

\input{../latex/peeter_prologue_print2.tex}

\usepackage{peeters_layout_exercise}
\usepackage{peeters_braket}
\usepackage{peeters_figures}
\usepackage{siunitx}
\usepackage{verbatim}
%\usepackage{mhchem} % \ce{}
%\usepackage{macros_bm} % \bcM
%\usepackage{macros_qed} % \qedmarker
%\usepackage{txfonts} % \ointclockwise

\beginArtNoToc

\generatetitle{Lagrangian for the Lorentz force equation.}
In my old classical mechanics notes it appears that I derived the Lorentz force equations a number of times, however,
none of these appear to have been done concisely.

\section{Euler-Lagrange equations.}
A relativistic action for a single particle system has the form
\begin{dmath}\label{eqn:lorentzForceCovariant:n}
S = \int d\tau L(x, \dot{x}),
\end{dmath}
where
\( x \) is the spacetime coordinate, \( \dot{x} = dx/d\tau \) is the four-velocity,
and \( \tau \) is proper time.
%\( A \) is the four-potential,

\maketheorem{Relativistic Euler-Lagrange equations.}{thm:lorentzForceCovariant:n}{
The Euler-Lagrange equations
\begin{equation*}
\delta L = \grad L - \frac{d}{d\tau} (\grad_v L) = 0.
\end{equation*}
follow from minimizing the action
\begin{equation*}
\delta S = \int d\tau \delta L(x, \dot{x}) = 0,
\end{equation*}
with respect to all possible variations.
} % theorem

We wish to show that the equations of motion follow by requiring that the variation of the action is zero for all variations of \( x, \dot{x} \).  Such a variation is a mapping \( x \rightarrow x + \epsilon \), where \( \epsilon = 0 \) at the boundaries of the action integral.

We expanding the Lagrangian in Taylor series
\begin{dmath}\label{eqn:lorentzForceCovariant:n}
S + \delta S
= \int d\tau L( x + \epsilon, \dot{x} + \dot{\epsilon})
=
\int d\tau \lr{
   L(x, \dot{x}) + \epsilon \cdot \grad L + \dot{\epsilon} \cdot \grad_v L
}.
\end{dmath}
Subtracting off \( S \) and integrating by parts, leaves
\begin{dmath}\label{eqn:lorentzForceCovariant:n}
\delta S =
\int d\tau \epsilon \cdot \lr{
   \grad L - \frac{d}{d\tau} \grad_v L
}
+
\int d\tau \frac{d}{d\tau} (\grad_v L ) \cdot \epsilon.
\end{dmath}
This last integral can be evaluated
\begin{dmath}\label{eqn:lorentzForceCovariant:n}
\int d\tau \frac{d}{d\tau} (\grad_v L ) \cdot \epsilon
=
\evalbar{(\grad_v L ) \cdot \epsilon}{\Delta \tau} = 0.
\end{dmath}
This is zero since the variation \( \epsilon \) is required to vanish on the boundaries.  So, if \( \delta S = 0 \), we must have
\begin{dmath}\label{eqn:lorentzForceCovariant:n}
0 =
\int d\tau \epsilon \cdot \lr{
   \grad L - \frac{d}{d\tau} \grad_v L
},
\end{dmath}
for all \( \epsilon \).  Setting
\begin{equation}\label{eqn:lorentzForceCovariant:n}
\delta L = \grad L - \frac{d}{d\tau} (\grad_v L),
\end{equation}
shows that four-vector form of the Euler-Lagrange equations follow from \( \delta L = 0 \).

\section{Lorentz force equations.}
The starting point in all cases is the action
where
\begin{dmath}\label{eqn:lorentzForceCovariant:n}
L = \inv{2} m v^2 + q A \cdot v.
\end{dmath}

%}
\EndArticle
%\EndNoBibArticle
