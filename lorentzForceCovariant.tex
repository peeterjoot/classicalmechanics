%
% Copyright � 2020 Peeter Joot.  All Rights Reserved.
% Licenced as described in the file LICENSE under the root directory of this GIT repository.
%
%{
\input{../latex/blogpost.tex}
\renewcommand{\basename}{lorentzForceCovariant}
%\renewcommand{\dirname}{notes/phy1520/}
\renewcommand{\dirname}{notes/ece1228-electromagnetic-theory/}
%\newcommand{\dateintitle}{}
%\newcommand{\keywords}{}

\input{../latex/peeter_prologue_print2.tex}

% proof:
\usepackage{amsthm}

\usepackage{peeters_layout_exercise}
\usepackage{peeters_braket}
\usepackage{peeters_figures}
\usepackage{siunitx}
\usepackage{verbatim}
%\usepackage{mhchem} % \ce{}
%\usepackage{macros_bm} % \bcM
%\usepackage{macros_qed} % \qedmarker
%\usepackage{txfonts} % \ointclockwise

\beginArtNoToc

\generatetitle{Lagrangian for the Lorentz force equation.}
In my old classical mechanics notes it appears that I derived the Lorentz force equations a number of times, however,
none of these appear to have been done concisely.

\section{Euler-Lagrange equations.}
A relativistic action for a single particle system has the form
\begin{dmath}\label{eqn:lorentzForceCovariant:20}
S = \int d\tau L(x, \dot{x}),
\end{dmath}
where
\( x \) is the spacetime coordinate, \( \dot{x} = dx/d\tau \) is the four-velocity,
and \( \tau \) is proper time.
%\( A \) is the four-potential,

\maketheorem{Relativistic Euler-Lagrange equations.}{thm:lorentzForceCovariant:40}{
Let \( x \rightarrow x + \delta x \) be any variation of the Lagrangian four-vector coordinates, where
\( \delta x = 0 \) at the boundaries of the action integral.  The variation of the action is
\begin{equation*}
\delta S = \int d\tau \delta x \cdot \delta L(x, \dot{x}),
\end{equation*}
where
\begin{equation*}
\delta L = \grad L - \frac{d}{d\tau} (\grad_v L).
\end{equation*}
The action is extremized when \( \delta S = 0 \), or when \( \delta L = 0 \).  This latter condition is called the Euler-Lagrange equations.
} % theorem
\begin{proof}
Let \( \epsilon = \delta x \), and expand the 
Lagrangian in Taylor series to first order
\begin{dmath}\label{eqn:lorentzForceCovariant:60}
S \rightarrow
S + \delta S
= \int d\tau L( x + \epsilon, \dot{x} + \dot{\epsilon})
=
\int d\tau \lr{
   L(x, \dot{x}) + \epsilon \cdot \grad L + \dot{\epsilon} \cdot \grad_v L
}.
\end{dmath}
Subtracting off \( S \) and integrating by parts, leaves
\begin{dmath}\label{eqn:lorentzForceCovariant:80}
\delta S =
\int d\tau \epsilon \cdot \lr{
   \grad L - \frac{d}{d\tau} \grad_v L
}
+
\int d\tau \frac{d}{d\tau} (\grad_v L ) \cdot \epsilon.
\end{dmath}
The boundary integral
\begin{equation}\label{eqn:lorentzForceCovariant:100}
\int d\tau \frac{d}{d\tau} (\grad_v L ) \cdot \epsilon
=
\evalbar{(\grad_v L ) \cdot \epsilon}{\Delta \tau} = 0,
\end{equation}
is zero since the variation \( \epsilon \) is required to vanish on the boundaries.  So, if \( \delta S = 0 \), we must have
\begin{dmath}\label{eqn:lorentzForceCovariant:120}
0 =
\int d\tau \epsilon \cdot \lr{
   \grad L - \frac{d}{d\tau} \grad_v L
},
\end{dmath}
for all variations \( \epsilon \).  Clearly, this requires that
\begin{equation}\label{eqn:lorentzForceCovariant:140}
\delta L = \grad L - \frac{d}{d\tau} (\grad_v L) = 0,
\end{equation}
or
\begin{equation}\label{eqn:lorentzForceCovariant:145}
\grad L = \frac{d}{d\tau} (\grad_v L),
\end{equation}
which is the coordinate free statement of the Euler-Lagrange equations.
\end{proof}
\makeproblem{Coordinate form of the Euler-Lagrange equations.}{problem:lorentzForceCovariant:1}{
Working in coordinates, use the action argument show that the Euler-Lagrange equations have the form
\begin{equation*}
   \PD{x^\mu}{L} = \frac{d}{d\tau} \PD{\dot{x}^\mu}{L}
\end{equation*}
Observe that this is identical to the statement of \cref{thm:lorentzForceCovariant:40} after contraction with \( \gamma^\alpha \).
} % problem
\makeanswer{problem:lorentzForceCovariant:1}{
\begin{proof}
In terms of coordinates, the first order Taylor expansion of the action is
\begin{dmath}\label{eqn:lorentzForceCovariant:180}
S \rightarrow
S + \delta S
= \int d\tau L( x^\alpha + \epsilon^\alpha, \dot{x}^\alpha + \dot{\epsilon}^\alpha)
=
\int d\tau \lr{
L(x, \dot{x}) + \epsilon^\mu \PD{x^\mu}{L} + \dot{\epsilon}^\mu \PD{\dot{x}^\mu}{L}
}.
\end{dmath}
As before, we integrate by parts to separate out a pure boundary term
\begin{dmath}\label{eqn:lorentzForceCovariant:200}
\delta S =
\int d\tau \epsilon^\mu
\lr{
   \PD{x^\mu}{L} - \frac{d}{d\tau} \PD{\dot{x}^\mu}{L}
}
+
\int d\tau \frac{d}{d\tau} \lr{ 
   \epsilon^\mu \PD{\dot{x}^\mu}{L}
}.
\end{dmath}
The boundary term is killed since \( \epsilon = 0 \) at the end points of the action integral, so extremization of the action (\( \delta S = 0 \), for all \( \epsilon^\mu \)) requires
\begin{dmath}\label{eqn:lorentzForceCovariant:220}
   \PD{x^\mu}{L} - \frac{d}{d\tau} \PD{\dot{x}^\mu}{L} = 0.
\end{dmath}
\end{proof}
} % answer
%
%
%
\section{Lorentz force equations.}
The starting point in all cases is the action
where
\begin{dmath}\label{eqn:lorentzForceCovariant:160}
L = \inv{2} m v^2 + q A \cdot v.
\end{dmath}

%\section{Problem solutions.}
%\shipoutAnswer
%}
%\EndArticle
\EndNoBibArticle
