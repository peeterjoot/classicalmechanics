%
% Copyright � 2020 Peeter Joot.  All Rights Reserved.
% Licenced as described in the file LICENSE under the root directory of this GIT repository.
%
%{
\input{../latex/blogpost.tex}
\renewcommand{\basename}{lorentzForceCovariant}
%\renewcommand{\dirname}{notes/phy1520/}
\renewcommand{\dirname}{notes/ece1228-electromagnetic-theory/}
%\newcommand{\dateintitle}{}
%\newcommand{\keywords}{}

\input{../latex/peeter_prologue_print2.tex}

\PassOptionsToPackage{answerdelayed}{exercise}

% proof:
\usepackage{amsthm}

%\usepackage[thmmarks]{ntheorem}
%\theoremheaderfont{\bfseries}
%\theorembodyfont{\normalfont}
%\theoremseparator{}
%\theoremsymbol{$\blacksquare$}
%\newtheorem*{answerproof}{}

\usepackage{peeters_layout_exercise}
\usepackage{peeters_braket}
\usepackage{peeters_figures}
\usepackage{siunitx}
\usepackage{verbatim}
%\usepackage{mhchem} % \ce{}
%\usepackage{macros_bm} % \bcM
%\usepackage{macros_qed} % \qedmarker
%\usepackage{txfonts} % \ointclockwise

\beginArtNoToc

\generatetitle{Lagrangian for the Lorentz force equation.}
\section{Motivation.}
In my old classical mechanics notes it appears that I derived the Lorentz force equations a number of times, using different trial Lagrangians (relativistic and non-relativistic).
However,
none of these appear to have been done concisely, and a number not even coherently.
Here I'll cover the derivation from the relativistic Lagrangian and some associated concepts.  Geometric algebra in its STA (Space Time Algebra) form will be used.
\section{Conventions.}
\makedefinition{Index conventions.}{dfn:lorentzForceCovariant:23}{
Latin indexes \( i, j, k, r, s, t, \cdots \) are used to designate values in the range \( \setlr{ 1,2,3 } \).  Greek indexes are \( \alpha, \beta, \mu, \nu, \cdots \) are used for indexes of spacetime quantities \( \setlr{0,1,2,3} \).
Summation is implied when greek indexes mixed upper and lower greek indexes are used, for example
\begin{equation*}
x^\alpha x_\alpha \equiv \sum_{\alpha = 0}^3 x^\alpha x_\alpha.
\end{equation*}
} % definition
\section{Space Time Algebra (STA.)}
\makedefinition{Space Time Algebra.}{dfn:lorentzForceCovariant:17}{
Let the four-vector standard basis be designated
\( \setlr{\gamma_0, \gamma_1, \gamma_2, \gamma_3 } \), where
\( \gamma_0^2 = -\gamma_i^2 = 1\), and \( \gamma_\alpha \cdot \gamma_\beta = 0, \forall \alpha \ne \beta \).
%We use a \((+,-,-,-)\) basis
} % definition
\makedefinition{Pseudoscalar.}{dfn:lorentzForceCovariant:540}{
The pseudoscalar for the space is denoted \( I = \gamma_0 \gamma_1 \gamma_2 \gamma_3 \).
} % definition
\makeproblem{Pseudoscalar.}{problem:lorentzForceCovariant:99}{
% use parts.
Show that the STA pseudoscalar \( I \) defined by \cref{dfn:lorentzForceCovariant:17} satisfies
\begin{equation*}
\tilde{I} = I,
\end{equation*}
where the tilde operator designates reversion.
Also show that \( I \) has the properties of an imaginary number
\begin{equation*}
I^2 = -1.
\end{equation*}
Finally, show that, unlike the spatial pseudoscalar that commutes with all grades, \( I \) anticommutes with any vector or trivector, and commutes with any bivector.
} % problem
\makeanswer{problem:lorentzForceCovariant:99}{
Since
\( \gamma_\alpha \gamma_\beta = -\gamma_\beta \gamma_\alpha \)
for any
\( \alpha \ne \beta \)
, any permutation of the factors of \( I \) changes the sign once.  In particular
\begin{dmath}\label{eqn:lorentzForceCovariant:680}
I =
\gamma_0
\gamma_1
\gamma_2
\gamma_3
=
-
\gamma_1
\gamma_2
\gamma_3
\gamma_0
=
-
\gamma_2
\gamma_3
\gamma_1
\gamma_0
=
+
\gamma_3
\gamma_2
\gamma_1
\gamma_0
= \tilde{I}.
\end{dmath}
Using this, we have
\begin{dmath}\label{eqn:lorentzForceCovariant:700}
I^2
= I \tilde{I}
=
(
\gamma_0
\gamma_1
\gamma_2
\gamma_3
)(
\gamma_3
\gamma_2
\gamma_1
\gamma_0
)
=
\lr{\gamma_0}^2
\lr{\gamma_1}^2
\lr{\gamma_2}^2
\lr{\gamma_3}^2
=
(+1)
(-1)
(-1)
(-1)
= -1.
\end{dmath}
Commutation: TODO
} % answer
\makeproblem{Pseudoscalar commutation.}{problem:lorentzForceCovariant:47}{
} % problem
\makeanswer{problem:lorentzForceCovariant:47}{
   TODO.
} % answer
\makedefinition{Reciprocal frame.}{dfn:lorentzForceCovariant:19}{
It will be useful to define a reciprocal basis defined by \( \gamma^\alpha \cdot \gamma_\beta = \delta^\alpha_\beta \).
} % definition
%It is easy to show that \( \gamma^0 = \gamma_0 \) and \(
%TODO... lots more.
\subsection{Solutions.}
\shipoutAnswer
\section{Euler-Lagrange equations.}
I'll start at ground zero, with the derivation of the relativistic form of the Euler-Lagrange equations from the action.
A relativistic action for a single particle system has the form
\begin{dmath}\label{eqn:lorentzForceCovariant:20}
S = \int d\tau L(x, \dot{x}),
\end{dmath}
where
\( x \) is the spacetime coordinate, \( \dot{x} = dx/d\tau \) is the four-velocity,
and \( \tau \) is proper time.
%\( A \) is the four-potential,

\maketheorem{Relativistic Euler-Lagrange equations.}{thm:lorentzForceCovariant:40}{
Let \( x \rightarrow x + \delta x \) be any variation of the Lagrangian four-vector coordinates, where
\( \delta x = 0 \) at the boundaries of the action integral.  The variation of the action is
\begin{equation*}
\delta S = \int d\tau \delta x \cdot \delta L(x, \dot{x}),
\end{equation*}
where
\begin{equation*}
\delta L = \grad L - \frac{d}{d\tau} (\grad_v L),
\end{equation*}
where \( \grad = \gamma^\mu \partial_\mu \), and \( \grad_v = \gamma^\mu \partial/\partial \dot{x}^\mu \).

The action is extremized when \( \delta S = 0 \), or when \( \delta L = 0 \).  This latter condition is called the Euler-Lagrange equations.
} % theorem
\begin{proof}
Let \( \epsilon = \delta x \), and expand the
Lagrangian in Taylor series to first order
\begin{dmath}\label{eqn:lorentzForceCovariant:60}
S \rightarrow
S + \delta S
= \int d\tau L( x + \epsilon, \dot{x} + \dot{\epsilon})
=
\int d\tau \lr{
   L(x, \dot{x}) + \epsilon \cdot \grad L + \dot{\epsilon} \cdot \grad_v L
}.
\end{dmath}
Subtracting off \( S \) and integrating by parts, leaves
\begin{dmath}\label{eqn:lorentzForceCovariant:80}
\delta S =
\int d\tau \epsilon \cdot \lr{
   \grad L - \frac{d}{d\tau} \grad_v L
}
+
\int d\tau \frac{d}{d\tau} (\grad_v L ) \cdot \epsilon.
\end{dmath}
The boundary integral
\begin{equation}\label{eqn:lorentzForceCovariant:100}
\int d\tau \frac{d}{d\tau} (\grad_v L ) \cdot \epsilon
=
\evalbar{(\grad_v L ) \cdot \epsilon}{\Delta \tau} = 0,
\end{equation}
is zero since the variation \( \epsilon \) is required to vanish on the boundaries.  So, if \( \delta S = 0 \), we must have
\begin{dmath}\label{eqn:lorentzForceCovariant:120}
0 =
\int d\tau \epsilon \cdot \lr{
   \grad L - \frac{d}{d\tau} \grad_v L
},
\end{dmath}
for all variations \( \epsilon \).  Clearly, this requires that
\begin{equation}\label{eqn:lorentzForceCovariant:140}
\delta L = \grad L - \frac{d}{d\tau} (\grad_v L) = 0,
\end{equation}
or
\begin{equation}\label{eqn:lorentzForceCovariant:145}
\grad L = \frac{d}{d\tau} (\grad_v L),
\end{equation}
which is the coordinate free statement of the Euler-Lagrange equations.
\end{proof}
\makeproblem{Coordinate form of the Euler-Lagrange equations.}{problem:lorentzForceCovariant:1}{
Working in coordinates, use the action argument show that the Euler-Lagrange equations have the form
\begin{equation*}
   \PD{x^\mu}{L} = \frac{d}{d\tau} \PD{\dot{x}^\mu}{L}
\end{equation*}
Observe that this is identical to the statement of \cref{thm:lorentzForceCovariant:40} after contraction with \( \gamma^\mu \).
} % problem
\makeanswer{problem:lorentzForceCovariant:1}{
%\begin{answerproof}
In terms of coordinates, the first order Taylor expansion of the action is
\begin{dmath}\label{eqn:lorentzForceCovariant:180}
S \rightarrow
S + \delta S
= \int d\tau L( x^\alpha + \epsilon^\alpha, \dot{x}^\alpha + \dot{\epsilon}^\alpha)
=
\int d\tau \lr{
L(x^\alpha, \dot{x}^\alpha) + \epsilon^\mu \PD{x^\mu}{L} + \dot{\epsilon}^\mu \PD{\dot{x}^\mu}{L}
}.
\end{dmath}
As before, we integrate by parts to separate out a pure boundary term
\begin{dmath}\label{eqn:lorentzForceCovariant:200}
\delta S =
\int d\tau \epsilon^\mu
\lr{
   \PD{x^\mu}{L} - \frac{d}{d\tau} \PD{\dot{x}^\mu}{L}
}
+
\int d\tau \frac{d}{d\tau} \lr{
   \epsilon^\mu \PD{\dot{x}^\mu}{L}
}.
\end{dmath}
The boundary term is killed since \( \epsilon^\mu = 0 \) at the end points of the action integral.  We conclude that extremization of the action (\( \delta S = 0 \), for all \( \epsilon^\mu \)) requires
\begin{dmath}\label{eqn:lorentzForceCovariant:220}
   \PD{x^\mu}{L} - \frac{d}{d\tau} \PD{\dot{x}^\mu}{L} = 0.
\end{dmath}
%\end{answerproof}
} % answer
\subsection{Solutions.}
\shipoutAnswer
%
\section{Lorentz force equation.}
\maketheorem{Lorentz force.}{thm:lorentzForceCovariant:2}{
The relativistic Lagrangian for a charged particle is
\begin{equation*}
L = \inv{2} m v^2 + q A \cdot v/c.
\end{equation*}
Application of the Euler-Lagrange equations to this Lagrangian yields the Lorentz-force equation
\begin{equation*}
\frac{dp}{d\tau} = q F \cdot v/c,
\end{equation*}
where \( p = m v \) is the proper momentum, \( F \) is the Faraday bivector \( F = \grad \wedge A \), and \( c \) is the speed of light.
} % theorem
\begin{proof}
To make life easier, let's take advantage of the linearity of the Lagrangian, and break it into the free particle Lagrangian \( L_0 = (1/2) m v^2 \) and a potential term \( L_1 = q A \cdot v \).  For the free particle case we have
\begin{dmath}\label{eqn:lorentzForceCovariant:240}
\delta L_0
= \grad L_0 - \frac{d}{d\tau} (\grad_v L_0)
=           - \frac{d}{d\tau} (m v)
= - \frac{dp}{d\tau}.
\end{dmath}
For the potential contribution we have
\begin{dmath}\label{eqn:lorentzForceCovariant:260}
\delta L_1
= \grad L_1 - \frac{d}{d\tau} (\grad_v L_1)
= \frac{q}{c} \lr{ \grad (A \cdot v) - \frac{d}{d\tau} \lr{ \grad_v (A \cdot v)} }
= \frac{q}{c} \lr{ \grad (A \cdot v) - \frac{dA}{d\tau} }.
\end{dmath}
The proper time derivative can be evaluated using the chain rule
\begin{equation}\label{eqn:lorentzForceCovariant:280}
\frac{dA}{d\tau}
=
\frac{\partial x^\mu}{\partial \tau} \partial_\mu A
= (v \cdot \grad) A.
\end{equation}
Putting all the pieces back together we have
\begin{dmath}\label{eqn:lorentzForceCovariant:300}
0 =
\delta L
=
-\frac{dp}{d\tau} + \frac{q}{c} \lr{ \grad (A \cdot v) -  (v \cdot \grad) A }
=
-\frac{dp}{d\tau} + \frac{q}{c} \lr{ \grad \wedge A } \cdot v.
\end{dmath}
\end{proof}
\makeproblem{Gradient of a squared position vector.}{problem:lorentzForceCovariant:13}{
Show that
\begin{equation*}
   \grad (a \cdot x) = a,
\end{equation*}
and
\begin{equation*}
   \grad x^2 = 2 x.
\end{equation*}
Note that special cases of this were used above, in particular \( \grad_v (v^2) = 2 v \), and \( \grad_v (A \cdot v) = A \).
} % problem
\makeanswer{problem:lorentzForceCovariant:13}{
%\begin{answerproof}
The first identity follows easily by expansion in coordinates
\begin{dmath}\label{eqn:lorentzForceCovariant:320}
\grad (a \cdot x)
=
\gamma^\mu \partial_\mu a_\alpha x^\alpha
=
\gamma^\mu a_\alpha \delta_\mu^\alpha
=
\gamma^\mu a_\mu
=
a.
\end{dmath}
The second identity follows by linearity of the gradient
\begin{dmath}\label{eqn:lorentzForceCovariant:340}
\grad x^2
=
\grad (x \cdot x)
=
\evalbar{\lr{\grad (x \cdot a)}}{a = x}
+
\evalbar{\lr{\grad (b \cdot x)}}{b = x}
=
\evalbar{a}{a = x}
+
\evalbar{b}{b = x}
=
2x.
\end{dmath}
%\end{answerproof}
} % answer

It is desirable to put this relativistic Lorentz force equation into the usual vector and tensor forms for comparision.
% The tensor form is pretty easy to extract.
\maketheorem{Tensor form of the Lorentz force equation.}{thm:lorentzForceCovariant:360}{
The tensor form of the Lorentz force equation is
\begin{equation*}
\frac{dp^\mu}{d\tau} = \frac{q}{c} F^{\mu\nu} v_\nu,
\end{equation*}
where the antisymmetric Faraday tensor is defined as \( F^{\mu\nu} = \partial^\mu A^\nu - \partial^\nu A^\mu \).
} % theorem
\begin{proof}
We have only to dot both sides with \( \gamma^\mu \).  On the left we have
\begin{dmath}\label{eqn:lorentzForceCovariant:380}
\gamma^\mu \cdot \frac{dp}{d\tau}
=
\frac{dp^\mu}{d\tau}.
\end{dmath}
On the right, we have
\begin{dmath}\label{eqn:lorentzForceCovariant:400}
\gamma^\mu \cdot \lr{ \frac{q}{c} F \cdot v }
=
 \frac{q}{c}  (( \grad \wedge A ) \cdot v ) \cdot \gamma^\mu
=
 \frac{q}{c}  ( \grad ( A \cdot v ) - (v \cdot \grad) A ) \cdot \gamma^\mu
=
 \frac{q}{c}  \lr{ (\partial^\mu A^\nu) v_\nu - v_\nu \partial^\nu A^\mu }
=
 \frac{q}{c} F^{\mu\nu} v_\nu.
\end{dmath}
\end{proof}
\makeproblem{Lorentz force direct tensor derivation.}{problem:lorentzForceCovariant:420}{
Instead of using the geometric algebra form of the Lorentz force equation as a stepping stone, we may derive the tensor form from the Lagrangian directly, provided the Lagrangian is put into tensor form
\begin{equation*}
L = \inv{2} m v^\mu v_\mu + q A^\mu v_\mu /c.
\end{equation*}
Evaluate the Euler-Lagrange equations in coordinate form and compare to \cref{thm:lorentzForceCovariant:360}.
} % problem
%
\makeanswer{problem:lorentzForceCovariant:420}{
%\begin{answerproof}
Let \( \delta_\mu L = \gamma_\mu \cdot \delta L \), so that we can write the
Euler-Lagrange equations as
\begin{equation}\label{eqn:lorentzForceCovariant:460}
0 = \delta_\mu L = \PD{x^\mu}{L} - \frac{d}{d\tau} \PD{\dot{x}^\mu}{L}.
\end{equation}
Operating on the kinetic term of the Lagrangian, we have
\begin{dmath}\label{eqn:lorentzForceCovariant:480}
\delta_\mu L_0 = - \frac{d}{d\tau} m x_\mu.
\end{dmath}
For the potential term
\begin{dmath}\label{eqn:lorentzForceCovariant:500}
\delta_\mu L_1
=
\frac{q}{c} \lr{
v_\nu \PD{x^\mu}{A^\nu} - \frac{d}{d\tau} A_\mu
}
=
\frac{q}{c} \lr{
v_\nu \PD{x^\mu}{A^\nu} - \frac{dx_\alpha}{d\tau} \PD{x_\alpha}{ A_\mu }
}
=
\frac{q}{c} v^\nu \lr{
\partial_\mu A_\nu - \partial_\nu A_\mu
}
=
\frac{q}{c} v^\nu F_{\mu\nu}.
\end{dmath}
Putting the pieces together gives
\begin{dmath}\label{eqn:lorentzForceCovariant:520}
\frac{d}{d\tau} m x_\mu = \frac{q}{c} v^\nu F_{\mu\nu},
\end{dmath}
which is identical to the tensor form that we found by expanding the geometric algebra form of Maxwell's equation in coordinates (after some minor index raising and lowering gymnastics.)
%\end{answerproof}
} % answer

Next we want to establish correspondence with the traditional vector form of the Lorentz force equation.  Before doing so, we must discuss the representation of spatial vectors.
\makedefinition{Spatial basis.}{dfn:lorentzForceCovariant:580}{
The position of the origin of a fixed observer frame is defined as \( \gamma_0 c \tau \).  This is a four vector that moves in time but not space.  Relative to this frame we decompose any four vector \( x \) into it's timelike and spacelike components as follows
\begin{equation*}
\Bx = x \wedge \gamma_0,
\end{equation*}
and
\begin{equation*}
x_0 = x \cdot \gamma_0.
\end{equation*}
} % definition
Illustrating by example, let \( x = x^0 \gamma_0 + x^1 \gamma_1 + x^2 \gamma_2 + x^3 \gamma_3 \).  The timelike component is just \( x^0 \), but the spacelike component is
\begin{dmath}\label{eqn:lorentzForceCovariant:600}
\Bx =  x^1 \gamma_1 \gamma_0 + x^2 \gamma_2  \gamma_0+ x^3 \gamma_3 \gamma_0.
\end{dmath}
We see that spatial vectors are represented as bivectors in the STA, with a basis \( \setlr{ \gamma_1 \gamma_0, \gamma_2 \gamma_0, \gamma_3 \gamma_0 } \).  The reader can show that the identification \( \sigma_i = \gamma_i \gamma_0 \) is sensible, as these STA bivectors have all the characteristics of Pauli matrices.  However, since the usual standard (orthonormal) basis for \R{3} also has those characteristics in geometric algebra, we can utilize a more natural notation.
\makedefinition{Spatial basis.}{dfn:lorentzForceCovariant:77}{
We define \( \Be_i = \gamma_i \gamma_0 \) as the spatial basis for \R{3}.
} % definition
\makeproblem{Orthonormality of the spatial basis.}{problem:lorentzForceCovariant:33}{
   Show that the basis \( \setlr{ \Be_1, \Be_2, \Be_3 } \), with \( \Be_i \) defined by
\cref{dfn:lorentzForceCovariant:77}, is an orthonormal basis.
} % problem
\makeanswer{problem:lorentzForceCovariant:33}{
\begin{dmath}\label{eqn:lorentzForceCovariant:620}
\Be_i \cdot \Be_j
= \gpgradezero{ \gamma_i \gamma_0 \gamma_j \gamma_0 }
= -\gpgradezero{ \gamma_i \gamma_j }.
\end{dmath}
If \( i = j \) the grade zero selection above is
} % answer
, with \( \setlr{ \Be_1, \Be_2, \Be_3 } \) as the standard (othnonormal) basis

 be loose with notation and set \( \Be_i = \gamma_i \gamma_0 \), and dispense with the association with the Pauli matrix algebra.
\makeproblem{Spatial pseudoscalar.}{problem:lorentzForceCovariant:11}{
Show that the STA pseudoscalar, and the spatial pseudoscalar
} % problem
\makeanswer{problem:lorentzForceCovariant:11}{
} % answer
\makeproblem{Characteristics of the Pauli matrices.}{problem:lorentzForceCovariant:7}{
The Pauli matrices obey the following commutation relations:
\begin{equation}\label{eqn:lorentzForceCovariant:640}
\antisymmetric{ \sigma_a}{ \sigma_b } = 2 i \epsilon_{a b c}\,\sigma_c,
\end{equation}
and anticommutation relations:
\begin{equation}\label{eqn:lorentzForceCovariant:660}
\symmetric{ \sigma_a}{\sigma_b } = 2 \delta_{a b}.
\end{equation}
} % problem
\makeanswer{problem:lorentzForceCovariant:7}{
} % answer

\maketheorem{Vector Lorentz force equation.}{thm:lorentzForceCovariant:540}{
Relative to a fixed observer's frame, the Lorentz force equation of \cref{thm:lorentzForceCovariant:2} splits into a spatial rate of change of momentum, and (timelike component) rate of change of energy, as follows
\begin{equation*}
\begin{aligned}
   \ddt{(\gamma m \Bv)} &= q \lr{ \BE + \Bv \cross \BB } \\
   \ddt{(\gamma m c^2)} &= q \Bv \cdot \BE,
\end{aligned}
\end{equation*}
where \( F = \BE + I c \BB \), \( \gamma = 1/\sqrt{1 - \Bv^2/c^2 }\).
} % theorem
\begin{proof}
TODO: ...
\end{proof}
\subsection{Solutions.}
\shipoutAnswer
%}
%\EndArticle
\EndNoBibArticle
